% !TEX root = ../main.tex

\chapter{Challenges, Criticisms, and Future Directions}
\label{chp:discussion}

\begin{itemize}
	\item Relationships with ABC
	\item Reinvention of priors in deep learning by generating data
	\item Is amortization just a learning a different decomposition of the joint 
	$p(\theta | \mathcal{D}_1)p(\mathcal{D}_2|\theta)$ that shifts more to the
	prior to make inference easier?  Does it actually implicitly define a different
	joint distribution that is otherwise difficult to express or is it just proposal
	adaptation as Tuan Anh's paper says.
	\item Randomness vs no randomness - blog post for Michi
	\item Fully automated experimental design discussion.
	\item Sort out PPS vs PPL and PPS vs PPSs
	\item Add reference somewhere to the paper with Andrew
\end{itemize}

\section{Do we need Random Numbers?}
	
Random numbers are a fundamental tool in the arsenal of all the mathematical sciences, especially in the realm of 
probabilistic numerics.  From basic cross-validation to advanced MCMC methods, randomness is at the core of 
many of our most prevalent and highest performance algorithms.  In his related post, Michael will 
try to convince you why we should eventually try and do away this randomness; here I will do my best to explain why
\texttt{if(rand<1-$\varepsilon$)\{we should not\}}.

My argument can be broken down into four main reasons that we need, and always will need, random numbers:
\begin{enumerate}
	\item Speed and simplicity - sometimes even if there if there is the information available to improve a computation, 
	uncovering or incorporating that information in a principled manner may require substantially more computational 
	power than not doing so.
	\item Honesty and reliability - the alternative to randomness is almost always approximation, introducing bias 
	and an error whose uncertainty is inherently subjective.
	\item Compossibility - methods such as Monte Carlo are ambivalent to how the samples are to be used, creating 
	a scalable ability to modularize, compose and reuse.  Non-compossible systems, on the other hand, are doomed to fail
	for large frameworks through the curse of repeated nested of estimators~\citep{rainforth2017pitfalls}
	\item Lack of repeatability - though it is always convenient when an experiment is repeatable, systematically 
	returning the same incorrect answer is far more dangerous.
\end{enumerate}
Of these, the speed and simplicity argument is perhaps the most commonly used to support the use of random 
numbers and is certainly a core reason for their prevalence.  However, I believe it is the other three, and in 
particular the honesty of random numbers, that is most critical for why random numbers are, will remain to be, 
fundamental to numerical computation.  All of these arguments are related to the following principle:

\emph{Once we have imparted all possible information upon a system, we must treat what is left as truly 
	stochastic or introduce bias.}

In other words, imagine we can construct a problem in a manner that utilizes all available prior information,
both in terms of the model and in how best to solve the model.
Any remaining uncertainty is now, at least for all practical purposes, inherent to the problem and so any
further attempt to remove randomness from the system is indirectly adding in further prior information we
did not want to impart.  Furthermore, when things we kept random, we can gauge the error in our approximation
or the level of our uncertainty by simply rerunning the system.  As soon as we replace this with an approximation
or fix the random number seed, we lose calibration of the uncertainty in our estimate because we can estimate the variance
resulting from randomness, but typically not the bias resulting from approximation.

Putting our frequentist hat on for the moment, then the notion of doing away with randomness is scary.
Frequentist statistics is all about \emph{repeatability} and focuses on the fact that any data we collect is random
in a many-worlds point of view,
because even with the same ground-truth underlying generative process, the inherent randomness
of the universe (or at least effective randomness if you have philosophical objects to this) means that multiple possible
datasets \emph{could} have been generated.  Regardless of your philosophical position on the Bayesian vs frequentist divide,
there are many times when such a frequentist approach is absolutely essential from a practical viewpoint.  For example, in medical tasks we need
to make sure that our confidence intervals are actually correct when we repeat a procedure -- we need guarantees on
their \emph{calibration}.  This simply is not possible in a Bayesian framework as by its very nature it presumes the data
is fixed and thus permits no concept of repetition.  Once we realize that we sometimes need to take a frequentist
approach, the need for randomness becomes obvious, as in the frequentist framework probability originates only
through random variables.  In other words, as powerful as the Bayesian framework is, it is inherently optimistic and subjective
(e.g. there are always unknown unknowns), and as such it can never really be the whole story.  If we try and
do away with randomness completely it is impossible to be objective -- we have to choose our approximation, sometimes
in an arbitrary fashion (e.g. by setting the random number seed) -- and so we can never provide properly calibrated confidence 
intervals that are not prone to subjective interpretation.

This honesty of Monte Carlo goes beyond specific applications.  One of its most powerful consequences is the compossibility
of Monte Carlo estimates: give a sample from a marginal, I can generate a valid sample of the joint by sampling from the
conditional.  Similarly, if I have samples from a joint, I also have samples from the marginal distributions.  This behavior
is essential when composing different components into a greater system as one can ensure each component takes in
samples as inputs and outputs corresponding samples as output that can be used by the next process in the pipeline.
If we instead make some deterministic approximation at each stage, we flaunt the flaw of averages and our biases
will conflagulate to give a result that might be substantially different to the truth.  Sticking with unbiased Monte Carlo estimation
(e.g. using importance sampling or sequential Monte Carlo~\citep{doucet2009tutorial}) then although it is of course possible that our 
variance will explode, we at least know that if our estimate has a poor accuracy.  If we do away with randomness,
we might generate large errors without even knowing we have.

This lack of repeatability is actually an essential advantage of Monte Carlo.  It is better for a system to give you
a different wrong answer each time it is queried than to always give the same wrong answer, particularly if the average
of the different answers is in fact the truth, as is often the case.  When we write scientific papers we run our experiments
multiple times to show the variability in the results.  Systems that always give the same answer and instead return
a single subjective uncertainty estimate (e.g. a Gaussian process) are not generally trustworthy because there is no
true calibration or sense of whether the experimenter simply got lucky with a system that does not actually work well
in practice.  Furthermore, such a setting it very open to intentional or unintentional abuse.  If we fix our randomness
a priori and then adapt our algorithm until it works, we may well simply be over-fitting to what works best for that
random number seed or approximation.  It horrifies me that some people fix their random number seeds when 
tunning or improving an algorithm and I expect that this practice has actually led to a plethora of statistically incorrect
results.  If we want to maintain scientific integrity then we have to report results on experiments that have not been
tested during the design of our algorithm and which are tested with multiple different choices for arbitrary or subjective
decisions, to show that those results are actually stable and representative.  If we do away with randomness, we lose, or at the very least
seriously hamper, this ability to repeat experiments in an objective manner.

To summarize, we need randomness because we do not know everything.  In the Bayesian framework we place distributions
on what we are not sure about to reflect this lack of knowledge.  A key part of this process is acknowledging that once we
have done this, what is left is random.  After all, this uncertainty is where the Bayesian definition of probability actually comes
from and what does it mean to have probability without randomness?  Yes we might be able to impart more knowledge on
problems that we currently do, such as by decorrelating our samples, but there always become a point where we have
imparted everything we know, after which what is left is truly random by the Bayesian definition of
probability.  If we go outside the Bayesian framework, the need for randomness becomes even more pronounced.  In
frequentist settings even our data is a random variable and probability is defined through variations in repeating an
experiment.  In short, regardless of whether we take a Bayesian or frequentist viewpoint, arguing for the 
removal of all randomness from our system is tantamount to arguing that we should do away with probability
theory entirely.
% !TEX root = ../main.tex

\chapter{Discussions, Conclusions, and Future Directions}
\label{chp:discussion}

In this thesis we have introduced a \emph{Bayesian} approach to machine learning and demonstrated
how \emph{probabilistic programming systems} (PPSs) provide an exciting framework for
encoding rich and expressive models.  We have shown how by \emph{automating inference} for
arbitrary models and allowing models to be written in a manner more in line with conventional
scientific coding, PPS also make effective statistical methods accessible to non-experts.
We highlighted the importance of taking such an approach in scenarios where one has
access to substantial \emph{prior knowledge} or domain specific expertise, but noted that taking
\emph{discriminative} machine learning approaches can be more effective, and typically substantially
easier to implement, when one has access to large quantities of data but little prior information.
When opting for a Bayesian approach, we explained that thee key challenges are in writing good models
and solving the resultant \emph{Bayesian inference} problem needed to characterize the posterior.
PPSs can in some ways help in addressing both of these challenges, providing an expressive framework
that allows users to write models true to their assumptions and removing the burden of inference
from the user.  However, in many ways, PPSs actually hamper the Bayesian inference problem, because
they require developers to design inference engines capable of working on any model.
 This problem is especially pronounced
for so-called \emph{universal} PPSs, which, as we explained, place almost no restrictions on the
models the user can write.  Consequently,
universal PPSs will rarely produce state-of-the-art performance on any particular task, though
by directly using the source code of the target, they are still often able to exploit many salient
features of models and thus still provide effective inference for wide array of tasks.
  Their main
utility is thus in \emph{automation} and in allowing models that would difficult, or even impossible,
to encode otherwise.  In particular, they allow us to go beyond
the confines of conventional Bayesian  modeling, allowing, for example, one to \emph{nest} 
inference problems within one another.

There have been two core thrusts to the novel contributions of this thesis: improving the inference
for PPSs and extending the range of problems to which they can be applied.  For
the former, we introduced the \emph{interacting particle Markov chain Monte Carlo} (iPMCMC)
algorithm~\citep{rainforth2016interacting}, a particle based inference method delivering
better per-sample performance than existing methods for a given memory budget, while also introducing the opportunity for
parallelization.  We then showed how iPMCMC is suitable for universal PPSs, detailing its implementation
in Anglican.  Towards the aim of increasing the range of problems to which PPSs can be applied, 
we have firstly introduced \emph{Bayesian optimization for probabilistic 
	programs} (BOPP)~\citep{rainforth2016bayesian}, providing
the first framework for mixed inference-optimization problems in probabilistic programs, allowing things
such as model learning and principled engineering design to be automated in the same manner a inference.
BOPP also provides a significant contribution to the Bayesian optimization literature, constituting a
convenient and efficient package for solving conventional optimization problems that exploits the
source code of its target to provide innovations in problem-independent hyperpriors, unbounded
 optimization, and implicit constraint satisfaction.
We have further undertaken theoretical work looking into the convergence of \emph{nested Monte Carlo} (NMC)~\citep{rainforth2017pitfalls}, for which \emph{nested inference} problems are an
example use case.   We demonstrated that the convergence
of NMC is possible, but that this some additional assumptions need
to be satisfied, e.g. the number of samples used for \emph{each} inner estimate needs to be driven
to infinity.  Our results go beyond the cases considered
by previous results and, in particular, apply to cases of repeated nesting, as can occur in PPSs.
We then carefully considered the implications of our results for PPSs and provided recommendations
about how to ensure consistency is maintain.  Our final contribution to extending PPSs was in
applying our theoretical results from nested estimation to consider the specific problem of Bayesian
experimental design, deriving an improved estimator for discrete problems and outlining a framework
for automating the solution of sequential design problems, as might be experienced in, for example,
psychological trials.  Our technical innovations were applied to automating
a specific class of problems relating to delay and probability discounting experiments by
introducing the DARC toolbox, perhaps providing a first step in demonstrating how one might
be able to construct a general purpose PPS for automating sequential design problems.

A natural question is now where do we go next?  In addition to the obvious ongoing challenges
of improving inference and range of problems covered by PPS, perhaps the most clear cut 
extensions for our
work is in its direct application.  We have introduced various frameworks which, while someway off
our lofty long term aims of tractability and automatically solving almost any problem the 
user might write, are nonetheless clearly already useful for a number of existing applications.  
More generally, the biggest
potential impact of PPSs is arguably not in what they provide for seasoned machine learning or 
statistics researchers, but in the democratization of Bayesian approaches to anybody with sufficient
scientific programming background to write a stochastic simulator.  The best example of this is
perhaps the BUGS system~\cite{spiegelhalter1996bugs}, which has seen effective widespread use across
many fields.
More recently, Stan~\cite{carpenter2015stan} and PyMC~\citep{salvatier2016probabilistic} 
are seeing particularly widespread usage that extends well beyond
the machine learning community.  Though some universal PPSs are already relatively 
mature and user-friendly with a small base of users, they are arguably still more
at the stage of predominantly being a framework for probabilistic programming
research than a commonly used tool by applied communities.  Perhaps this is not surprising given
their inference engines inevitably tend to be less efficient than more inference-driven packages.
However, we would argue that as inference engines improve and people's modeling ambitions
increase, the flexibility that such systems provide will become increasing important,
particularly if we want to achieve our long term aims regarding arbitrary simulators.
Existing applications requiring this flexibility are already to be found in the form of Bayesian
nonparametrics~\citep{dhir2017interpreting} and theory-of-mind style 
modeling~\citep{stuhlmuller2014reasoning}.

We now finish by outlining a number of more specific possible future directions for research
and by briefly discussing some interesting high level questions.

\section{A General Purpose Experimental Design Toolbox}
\label{sec:disc:design}

One direct possible extension of our work would be to develop the framework we introduced 
in Chapter~\ref{chp:design} to a general purpose tool for automating sequential design problems.
The potential applications for such a system would be huge,  ranging from designing clinical trials to
smart online surveys and classic autonomous agent problems.  In  the same way all Bayesian modeling
problems can be reduced to a common inference framework, such problems can be reduced to a
common Bayesian experimental design framework, suggesting that they are highly suited to a
probabilistic programming style approach.  However, their are a number of reasons existing systems
are not suited to such problems.  Firstly, such problems require concurrent inference 
an optimization.  Our introduction of BOPP could be a big help for overcoming this problem,
but its computationally intensive nature means that it is perhaps not suitable for
some scenarios, e.g. the psychological trials we considered in Section~\ref{sec:design:darc},
for which the approach we outlined in Section~\ref{sec:design:seq}, or an extension thereof,
might be more suitable.  If BOPP were to be used, suitable adjustments would also be necessary
to account for \emph{sequential} design problems.
Secondly, Bayesian experiment design problems, in general, require the calculation of a nested 
estimation.  For discrete outputs, we showed how a superior estimator can be derived and so
it would be important to exploit this when possible in any hypothetical system.  For continuous problems that
cannot be rearranged, then as we explained in Section~\ref{sec:nest:imp}, we need to be careful
to avoid the pitfalls of nested estimation, for which existing systems potentially need suitable
adaptation.  It is may also to be beneficial to develop application specific estimation schemes
for this setting, potentially exploiting the known form of the nonlinear mapping to
devise methods for bias reduction.  Finally, to make such a system user friendly, it would
be important to provide a convenient means of encoding a suitable design space alongside the
model.  It might, for example, be possible to do in a similar manner as constraints were enforced
in BOPP, noting that each each iteration in a sequential design is effectively a maximum marginal likelihood
problem.

\section{Amortizing Inference}

An interesting recent advancement in PPSs is the idea of \emph{amortizing} the process of 
inference~\citep{paige2016inference,ritchie2016deep,le2017inference}.  At a high level, 
that we often repeatedly use the same models for multiple different datasets and therefore
it can be wasteful to attempt to do inference from scratch each time.  Instead, one can, before any data is
actually seen, look to learn a regressor from the space of data to, for example, proposals, such that
when any particular dataset is actually observed, one can use this regressor to produce an
artifact to assist with the resulting inference.  To do this, methods exploit the fact that generative
approaches specify a distribution over both data and parameters, such that it is theoretically possible
to learn the form of the posterior as a function of the dataset by sampling parameter-data pairs from
the joint distribution and using these to train a regressor from data to some characterization of
parameters.  In practice, doing this is can be very challenging -- learning a posterior for any possible
dataset is inevitably far harder than learning one for a particular dataset -- but in applications where
the same model is repeatedly used, the additional cost can be easily justified.  Current methods share
a number of weaknesses, such as being restricted by the fidelity of the underlying regressor, mostly
only allowing importance sampling for the final inference, and being potentially even more sensitive to
model misspecification than conventional Bayesian approaches.  Nonetheless, they represent a very interesting avenue
for future research and offer the potential for substantial long term impact.  In particular, as current methods
are already effective at amortizing inference for low dimensional problems, the utility of such approaches
might be dramatically increased if one can develop methods for breaking an overall inference problem down
into an number of smaller problems that can be amortized separately and then usefully combined at run time.

\section{Shouldn't we All Just be Using Deep Learning?}

No.  As successful as deep learning, or artificial neural network (ANN) training as it was known before the hype,
has been in recent years, it is not a panacea for machine learning.  Instead, it is a particular discriminative
machine learning approach that has been particularly successful on certain kinds of machine learning tasks,
namely those with huge quantities of, usually high dimensional, data with rich underlying structure.   
To suggest that ANNs are universally dominant even among discriminative machine learning tasks is
somewhat preposterous, e.g. they are often, if not usually, inferior to Gaussian processes on low dimensional,
small-data regime, tasks, and still arguably trail behind decision tree ensemble approaches for ``out-of-the-box''
usage on, for example, generic classification tasks~\citep{rainforth2015canonical}.   Perhaps more significantly,
many problems call for a generative, rather than discriminative, approach,
particularly when data is scarce or significant prior information is available as we explained
in Section~\ref{sec:bayes:discrim}.  Interestingly, however, this is perhaps
where a lot of the success for ANNs actually originates.  Compared to say random forests, ANNs constitute a very
flexible and composible framework for discriminative machine learning and perhaps provide a means of 
indirectly imposing prior knowledge on the ANN through its structuring.  Arguably, a major weakness of
declarative generative approaches, such as those employed by PPSs, is that we almost
always have to impose more assumptions on the model then we would like -- all models will inevitably be
misspecified.  For ANNs, on the other hand, when there is not an abundance of training data,
 one often struggles to impose enough assumptions, sometimes
for example, resorting to generating synthetic data to train the network.  Doing this is, in effect, a somewhat convoluted,
though often very effective, method for imposing prior information on the model~\citep{le2017using}.  From this
perspective, the Bayesian approach, and in particular probabilistic programming, is perhaps also of significant
consequence to the ANN literature, as for example characterized  by the rise of so called 
``Bayesian deep-learning''~\citep{gal2016uncertainty}.
  In particular, there may be a bright future for methods which
take a model based approach, but without having to assumptions as strict and prone to misspecification
as conventional Bayesian modeling, for which ANNs are likely to be highly useful tool~\citep{siddharth2017learning}.

\section{Do we need Random Numbers?}
	
Random numbers are a fundamental tool in the arsenal of all the mathematical sciences, especially in the realm of 
probabilistic numerics.  From basic cross-validation to advanced MCMC methods, randomness is at the core of 
many of our most prevalent and highest performance algorithms.  In his related post, Michael will 
try to convince you why we should eventually try and do away this randomness; here I will do my best to explain why
\texttt{if(rand<1-$\varepsilon$)\{we should not\}}.

My argument can be broken down into four main reasons that we need, and always will need, random numbers:
\begin{enumerate}
	\item Speed and simplicity - sometimes even if there if there is the information available to improve a computation, 
	uncovering or incorporating that information in a principled manner may require substantially more computational 
	power than not doing so.
	\item Honesty and reliability - the alternative to randomness is almost always approximation, introducing bias 
	and an error whose uncertainty is inherently subjective.
	\item Compossibility - methods such as Monte Carlo are ambivalent to how the samples are to be used, creating 
	a scalable ability to modularize, compose and reuse.  Non-compossible systems, on the other hand, are doomed to fail
	for large frameworks through the curse of repeated nested of estimators~\citep{rainforth2017pitfalls}
	\item Lack of repeatability - though it is always convenient when an experiment is repeatable, systematically 
	returning the same incorrect answer is far more dangerous.
\end{enumerate}
Of these, the speed and simplicity argument is perhaps the most commonly used to support the use of random 
numbers and is certainly a core reason for their prevalence.  However, I believe it is the other three, and in 
particular the honesty of random numbers, that is most critical for why random numbers are, will remain to be, 
fundamental to numerical computation.  All of these arguments are related to the following principle:

\emph{Once we have imparted all possible information upon a system, we must treat what is left as truly 
	stochastic or introduce bias.}

In other words, imagine we can construct a problem in a manner that utilizes all available prior information,
both in terms of the model and in how best to solve the model.
Any remaining uncertainty is now, at least for all practical purposes, inherent to the problem and so any
further attempt to remove randomness from the system is indirectly adding in further prior information we
did not want to impart.  Furthermore, when things we kept random, we can gauge the error in our approximation
or the level of our uncertainty by simply rerunning the system.  As soon as we replace this with an approximation
or fix the random number seed, we lose calibration of the uncertainty in our estimate because we can estimate the variance
resulting from randomness, but typically not the bias resulting from approximation.

Putting our frequentist hat on for the moment, then the notion of doing away with randomness is scary.
Frequentist statistics is all about \emph{repeatability} and focuses on the fact that any data we collect is random
in a many-worlds point of view,
because even with the same ground-truth underlying generative process, the inherent randomness
of the universe (or at least effective randomness if you have philosophical objects to this) means that multiple possible
datasets \emph{could} have been generated.  Regardless of your philosophical position on the Bayesian vs frequentist divide,
there are many times when such a frequentist approach is absolutely essential from a practical viewpoint.  For example, in medical tasks we need
to make sure that our confidence intervals are actually correct when we repeat a procedure -- we need guarantees on
their \emph{calibration}.  This simply is not possible in a Bayesian framework as by its very nature it presumes the data
is fixed and thus permits no concept of repetition.  Once we realize that we sometimes need to take a frequentist
approach, the need for randomness becomes obvious, as in the frequentist framework probability originates only
through random variables.  In other words, as powerful as the Bayesian framework is, it is inherently optimistic and subjective
(e.g. there are always unknown unknowns), and as such it can never really be the whole story.  If we try and
do away with randomness completely it is impossible to be objective -- we have to choose our approximation, sometimes
in an arbitrary fashion (e.g. by setting the random number seed) -- and so we can never provide properly calibrated confidence 
intervals that are not prone to subjective interpretation.

This honesty of Monte Carlo goes beyond specific applications.  One of its most powerful consequences is the compossibility
of Monte Carlo estimates: give a sample from a marginal, I can generate a valid sample of the joint by sampling from the
conditional.  Similarly, if I have samples from a joint, I also have samples from the marginal distributions.  This behavior
is essential when composing different components into a greater system as one can ensure each component takes in
samples as inputs and outputs corresponding samples as output that can be used by the next process in the pipeline.
If we instead make some deterministic approximation at each stage, we flaunt the flaw of averages and our biases
will conflagulate to give a result that might be substantially different to the truth.  Sticking with unbiased Monte Carlo estimation
(e.g. using importance sampling or sequential Monte Carlo~\citep{doucet2009tutorial}) then although it is of course possible that our 
variance will explode, we at least know that if our estimate has a poor accuracy.  If we do away with randomness,
we might generate large errors without even knowing we have.

This lack of repeatability is actually an essential advantage of Monte Carlo.  It is better for a system to give you
a different wrong answer each time it is queried than to always give the same wrong answer, particularly if the average
of the different answers is in fact the truth, as is often the case.  When we write scientific papers we run our experiments
multiple times to show the variability in the results.  Systems that always give the same answer and instead return
a single subjective uncertainty estimate (e.g. a Gaussian process) are not generally trustworthy because there is no
true calibration or sense of whether the experimenter simply got lucky with a system that does not actually work well
in practice.  Furthermore, such a setting it very open to intentional or unintentional abuse.  If we fix our randomness
a priori and then adapt our algorithm until it works, we may well simply be over-fitting to what works best for that
random number seed or approximation.  It horrifies me that some people fix their random number seeds when 
tunning or improving an algorithm and I expect that this practice has actually led to a plethora of statistically incorrect
results.  If we want to maintain scientific integrity then we have to report results on experiments that have not been
tested during the design of our algorithm and which are tested with multiple different choices for arbitrary or subjective
decisions, to show that those results are actually stable and representative.  If we do away with randomness, we lose, or at the very least
seriously hamper, this ability to repeat experiments in an objective manner.

To summarize, we need randomness because we do not know everything.  In the Bayesian framework we place distributions
on what we are not sure about to reflect this lack of knowledge.  A key part of this process is acknowledging that once we
have done this, what is left is random.  After all, this uncertainty is where the Bayesian definition of probability actually comes
from and what does it mean to have probability without randomness?  Yes we might be able to impart more knowledge on
problems that we currently do, such as by decorrelating our samples, but there always become a point where we have
imparted everything we know, after which what is left is truly random by the Bayesian definition of
probability.  If we go outside the Bayesian framework, the need for randomness becomes even more pronounced.  In
frequentist settings even our data is a random variable and probability is defined through variations in repeating an
experiment.  In short, regardless of whether we take a Bayesian or frequentist viewpoint, arguing for the 
removal of all randomness from our system is tantamount to arguing that we should do away with probability
theory entirely.


\begin{itemize}
	\item Relationships with ABC
	\item Reinvention of priors in deep learning by generating data
	\item Is amortization just a learning a different decomposition of the joint 
	$p(\theta | \mathcal{D}_1)p(\mathcal{D}_2|\theta)$ that shifts more to the
	prior to make inference easier?  Does it actually implicitly define a different
	joint distribution that is otherwise difficult to express or is it just proposal
	adaptation as Tuan Anh's paper says.
	\item Fully automated experimental design discussion.
	\item Sort out PPS vs PPL and PPS vs PPSs
	\item Add reference somewhere to the paper with Andrew
	\item Frequentist probabilistic programming
	\item Learning intermediate targets in SMC
\end{itemize}
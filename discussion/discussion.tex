% !TEX root = ../main.tex

\chapter{Discussions, Conclusions, and Future Directions}
\label{chp:discussion}

In this thesis, we have introduced a \emph{Bayesian} approach to machine learning and demonstrated
how \emph{probabilistic programming systems} (PPSs) provide an exciting framework for
encoding rich and expressive models in the Bayesian framework and then automating the process
of drawing inferences from these models.  We explained how the 
long-term
aim of probabilistic programming is to automate the process of drawing inferences from any simulator
the user might write, thereby streamlining the modeling process and breaking down barriers to carrying out such
analysis.  We have made steps towards this eventual aim by improving the inference
engines used by PPSs and expanding the probabilistic programming framework beyond its conventional
Bayesian inference setting to include, for example, 
mixed inference-optimization problems, thus
allowing automation of tasks such as model learning and engineering design.

We highlighted the importance of taking a Bayesian approach in scenarios where one has
access to substantial \emph{prior knowledge} or domain-specific expertise, but noted that
\emph{discriminative} machine learning approaches can be more effective, and typically substantially
easier to implement, when one has access to large quantities of data, but little prior information.
When opting for a Bayesian approach, we explained that two key challenges are in writing good models
and solving the resultant \emph{Bayesian inference} problem.
PPSs can help in addressing both of these challenges.  Firstly, they provide an expressive framework
that allows users to write models true to their assumptions in a manner more in line with conventional
scientific coding.  Secondly, they remove the burden of inference from the user by providing general
purpose \emph{inference engines} capable of operating on arbitrary models.
Together these mean that PPSs make effective statistical methods accessible to non-experts, while
streamlining the process of writing models and inference algorithms for seasoned researchers.
However, from a developers perspective, designing inference engines for PPSs is challenging because of
 the need for them to work on any model.
 This problem is especially pronounced
for so-called \emph{universal} PPSs, which, as we explained, place almost no restrictions on the
models the user can write.  Consequently,
universal PPSs will rarely produce state-of-the-art performance on any particular task, though
by directly using the source code of the target, they are still often able to exploit many salient
features of models and thus still provide effective inference for wide array of tasks.
  Their main
utility is thus in \emph{automation} and in allowing models that would difficult, or even impossible,
to encode otherwise.  In particular, they allow us to go beyond
the confines of conventional Bayesian  modeling, allowing, for example, one to \emph{nest} 
inference problems within one another.

There have been two core thrusts to the novel contributions of this thesis: improving the inference engines
for PPSs and extending the range of problems to which they can be applied.  For
the former, we introduced the \emph{interacting particle Markov chain Monte Carlo} (iPMCMC)
algorithm~\citep{rainforth2016interacting}, a particle-based inference method delivering
better per-sample performance than existing methods for a given memory budget, while also introducing the opportunity for
parallelization.  We then showed how iPMCMC is suitable for inference in universal PPSs, detailing its implementation
in the PPS \emph{Anglican}~\citep{tolpin2016design}.  Towards the aim of increasing the range of problems 
to which PPSs can be applied, 
we have firstly introduced \emph{Bayesian optimization for probabilistic 
	programs} (BOPP)~\citep{rainforth2016bayesian}, providing
the first framework for mixed inference-optimization problems in probabilistic programs, allowing things
such as model learning and principled engineering design to be automated in the same manner as inference.
BOPP also provides a significant contribution to the Bayesian optimization literature, constituting a
convenient and efficient package for solving conventional optimization problems by exploiting the
source code of its target to provide innovations in problem-independent hyperpriors, unbounded
 optimization, and implicit constraint satisfaction.
We have further undertaken theoretical work looking into the convergence of \emph{nested Monte Carlo} (NMC)~\citep{rainforth2017pitfalls}, for which \emph{nested inference} problems are a particular example.
We demonstrated that the convergence
of NMC is possible, but that some additional assumptions need
to be satisfied, e.g. the number of samples used for \emph{each} inner estimate needs to be driven
to infinity.  Our results go beyond the cases considered
by previous results and, in particular, apply to cases of repeated nesting, as can occur in PPSs.
We then carefully considered the implications of our results for PPSs and provided recommendations
about how to ensure consistency is maintained.  Our final contribution to extending PPSs was in
applying our theoretical results from nested estimation to consider the specific problem of Bayesian
experimental design, deriving an improved estimator for discrete problems and outlining a framework
for automating the solution of adaptive sequential design problems, as might be experienced in, for example,
psychological trials.  Our technical innovations were applied to automating
a specific class of problems relating to delay and probability discounting experiments by
introducing the DARC toolbox, perhaps providing a first step in demonstrating how one might
be able to construct a general purpose language for automating adaptive design problems.

A natural question is now where do we go next?  In addition to the obvious ongoing challenges
of improving inference and range of problems covered by PPSs, perhaps the most clear-cut 
extensions for our
work is in its direct application.  We have introduced various frameworks which, while someway off
our lofty long-term aims of tractably and automatically solving almost any problem the 
user might write, are nonetheless clearly already useful for a number of existing applications.  
More generally, the biggest
potential impact of PPSs is arguably not in what they provide for seasoned machine learning or 
statistics researchers, but in the democratization of advanced statistical approaches to anybody with sufficient
scientific programming background to write a stochastic simulator.  The best example of this is
perhaps the BUGS system~\cite{spiegelhalter1996bugs}, which has seen effective widespread use across
many fields.
More recently, Stan~\citep{carpenter2015stan} and PyMC~\citep{salvatier2016probabilistic} 
are seeing particularly widespread usage that extends well beyond
the machine learning community.  Though some universal PPSs are already relatively 
mature and user-friendly with a small base of users, they are arguably still more
at the stage of predominantly being a framework for probabilistic programming
research, than a commonly used tool by applied communities.  Perhaps this is not surprising given
their inference engines inevitably tend to be less efficient than more inference-driven packages.
However, we believe that as inference engines improve and people's modeling ambitions
increase, the flexibility that such systems provide will become increasingly important,
particularly if we want to achieve our long-term aims regarding arbitrary simulators.
Existing applications requiring this flexibility are already to be found in the form of Bayesian
nonparametrics~\citep{dhir2017interpreting} and theory-of-mind style 
modeling~\citep{stuhlmuller2014reasoning}.

We now finish by briefly discussing some interesting high-level questions and 
outlining some more specific possible future directions for research. 

\section{Shouldn't We Just Use Deep Learning?}

In a word, no.  As successful as deep learning, or artificial neural network (ANN) training~\citep{bishop1995neural} 
as it was known before the hype,
has been in recent years, it is not a panacea for machine learning.  Instead, it is a particular discriminative
machine learning approach that has been especially successful on certain kinds of machine learning tasks,
namely those with huge quantities of, usually high dimensional, data with rich underlying structure.   
To suggest that ANNs are universally dominant even among discriminative machine learning tasks is
somewhat preposterous, e.g. they are often, if not usually, inferior to Gaussian processes on low-dimensional,
small-data regime, tasks, and still arguably trail behind decision tree ensemble approaches for ``out-of-the-box''
usage on, for example, generic classification tasks~\citep{rainforth2015canonical}.   Perhaps more significantly,
many problems call for a generative, rather than discriminative, approach,
particularly when data is scarce or significant prior information is available as we explained
in Section~\ref{sec:bayes:discrim}.  Interestingly, however, this is perhaps
where a lot of the successes of ANNs actually originate.  Compared to say random forests, ANNs constitute a very
flexible and composable framework for discriminative machine learning and perhaps provide a means of 
indirectly imposing prior knowledge on the ANN through its structuring.  A major weakness of
declarative generative approaches, such as those employed by PPSs, is that we almost
always have to impose more assumptions on the model then we would like -- all models will inevitably be
misspecified.  For ANNs, on the other hand, when there is not an abundance of training data,
one often struggles to impose enough assumptions, sometimes,
for example, resorting to generating synthetic data to train the network~\citep{von2008recaptcha}.  
Doing this is, in effect, a somewhat convoluted,
though often very effective, method for imposing prior information on the model~\citep{le2017using}.  From this
perspective, the Bayesian approach, and in particular probabilistic programming, is perhaps also of significant
consequence to the ANN literature~\citep{gal2016uncertainty}.
In particular, there may be a bright future for methods which
take a model-based approach, but without having assumptions as strict and prone to misspecification
as conventional Bayesian modeling, for which ANNs are likely to be a highly useful tool~\citep{siddharth2017learning}.

\section{Do We Need Random Numbers?}

An interesting question was recently raised by the probabilistic numerics~\citep{hennig2015probabilistic}
community about whether we should, at least as a long-term idealistic vision, look to do away with randomization entirely~\citep{schober2017thesis}.
This in many ways corresponds to taking the Bayesian philosophy to its most extreme possible
limit, by which we shun all notions of drawing from probability distributions and view probability only as a measure
of uncertainty.  Given the heavy use of Monte Carlo methods throughout this thesis,
it perhaps comes as no surprise that we are vehemently opposed to this idea.  Nonetheless, it is still an insightful
question that deserves serious consideration.  We here present arguments for why we believe randomization is essential
on a fundamental level. The intention is that this, inevitably subjective,
discussion should be seen as a pairing to~\citep{schober2017thesis}, who provides the other side of the argument.

 At a high-level, the crux of the argument against randomization is that, 
 unless our aim 
is sampling itself, we always have a certain amount of information available and a final aim we are trying to achieve (e.g.
calculating an expectation).  Therefore, at least at a philosophical level, randomization only serves to add noise to this
process.  
Similarly to the likelihood principle, one might argue that given the same data and model, we should 
always carry out the same calculations and thus always reach the same conclusions.  Doing otherwise,
the argument follows, must always be suboptimal, as it involves inconsequential information affecting the final
decision.  One compelling motivation for this is that, in certain situations, some deterministic
methods can deliver substantially better convergence rates than Monte Carlo~\citep{briol2016fwbq,caflisch1998monte}, 
though these are typically crippled by the curse of dimensionality in a way that Monte Carlo is not.

Though eminently reasonable, our opinion is that this argument, which almost completely dismisses the entirety
of frequentist statistics, is fundamentally flawed.  Our argument can be broken down into four main reasons 
why we need, and always will need, random numbers:
\begin{description}[align=left]
	\vspace{-8pt}
	\setlength\itemsep{-0.1em}
	\item[Honesty and reliability] -- The alternative to randomness is almost always approximation, introducing bias 
	that might be impossible to effectively quantify.
	\item[Lack of repeatability] -- Though it is always convenient when an experiment is repeatable, systematically 
	returning the same incorrect answer is far more dangerous.
	\item[Composability] -- Methods such as Monte Carlo are ambivalent to how the samples are to be used, creating 
	a scalable ability to modularize, compose and reuse.  Non-composable systems, on the other hand, are doomed to fail
	for large frameworks through the curse of repeated nesting of estimators as we discussed in Chapter~\ref{chp:nest}.
	\item[Speed and simplicity] -- Sometimes even if there is the information available to improve a computation, 
	uncovering or incorporating that information in a principled manner may require substantially more computational 
	power than not doing so.
	\vspace{-10pt}
\end{description}
While the latter two arguments are computational, the others are related to the following principle:
\begin{quote}
	\vspace{-4pt}
	\emph{Once we have imparted all possible information upon a system, we must treat what is left as truly 
		stochastic or introduce bias.}
	\vspace{-8pt}
\end{quote}
In other words, imagine we can construct a problem in a manner that utilizes all available prior information,
both in terms of the model itself and in how best to draw inferences from that model.
Any remaining uncertainty is now, at least for all practical purposes, inherent to the problem and so any
further attempt to remove randomness from the system is indirectly adding additional prior information  that we
did intend to impart.  Furthermore, as soon as we make approximations to our model,
we lose \emph{calibration} in our estimate: we can estimate the variance
resulting from randomness through repetition, but there is usually no way to estimate the bias
resulting from approximation.

%Putting our frequentist hat on for the moment, then the notion of doing away with randomness is scary.
As we explained in Section~\ref{sec:bayes:religions},
frequentist statistics is all about \emph{repeatability} and focuses on the fact that any data we collect is random
in a many-worlds point of view: multiple possible
datasets \emph{could} have been generated.  We demonstrated that there are many times when this frequentist
approach is absolutely essential.  For example, in medical tasks we need
to make sure that our confidence intervals are actually correct when we repeat a procedure -- we need guarantees on
their calibration.  This is not possible in a Bayesian framework as, by its very nature, it presumes the data
is fixed and thus permits no concept of repetition.  Once we realize that we sometimes need to take a frequentist
approach, the need for randomness becomes obvious, as in the frequentist framework probability originates only
through random variables.  As powerful as the Bayesian framework is, it is inherently optimistic and subjective
(e.g. there are always unknown unknowns), and as such, it can never really be the whole story.  If we try to
do away with randomness completely it is impossible to be objective -- we have to choose our approximation, sometimes
in an arbitrary fashion -- and so we can never provide properly calibrated confidence 
intervals that are not prone to subjective interpretation.

One of its most powerful consequences of the honesty of Monte Carlo is the composability
of estimates: give a sample from a marginal, one can generate a valid sample of the joint by sampling from the
conditional.  Similarly, if we have samples from a joint, we also have samples from the marginal distributions.  This behavior
is essential when composing different components into a greater system, as one can ensure each component takes in
samples as inputs and outputs corresponding samples that can be used by the next process in the pipeline.
If we instead make some deterministic approximation at each stage, we flaunt the flaw of averages and our biases
will conflagulate to give a result that might be substantially different to the truth.  If we stick with unbiased Monte Carlo estimation
methods then although it is, of course, possible that our variance will explode, this can almost always be checked.
If we do away with randomness, we might generate large errors without even knowing we have.

This lack of repeatability is actually an essential advantage of Monte Carlo.  It is better for a system to give you
a different wrong answer each time it is queried, particularly if these answers
are unbiased, than to always give the same wrong answer.  When we write scientific papers we run our experiments
multiple times to show the variability in the results.  Systems that always give the same answer and instead return
a single subjective uncertainty estimate (e.g. a Gaussian process) are not generally trustworthy because there is no
true calibration or sense of whether the experimenter simply got lucky with a system that will not generalize.  
Furthermore, such a setting is very open to intentional or unintentional abuse.  If we fix our randomness
a priori and then adapt our algorithm until it works, we may well simply be over-fitting to what works best for that
random number seed or approximation. To maintain scientific integrity, we have to report results on 
experiments that have not been
tested during the design of our algorithm and which are tested with multiple different choices for arbitrary or subjective
decisions, to show that those results are actually stable and representative.  If we do away with randomness, we lose, or at the very least
seriously hamper, this ability to repeat experiments in an objective manner.

To summarize, we need randomness because we do not know everything.  In the Bayesian framework, we place distributions
on what we are not sure about to reflect this lack of knowledge.  A key part of this process is acknowledging that once we
have done this, what is left is random.   For example, if we have a sampler that converts $u\sim\textsc{Uniform}(0,1)$ draws
to samples from the posterior, then the $u$ corresponding to the ground truth parameters is equally likely to be any value
between $0$ and $1$ and is truly random.
Yes, we might be able to impart more knowledge on
problems that we currently do, such as by decorrelating our samples, but there always become a point where we have
imparted everything we know, after which what is left is truly random.


\section{Amortizing Inference}

An interesting recent advancement in PPSs is the idea of \emph{amortizing} the process of 
inference~\citep{paige2016inference,ritchie2016deep,le2017inference}.  The  high-level idea is that  
we often repeatedly use the same models for multiple different datasets and therefore
it can be wasteful to attempt to do inference from scratch each time.  Instead, one can, before any data is
actually seen, look to learn a regressor from the space of data to, for example, proposals, such that
when any particular dataset is actually observed, one can use this regressor to produce an
artifact to assist with the resulting inference.  To do this, amortization methods exploit the fact that generative
approaches specify a distribution over both data and parameters.  It is thus theoretically possible
to learn the form of the posterior as a function of the dataset by sampling parameter-data pairs from
the joint distribution and then using these to train a regressor from data to some variables providing a characterization of
posterior.  In practice, doing this is can be very challenging -- learning a posterior for any possible
dataset is inevitably far harder than learning one for a particular dataset -- but in applications where
the same model is repeatedly used, the additional cost can be easily justified.  Current methods share
a number of weaknesses, such as being restricted by the fidelity of the underlying regressor, mostly
only allowing importance sampling for the final inference, and being potentially even more sensitive to
model misspecification than conventional Bayesian approaches.  Nonetheless, they represent a very interesting avenue
for future research and offer the potential for substantial long-term impact.  In particular, as current methods
are already effective at amortizing inference for low-dimensional problems, the utility of such approaches
might be dramatically increased if one can develop methods for breaking an overall inference problem down
into a number of smaller problems that can be amortized separately and then usefully combined at run time.

\section{A General Purpose Experimental Design Language}
\label{sec:disc:design}

We finish by discussing an exciting and direct possible extension of our work: developing the framework we introduced 
in Chapter~\ref{chp:design} into a general purpose tool for automating sequential design problems.
The potential applications for such a system would be huge,  ranging from designing clinical trials to
smart online surveys and classic autonomous agent problems.  In  the same way all Bayesian modeling
problems can be reduced to a common inference framework, such problems can be reduced to a
common Bayesian experimental design framework. This suggests that they are highly suited to a
probabilistic programming style approach.  However, there are a number of reasons existing systems
are not suited to such problems.  Firstly, such problems require coincident inference 
and optimization.  Our introduction of BOPP could be of help for overcoming this problem,
but its computationally intensive nature means that it is perhaps not suitable for
some scenarios, e.g. the psychological trials we considered in Section~\ref{sec:design:darc},
for which the approach we outlined in Section~\ref{sec:design:seq}, or an extension thereof,
might be more suitable.  If BOPP were to be used, suitable adjustments would also be necessary
to account for \emph{sequential} design problems.
Secondly, Bayesian experiment design problems, in general, require the calculation of a nested 
estimation.  For discrete outputs, we showed how a superior estimator can be derived and so
it would be important to exploit this, when possible, in any hypothetical system.  For continuous problems that
cannot be rearranged, then as we explained in Section~\ref{sec:nest:imp}, we need to be careful
to avoid the pitfalls of nested estimation, for which existing systems appear to need suitable
adaptation.  It may also to be beneficial to develop application specific estimation schemes
for this setting, potentially exploiting the known form of the nonlinear mapping to
devise methods for bias reduction.  Finally, to make such a system user-friendly, it would
be important to provide a convenient means of encoding a suitable design space alongside the
model.  It might, for example, be possible to do this in a similar manner as constraints were enforced
in BOPP, noting that each iteration in a sequential design is effectively a maximum marginal likelihood
problem.

%
%\begin{itemize}
%	\item Relationships with ABC
%	\item Reinvention of priors in deep learning by generating data
%	\item Is amortization just a learning a different decomposition of the joint 
%	$p(\theta | \mathcal{D}_1)p(\mathcal{D}_2|\theta)$ that shifts more to the
%	prior to make inference easier?  Does it actually implicitly define a different
%	joint distribution that is otherwise difficult to express or is it just proposal
%	adaptation as Tuan Anh's paper says.
%	\item Fully automated experimental design discussion.
%	\item Sort out PPS vs PPL and PPS vs PPSs
%	\item Add reference somewhere to the paper with Andrew
%	\item Frequentist probabilistic programming
%	\item Learning intermediate targets in SMC
%\end{itemize}
% !TEX root = ../main.tex

At its core, the Bayesian paradigm is simple, intuitive, and compelling: for any task involving
learning from data we start with some prior knowledge and then update that knowledge to
incorporate information from the data.  Imagine we are trying to reason about some variables
or parameters $\theta$.  We can encode our initial belief as relative probabilities for different
possible instances of $\theta$, this is known as a \emph{prior} $p(\theta)$.  Given observed data
$\mathcal{D}$, we can characterize how likely different values of $\theta$ are to have given rise
to that data using a \emph{likelihood function} $p(\mathcal{D}|\theta)$.  These can then be
combined using Bayes' rule to give a \emph{posterior}, $p(\theta | \mathcal{D})$ that 
represents our updated belief about $\theta$ once the information from the data has been
incorporated
\begin{align}
	\label{eq:bayes}
	p(\theta | \mathcal{D}) = \frac{p(\mathcal{D} | \theta)p(\theta)}{\int p(\mathcal{D} | \theta)p(\theta) d\theta} 
	= \frac{p(\mathcal{D} | \theta)p(\theta)}{p(\mathcal{D})}.
\end{align}
Here the denominator, $p(\mathcal{D})$, is a normalization constant know as the \emph{marginal
	likelihood} and is necessary to ensure $p(\theta | \mathcal{D})$ is a valid probability distribution
(or probability density for continuous problems).  The high level interpretation of~\eqref{eq:bayes} is
thus that the posterior is proportional the to the prior times the likelihood.  

A key feature of Bayes' rule is that it can be used in a self-similar fashion where the posterior from
one task becomes the prior when the model is updated with more data, i.e.
\begin{align}
	\label{eq:intro:repeat-bayes}
p(\theta | \mathcal{D}_1, \mathcal{D}_2) = 
\frac{p(\mathcal{D}_2 | \theta, \mathcal{D}_1)p(\theta | \mathcal{D}_1)}{p(\mathcal{D}_2 | \mathcal{D}_1)}
\frac{p(\mathcal{D}_2 | \theta, \mathcal{D}_1)p(\mathcal{D}_1 | \theta) p(\theta)}
{p(\mathcal{D}_2 | \mathcal{D}_1) p(\mathcal{D}_1)} 
\end{align}
This means their is something quintessentially human about the Bayesian paradigm: we learn
from our experiences by updating our beliefs after making observations.  Our model of the world
is constantly evolving over time and is the cumulation of experiences over a lifetime.  
If we make an observation that goes against our prior experience, we do not suddenly make
drastic changes to our underlying belief, but if we see multiple corroborating observations our
view will change.
This also sometimes leads to less savory elements of human behavior -- once we have developed
a strong prior belief about something, we can take substantial convincing to change our mind, even
if that prior belief is high illogical.

There is similarly something distinctively Bayesian to the scientific process itself.  In science we construct models
to explain observed phenomena and then run experiments to validate how well our model matches
real observations.  We then update and improve our model accordingly in a never ending process of
increasing understanding for the world around us.  We can never hope to truly understand the workings
of the universe  -- after all it is, at least for practical purposes, fundamentally random~\citep{rainforth2013random} 
-- and so we can hope only to construct increasingly accurate and pertinent models.

\todo[inline]{To give a more concrete example of Bayesian inference ....}

For such a fundamental theorem, Bayes' rule has a remarkably simple derivation, following almost
directly from axioms of probability theory.  Though not technically axiomatic, one can summarize
the basic laws of probability using the sum rule and the product rule defined as follows.


\subsection{Graphical Models}

\subsection{Challenges and Shortcomings}
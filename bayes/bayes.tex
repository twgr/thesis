% !TEX root = ../main.tex
%
%\begin{savequote}[8cm]
%	\textlatin{Le doute n'est pas une état bien agréable, mais l'assurance est un état ridicule.}
%	
%	Doubt is not a very agreeable status, but certainty is a ridiculous one.
%	\qauthor{--- Voltaire}
%\end{savequote}


\chapter{Probabilistic Machine Learning}
\label{chp:bayes}

In this chapter we will provide a high-level introduction to  some of the core approaches to
machine learning.  We will distinguish between discriminative and generative approaches,
outlining some of the key features that indicate when problems are more suited to one approach
or the other.  Our focus then settles on probabilistic generative approaches, which
will be the main focus of this thesis.  We will explain how the \emph{Bayesian paradigm} provides
a powerful framework for generative machine learning that allows us to combine data with existing
expertise.  We will then go on to show how \emph{graphical models} can be used as a convenient
framework to express Bayesian models and extract important features.  
We continue by introducing one of the main competitors to the Bayesian approach -- frequentist
modeling -- and present arguments for why neither a Bayesian nor a frequentist approach is
entirely satisfactory.  In particular, we will carefully outline the, oft forgotten, fundamental underlying
assumptions made by each approach and explain why the differing suitability of these
assumptions to different tasks means that both are  essential tools in the machine learning
arsenal.
Though the focus of this thesis will be on Bayesian approaches, understanding its limitations
is essential for understanding when the methods we discuss should be used and critically when they
should definitely not be.  We finish the chapter by discussing some of the key practical challenges for Bayesian modeling and outline how we hope to address these in the present
work.

% !TEX root = ../main.tex

\section{Discriminative vs Generative Machine Learning}
\label{sec:bayes:discrim}

In some machine learning applications, huge quantities of data are available that dwarf the information
that can be provided from human expertise.  In such situations, the main challenge is in processing
and extracting all the desired information from the data to form a useful characterization,
typically an artifact providing accurate predictions at previous unseen inputs. 
Such problems are typically suited to \emph{discriminative} 
machine learning approaches~\citep{breiman2001statistical,vapnik1998statistical}, such as neural
networks~\citep{rumelhart1986learning,bishop1995neural}, 
support vector machines~\citep{cortes1995support,scholkopf2002learning}, and decision tree 
ensembles~\citep{breiman2001random,rainforth2015canonical}.  Discriminative machine learning approaches
focus on directly learning a predictive model: given training data $\mathcal{D} = \left\{x_n,y_n\right\}_{n=1}^N$
they learn a parametrized mapping $f_{\theta}$ from the inputs $x \in \mathcal{X}$ to the 
outputs $y\in\mathcal{Y}$ that can 
be used directly to make predictions 
for new inputs $\tilde{x} \notin \left\{x_n\right\}_{n=1}^N$.  \emph{Training}
uses the data $\mathcal{D}$ to estimate optimal values of the parameters $\theta^*$. \emph{Prediction}
at a new input $\tilde{x}$ involves applying the mapping with the optimal parameters giving an estimate for the output
$\tilde{y} = f_{\theta^*}(\tilde{x})$.  Perhaps the simplest example of this is linear regression: one finds
the hyperplane that best represents the data and then uses this hyperplane to interpolate or extrapolate
to previously unseen points.  
As a more advanced example, in a neural network one uses training to learn the
weights of the network, after which prediction can be done by running the network forwards.  

There are many intuitive reasons to take a discriminative machine learning 
approach.  Perhaps most compelling is the
idea that if our objective is prediction, then it is simplest to solve that problem directly, rather
than try and solve some more general problem such as learning an underlying generative 
process~\citep{vapnik1998statistical,breiman2001statistical}. Furthermore, if sufficient
data is provided, discriminant approaches can be spectacularly successful in term of predictive
performance.  Discriminant methods are typically highly flexible and can capture intricate structure in the data that
would be hard or even impossible to establish manually.  Many approaches can also be run with little
or no input on behalf of the user, delivering state-of-the-art performance when used
``out-of-the-box'' with default parameters~\citep{rainforth2015canonical}.  

However, this black-box nature is also often their downfall.  Discriminative methods typically make
such weak assumptions about the underlying process that is difficult to impart prior knowledge
or domain-specific expertise.  This can be disastrous if insufficient data is available, as the data
alone is unlikely to possess the required information to make adequate predictions.  Even when
substantial data is available, there may be significant prior information available that needs to be
exploited for effective performance.  For example, in time series modelling the sequential nature
of the data is critically important information~\citep{liu1998sequential}, while in vision tasks the 
knowledge that scenes are generated from objects can be invaluable~\citep{kulkarni2015picture}.
Many problems also increase in complexity as more data is added -- ``big data'' problems are often
actually a collection, or sometimes hierarchy, of many small problems, such that the complexity of the
required parametrization increases are more data is added.  Consider, for example, modeling interactions in
a social network.  Adding a new user into the model increases the amount of data, but also
requires the model to grow and accommodate the new user~\citep{ravasz2003hierarchical}.  In
this situation it is essential to
use an approach that respects the known structure, while the amount of data available
for each individual user is often quite small, such that it will essential to use prior information
by transferring insights gathered from some users to others.  Therefore even for such large-scale
problems, the inflexibility of many discriminative approaches to incorporate known characteristics
of the target problem can be problematic.

Not only does the black-box nature of many discriminative methods restrict the level of
human input that can be imparted on the system, it often restricts the amount of insight
and information that can be extracted from the system once trained.  The parameters in most discriminative
algorithms do not have physical meaning that can be queried by a user, making their operation
difficult to interpret and hampering the process of improving the system through manual
revision of the algorithm.  Furthermore, this typically makes them inappropriate for more
statistics orientated tasks, where it is the parameters themselves which are of interest, rather
than the ability for the system itself to make predictions.  For example, the parameters may
have real-world physical interpretations we wish to learn about.

Most discriminative methods are also poor at providing realistic uncertainty estimates.
Because they are typically trained in a manner that optimizes the parameters to minimize
some loss criterion (e.g. the predictive error), they do not, in general, encode any uncertainty
in either their parameters or the subsequent predictions.  Though many methods can
produce uncertainty estimates either as a by-product or from a post-processing step,
these are typically heuristic based, rather than stemming naturally from a statistically
principled estimate of the target uncertainty distribution.   This lack of reliable uncertainty
estimates can lead to overconfidence and can make discriminative methods inappropriate in
many scenarios.  It can also reduce the composability of discriminative methods within
larger systems, as information is lost when only providing a point estimate.
Not representing uncertainty in the parameters can also restrict the power of the resultant
models, compared with approaches that can average over different possible parameter values.

These shortfalls mean that many tasks instead call for a \emph{generative} machine learning
approach~\citep{ng2002discriminative,bishop2006pattern}.  Rather than directly learning a 
predictor, generative methods look to explain the observed data using a \emph{probabilistic model}.
Whereas discriminative approaches aim only to make predictions, generative approaches model
how the data is actually generated; they model the joint probability $p(X,Y)$ of the inputs 
$X$ and outputs $Y$.  By comparison, we can think of discriminative approaches as
only modeling the outputs given the inputs $p(Y|X)$.  

A key upshot of this difference
is that generative approaches generally make stronger modeling assumptions about the problem.  Though
this can be problematic when the model assumptions are wrong and is often unnecessary in
the limit of large data, it is essential for combining prior information with data
and therefore for constructing systems that exploit application-specific human expertise.
In the words of the great George Box, ``\textit{all models are wrong, but some are useful}''
\citep{box1979robustness,box2005statistics}.  In a way, this is a self-fulfilling statement: a model for
any real phenomena
is by definition an approximation and so is never exactly correct no matter how powerful.  However,
it is still an essential point that is all too often forgotten, particularly by academics trying to convince
the world that only their approach is correct.  Only in artificial situations can we construct exact models
and so we must remember, particularly in generative machine learning, that the first, and often largest,
error is in our original mathematical abstraction of the problem.  On the other hand, real situations
have access to finite and often highly restricted data, so it is equally preposterous to suggest that a
method is superior simply due to better asymptotic behavior in the limit of large data, or that if our approach
does not work then the solution is simply to get more data.\footnote{It should, of course, be noted
	that the availability of data is typically the biggest bottleneck in machine learning.  At times, it feels 
	like the machine learning community would we well served to remember that the differences in performance between
	machine learning approaches is often, if not usually, dominated by variations in the inherently difficultly of the
	problem, which is itself not usually known up front, rather than differences between approaches.}  As such, the ease of which domain-specific
expertise can be included in generative approaches is often essential to achieving effective performance
on real world tasks.

To highlight the difference between discriminative and generative machine learning, we consider the
example of the differences between logistic regression (a discriminative classifier) and na\"{i}ve Bayes 
(a generative classifier).  We will consider the binary classification case for simplicity.  Logistic regression is a linear
classification where the class label $y \in \{0,1\}$ is predicted from the input features $x \in \real^D$ using
\begin{align}
\label{eq:bayes:logistic}
p(y|x,a,b) = \frac{1}{1+\exp(-y(a+b^Tx))},
\end{align}
and where $a \in \real^D$ and $b \in \real^D$ are the parameters of the model.  The model is trained by finding the values
for $a$ and $b$ that minimize a loss function on the training data.  For example, a common approach
is to find the \emph{most likely} parameters $a^*$ and $b^*$ by minimizing the negative log-likelihood
(which is equivalent to the cross-entropy loss function)
\begin{align}
\{a^*,b^*\} &= \argmin_{a\in \real^D,b\in \real^D} -\log \left(\prod_{n=1}^{N} p(y_n|x_n,a,b)\right).
\end{align}
Once found, $a^*$ and $b^*$ can be used with~\eqref{eq:bayes:logistic} to make predictions at
any possible $x$.  Logistic regression is a discriminative approach as we have directly calculated
a characterization for the predictive distribution, bypassing any direct reasoning about the joint
probability of the inputs and the outputs.

The na\"{i}ve Bayes classifier, on the other hand, starts by making the assumption that each feature
is \emph{independent} given the class label.  As such it reasons about both the distribution over both
inputs and outputs, unlike logistic regression which only reasons about the conditional probability
of the output given the input.   If we let $\{x^1,\dots,x^D\} =: x$ represent the dimensions of $x$, then
the na\"{i}ve Bayes assumption is that
\begin{align}
p(x|y) = \prod_{i=1}^D p(x^i |y)
\end{align}
which in turn leads to the joint distribution
\begin{align}
p(x,y) = p(y) \prod_{i=1}^D p(x^i |y).
\end{align}
Here we are free to choose the form for both $p(x^i |y)$ and $p(y)$ %(defining these also indirectly defines $p(x)$)
and we will use the data to learn their parameters.
This freedom is both a blessing and a curse: it allows us to impart our own knowledge about the problem to
the model, but we may be forced to make assumptions without proper justification in the interest
of tractability, for convenience, in error, or simply because it is challenging to specify a sufficiently
general purpose model that can cover all possible cases.
Further, even once the forms of $p(x^i |y)$ and $p(y)$ have been defined, there are still decisions to be
made: do we take a Bayesian or frequentist view for making predictions? What is the best way
to calculate the information required to make predictions?  We will go into these questions in
more depth in Section~\ref{sec:bayes:religions}.

As we have shown, generative approaches are inherently probabilistic.  This is highly convenient
when it comes to calculating uncertainty estimates or gaining insight from our trained model.
They are generally more intuitive than discriminative methods, as, in essence, they constitute an explanation for how the data is
generated.  As such, the parameters tend to have physical interpretation in the generative process and
therefore provide not only prediction, but also insight.  Generative approaches will not always be preferable,
particularly when there is an abundance of data available, but they provide a very powerful framework
that is essential in many scenarios.  Perhaps their greatest strength is in allowing the use of so-called
Bayesian approaches which we now introduce.
% !TEX root = ../main.tex

At its core, the Bayesian paradigm is simple, intuitive, and compelling: for any task involving
learning from data we start with some prior knowledge and then update that knowledge to
incorporate information from the data.  Imagine we are trying to reason about some variables
or parameters $\theta$.  We can encode our initial belief as relative probabilities for different
possible instances of $\theta$, this is known as a \emph{prior} $p(\theta)$.  Given observed data
$\mathcal{D}$, we can characterize how likely different values of $\theta$ are to have given rise
to that data using a \emph{likelihood function} $p(\mathcal{D}|\theta)$.  These can then be
combined using Bayes' rule to give a \emph{posterior}, $p(\theta | \mathcal{D})$ that 
represents our updated belief about $\theta$ once the information from the data has been
incorporated
\begin{align}
	\label{eq:bayes:bayes}
	p(\theta | \mathcal{D}) = \frac{p(\mathcal{D} | \theta)p(\theta)}{\int p(\mathcal{D} | \theta)p(\theta) d\theta} 
	= \frac{p(\mathcal{D} | \theta)p(\theta)}{p(\mathcal{D})}.
\end{align}
Here the denominator, $p(\mathcal{D})$, is a normalization constant know as the \emph{marginal
	likelihood} and is necessary to ensure $p(\theta | \mathcal{D})$ is a valid probability distribution
(or probability density for continuous problems).  The high level interpretation of~\eqref{eq:bayes:bayes} is
thus that the posterior is proportional the to the prior times the likelihood.  

A key feature of Bayes' rule is that it can be used in a self-similar fashion where the posterior from
one task becomes the prior when the model is updated with more data, i.e.
\begin{align}
	\label{eq:bayes:repeat-bayes}
p(\theta | \mathcal{D}_1, \mathcal{D}_2) = 
\frac{p(\mathcal{D}_2 | \theta, \mathcal{D}_1)p(\theta | \mathcal{D}_1)}{p(\mathcal{D}_2 | \mathcal{D}_1)}
\frac{p(\mathcal{D}_2 | \theta, \mathcal{D}_1)p(\mathcal{D}_1 | \theta) p(\theta)}
{p(\mathcal{D}_2 | \mathcal{D}_1) p(\mathcal{D}_1)} 
\end{align}
This means their is something quintessentially human about the Bayesian paradigm: we learn
from our experiences by updating our beliefs after making observations.  Our model of the world
is constantly evolving over time and is the cumulation of experiences over a lifetime.  
If we make an observation that goes against our prior experience, we do not suddenly make
drastic changes to our underlying belief, but if we see multiple corroborating observations our
view will change.
This also sometimes leads to less savory elements of human behavior -- once we have developed
a strong prior belief about something, we can take substantial convincing to change our mind, even
if that prior belief is high illogical.

There is similarly something distinctively Bayesian to the scientific process itself.  In science we construct models
to explain observed phenomena and then run experiments to validate how well our model matches
real observations.  We then update and improve our model accordingly in a never ending process of
increasing understanding for the world around us.  We can never hope to truly understand the workings
of the universe  -- after all it is, at least for practical purposes, fundamentally random~\citep{rainforth2013random} 
-- and so we can hope only to construct increasingly accurate and pertinent models.

\todo[inline]{To give a more concrete example of Bayesian inference ....}

For such a fundamental theorem, Bayes' rule has a remarkably simple derivation, following almost
directly from axioms of probability theory.  Though not technically axiomatic, one can summarize
the basic laws of probability using the sum rule and the product rule defined as follows.


\subsection{Graphical Models}

\subsection{Challenges and Shortcomings}
% !TEX root = ../main.tex

\section{Graphical Models}
\label{sec:bayes:paradigm:graph}

Generative models will typically have many variables and a complex \emph{dependency structure}.
In other words, many of variables will be conditionally independent of one another given values for
other variables.  Graphical models are a ubiquitously used method for representing and reasoning
about generative models with a particular focus on the dependency structure.  At a high-level, they
capture how the joint probability distribution can be broken down into a product of different factors, 
each defined over
a subset of the variables.  They are formed of a number of connected nodes, where each node
represents a random variable in the model and each variable has its own node.  Links between nodes in
the model represent dependencies: any two connected nodes have an explicit dependency.
Various independence assumptions can be deduced from the graphical model, though the exact nature
of these deductions will depend or the type of graphical model -- nodes without direct links
between them will often still be dependent.

Graphical models can be seperated into two distinct classes: directed graphical
models and undirected graphical models.  Undirected graphical models, also known as Markov random
fields, imply no ordering on their factorization and are used only to express conditional independences
between variables.  They are used in scenarios where it is difficult to specify the target distribution in a
structured generative way.  For example, if modeling whether it will rain at various locations then there
is a clear dependence between nearby locations, but not a natural ordering to the joint probability
distribution of where it will rain.  Boltzmann machines are a common example of undirected graphical
models~\citep{ackley1985learning}.  Independence in undirected graphical models can be deduced
through the \emph{global Markov property} which states that any two non-intersecting subsets of the 
variables $A$ and $B$ are
conditionally independent given a third, separating, subset $C$ if there is no path between $A$ and
$B$ that does not pass through $C$.  This means, for example, that each variable is conditionally
independent of all the other variables given its neighbors.

Our main focus, though, will instead be on directed graphical models and in particular directed acyclic 
graphical models (DAGs), i.e. directed graphical models containing no cycles or loops one can follow 
and arrive back in the starting position.  DAGs, also known as Bayesian networks, are particularly
useful in the context of Bayesian modeling because they can be used to express \emph{casual relationships}.
As such, they can be used as a piecewise explanation for how the joint distribution is generated.
Not only does this form a natural means to describe and design models in the first place, 
we can carefully order the breakdown to factorize the distribution into only terms we know.  For example,
in the linear regression model we did not know (at least when the model was first defined) 
$p(\mathcal{D})$ but we did know $p(\mathbf{w})$ and $p(\mathcal{D} | \mathbf{w})$.  Therefore even
though we could factorize our joint $p(\mathcal{D}, \mathbf{w})$ as 
$p(\mathbf{w} | \mathcal{D})p(\mathcal{D})$ and construct a DAG this way, it is much more convenient
to factorize and construct the DAG the other way round, namely as 
$p(\mathcal{D} | \mathbf{w})p(\mathbf{w})$.
As a good rule of thumb, when we define a model using a DAG, we need to be able to define the 
probability of each variable given its \emph{parents}, i.e. all the nodes with links from that node
towards the node the question.  We will generally not have access to also possible factorizations
in an analytic form as otherwise there would be no need to perform inference.

The demonstrate this factorization more explicitly and give a concrete example of a DAG, consider a joint model
$p(a,b,c)$.  By the product rule, we can break down this joint distribution in to a number of different
factorizations.  However, some will typically be more useful then others.  Imagine a medical diagnostic
example where $a$ represents lifestyle and genetic factors of a patient such as whether they smoke
or have unknown preexisting conditions, 
$b$ represents the presence of lung cancer, and $c$ represents symptoms 
such as a persistent cough.  Here we have the following natural breakdown of the joint distribution
\begin{align}
\label{eq:bayes:example-graph}
p(a,b,c) = p(a) p(b|a) p(c|a,b).
\end{align}
The lifestyle and genetic factors will generally either be known or can be estimated from tests or
simply prevalence within the population.  These factors, the likelihood of somebody developing
lung cancer can be modeled using existing data and domain expertise.   Given the lifestyle and
genetic factors and the knowledge of whether lung cancer is present, we can predict the likelihood
of observing the observed symptoms.  We can express our model and this factorization using
the DAG shown in Figure INSERT.\todo{Add DAG}
  Here we have shaded in $c$ to express the fact that this is
observed.  The graphical model expresses our dependency structure as we have the probability
of each node given its parents.  As shown in~\eqref{eq:bayes:example-graph}, the product of
all these factors is equal to our joint distribution.  The DAG has thus formed a convenient means
of representing our generative process.

Clearly our aim for this problem will be to find the marginal probability $p(b|c)$.  To calculate this
we will need to perform Bayesian inference as explained.
An importance feature of the breakdown of graphical models will become apparent when we
consider how we can conduct Bayesian inference more a general class of 
models where the solution is no analytic in Chapter~\ref{chp:inf}.  Here knowing the dependency
structure, and in particular the independence relationships, will be essential to many inference 
schemes such as sequential Monte Carlo and message
passing schemes.\todo{Add refs to relevant sections}

A natural question is now how can we deduce the independence relationships from DAG?
This can be done by introducing the notion of \emph{d-separation}~\citep{pearl2014probabilistic}.
Consider three arbitrary, non-intersecting, subsets of our DAG $A$, $B$, and $C$.  $A$ and $B$
are conditionally independent given $C$ if there is no \emph{unblocked} paths from $A$ to $B$
(or equivalently from $B$ to $A$), in which case $A$ is said to be d-separated from $B$ by $C$.  
Paths do not need to be in the directions defined by the DAG but are blocked if either
\begin{enumerate}
	\item Two consecutive arrows in the path both point towards a node that is not $C$ and
	has no descendants in $C$, i.e. we cannot get to any of nodes in $C$ by following the arrows
	from this node.
	\item Two consecutive arrows in the path meet at a node in $C$ and one of them
	points away from the node.
\end{enumerate}
Examples of blocked paths are shown in Figure INSERT while examples of unblocked paths
are shown in Figure INSERT, explanations for which are shown in the captions.\todo{Add Figures and captions}
For a more comprehensive introduction to establishing independence in DAGs we
refer the reader to Section 8.2 of~\cite{bishop2006pattern}.

In the simple example of Figure INSERT there were no independence relationships and so
we gain little from working with the DAG compared to just the joint probability distribution.
A more advanced example where there are substantial independene relationships which can
be exploited is shown in Figure INSERT.  This model is known as a HMM\dots Markov property
etc \dots

Improving models - addding hyperparameters to the HMM example\dots
% !TEX root = ../main.tex

\section{Bayesianism vs Frequentism}
\label{sec:bayes:religions}

We have just introduced the Bayesian approach to generative modeling, but this is far from
the only possible approach.  In this section, we will briefly introduce and compare the alternative, \emph{frequentist}, approach.
As a community, we have a come a long way from the absolutism of~\cite{feller1950introduction}, with most researchers 
appreciating that
both Bayesian and frequentist approaches are a necessary component of the general statistical method.
Nonetheless, the divide between those within 
the statistics and machine learning communities
who advocate, at least at the philosophical level, the use Bayesian or frequentist methods 
is at times surprisingly large.  Many researchers
have strong views one way or another and it is easy, certainly as a graduate student,
to be sufficiently isolated to develop a somewhat warped view of the overall picture.
%It sometimes feels that the beliefs of some advocates of each side are so deep-rooted and 
%fundamental that this almost a religious, rather than scientific divide.  
The actual statistical differences between the approaches are somewhat distinct to the 
well-known philosophical differences we touched
on in Section~\ref{sec:prob:prob}, even though the latter are often dubiously used for justification 
for the practical application of a particular approach. These statistical differences are arguably
less well-known, in general, by those in the early stages of their research careers without statistics
backgrounds.
Our aim in this section is not to advocate the use of one approach over the other, but to (hopefully objectively)
highlight these statistical differences and demonstrate that \emph{both} approaches have advantages and
drawbacks, such that ``head in the sand''
mentalities either way can be highly detrimental, with effective modeling often requiring us to draw
on both.
We note that whereas Bayesian methods are always, at least in theory, generative~\citep[Section~14.1]{gelman2014bayesian},
frequentist methods can be either generative or discriminative. 
As we have already discussed differences between generative and
discriminative modeling in Section~\ref{sec:bayes:discrim}, we will mostly omit this difference from our subsequent discussion.

At their root, the statistical differences between Bayesian and frequentist methods\footnote{At least in their decision-theoretic
frameworks.  It is somewhat inevitable that delineation here and later will be a simplification on what is, in truth, not a 
clear-cut divide~\citep{gelman2011induction}.}
stem from distinct fundamental
assumptions: frequentist modeling presumes fixed parameters, Bayesian modeling assumes fixed data~\citep{jordan2009you}.  In many ways,
both of these assumptions are somewhat dubious. Why assume fixed parameters when we do not have enough
information from the data to be certain of the correct value?  Why ignore that fact that other data could have 
been generated by the same underlying true parameters?  However,
making such assumptions can sometimes be unavoidable for carrying out particular analyses.
%Perhaps this is why people have
%such strong views either way -- the arguments against both methods are very strong, so much so that advocates are
%often, sometimes unintentionally, willing to overlook the shortcomings of their favored approach.  

To elucidate the
different assumptions further and start looking into why they are made, we will now step into a decision-theoretic
framework.  Let's presume that the universe gives us some data $X$ and some true parameter $\theta$, the former of which
we can access, but the latter of which is unknown.  We can alternatively think in terms of $X$ being some information
that we actually receive and $\theta$ being some underlying truth or oracle from which we could make optimal predictions, noting 
that there is
no need for $\theta$ to be some explicit finite parameterization.  Any machine learning approach will take the data as input and
return some artifact or decision, for example, predictions for previously unseen inputs.  
Let's call this process the decision rule $\delta$, which we presume, for the sake of argument, to be 
deterministic for a giving dataset, producing decisions $\delta(X)$.\footnote{If we allow our predictions to be probability
distributions this assumption is effectively equivalent to assuming we can solve any analysis required by our approach exactly.}
Presuming that our analysis is not frivolous, there will be some loss function $L(\delta(X),\theta)$ associated with the action we take
and the true parameter $\theta$, even if this loss function is subjective or unknown.  At a high level, our aim is always to
minimize this loss, but what we mean by minimizing the loss changes between the Bayesian and frequentist settings.  

In the
frequentist setting, $X$ is a random variable but $\theta$ is not.  Therefore, one takes an expectation over possible data
that could have been generated, giving the frequentist risk~\citep{vapnik1998statistical}
\begin{align}
\label{eq:bayes:freq-risk}
R(\theta,\delta)  = \E\left[L(\delta(X),\theta) | \theta \right]
\end{align}
which is thus a function of theta and our decision rule.
The frequentist focus is therefore on \emph{repeatability}, i.e.
the generalization of the approach to different datasets that \emph{could} have been generated.
Choosing the parameters $\theta$ is thus based on optimizing for the best average performance over all possible datasets.

In the Bayesian setting, $\theta$ is a random variable but $X$ is fixed: the focus of the Bayesian approach 
is on generalizing over possible values of
the parameters and using all the information at hand.  Therefore one takes an expectation over $\theta$
to make predictions conditioned on the value of $X$, giving the \emph{posterior expected loss}~\citep{robert2007bayesian}
\begin{align}
\label{eq:bayes:bayes-est}
\varrho(\pi,\delta(X) | X) = \E_{\pi(\theta | X)} [L(\delta(X),\theta) | X],
\end{align}
where $\pi(\theta | X)$ is our posterior distribution on $\theta$.  Although $\varrho(\pi,\delta(X) | X)$ 
is a function of the data, the Bayesian
approach takes the data as given (after all we have a particular dataset) and so for a given prior and decision rule, the posterior
expected loss is a fixed value and, unlike in the frequentist case, further assumptions are not required to
calculate the optimal decision rule $\delta$.  To see this, we can consider calculating the 
\emph{Bayes risk}~\citep{robert2007bayesian}, 
also known as the \emph{integrated risk}, which averages over both data and parameters
\begin{align}
	\label{eq:bayes:bayes-risk}
r(\pi,\delta) = \E \left[\E_{\pi(\theta | X)} [L(\delta(X),\theta) | X]\right] = 
\E_{\pi(\theta | X)} \left[\E\left[L(\delta(X),\theta) | \theta \right] \right].
\end{align}
Here we have noted that we could have equivalently taken the expectation of the frequentist
risk over the posterior, such that, despite the name, the Bayes risk is neither wholly Bayesian
nor frequentist.  It is now straightforward to show (see e.g.~\citep{robert2007bayesian}) that the
decision function which minimizes $r(\pi,\delta)$ is obtained by, for each possible dataset $X\in\mX$,
choosing the decision that minimizes the posterior expected loss, i.e. $\delta(X) = \argmin_{\delta(X)} \varrho(\pi,\delta(X) | X)$.
By comparison, because the frequentist risk is still a function of the parameters,
 further work is required to define the optimal decision rule, e.g. by taking a \emph{minimax}
approach~\citep{vapnik1998statistical}.
We now see that the Bayesian approach can be relatively optimistic, as it is constrained to choose
decisions that optimize the expected loss,
whereas the frequentist approach allows, for example, $\delta$ to be chosen in a manner that
optimizes for the worst case $\theta$.

We now introduce some shortfalls that originate from taking each approach.  We emphasize that we are only
scratching the surface of one of the most hotly debated issues in statistics and do not even come close to doing
the topic justice.  Our aim is less to provide a comprehensive explanation of the relative merits of the two approaches, but
more to make the reader aware that there are a vast array of complicated, and sometimes controversial,
issues associated with whether to use a Bayesian or
frequentist approach, most of which have no simple objective conclusion.


\subsection{Shortcomings of the Frequentist Approach}
\label{sec:bayes:religion:freq}

One of the key criticisms of the frequentist approach is that predictions depend on the experimental procedure and
can violate the \emph{likelihood principle}.  The likelihood principle states that, for a given model, 
the only information relevant
to the parameters $\theta$ conveyed by the data is encoded through the likelihood function~\citep{robert2007bayesian}.  
In other words, the same data and 
the same model should always lead to the same inferences about $\theta$.  Though this sounds intuitively obvious, it is actually violated by
taking an expectation of $X$ in frequentist methods, as this introduces a dependency from the experimental procedure.

As a classic example, 
imagine that our data from flipping a coin is $3$ heads and $9$ tails.
In a frequentist setting, we make different inferences about whether the coin is biased 
depending on whether our data originated from flipping the coin $12$ times and counting the number of heads, or if we 
flipped the coin until we got $3$ heads.  For example, at the $5\%$ level of a \emph{significance test}, we can reject the \emph{null
hypothesis} that the coin is unbiased in the latter case, but not the former.  This is obviously somewhat problematic, but it
can be used to argue both for and against frequentist methods.  Using it
to argue against frequentist methods, and in particular significance tests, is quite straightforward: the subjective
differences in our experiment should not affect our conclusions about whether the coin is fair or not.  We can also take things
further and make the results change for absurd reasons.  For example, imagine our experimenter had intended to flip until she got
$3$ heads, but was then attacked and killed by a bear while the twelfth flip was in the air, such that further flips would not
have been possible regardless of the outcome.  In the frequentist setting, this again changes our conclusions
about whether the coin is biased.  Clearly, it is ridiculous that the occurrence or lack of a bear attack during the experiment
should change our inferences, but that is need-to-know information for frequentist approaches.

As we previously suggested though, one can also use this example to argue for frequentist methods as one can argue that it actually
suggests the likelihood principle is incorrect.  Although significance tests are a terribly abused tool whose misuse has had
severe detrimental impact on many applied communities~\citep{goodman1999toward,ioannidis2005most}, they are not incorrect,
and extremely useful, if 
interpreted correctly.  If one very carefully considers
the definition of a \emph{p-value} as being the probability that a given, or more extreme, event is observed if the
\emph{experiment is repeated}, we see that our bear attack does actually affect the outcome.  Namely, the chance of getting the same
or more extreme data from repeating the experiment of ``\textit{flip the coin until you get $3$ heads}'' is different to 
the chance of getting the same or a more extreme result from repeating the experiment 
``\textit{flip the coin until you get $3$ heads or make $12$ flips (at which point you will be killed by a bear)}''.  
As such, one can argue that the apparently absurd
changes in conclusions originate from misinterpreting the results and that, in fact, these changes actually demonstrate
that the likelihood principle is flawed because, without a notion of an experimental procedure, 
we have no concept of repeatability.  Imagine instead the more practical scenario where a suspect researcher stops their experiment
early as it looks like the results are likely to support their hypothesis and they do not want to take the risk that if they
keep it running as long as they intended, then the results might no longer be so good.  Here the researcher has clearly
biased their results in a way that ostensibly violates the likelihood principle.

Whichever view you take, two things are relatively indisputable.  Firstly a number of a frequentist concepts, such as p-values,
are not compatible with the likelihood principle.  Secondly, frequentist methods are not always \emph{coherent}, such that they can
return answers that are not consistent with each other, e.g. probabilities that do not sum to one.  
%For example, p-values
%for a hypothesis and its complement do not, in general, sum to one.  

Another major criticism of the frequentist approach is that it takes a point estimate for $\theta$, rather than averaging
over different possible parameters.  This can be somewhat inefficient in the finite data case, as it limits the information
gathered from the learning process to that encoded by the calculated point estimate for $\theta$, which is then wasteful
when making predictions.  Part of the reason that this
is done is to actively avoid placing a prior distribution on the parameters, either because this prior distribution might
be ``wrong''\footnote{Whether a prior can be wrong, or what that even means, is a somewhat open question except in the
	case where it fails to put any probability mass (or density for continuous problems) on the ground truth value of
	the parameters.} or because, at a more philosophical level, they are not random variables under the frequentist definition
of probability.  Some people thus object to placing a distribution over them at a fundamental level (we will see this objection
mirrored by Bayesians for the data in the next section).  For the Bayesian perspective (and a viewpoint we actively 
argue for elsewhere in the paper), this is itself also a weakness of the frequentist approach as incorporating prior
information is often essential for effective modeling.


\subsection{Shortcomings of the Bayesian Approach}
\label{sec:bayes:religion:bayes}

Unfortunately, the Bayesian approach is also not without its shortcomings.  We have already discussed one
key criticism in the last section in that the Bayesian approach relies on the likelihood principle which itself may not
be sound, or at the very least ill-suited for some statistical modeling problems.  More generally, it can be seen as
foolhardy to not consider other possible datasets that might have been generated.  Taking a very strict stance, then
even checking the performance of a Bayesian method on test data is fundamentally frequentist, as we are assessing how
well our model generalizes to other data.  Pure Bayesianism, which is admittedly not usually carried out in practice, shuns
empiricism as empiricism, by definition, is rooted in the concept of repeated trials which is not possible if the data is kept fixed.
The rationale typically given for
this is that we should use all the information available in the data and by calculating a frequentist risk we are throwing
some of this away.  For example, cross-validation approaches only ever use a subset of the data when training the model.
However, a common key aim of statistics is generalization and repeatability.  Pure Bayesian approaches include no
consideration for \emph{calibration}, which means that, even if our likelihood model is correct, there is still no reason
that any probabilities or confidence intervals must be also.  This at odds with frequentist approaches, for which
we can often derive absolute guarantees.

A related issue is that Bayesian approaches will often reach
spectacularly different conclusions for ostensibly inconsequential changes between datasets.\footnote{Though this is arguably more
of an issue with generative approaches than Bayesian methods in particular.}
At least when making the standard assumption of i.i.d.~data in Bayesian analysis, then likelihood terms are multiplicative 
and so typically when one adds more data, the relative probabilities of two parameters quickly diverge.  This divergence is
necessary for Bayesian methods to converge to the correct ground truth parameters for data distributed exactly as per the
model, but it also means any slight misspecifications in the likelihood model
become heavily emphasized very quickly.  As a consequence, Bayesian methods can chronically underestimate uncertainty 
in the parameters, particularly for large datasets, because they do not account for
the \emph{unknown unknowns}.  This means that ostensibly  inconsequential features of the likelihood model
can lead to massively different conclusions about the relative probabilities of different parameters.
In terms of the posterior expected loss, this is often not much of a problem as the assumptions might be similarly
inconsequential for predictions.  However, if our aim is actually to learn 
about the parameters themselves then this is quite worrying.  At the very least it shows why we should view posteriors with
a serious degree of skepticism (particularly in their uncertainty estimates), rather than taking them as ground truth.

Though techniques such as cross-validation can reduce sensitivity to model
misspecification, generative frequentist methods still often do not fare much better than Bayesian
methods for misspecified models (after all they produce no uncertainty estimates on $\theta$).
Discriminative methods, on the other hand, do not have
an explicit model to specify in the same way and so are far less prone to the same issues.
Therefore, though  much of the criticisms of Bayesian modeling stem
from the use of a prior, issues with model (i.e. likelihood) misspecification are often much more severe and predominantly
shared with generative frequentist approaches~\citep{gelman2013not}.  It is, therefore, often the question of discriminative
vs generative machine learning that is most critical~\citep{breiman2001statistical}.

%
%Consider, for example, a binary
%classification problem where, as in common in the Bayesian framework, we will presume
%that our observations $y_{1:N}$ are independent such that our likelihood is a product of 
%likelihood terms for individual datapoints, i.e.
%$p(y_{1:N} | \theta) = \prod_{n=1}^{N} p(y_n | \theta)$.  We now presume that using
%hyperparameter\footnote{Mathematically is no hard distinction between a hyperparameter and a parameter, but the
%	former is often used to refer to ``higher-level'' parameters.  For example, we might have a prior over model parameters, with
%	the parameters of the prior itself referred to as hyperparameters.  Here we are implicitly presuming that there are some latent
%	variables that have been marginalized over to make the setup realistic, hence $\theta$ corresponding to a ``hyperparameter''.}
% $\theta_1$ classifies four billion of the five billion points in the dataset ``correctly''
%while using hyperparameter $\theta_2$, which has the same prior probability,
%classifies four billion and one thousand of data points ``correctly'', where by correct we mean that $p(y_n | \theta) = 0.8$
%(otherwise $p(y_n | \theta) = 0.2$).  We thus have that the difference in classification accuracies of the
%approaches is $0.00002\%$, but under our posterior $\theta_2$ is $(0.8/0.2)^{1000}\approx10^{602}$ times
%more probable than $\theta_1$.  Intuitively this is preposterous and no sensible gambler would ever take odds
%that $\theta_2$ is even twice as likely to give better performance on the a test dataset of the same size.  
%Though somewhat oversimplified, this example is not actually unrepresentative to behavior often experienced.  
%To explain the dichotomy between the theoretical correctness of Bayes' rule and our intuitions, we consider what
%happens if we let the universe regenerate the dataset.  We will draw datapoints independently
%and we presume that the probability of a new datapoint being correctly classified using $\theta_1$ is independent to the probability of
%it being correctly classified by $\theta_2$, and that these probabilities are equal to the empirical misclassification
%rates from our first dataset.  The probability that $\theta_1$ gives a better misclassification rate for the new dataset is
%almost $50\%$, even though the previous posterior suggested that $\theta_2$ was astronomically better than $\theta_1$.
%Furthermore, there is almost a $50\%$ chance that $\theta_1$ will have an equally astronomically higher
%posterior mass for our second dataset!  The problem here is that the Bayesian approach does not taken into 
%account that datasets are usually sampled from a population or possible samples that \emph{could} have been generated.
%Our analysis was perfectly reasonable under the assumption that the our dataset was fixed and absolute, but it
%lead to somewhat preposterous conclusions when we generalized to other possible data.
%In terms of the posterior expected loss, this is not much of a problem
%as for most sensible loss functions this will not substantially change.  However, if our aim is actually to learn 
%about $\theta$ then this is quite worrying.  At the very least it shows why we should view posteriors with
%a degree of skepticism, rather than taking them as ground truth.

Naturally one of the biggest concerns with Bayesian approaches is their use of a prior, with this being
one of the biggest differences to generative frequentist approaches.  The prior is typically a double-edged
sword.  On the one hand, it is necessary for combining existing knowledge and information from data 
in a principled manner, on the other hand, priors are inherently subjective and so all produced posteriors are
similarly subjective.  Given the same problem, different practitioners will use different priors and reach
potentially different conclusions.  In the Bayesian framework, there is no such thing
as a correct posterior for a given likelihood (presuming finite data).
Consequently, ``all bets are off'' for repeated experimentation with Bayesian methods as there is no
quantification for how wrong our posterior predictive might be compared with the true generating
distribution.  This can mean they  are somewhat inappropriate for tasks where the focus is on repeated use,
e.g. providing reliable confidence intervals for medical trials, though many Bayesian methods retain
good frequentist properties.  Such cases both predicate a need for
considering that there are many possible datasets that might occur and, ideally, an objective approach
that means the conclusions are independent of the whims of the analyst.

In particular, there is no (objective) Bayesian alternative to frequentist
\emph{falsification} methods such as the aforementioned significance tests.\footnote{This is not to say one cannot 
	perform falsification as part of Bayesian modeling, in fact avoiding doing so would
	be ridiculous, but that providing objective statistical guarantees to this falsification requires the use of
	frequentist methods.  For example, even the essential research process of informally
	rejecting and improving models undermines Bayesian coherence~\citep{gelman2011induction}.  On the other hand, the
	process of peer review effectively
	invalidates frequentist calibration by ensuring some studies never make it to the public domain.}
  Both Bayesian and frequentist methods require assumptions
and neither can ever truly prove that a particular model or prediction is correct, but frequentist methods do
allow one to indisputably \emph{disprove} hypotheses to within some confidence interval.  The real power
of this is realized when we consider disproving \emph{null hypotheses}, e.g. the hypothesis that an
output is independent of a potential driving factor.  This is why significance tests are so widely
used through the sciences as, although they cannot be used to prove
a model is correct (as much as people might try), they can certainly be used to show 
particular assumptions or models are wrong.

The use of a prior in Bayesian modeling can also be problematic because it is often easy to end up 
``using the data twice''~\citep{gelman2008objections}.  The Bayesian paradigm requires that the
prior is independent from the data and this means that there should not be any human-induced
dependency.  In other words, it is necessary that the user sets their prior before
observing the data they condition on, otherwise, the data will both influence their choice of prior
and be incorporated through the likelihood.  This is not a problem with Bayesian 
methods per se, but it can be a common shortfall in their application.

Another practical issue with Bayesian methods is their computational complexity at both
train time and test time.  Firstly, using Bayesian methods requires one to carry out inference,
which, as we explain in Chapter~\ref{chp:inf}, is typically a challenging and computationally
expensive process, often prohibitively so.  Some frequentist approaches can also
be similarly expensive at train time, but others can be substantially cheaper, particularly
discriminative approaches.  Further, Bayesian methods tend to also be somewhat expensive
for making predictions with the posterior predictive itself being an integral.  Frequentist
methods tend to instead predict using point estimates for $\theta$, such that prediction is typically
much cheaper.

\subsection{Practical Usage}
\label{sec:bayes:religion:correct}

Although Bayesianism and frequentism are both exact frameworks, their application to real problems is not.
Both frameworks have their strengths and weaknesses and so perhaps the key question is not which framework
is correct, but when should we use each.  In
particular, the Bayesian approach is often essential when working with small datasets but where we have substantial
prior expertise.  On the other hand, a frequentist approach is essential to providing guarantees and ensuring repeatability.  
Bayesian and frequentist methods are also by no means mutually 
exclusive and effective modeling often requires elements of both
to be used concurrently.  
For example, one could be Bayesian about the results from a cross-validation test or look to calculate
frequentist guarantees for a Bayesian model.
In essence, Bayesian and frequentist analysis have different aims -- Bayesianism is about updating
subjective beliefs and frequentism is about creating long run, or repeated application, guarantees.
We often we care about both.  It is also worth noting that a number of Bayesian methods exhibit
good frequentist properties, see e.g.~\cite{mcallester2013pac} and the references therein.

We finish by noting a critical assumption made by both Bayesian and generative frequentist methods --
that there is some true underlying value for the parameters.  Because
all models are approximations of the real world, this is often a misguided and harmful assumption.  That this assumption
is made is clear in the frequentist setting, but is somewhat subtle for Bayesian approaches.
Bayesian methods allow for multiple hypotheses or parameter values, but this originates from our own uncertainty
about which parameter or hypothesis is correct, thereby still implicitly assuming that \emph{one} of them is correct.
Namely, as we showed with the Bernstein-Von Mises theorem,  in the limit of large data, Bayesian methods with finite numbers of parameters will collapse to a point estimate, 
corresponding to the ``true parameter values'' (assuming the model is correct).
Consequently, a Bayesian approach does not fundamentally enrich the model space by averaging over parameters -- it
is still necessary that exactly one set of parameters lead to the data, but we are not exactly sure which one~\citep{minka2000bayesian}.

Consider as an example, Bayesian modeling for decision trees~\citep{chipman1998bayesian,lakshminarayanan2013top}
compared to (discriminative) ensemble-based approaches~\citep{breiman2001random,rainforth2015canonical}.
The Bayesian approaches explicitly assume that our data is generated by one single decision tree and so in the limit of
large data the relative probabilities of different trees in the posterior diverge and will collapse to a
single tree.\footnote{Technically we only get a single tree 
	if we limit the tree depth to ensure a finite parameterization.  Nonetheless, the argument still
	holds from a practical perspective even when trees are unbounded.}  The ensemble approaches, on the other hand, maintain
a full ensemble of trees in the limit of large data.  They are, in fact, a fundamentally more general model class
as they do not require the data to have been generated by a single tree.  Here the averaging over trees enriches the
model class itself, rather than just representing uncertainty about which of the trees in the ensemble is the correct 
one~\citep{domingos1997does}, typically leading to better performing algorithms for large datasets than Bayesian approaches.


\section{Challenges of the Bayesian Approach}
\label{sec:bayes:challenges}

Di Finetti and i.i.d. assumption of data.

More data means more likely the model is wrong.
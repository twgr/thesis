% !TEX root = ../main.tex

\section{Graphical Models}
\label{sec:bayes:paradigm:graph}

Generative models will typically have many variables and a complex \emph{dependency structure}.
In other words, many of variables will be conditionally independent of one another given values for
other variables.  Graphical models are a ubiquitously used method for representing and reasoning
about generative models, with a particular focus on the dependency structure.  At a high-level, they
capture how the joint probability distribution can be broken down into a product of different factors, 
each defined over
a subset of the variables.  They are formed of a number of connected nodes, where each node
represents a random variable (or collection of random variables) in the model.  Links between nodes
represent dependencies: any two connected nodes have an explicit dependency, though unconnected
nodes may still be dependent.
Various independence assumptions can be deduced from the graphical model, though the exact nature
of these deductions will depend on the type of graphical model -- nodes without direct links
between them will often still be dependent.

Graphical models can be separated into two distinct classes: directed graphical
models and undirected graphical models.  Undirected graphical models, also known as Markov random
fields, imply no ordering on their factorization and are used only to express conditional independences
between variables.  They are used in scenarios where it is difficult to specify the target distribution in a
generative way, e.g.  Boltzmann machines~\citep{ackley1985learning}.  To give 
a more concrete example, if modeling whether it will rain at various locations, then there
is a clear dependence between nearby locations, but not a natural ordering to the joint probability
distribution of where it will rain.   Independence in undirected graphical models can be deduced
through the \emph{global Markov property} which states that any two non-intersecting subsets of the 
variables $A$ and $B$ are
conditionally independent given a third, separating, subset $C$ if there is no path between $A$ and
$B$ that does not pass through $C$.  This means, for example, that each variable is conditionally
independent of all the other variables given its neighbors.

Our main focus, though, will instead be on directed graphical models and in particular directed acyclic 
graphical models (DAGs), i.e. directed graphical models containing no cycles or loops one can follow 
and arrive back in the starting position.  DAGs, also known as Bayesian networks, are particularly
useful in the context of Bayesian modeling because they can be used to express \emph{casual relationships}.
As such, they can be used as a piecewise explanation for how the joint distribution is generated.
This form a natural means to describe and design models as
we can carefully order the breakdown to factorize the distribution into only terms we know.  For example,
in the linear regression model, we did not know (at least when the model was first defined) 
$p(\mathcal{D})$ but we did know $p(\mathbf{w})$ and $p(\mathcal{D} | \mathbf{w})$.  Therefore even
though we could factorize our joint $p(\mathcal{D}, \mathbf{w})$ as 
$p(\mathbf{w} | \mathcal{D})p(\mathcal{D})$ and construct a DAG this way, it is much more convenient
to factorize and construct the DAG the other way round, namely as 
$p(\mathcal{D} | \mathbf{w})p(\mathbf{w})$.
We will generally not have access to all possible factorizations
in an analytic form as otherwise there would be no need to perform inference.
As a rule-of-thumb, when we define a model using a DAG, we need to be able to define the 
probability of each variable given its \emph{parents}, i.e. all the nodes with arrows, representing a link and its direction, pointing
to the node the question.

To demonstrate this factorization more explicitly and give a concrete example of a DAG, 
%consider a joint model
%$p(a,b,c)$.  
%By the product rule, we can break this down into a number of different
%factorizations, but some will typically be more useful than others.  
imagine a medical diagnostic problem where we wish to predict if a patient has lung cancer.
et $a$ denote lifestyle and genetic factors of the patient such as whether they smoke
or have (potentially unknown) preexisting conditions. 
These will generally either be known or can reasonably estimated by other means, e.g. using tests or
by considering prevalence within the population, allowing definition of a prior marginal on $a$, $p(a)$.
Given these factors, we can develop a model for the probability that a patient will develop
lung cancer, which we can denote $p(b|a)$ where $b=1$ indicates cancer is the present.  Given the lifestyle and
genetic factors and the knowledge of whether lung cancer is present, we can predict what
symptoms, $c$, might be observed, e.g. a persistent cough, encoding this using $p(c|a,b)$.
We thus have the following breakdown of the joint distribution
\begin{align}
	\label{eq:bayes:example-graph}
	p(a,b,c) = p(a) p(b|a) p(c|a,b).
\end{align}
\begin{wrapfigure}{r}{0.35\textwidth}
	\vspace{-12pt}
	\centering 
	\resizebox{.32\textwidth}{!}{
		\input{bayes/figures/simple_graph.tex}
	}
	\caption{Simple example DAG corresponding to~\eqref{eq:bayes:example-graph}
		\label{fig:bayes:simple-graph}}
	\vspace{-10pt}
\end{wrapfigure}
which can be expressed using
the DAG  shown in Figure~\ref{fig:bayes:simple-graph}. Here we have shaded in $c$ to express the fact that this is
observed.  The graphical model expresses our dependency structure as we have the probability
of each node given its parents.  As shown in~\eqref{eq:bayes:example-graph}, the product of
all these factors is equal to our joint distribution.  The DAG has thus formed a convenient means
of representing our generative process.
Our aim for this problem was to find the probability cancer is present given
the observed symptoms, i.e. $p(b|c)$, which will
require Bayesian inference.  In our previous simple examples, the posterior had an analytic form. 
However, In general this will not be the case and we will need to develop strategies for carrying
out the inference as explained in Chapter~\ref{chp:inf}.  For these, knowing the dependency
structure and, in particular, the independence relationships, of a model will often be very helpful
information, for which DAGs can be very useful.

A natural question is now how can we deduce the independence relationships from a DAG?
This can be done by introducing the notion of \emph{d-separation}~\citep{pearl2014probabilistic}.
Consider three arbitrary, non-intersecting, subsets $A$, $B$, and $C$ of a DAG.  $A$ and $B$
are conditionally independent given $C$ if there are no \emph{unblocked} paths from $A$ to $B$
(or equivalently from $B$ to $A$), in which case $A$ is said to be d-separated from $B$ by $C$.  
Paths do not need to be in the directions defined by the DAG but are blocked if either
\vspace{-5pt}
\begin{enumerate}
		\setlength\itemsep{-0.1em}
	\item Consecutive arrows both point towards a node that is not in and
	has no descendants in $C$, i.e. we cannot get to any of the nodes in $C$ by following the arrows
	from this node.
	\item Consecutive arrows meet at a node in $C$ and one of them
	points away from the node.
\end{enumerate}
\vspace{-5pt}
Note that only the first of these rules is necessary for establishing marginal independence
between nodes as this rule can still be used when $C$ is empty.
Examples of blocked paths are shown in Figure~\ref{fig:bayes:blocked-graphs} while examples of unblocked paths
are shown in Figure~\ref{fig:bayes:unblocked-graphs}, explanations for which are given in the captions.
For a more comprehensive introduction to establishing independence in DAGs, we
refer the reader to~\cite[Section 8.2]{bishop2006pattern}.

\begin{figure}[t]
	\centering 
	\begin{subfigure}[t]{0.32\textwidth}
		\centering
		\resizebox{0.9\textwidth}{!}{
		% !TEX root = ../../main.tex

\begin{tikzpicture}

\node[obs]                               (c) {$c$};
\node[latent, above=of c, xshift=-1.2cm] (a) {$a$};
\node[latent, above=of c, xshift=1.2cm]  (b) {$b$};

\edge {a} {c} ; %
\edge {c} {b} ; %

\end{tikzpicture}}
		\caption{\label{fig:bayes:block1}}
	\end{subfigure}
	\begin{subfigure}[t]{0.32\textwidth}
		\centering
		\resizebox{0.9\textwidth}{!}{
		% !TEX root = ../../main.tex

\begin{tikzpicture}

\node[obs]                               (c) {$c$};
\node[latent, below=of c, xshift=-1.2cm] (a) {$a$};
\node[latent, below=of c, xshift=1.2cm]  (b) {$b$};

\edge {c} {a} ; %
\edge {c} {b} ; %

\end{tikzpicture}}
		\caption{\label{fig:bayes:block2}}
	\end{subfigure}
	\begin{subfigure}[t]{0.32\textwidth}
		\centering
		\resizebox{0.9\textwidth}{!}{
		% !TEX root = ../../main.tex

\begin{tikzpicture}

%\node[latent]  (a) {$a$};
%\node[latent, right=2.4cm of a]  (b) {$b$};
%\node[latent, below= of a, xshift=1.2cm] (d) {$d$};
%%\node[latent, right=1.8cm of d]             (e) {$e$};
%
%%\edge {d} {e} ; %
%\edge {a} {d} ; %
%\edge {b} {d} ; %

\node[latent]                               (d) {$d$};
\node[latent, above=of d, xshift=-1.2cm] (a) {$a$};
\node[latent, above=of d, xshift=1.2cm]  (b) {$b$};

\edge {a} {d} ; %
\edge {b} {d} ; %

\end{tikzpicture}}
		\caption{\label{fig:bayes:block3}}
	\end{subfigure}
	\caption{Examples of DAGs blocked between $a$ and $b$.  \textbf{(a)} is blocked
		by the second rule of d-separation, while
		\textbf{(b)} is blocked by the first rule of d-separation.  Consequently, for \textbf{(a)} 
		and \textbf{(b)} then $a$ and $b$ are conditionally independent given $c$ and thus
		$p(b|a,c)=p(b|c)$ and $p(a|b,c)=p(a|c)$.
		\textbf{(c)} is an instead example of where $a$ and $b$ are \emph{marginally} independent.  Here
		the path between $a$ and $b$ is blocked because the the arrows meet head-to-head at 
		$d$ and neither $d$ nor any of its
		descendants are observed and so $p(b|a) = p(b)$.
		Note though that $a$ and $b$ are not conditionally independent given $d$ as
		the path becomes unblocked if this is observed (see Figure~\ref{fig:bayes:unblocked-graphs}).
		\label{fig:bayes:blocked-graphs}
		\vspace{5pt}}
\end{figure}

\begin{figure}[t]
	\centering 
	\hspace{-15pt}
	\begin{subfigure}[t]{0.32\textwidth}
		\centering
		\resizebox{0.9\textwidth}{!}{
		% !TEX root = ../../main.tex

\begin{tikzpicture}

\node[latent]                               (d) {$d$};
\node[latent, above=of d, xshift=-1.2cm] (a) {$a$};
\node[latent, above=of d, xshift=1.2cm]  (b) {$b$};

\edge {a} {d} ; %
\edge {d} {b} ; %

\end{tikzpicture}}
		\caption{\label{fig:bayes:unblock1}}
	\end{subfigure}
	\begin{subfigure}[t]{0.32\textwidth}
		\centering
		\resizebox{0.9\textwidth}{!}{
		% !TEX root = ../../main.tex

\begin{tikzpicture}

\node[latent]                               (d) {$d$};
\node[latent, below=of d, xshift=-1.2cm] (a) {$a$};
\node[latent, below=of d, xshift=1.2cm]  (b) {$b$};

\edge {d} {a} ; %
\edge {d} {b} ; %

\end{tikzpicture}}
		\caption{\label{fig:bayes:unblock2}}
	\end{subfigure}
	\begin{subfigure}[t]{0.32\textwidth}
		\centering
		\resizebox{1.1\textwidth}{!}{
		% !TEX root = ../../main.tex

\begin{tikzpicture}

\node[latent]                               (d) {$d$};
\node[latent, above=of d, xshift=-1.2cm] (a) {$a$};
\node[latent, above=of d, xshift=1.2cm]  (b) {$b$};
\node[obs, right=1.2cm of d]             (c) {$c$};

\edge {d} {c} ; %
\edge {a} {d} ; %
\edge {b} {d} ; %

\end{tikzpicture}}
		\caption{\label{fig:bayes:unblock3}}
	\end{subfigure}
	\caption{Examples of DAGs unblocked between $a$ and $b$. For \textbf{(a)} then the path from $a$ to
		$b$ is not blocked by the d-separation rules.  Perhaps more intuitively, we have that
		$p(b|a) = \int p(b,d|a) dd = \int p(b|d)p(d|a) dd \neq p(b)$ unless $p(d|a)=p(d)$.
		Similarly there is an unblocked path for \textbf{(b)} from $a$ to $b$ as the path
		that does not pass through any observed nodes or nodes with both arrows points towards it.
		The path in \textbf{(c)} is unblocked because of the phenomenon of \emph{explaining away}.
		The first rule of d-separation does not apply here because $c$ is an observed descendant of
		$d$.  We thus have that $a$ and $b$ are marginally independent as per Figure~\ref{fig:bayes:block3},
		but not conditionally independent given $c$ (and or $d$). The rationale for explaining away can be thought
		of in terms of events needing an explanation -- if two precursor events (here $a$ and $b$) can give rise 
		to a third event (here $c$), then
		the third event occurring but not the first precursor event implies that the second precursor
		event occurs.  Thus the two precursor events are correlated because one \emph{explains away} the other.
		As described in Section~\ref{sec:prob:cond}, one example of this is that if a speed camera is
		triggered and the camera is not malfunctioning, this implies the vehicle is speeding, even though
		the vehicle speeding and the camera malfunctioning are marginally independent.	
		\label{fig:bayes:unblocked-graphs}}
\end{figure}

In the simple example of Figure~\ref{fig:bayes:simple-graph} there were no independence relationships and so
we gain little from working with the DAG compared to just the joint probability distribution.
A more advanced example where there are substantial independence relationships which can
be exploited is shown in Figure~\ref{fig:bayes:hmm}.
  This model is known as a hidden Markov 
model\footnote{The use of the tern HMM in the literature (and later in this paper) often also implies that
	the latent states are discrete variables, with the equivalent continuous model referred to as a
	(Markovian) state space model.} (HMM)
and has $T$ latent variables $x_{1:T}$ and $T$ observations $y_{1:T}$.  The joint distribution is as follows
\begin{align}
\label{eq:bayes:hmm}
p(x_{1:T},y_{1:T}) = p(x_1) p(y_1|x_1)\prod_{t=2}^{T} p(x_t|x_{t-1})p(y_t|x_t),
\end{align}
where each $x_t$ is
independent of $x_{1:{t-2}}$ and $y_{1:t-1}$ given $x_{t-1}$ and of $x_{t+2:T}$ and $y_{{t+1}:T}$ given $x_{t+1}$.
This is known as the Markov property and means that each latent variable only depends on the other
variables and observations through its immediate neighboring states.  In essence, the model has no memory as information
is passed forwards (or backwards) only through the value of the previous (or next) latent state.  A number of stochastic
processes and dynamical systems obey the Markov property and HMMs and their extensions are extensively used for
a number of tasks involving sequential data, such as DNA sequencing~\citep{durbin1998biological} and tracking
animals~\citep{dhir2016tracking,dhir2017interpreting} to name but a few. 
%For example, in a dynamical system, then the next state
%depends only on the dynamics of the system and its  current state (e.g. position velocity etc).

\begin{figure}[t]
	\centering 
	% !TEX root =  bopp.tex

\begin{figure*}[t]
	\centering
	\includegraphics[width=2.72in]{hmm/hmm_ML}
	\includegraphics[width=2.72in]{hmm/hmm_distance}
	\caption{Convergence for HMM in terms of the cumulative best $\log p\left(Y,\theta\right)$ (\emph{left}) and distance to the ``true" $\theta$ used in generating the data (\emph{right}). Solid line shows median over 100 runs, whilst the shaded region the 25/75\% quantiles.  Note that for the distance to true $\theta$ was calculated by selecting which three states (out of the 5 generates) that were closest to the true parameters.  \label{fig:hmm}}
\end{figure*}

We finally consider a hidden Markov model (HMM) with an unknown number of states.  This example demonstrates how BOPP can be applied to models which conceptually have an unknown number of variables, by generating all possible variables that might be needed, but then leaving some variables unused for some execution traces.  This avoids problems of varying base measures so that the MMAP problem is well defined  and provides a function with a fixed number of inputs as required by the BO scheme.  From the BO perspective, the target function is simply constant for variations in an unused variable.

HMMs are Markovian state space models with discrete latent variables.  Each latent state $x_t \in\{1,\dots,K\}, t=1,\dots,T$ is defined conditionally on $x_{t-1}$ through a set of discrete transition probabilities, whilst each output $y_t\in\real$ is considered to be generated i.i.d. given $x_t$.  We consider the following HMM, in which the number of states $K$, is also a random variable: 
\begin{align}
\label{eq:hmm}
K & \sim \text{Discrete}\{1,2,3,4,5\} \\\displaybreak[0]
T_k &\sim \text{Dirichlet}\{{1}_{1:K}\}, \quad \forall k=1,\dots,K \\\displaybreak[0]
\phi_k &\sim \text{Uniform}[0,1], \quad \forall k=1,\dots,K \\\displaybreak[0]
\mu_0 &\leftarrow \min \{y_{1:T}\} \\\displaybreak[0]
\mu_k &\leftarrow \mu_{k-1}+\phi_k \cdot (\max \{y_{1:T}\} -\mu_{k-1}), \quad \forall k=1,\dots,K \\\displaybreak[0]
x_1 &\leftarrow 1 \\\displaybreak[0]
x_t | x_{t-1} &\sim \text{Discrete}\{T_{x_{t-1}}\}\\\displaybreak[0]
y_t | x_t &\sim\mathcal{N}(\mu(x_{t-1}),0.2).
\end{align}
Our experiment is based on applying BOPP to the above model to do MMAP estimation with a single synthetic dataset, generated using $K=3, \;\mu_1 = -1, \;\mu_2 = 0, \;\mu_3 = 4, \;T_1 = [0.9,0.1,0], \;T_2=[0.2,0.75,0.05]$ and $T_3=[0.1,0.2,0.7]$.  

We use BOPP to optimize both the number of states $K$ and the stick-breaking parameters $\phi_k$, with full inference performed on the other parameters.  BOPP therefore aims to maximize
\begin{align}
\label{eq:hmm-marginal}
p(K,\phi_{k=1:5}|y_{t=1:T}) = \iint p(K,\phi_{k=1:5},x_{t=1:T},T_{k=1:K}|y_{t=1:T}) \mathrm{d}x_{t=1:T} \mathrm{d}T_{k=1:K}.
\end{align}
As with the chaotic Kalman filter example, we compare to two PMMH variants using the same code transformations.  The results, given in Figure~\ref{fig:hmm}, again show that BOPP outperforms these PMMH alternatives.
	\caption{DAG for a hidden Markov model.
		\label{fig:bayes:hmm}}
\end{figure}

A key part of the appeal of HMMs is that the structure of the
DAG can be exploited to give analytic solutions to the resulting Bayesian inference whenever each $p(y_t | x_t)$ 
and $p(x_t | x_{t-1})$ are either a categorical or Gaussian distribution. Even when this does not hold, there 
are still a number of features of the dependency structure that can make the inference substantially easier.
As we will show in Chapter~\ref{chp:inf}, Bayesian inference is generally a challenging problem, often prohibitively so.
Therefore the (fast) analytic inference for HMMs is highly convenient.  However, it can mean that HMMs are perhaps overused.
More generally, simplifying approximations or unjustified assumptions are often made by Bayesian practitioners 
for tractability, e.g. by using an off-the-shelf model like an HMM with known analytic solution. Though often necessary, 
this must be done with extreme care and the implications of the approximations should carefully considered.  Unfortunately, 
quantifying the implications of approximations can be very difficult, given that they are typically made in the interest
of tractability in the first place.  
%This is a serious practical weakness of Bayesian modeling and should be factored 
%into the decision processes of whether taking a Bayesian approach or not.  
%While taking a Bayesian approach for large projects, 
%one should, in general, avoid making approximations or non-clearcut assumptions until absolutely necessary, but also 
%keep in mind that it may be necessary to do so further down the line and account for this appropriately.
% !TEX root = ../main.tex

\section{Graphical Models}
\label{sec:bayes:paradigm:graph}

Generative models will typically have many variables and a complex \emph{dependency structure}.
In other words, many of variables will be conditionally independent of one another given values for
other variables.  Graphical models are a ubiquitously used method for representing and reasoning
about generative models with a particular focus on the dependency structure.  At a high-level, they
capture how the joint probability distribution can be broken down into a product of different factors, 
each defined over
a subset of the variables.  They are formed of a number of connected nodes, where each node
represents a random variable in the model and each variable has its own node.  Links between nodes in
the model represent dependencies: any two connected nodes have an explicit dependency.
Various independence assumptions can be deduced from the graphical model, though the exact nature
of these deductions will depend or the type of graphical model -- nodes without direct links
between them will often still be dependent.

Graphical models can be seperated into two distinct classes: directed graphical
models and undirected graphical models.  Undirected graphical models, also known as Markov random
fields, imply no ordering on their factorization and are used only to express conditional independences
between variables.  They are used in scenarios where it is difficult to specify the target distribution in a
structured generative way.  For example, if modeling whether it will rain at various locations then there
is a clear dependence between nearby locations, but not a natural ordering to the joint probability
distribution of where it will rain.  Boltzmann machines are a common example of undirected graphical
models~\citep{ackley1985learning}.  Independence in undirected graphical models can be deduced
through the \emph{global Markov property} which states that any two non-intersecting subsets of the 
variables $A$ and $B$ are
conditionally independent given a third, separating, subset $C$ if there is no path between $A$ and
$B$ that does not pass through $C$.  This means, for example, that each variable is conditionally
independent of all the other variables given its neighbors.

Our main focus, though, will instead be on directed graphical models and in particular directed acyclic 
graphical models (DAGs), i.e. directed graphical models containing no cycles or loops one can follow 
and arrive back in the starting position.  DAGs, also known as Bayesian networks, are particularly
useful in the context of Bayesian modeling because they can be used to express \emph{casual relationships}.
As such, they can be used as a piecewise explanation for how the joint distribution is generated.
Not only does this form a natural means to describe and design models in the first place, 
we can carefully order the breakdown to factorize the distribution into only terms we know.  For example,
in the linear regression model we did not know (at least when the model was first defined) 
$p(\mathcal{D})$ but we did know $p(\mathbf{w})$ and $p(\mathcal{D} | \mathbf{w})$.  Therefore even
though we could factorize our joint $p(\mathcal{D}, \mathbf{w})$ as 
$p(\mathbf{w} | \mathcal{D})p(\mathcal{D})$ and construct a DAG this way, it is much more convenient
to factorize and construct the DAG the other way round, namely as 
$p(\mathcal{D} | \mathbf{w})p(\mathbf{w})$.
As a good rule of thumb, when we define a model using a DAG, we need to be able to define the 
probability of each variable given its \emph{parents}, i.e. all the nodes with links from that node
towards the node the question.  We will generally not have access to also possible factorizations
in an analytic form as otherwise there would be no need to perform inference.

The demonstrate this factorization more explicitly and give a concrete example of a DAG, consider a joint model
$p(a,b,c)$.  By the product rule, we can break down this joint distribution in to a number of different
factorizations.  However, some will typically be more useful then others.  Imagine a medical diagnostic
example where $a$ represents lifestyle and genetic factors of a patient such as whether they smoke
or have unknown preexisting conditions, 
$b$ represents the presence of lung cancer, and $c$ represents symptoms 
such as a persistent cough.  Here we have the following natural breakdown of the joint distribution
\begin{align}
\label{eq:bayes:example-graph}
p(a,b,c) = p(a) p(b|a) p(c|a,b).
\end{align}
\begin{wrapfigure}{r}{0.35\textwidth}
	\vspace{-12pt}
	\centering 
	\resizebox{.32\textwidth}{!}{
		% !TEX root = ../../main.tex

\begin{tikzpicture}

\node[obs]                               (c) {$c$};
\node[latent, above=of c, xshift=-1.2cm] (a) {$a$};
\node[latent, above=of c, xshift=1.2cm]  (b) {$b$};

\edge {a,b} {c} ; %
\edge {a} {b} ; %

\end{tikzpicture}
	}
	\caption{Simple example DAG corresponding to~\eqref{eq:bayes:example-graph}
		\label{fig:bayes:simple-graph}}
	\vspace{-10pt}
\end{wrapfigure}
The lifestyle and genetic factors will generally either be known or can be estimated from tests or
simply prevalence within the population.  These factors, the likelihood of somebody developing
lung cancer can be modeled using existing data and domain expertise.   Given the lifestyle and
genetic factors and the knowledge of whether lung cancer is present, we can predict the likelihood
of observing the observed symptoms.  We can express our model and this factorization using
the DAG shown in Figure~\ref{fig:bayes:simple-graph}.
  Here we have shaded in $c$ to express the fact that this is
observed.  The graphical model expresses our dependency structure as we have the probability
of each node given its parents.  As shown in~\eqref{eq:bayes:example-graph}, the product of
all these factors is equal to our joint distribution.  The DAG has thus formed a convenient means
of representing our generative process.

Clearly our aim for this problem will be to find the marginal probability $p(b|c)$.  To calculate this
we will need to perform Bayesian inference as explained.
An importance feature of the breakdown of graphical models will become apparent when we
consider how we can conduct Bayesian inference more a general class of 
models where the solution is no analytic in Chapter~\ref{chp:inf}.  Here knowing the dependency
structure, and in particular the independence relationships, will be essential to many inference 
schemes such as message passing schemes.

A natural question is now how can we deduce the independence relationships from DAG?
This can be done by introducing the notion of \emph{d-separation}~\citep{pearl2014probabilistic}.
Consider three arbitrary, non-intersecting, subsets of our DAG $A$, $B$, and $C$.  $A$ and $B$
are conditionally independent given $C$ if there is no \emph{unblocked} paths from $A$ to $B$
(or equivalently from $B$ to $A$), in which case $A$ is said to be d-separated from $B$ by $C$.  
Paths do not need to be in the directions defined by the DAG but are blocked if either
\begin{enumerate}
	\item Two consecutive arrows in the path both point towards a node that is not $C$ and
	has no descendants in $C$, i.e. we cannot get to any of the nodes in $C$ by following the arrows
	from this node.
	\item Two consecutive arrows in the path meet at a node in $C$ and one of them
	points away from the node.
\end{enumerate}
Note that only the first of these rules is necessary for establishing marginal independence
between nodes as this rule can still be used when $C$ is empty.
Examples of blocked paths are shown in Figure~\ref{fig:bayes:blocked-graphs} while examples of unblocked paths
are shown in Figure~\ref{fig:bayes:unblocked-graphs}, explanations for which are given in the captions.
For a more comprehensive introduction to establishing independence in DAGs we
refer the reader to Section 8.2 of~\cite{bishop2006pattern}.

\begin{figure}[p]
	\centering 
	\begin{subfigure}[t]{0.32\textwidth}
		\centering
		\resizebox{0.9\textwidth}{!}{
		% !TEX root = ../../main.tex

\begin{tikzpicture}

\node[obs]                               (c) {$c$};
\node[latent, above=of c, xshift=-1.2cm] (a) {$a$};
\node[latent, above=of c, xshift=1.2cm]  (b) {$b$};

\edge {a} {c} ; %
\edge {c} {b} ; %

\end{tikzpicture}}
		\caption{\label{fig:bayes:block1}}
	\end{subfigure}
	\begin{subfigure}[t]{0.32\textwidth}
		\centering
		\resizebox{0.9\textwidth}{!}{
		% !TEX root = ../../main.tex

\begin{tikzpicture}

\node[obs]                               (c) {$c$};
\node[latent, below=of c, xshift=-1.2cm] (a) {$a$};
\node[latent, below=of c, xshift=1.2cm]  (b) {$b$};

\edge {c} {a} ; %
\edge {c} {b} ; %

\end{tikzpicture}}
		\caption{\label{fig:bayes:block2}}
	\end{subfigure}
	\begin{subfigure}[t]{0.32\textwidth}
		\centering
		\resizebox{0.9\textwidth}{!}{
		% !TEX root = ../../main.tex

\begin{tikzpicture}

%\node[latent]  (a) {$a$};
%\node[latent, right=2.4cm of a]  (b) {$b$};
%\node[latent, below= of a, xshift=1.2cm] (d) {$d$};
%%\node[latent, right=1.8cm of d]             (e) {$e$};
%
%%\edge {d} {e} ; %
%\edge {a} {d} ; %
%\edge {b} {d} ; %

\node[latent]                               (d) {$d$};
\node[latent, above=of d, xshift=-1.2cm] (a) {$a$};
\node[latent, above=of d, xshift=1.2cm]  (b) {$b$};

\edge {a} {d} ; %
\edge {b} {d} ; %

\end{tikzpicture}}
		\caption{\label{fig:bayes:block3}}
	\end{subfigure}
	\caption{Examples of DAGs blocked between $a$ and $b$.  \textbf{(a)} is blocked
		by the second rule of d-separation.  More specifically we have 
		$p(b|a,c)=\frac{p(a,b,c)}{p(a,c)} = \frac{p(a)p(c|a)p(b|c)}{p(a)p(c|a)} = p(b|c)$.
		\textbf{(b)} is blocked by the first rule of d-separation.  Here we have
		$p(b|a,c)=\frac{p(c)p(a|c)p(b|c)}{p(c)p(a|c)} = p(b|c)$.  Consequently, for \textbf{(a)} 
		and \textbf{(b)} then $a$ and $b$ are conditionally independent given $c$ .
		\textbf{(c)} is an instead example of where $a$ and $b$ are marginally independent.  Here
		the path between $a$ and $b$ is blocked because the the arrows meet head-to-head at 
		$d$ and neither $d$ nor any of its
		descendants are observed.  We thus have $p(b|a) = \frac{p(a)p(b)p(d|a,b)}{p(a)p(d|a,b)} = p(b)$.
		Note though that $a$ and $b$ are note conditionally independent given $d$ as
		the path becomes unblocked if this is observed (see Figure~\ref{fig:bayes:unblocked-graphs}).
		\label{fig:bayes:blocked-graphs}}
\end{figure}

\begin{figure}[p]
	\centering 
	\hspace{-15pt}
	\begin{subfigure}[t]{0.32\textwidth}
		\centering
		\resizebox{0.9\textwidth}{!}{
		% !TEX root = ../../main.tex

\begin{tikzpicture}

\node[latent]                               (d) {$d$};
\node[latent, above=of d, xshift=-1.2cm] (a) {$a$};
\node[latent, above=of d, xshift=1.2cm]  (b) {$b$};

\edge {a} {d} ; %
\edge {d} {b} ; %

\end{tikzpicture}}
		\caption{\label{fig:bayes:unblock1}}
	\end{subfigure}
	\begin{subfigure}[t]{0.32\textwidth}
		\centering
		\resizebox{0.9\textwidth}{!}{
		% !TEX root = ../../main.tex

\begin{tikzpicture}

\node[latent]                               (d) {$d$};
\node[latent, below=of d, xshift=-1.2cm] (a) {$a$};
\node[latent, below=of d, xshift=1.2cm]  (b) {$b$};

\edge {d} {a} ; %
\edge {d} {b} ; %

\end{tikzpicture}}
		\caption{\label{fig:bayes:unblock2}}
	\end{subfigure}
	\begin{subfigure}[t]{0.32\textwidth}
		\centering
		\resizebox{1.1\textwidth}{!}{
		% !TEX root = ../../main.tex

\begin{tikzpicture}

\node[latent]                               (d) {$d$};
\node[latent, above=of d, xshift=-1.2cm] (a) {$a$};
\node[latent, above=of d, xshift=1.2cm]  (b) {$b$};
\node[obs, right=1.2cm of d]             (c) {$c$};

\edge {d} {c} ; %
\edge {a} {d} ; %
\edge {b} {d} ; %

\end{tikzpicture}}
		\caption{\label{fig:bayes:unblock3}}
	\end{subfigure}
	\caption{Examples of DAGs unblocked between $a$ and $b$. For \textbf{(a)} then the path from $a$ to
		$b$ is not blocked by the d-separation rules.  Perhaps more intuitively, we have that
		$p(b|a) = \int p(b,d|a) dd = \int p(b|d)p(d|a) dd \neq p(b)$ unless $p(d|a)=p(d)$.
		Similarly there is an unblocked path for \textbf{(b)} from $a$ to $b$ as the path
		that does not pass through any observed nodes or nodes with both arrows points towards it.
		Here we have $p(b|a) = \frac{p(d)p(a|d)p(b|d)}{p(a)p(d|a,b)} \neq p(b)$ again in general.
		The path in \textbf{(c)} is unblocked because of the phenomenon of \emph{explaining away}.
		The first rule of d-separation does not apply here because $c$ is an observed descent of
		$d$.  We thus have that $a$ and $b$ are marginally independent as per Figure~\ref{fig:bayes:block3},
		but not conditionally independent given $c$ (and/or $d$). The rationale for explaining away can thought
		of in terms of events needing an explanation -- if two precursor events (here $a$ and $b$) can give rise 
		to a third event (here $c$), then
		the third event occurring but not the first precursor event implies that the second precursor
		event occurs.  Thus the two precursor events are correlated because one \emph{explains away} the other.
		As described in Section~\ref{sec:prob:cond}, one example of this is that if a speed camera is
		triggered and the camera is not malfunctioning, this implies the vehicle is speeding, even though
		the vehicle speeding and the camera malfunctioning are marginally independent.	
		\label{fig:bayes:unblocked-graphs}}
\end{figure}

In the simple example of Figure~\ref{fig:bayes:simple-graph} there were no independence relationships and so
we gain little from working with the DAG compared to just the joint probability distribution.
A more advanced example where there are substantial independence relationships which can
be exploited is shown in Figure~\ref{fig:bayes:hmm}.
  This model is known as a hidden Markov model 
(HMM)\footnote{The use of the tern HMM in the literature (and later in this paper) often also implies that
	the latent states are discrete variables, with the equivalent continuous model referred to as a
	(Markovian) state space model.}
and has $T$ latent variables $x_{1:T}$ and $T$ observations $y_{1:T}$.  The joint distribution is as follows
\begin{align}
\label{eq:bayes:hmm}
p(x_{1:T},y_{1:T}) = p(x_1) p(y_1|x_1)\prod_{t=2}^{T} p(x_t|x_{t-1})p(y_t|x_t).
\end{align}
where each $x_t$ is
independent of $x_{1:{t-2}}$ and $y_{1:t-1}$ given $x_{t-1}$ and of $x_{t+2:T}$ and $y_{{t+1}:T}$ given $x_{t+1}$.
This is known as the Markov property and means that each latent variable only depends on the other
variables and observations through its immediate neighbours.  In essence, the model has no memory as information
is passed forwards (or backwards) only through the value of the previous latent state.  A number of stochastic
processes and dynamical systems obey the Markov property and HMMs and their extensions are extensively used for
a number of tasks involving sequential data from DNA sequencing~\citep{durbin1998biological} to tracking
animals~\citep{dhir2016tracking,dhir2017interpreting}.  A key part to their appeal is that the structure of the
DAG can be exploited to give analytic solutions to resulting Bayesian inference whenever each $p(y_t | x_t)$ 
and $p(x_t | x_{t-1})$ is either a discrete or Gaussian distribution. Even when this does not hold, there 
are still a number of features of the dependency structure that can make the inference substantially easier.

\begin{figure}[t]
	\centering 
	% !TEX root = ../../main.tex

\begin{tikzpicture}

\node[latent, minimum size=27pt] (x1) {$x_1$};
\node[latent, right=1.4cm of x1, minimum size=27pt] (x2) {$x_2$};
\node[right=1.4cm of x2] (x3) {{\tiny $\bullet \; \bullet \; \bullet$}};
\node[latent, right=1.4cm of x3, minimum size=27pt] (x4) {$x_{T\text{-}1}$};
\node[latent, right=1.4cm of x4, minimum size=27pt] (xT) {$x_T$};

\node[obs, below=of x1, minimum size=27pt] (y1) {$y_1$};
\node[obs, below=of x2, minimum size=27pt] (y2) {$y_2$};
%\node[obs, below=of x3] (y3) {$y_3$};
\node[obs, below=of x4, minimum size=27pt] (y4) {$y_{T\text{-}1}$};
\node[obs, below=of xT, minimum size=27pt] (yT) {$y_T$};

%\node[latent, above=of x1, xshift=-1.2cm, minimum size=27pt] (t) {$\theta$};

\edge {x1} {x2,y1} ; %
\edge {x2} {x3,y2} ; %
\edge {x3} {x4} ; %
\edge {x4} {xT,y4} ; %
\edge {xT} {yT} ; %

%\edge {t} {x1,x2};
%\edge[bend left=0] {t} {x4};
%\edge[bend left=10] {t} {xT};

\end{tikzpicture}
	\caption{DAG for a hidden Markov model.
		\label{fig:bayes:hmm}}
\end{figure}

We finish the section by noting a potential problem with the inference for models such as the HMM being analytic,
or more specifically a problem with the lack of analytical solutions for other problems.  As we will show
in Chapter~\ref{chp:inf}, Bayesian inference is generally a challenging problem, often prohibitively so.  As
such, simplifying approximations or unjustified assumptions are often made for tractability so that a model 
like the HMM can be used. Though often necessary, this must be done with extreme care and implications of 
the approximations carefully considered.
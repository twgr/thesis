% !TEX root = ../main.tex

\section{Discriminative vs Generative Machine Learning}
\label{sec:bayes:discrim}

In some machine learning applications, huge quantities of data are available that dwarf the information
that can be provided from human expertise.  In such situations, the main challenge is in processing
and extracting all the desired information from the data to form a useful characterization,
typically an artifact providing accurate predictions at previous unseen inputs. 
Such problems are typically suited to \emph{discriminative} 
machine learning approaches~\citep{breiman2001statistical,vapnik1998statistical}, such as neural
networks~\citep{rumelhart1986learning,bishop1995neural}, 
support vector machines~\citep{cortes1995support,scholkopf2002learning}, and decision tree 
ensembles~\citep{breiman2001random,rainforth2015canonical}.  Discriminative machine learning approaches
focus on directly learning a predictive model: given training data $\mathcal{D} = \left\{x_n,y_n\right\}_{n=1}^N$
they learn a parametrized mapping $f_{\theta}$ from the inputs $x \in \mathcal{X}$ to the 
outputs $y\in\mathcal{Y}$ that can 
be used directly to make predictions 
for new inputs $\tilde{x} \notin \left\{x_n\right\}_{n=1}^N$.  \emph{Training}
uses the data $\mathcal{D}$ to estimate optimal values of the parameters $\theta^*$. \emph{Prediction}
at a new input $\tilde{x}$ involves applying the mapping with the optimal parameters given an estimate for the output
$\tilde{y} = f_{\theta^*}(\tilde{x})$.  Perhaps the simplest example of this is linear regression: one finds
the hyperplane that best represents the data and then uses this hyperplane to interpolate or extrapolate
to predict the output at previously unseen points.  
As a more advanced example, in a neural network one uses training to learn the
weights of the network, after which prediction can be done by running the network forwards.  

There are many intuitive reasons to take a discriminative machine learning 
approach.  Perhaps most compelling is the
idea that if our objective is prediction, then it is simplest to solve that problem directly, rather
than try and solve some more general problem such as learning an underlying generative 
process~\citep{vapnik1998statistical,breiman2001statistical}. Furthermore, if sufficient
data is provided, discriminant approaches can be spectacularly successful in term of predictive
performance.  Methods are typically highly flexible and can capture intricate structure in the data that
would be hard or even impossible to establish manually.  Many approaches can also be run with little
or no input on behalf of the user, delivering state-of-the-art performance when used
``out-of-the-box'' with default parameters~\citep{rainforth2015canonical}.  

However, this black-box nature is also often their downfall.  Discriminative methods typically make
such weak assumptions about the underlying process that is difficult to impart prior knowledge
or domain-specific expertise.  This can be disastrous if insufficient data is available as the data
alone is unlikely to possess the required information to make adequate predictions.  Even when
substantial data is available, there may be significant prior information available that needs to be
exploited for effective performance.  For example, in time series modelling the sequential nature
of the data is critically important information~\citep{liu1998sequential}, while in vision tasks the 
knowledge that scenes are generated from objects can be invaluable~\citep{kulkarni2015picture}.
Many problems also increase in complexity as more data is added -- ``big data'' problems are often
actually a collection, or sometimes hierarchy, of many small problems, such that the complexity of the
required parametrization increases are more data is added.  Consider, for example, modeling interactions in
a social network.  Adding a new user into the model increases the amount of data, but also
requires the model to grow and accommodate the new user~\citep{ravasz2003hierarchical}.  In
this situation it is essential to
use an approach that respects the structure of the model, while the amount of data available
for each individual user is often quite small, such that it will essential to use prior information
by transferring insights gathered from some users to others.  Therefore even for such large-scale
problems, the inflexibility of many discriminative approaches to incorporate known characteristics
of the target problem can be problematic.

Not only does the black-box nature of many discriminative methods restrict the level of
human input that can be imparted on the system, it often restricts the amount of insight
and information that can be extracted from the system once trained.  The parameters in most discriminative
algorithms do not have physical meaning that can be queried by a user, making their operation
difficult to interpret and hampering the process of improving the system through manual
revision of the algorithm.  Furthermore, this typically makes them inappropriate for more
statistics orientated tasks, where it is the parameters themselves which are of interest, rather
than the ability for the system itself to make predictions.  For example, the parameters may
have real-world physical interpretations we wish to learn about.

Most discriminative methods are also poor at providing realistic uncertainty estimates.
Because they are typically trained in a manner that optimizes the parameters to minimize
some loss criterion (e.g. the predictive error), they do not, in general, encode any uncertainty
in either their parameters or the subsequent predictions.  Though many methods can
produce uncertainty estimates either as a by-product or from a post-processing step,
these are typically mostly heuristic based, rather than stemming naturally from a statistically
principled estimate of the target uncertainty distribution.   The lack of reliable uncertainty
estimates can lead to overconfidence and can make discriminative methods inappropriate in
many scenarios.  This can also reduce the composability of discriminative methods within
larger systems, as information is lost when only providing a point estimate.
Not representing uncertainty in the parameters can also restrict the power of the resultant
models, compared with approaches that can average over different possible parameter values.

These shortfalls mean that many tasks instead call for a \emph{generative} machine learning
approach~\citep{ng2002discriminative,bishop2006pattern}.  Rather than directly learning a 
predictor, generative methods look to explain the observed data using a \emph{probabilistic model}.
Whereas discriminative approaches aim only to make predictions, generative approaches model
how the data is actually generated; they model the joint probability $p(X,Y)$ of the inputs 
$X$ and outputs $Y$.  By comparison, we can think of discriminative approaches as
only modeling the outputs given the inputs $p(Y|X)$.  

A key upshot of this difference
is that generative approaches generally make stronger modeling assumptions about the problem.  Though
this can be problematic when the model assumptions are wrong and is often unnecessary in
the limit of large data, it is essential for combining prior information with information from data
and therefore for constructing systems that exploit application-specific human expertise.
In the words of the great George Box, ``\textit{all models are wrong, but some a useful}''
\citep{box1979robustness,box2005statistics}.  In a way, this is a self-fulfilling statement: a model for
any real phenomena
is by definition an approximation and so is never exactly correct no matter how powerful.  However,
it is still an essential point that is all too often forgotten, particularly by academics trying to convince
the world that only their approach is correct.  Only in artificial situations can we construct exact models
and so we must remember, particularly in generative machine learning, that the first, and often largest,
error is in our original mathematical abstraction of the problem.  On the other hand, real situations
have access to finite and often highly restricted data, so it is equally preposterous to suggest that a
method is superior simply due to its asymptotic behavior in the limit of large data or that if our approach
does not work then the solution is simply to get more data.\footnote{It should, of course, be noted
	that the availability of data is typically the biggest bottleneck in machine learning.  At times, it feels 
	like the machine learning community would we well served to remember that the differences in performance between
	machine learning approaches is often, if not usually, dominated by variations in the inherently difficultly of the
	problem, which is itself not usually known up front, rather than differences between approaches.}  As such, the ease of which domain-specific
expertise can be included in generative approaches is often essential to achieving effective performance
on real world tasks.

To highlight the difference between discriminative and generative machine learning, we consider the
example of the differences between logistic regression (a discriminative classifier) and na\"{i}ve Bayes 
(a generative classifier).  We will consider the binary classification case for simplicity.  Logistic regression is a linear
classification where the class label $y \in \{0,1\}$ is predicted from the input features $x \in \real^D$ using
\begin{align}
\label{eq:bayes:logistic}
p(y|x,a,b) = \frac{1}{1+\exp(-y(a+b^Tx))}
\end{align}
where $a \in \real^D$ and $b \in \real^D$ are the parameters of the model.  The model is trained by finding the values
for $a$ and $b$ that minimize a loss function on the training data.  For example, a common approach
is to find the \emph{most likely} parameters $a^*$ and $b^*$ by minimizing the negative log-likelihood
(which is equivalent to the cross-entropy loss function)
\begin{align}
\{a^*,b^*\} &= \argmin_{a\in \real^D,b\in \real^D} -\log \left(\prod_{n=1}^{N} p(y_n|x_n,a,b)\right).
\end{align}
Once found, $a^*$ and $b^*$ can be used with~\eqref{eq:bayes:logistic} to make predictions at
any possible $x$.  Logistic regression is a discriminative approach as we have directly calculated
a characterization for the predictive distribution, bypassing any direct reasoning about the joint
probability of the inputs and the outputs.

The na\"{i}ve Bayes classifier, on the other hand, starts by making the assumption that each feature
is \emph{independent} given the class label.  As such it reasons about both the distribution over both
inputs and outputs, unlike logistic regression which only reasons about the conditional probability
of the output given the input.   If we let $\{x^1,\dots,x^D\} =: x$ represent the dimensions of $x$, then
the na\"{i}ve Bayes assumption is that
\begin{align}
p(x|y) = \prod_{i=1}^D p(x^i |y)
\end{align}
which in turn leads to the following joint distribution
\begin{align}
p(x,y) = p(y) \prod_{i=1}^D p(x^i |y).
\end{align}
Here we are free to choose the form for both $p(x^i |y)$ and $p(y)$ %(defining these also indirectly defines $p(x)$)
and we will use the data to learn their parameters.
This freedom is both a blessing and curse: it allows us to impart our own knowledge about the problem to
the model, but we may be forced to make assumptions without proper justification in the interest
of tractability, for convenience, in error, or simply because it is challenging to specify a sufficiently
general purpose model that can cover all possible cases.
Further, even once the forms of $p(x^i |y)$ and $p(y)$ have been defined, there are still decisions to be
made: do we take a Bayesian or frequentist view for making predictions? What is the best way
to calculate the information required to make predictions?  We will go into these questions in
more depth in Section~\ref{sec:bayes:religions}.

As we have shown, generative approaches are inherently probabilistic.  This is highly convenient
when it comes to calculating uncertainty estimates or gaining insight from our trained model.
They are generally more intuitive than discriminative methods as models are they denote the probability
of events occurring, while the generative model is, in essence, an explanation for how the data is
generated.  As such, the parameters tend to have physical interpretation in the generative process and
therefore provide not only prediction, but also insight.  Generative approaches will not always be preferable,
particularly when there is an abundance of data available, but they provide a very powerful framework
that is essential in many scenarios.  Perhaps their greatest strength is in allowing the use of so-called
Bayesian approaches which we now introduce.
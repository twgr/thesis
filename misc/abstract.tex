% !TEX root = ../main.tex

\vspace{20pt}
Imagine a world where computational simulations can be inverted as easily as running them forwards, where data
can be used to refine models automatically, and where the only expertise one needs to carry
out powerful statistical analysis is a basic proficiency in scientific coding.  Creating such a
world is the ambitious long-term aim of \emph{probabilistic programming}.

The bottleneck for improving the probabilistic models, or simulators, used throughout the quantitative sciences,
is often not an ability to devise better models conceptually,  but a lack of expertise,
time, or resources to realize such innovations.
\emph{Probabilistic programming systems} (PPSs) help alleviate this bottleneck 
by providing an expressive and accessible modeling framework,
%, often more in line we conventional scientific simulation than mainstream statistical approaches,
 then
automating the required computation to draw inferences from the model, such as finding
the model parameters likely to give rise to a certain output.
By decoupling model specification and inference, PPSs 
streamline the process of developing, and drawing inferences from, new models
and open up powerful statistical methods to non-experts.
%open up powerful statistical methods 
%to non-experts, while
%streamlining the process of developing new models or inference algorithms
%those within the machine learning and statistics communities.  
Many systems further provide
the flexibility to write new and exciting models which would be hard, or even impossible, to convey using 
conventional statistical frameworks.

The central goal of this thesis is to improve and extend PPSs.
%, providing some of the many steps
%that will be necessary to achieve the lofty long-term aims of the field.  
In particular, we will
make advancements to the underlying inference engines and increase the
range of problems which can be tackled.  For example, we will extend PPSs to a mixed inference-optimization
framework, thereby providing automation of tasks
such as model learning and engineering design, and make inroads into constructing systems
for automating adaptive sequential design problems, providing potential applications across the sciences.
Furthermore, the contributions of the work reach far beyond probabilistic programming, as 
achieving our goal will require us to make
advancements in a number of related fields such as particle Markov chain Monte Carlo methods,
Bayesian optimization, and Monte Carlo fundamentals. %, and Bayesian experimental design.

%
%Specifically, we will introduce \emph{interacting particle
%	Markov chain Monte Carlo},
%a new inference algorithm suitable for large-scale computation, and 
%detail is implementation as a general purpose inference engine for the PPS \emph{Anglican}. 
%%explain how it can be
%%used as general purpose inference engine for PPSs by detailing its implementation
%%in the PPS \emph{Anglican}.  
%We will extend PPSs beyond their typical inference setting
%to a more general mixed inference-optimization framework by introducing~\emph{Bayesian
%	optimization for probabilistic programs}, thereby providing automation of tasks
%such as model learning and engineering design.
%%in the same manner as inference is automated in existing systems.  
%We will develop theoretical 
%results on \emph{nesting Monte Carlo
%	estimators} and explain the important implications these have for nesting models in PPSs.
%Finally, we will 
%%examine a particular
%%class of nested estimation problems, those of Bayesian experimental design, and 
%introduce a high-level framework for automating adaptive sequential design problems, 
%%a particle example 
%providing application of our technique to psychological trials.
%%Finally, we will examine a particular
%%class of nested estimation problems, those of Bayesian experimental design, and introduce
%%a high-level framework for automating arbitrary adaptive sequential design problems,
%%providing application our technique to psychological trials.
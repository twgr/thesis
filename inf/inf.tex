% !TEX root = ../main.tex

\chapter{An Introduction to Bayesian Inference}
\label{chp:inf}

% !TEX root = ../main.tex

\section{The Challenge of Bayesian Inference}
\label{sec:inf:challenge}

In the previous chapter we introduced the concept of Bayesian modelling and showed how we
can combine prior information $p(\theta)$ and a likelihood model $p(\mathcal{D}|\theta)$ using Bayes' rule 
(i.e.~\eqref{eq:bayes}), to produce a posterior $p(\theta|\mathcal{D})$ on variables $\theta$ that
characterizes both our prior information and information from the data $\mathcal{D}$.  We now consider the
problem of how to calculate (or more typically approximate) this posterior, a process 
known as Bayesian \emph{inference}.
At first this may seem like a straight forward problem: by Bayes' rule we have that
$p(\theta|\mathcal{D})\propto p(\mathcal{D}|\theta)p(\theta)$ and so we already know the relative probability of any one
value of $\theta$ compared to another.  In practice, this could hardly be further from the
truth.  Bayesian inference for the general class of graphical models is in fact an 
NP-hard problem \citep{cooper1990computational,dagum1993approximating}.  We can break
it down into key challenges: calculating the normalization constant
$p(\mathcal{D}) = \int p(\mathcal{D}|\theta)p(\theta)d\theta$ and providing a useful characterization of the posterior, for
example an object we can draw samples from.  Many inference schemes, for example Markov
chain Monte Carlo (MCMC) methods \citep{hastings1970monte}, will not
try to tackle these challenges directly and instead look to generate samples directly from 
the posterior.  However, this breakdown will still prove useful in illustrating the intuitions
about the difficulties presented by Bayesian inference.

\subsection{The Normalization Constant}
\label{sec:inf:challenge:norm}

Calculating the normalization constant in Bayesian inference is essentially a problem of
integration.  Our target, $p(\mathcal{D})$, is the expectation of the likelihood under the prior,
hence the name \emph{marginal likelihood}.  When $p(\mathcal{D})$ is known, the posterior can be evaluated
exactly at any possible input point using~\eqref{eq:bayes} directly.  When it is unknown, we lack
a scaling in the evaluation of any point and so we have no concept of how relatively 
significant that point is relative to the distribution as a whole.  For example, for a discrete
problem then if we know the normalizing constant, we can evaluate the exact probability of any
particular $\theta$ by evaluating that point alone.  If we do not know the normalizing constant, we do
not know if there are other substantially more probable events that we have thus-far missed, which
would in turn imply that the queried point has a negligible chance of occurring.
 
To give a more explicit example, consider a model where $\theta \in \{1,2,3\}$ with a corresponding uniform prior $P(\theta) = 1/3$
for each $\theta$.  Now presume that for some reason that we are only able to evaluate the likelihood at 
$\theta=1$ and $\theta=2$, giving $p(\mathcal{D}|\theta=1)=1$ and $p(\mathcal{D}|\theta=2)=10$ respectively.  Depending on the marginal
likelihood $p(\mathcal{D})$, the posterior probability of $P(\theta=2 | \mathcal{D})$ will vary wildly.  For example,
$p(\mathcal{D})=4$ gives $P(\theta=2 | \mathcal{D}) = 5/6$, while $p(\mathcal{D})=1000$ gives $P(\theta=2 | \mathcal{D}) = 1/100$.  Though this
example may seem far-fetched, this is the scenario almost always seen in practice for realistic
models, at least those with non-trivial solutions.   Typically it is not
possible to enumerate all the possible values of $\theta$ in reasonable time and we are left 
wondering - how much probability mass is left that we have not seen?  The problem is even worse 
in the setting where $\theta$ is continuous, for
which it is naturally impossible to evaluate all possible values for $\theta$.  
Knowing the posterior only up to a normalization constant is deceptively unhelpful - we never
know how much of the probability mass we have missed and therefore whether the probability (or
probability density) where we have looked so far is tiny compared to some other dominant region
we are yet to explore.  At its heart, the problem of Bayesian inference is a problem
of where to concentrate our finite computational resources so that we can effectively characterize
the posterior.  If $p(\mathcal{D})$ is known, then we immediately know whether we are looking in the right
place or whether there are places left to look that we are yet to find.  This brings us onto
our second challenge -- knowing the posterior in closed form is often not enough.

\subsection{Characterizing the Posterior}
\label{sec:inf:challenge:post}

Once with have the normalization constant, it might seem that we are done; after all
we now have the exact form the posterior using in Bayes' rule.  Unfortunately, it tends to
be the case, at least when $\theta$ is continuous, that is this insufficient to carry out most
tasks that we might want to use our posterior for.  There are a number of different, often
overlapping, reasons for wanting to calculate a posterior including
\begin{itemize}
	\item To calculate the posterior probability or probability density for one or more particular
	instances of the variables.
	\item To calculate the expected value of some function, $\mu_f = \E_{p(\theta|\mathcal{D})}\left[f(\theta)\right]$.
	For example, we might want to calculate the expected values of the variables themselves
	$\mu_\theta = E_{p(\theta|\mathcal{D})} \left[\theta\right]$ as a point estimate.
	\item To make predictions.  For example, in a supervised learning task then our data typically
	comprises of a series of input output pairs $\mathcal{D} = \{x_n,y_n\}_{n=1:N}$ and we
	wish to predict the output at some new input $\tilde{x}$.  In the fully Bayesian framework, one
	does this using the \emph{posterior predictive distribution}
	\[
	p(\tilde{y} | \tilde{x}) = \int p(\tilde{y}|\tilde{x},\theta) p(\theta | \mathcal{D}) d\theta.
	\]
	Note that this is a particular case of calculating an expectation under the posterior.
	\item To find the most probable variable values $\theta^* = \argmax_{\theta} p(\theta|\mathcal{D})$.  
	This is known
	as maximum a posteriori estimation and will be discussed in Chapter~\ref{chp:opt}.
	\item To produce samples from, or form some other useful characterization of, the posterior that can
	then passed on to another part of a computational pipeline or directly observed by a user.
	\item To estimate a marginal probability distribution over some variables of particular
	interest.  For example, if $\theta=\{u,v\}$ then we might be interest in the marginal
	$p(u|\mathcal{D})$.
\end{itemize}
If $\theta$ is continuous or some elements of $\theta$ are continuous then only the first of
these can be carried out directly using the form of the posterior provided by Bayes' rule
with known normalization constant.  We, therefore, see that knowing the normalization alone
will not be enough to fully solve the Bayesian inference problem in a useful manner.  In
particular, it will generally not sufficient in order to be able to \emph{sample} from the
posterior.  As we will see later, the ability to sample will be at the core of most practical uses
for the posterior as it allows use of Monte Carlo
 methods~\citep{metropolis1949monte,robert2004monte,rubinstein2016simulation}, which
 can in turn be used to carry out many of the outlined tasks.

To further demonstrate why knowing the normalization constant is insufficient
for most Bayesian inference tasks, we consider the following a simple example
\begin{subequations}
\label{eq:inf:example}
\begin{align}
p(\theta) &= \textsc{Gamma}\left(\theta; 3, 1\right) = \frac{\theta^2 \exp(-\theta)}{2},
 \quad \theta\in\left(0,\infty\right), \\
p(y=5|\theta) &= \textsc{Student-t}\left(5-\theta;2\right) = 
\frac{\Gamma (1.5)}{\sqrt{2\pi}}\left(1+\frac{\left(y-\theta\right)^2}{2}\right)^{-1}, \\
p(\theta |y=5) &\approx \frac{\theta^2
	 \exp(-\theta)}{0.36629891\left(2+(5-\theta)^2\right)}.
\end{align}
\end{subequations}
Here we have that the prior on $\theta$ is distributed according to a gamma
distribution with shape parameter $3$ and scale parameter $1$.  The likelihood
function is a student-t distribution on the difference between $\theta$ and the
output $y=5$.  Using a numerical integration over $\theta$ the normalization
constant can be calculated to a high accuracy, giving the provided closed-form
equation for the posterior.  This posterior, along with the prior and likelihood are shown Figure~\ref{fig:inf:inf-example}.

\begin{figure}[t]
	\centering
	\includegraphics[width=0.65\textwidth]{inf_example}
	\caption{Example inference for problem given
		in~\eqref{eq:inf:example}. Also shown is the cumulative distribution for
		the posterior as per~\eqref{eq:inf:cum-example}.  This is scaled by a factor
		of $1/2$ for visualization.  \label{fig:inf:inf-example}}
\end{figure}

Here knowing the marginal likelihood means that we have a 
closed-form equation for the posterior.  Image though that we 
wish to sample from it.  As it does not correspond to a standard distribution,
we cannot simply call a standard sampling procedure.  There is, in fact, no way
to directly sampling from this distribution without doing further calculations.  If
we also know the inverse of the cumulative density function of the posterior
\begin{align}
\label{eq:inf:cum-example}
P(\Theta<\theta | y=5) = \int_{\Theta=0}^{\Theta=\theta}  p(\theta=\Theta | y=5) d\Theta,
\end{align}
then we can sample from the posterior by sampling $\hat{u} \sim \textsc{Uniform}(0,1)$ and
then taking as our sample $\hat{\theta} = P^{-1}(\hat{u})$ such that 
$\hat{u} = P(\Theta<\hat{\theta} | y=5)$.  However, the cumulative distribution function
and its inverse cannot be calculated analytically.  Though in this simple one dimensional
problem they can be easily estimated numerically, this will prove prohibitively difficult
for most problems where $\theta$ has more than a few dimensions.  Similarly, if we
wish to estimate an expectation with respect to this posterior we could do this relatively
easily numerically for this simple problem, for example using Simpson's rule, but in higher
dimensions this will be impractical.

There are a number of indirect methods we could use instead to sample from the posterior
such as rejection sampling, importance sampling, and MCMC.  However, as we will show
in the Section~\ref{sec:inf:foundation}, these all only require that we can evaluate an unnormalized version of
target distribution, such that they side-step the need to calculate the marginal 
likelihood.\footnote{It should be noted though that for adaptive sampling schemes,
	then the normalization constant
	can still be useful information for such methods for the reasons outlined in 
	Section~\ref{sec:inf:challenge:norm}.}
% !TEX root = ../main.tex

\section{Monte Carlo}
\label{sec:inf:mc}

Monte Carlo~\citep{metropolis1949monte} is the characterization of a probability distribution 
through random sampling. It is the foundation for a huge array of methods for numerical 
integration, optimization, and scientific simulation; forming the underlying principle 
for all stochastic computation.
\mc provides us with a means of dealing with complex models and problems in a
statistically principled manner.  Without it, one would have to resort to deterministic
approximation methods whenever the target problem is too complex to permit an analytic
solution.  As we will show, it is a highly composable framework that will allow the output
of one system to be input directly to another.  For example, the \mc samples from a joint
distribution will also have the correct marginal distribution over any of its individual components,
while sampling from the marginal distribution then sampling from the conditional distribution
given these samples, will give samples distributed according to the joint.  As \mc
will be key to most methods for Bayesian inference that we will consider, we take the time
in this section to introduce Monte Carlo at a foundational level.

The most common usage of \mc in this work will be the \mc estimation of expectations, 
sometimes known as \mc integration.  
The critical importance of \mc estimation stems from the fact that most of the example
target tasks laid out in~\ref{sec:inf:challenge:post} can be formulated as expectations.
Even when our intention is simply to generate samples from a target distribution, we can
usually think of this as being an implicit expectation of an, as yet unknown, target function.
Here our implicit aim is to minimize the bias and variance of whatever process the samples are eventually
used for, even if that process is simply visual inspection.  

Consider the problem of calculating the expectation of some function
$f(\theta)$ under the distribution $\theta\sim \pi(\theta)$ ($= p(\theta | \mathcal{D})$ for the Bayesian
inference case), which we will denote 
as
\begin{align}
	\label{eq:inf:expt}
I:=\E_{\pi(\theta)} \left[f(\theta)\right]=\int f(\theta) \pi(\theta) d\theta.
\end{align}
This can be approximated using the \mc estimator $I_N$ where
\begin{align}
	\label{eq:inf:mc-est}
	I \approx I_N := \frac{1}{N} \sum_{n=1}^{N}f(\hat{\theta}_n)
	\quad \text{and} \quad \hat{\theta}_n \sim \pi(\theta).
\end{align}
The first result we note is that~\eqref{eq:inf:mc-est} is an \emph{unbiased} estimator for $I$, i.e. we have
\begin{align}
\label{eq:inf:unbiased}
\E \left[I_N\right] = \E \left[\frac{1}{N} \sum_{n=1}^{N}f(\hat{\theta}_n)\right]
= \frac{1}{N} \sum_{n=1}^{N} \E \left[f(\hat{\theta}_n)\right]
= \frac{1}{N} \sum_{n=1}^{N} \E \left[f(\hat{\theta}_1)\right]
= I
\end{align}
where we have first moved the sum outside of expectation using
linearity,\footnote{Note that this presumes that $N$ is independent
	of the samples.  This is usually the case, but care is necessary in some situations, namely when
	the number of samples taken is adaptively chosen based on the sample values, for example in
	adaptive stratified sampling~\citep{etore2010adaptive}.}
then the fact that each $\hat{\theta}_n$ is identically distributed to note that
each $\E \left[f(\hat{\theta}_n)\right]= \E \left[f(\hat{\theta}_1)\right]$, and finally
that $\E \left[f(\hat{\theta}_1)\right] = I$ by the definition of $I$ and the distribution
on $\hth_1$.  This is an important result as it means that \mc does not introduce
any systematic error, i.e. bias, into the approximation: in expectation, it does not
pathologically overestimate or underestimate the target.  This is not to say though that it is
equally likely to overestimate or underestimate as it may, for example, typically underestimate
by a small amount and then rarely overestimate by a large amount.  Instead, it means that if we
were to repeat the estimation an infinite number of times and average the results, we would
get the true value of $I$.  This now hints at another important question -- do we also
recover the true value of $I$ when we conduct one infinitely large estimation, namely if we
take $N\rightarrow\infty$?  This is known as \emph{consistency} of a statistical estimator,
which we will now consider next.  

Before moving
on, we make the important note that many common \mc inference methods, for example MCMC, are in fact biased.  
This is because it is often not possible to independently sample $\hth_n \sim \pi(\theta)$
exactly as we have assumed in~\eqref{eq:inf:mc-est}, with the bias resulting from
the approximation.  The convergence of such methods relies on the bias 
diminishing to $0$ as $N\rightarrow\infty$, such that they remain unbiased in the limit.

\subsection{The Law of Large Numbers}
\label{sec:inf:mc:law}

A key mathematical idea underpinning the convergence of many Monte Carlo methods is the 
law of large numbers (LLN).  Informally, the LLN states that the empirical average of 
independent and identically distributed (i.i.d.)  random variables converges to 
the true expected value of the underlying process as the number of samples in the
average increases.  We can, therefore, use it to prove the consistency of Monte Carlo estimators
where the samples are drawn independently from the same distribution, as is the case
in for example rejection sampling and importance sampling.  The high-level idea for the LLN can be shown by
considering the  \emph{mean squared error} of a \mc estimator as 
follows
%\footnote{Note here that we use the notation of expectation typically used within the statistics literature,
%whereby it corresponds to the expectation over all randomness contained in the system.}
\begin{align}
\E &\left[(I_N-I)^2\right] = \E\left[\left(\frac{1}{N}\sum_{n=1}^{N}f(\hth_n) - I\right)^2\right]
= \frac{1}{N^2}\E\left[\left(\sum_{n=1}^{N} \left(f(\hth_n) - I\right)\right)^2\right] \nonumber \\
&= \frac{1}{N^2}\sum_{n=1}^{N} \E\left[ \left(f(\hth_n)-I\right)^2\right] + 
\frac{1}{N^2}\sum_{n=1}^{N}\sum_{m=1,m\neq n}^{N} \E\left[ (f(\hth_n)-I)(f(\hth_m)-I)\right] \nonumber \\
&= \frac{1}{N^2}\sum_{n=1}^{N} \E\left[ \left(f(\hth_1)-I\right)^2\right] + 
\frac{1}{N^2}\sum_{n=1}^{N}\sum_{m=1,m\neq n}^{N} \cancelto{0}{\left(\E\left[(f(\hth_1)-I)\right]\right)^2} \nonumber \\
&= \frac{\sigma_{\theta}^2}{N}  \quad \text{where} \quad \sigma_{\theta}^2 := \E\left[ \left(f(\hth_1)-I\right)^2\right]
= \var \left[f(\theta)\right].\label{eq:inf:LLN-informal}
\end{align}
Here the second line follows from the first simply by expanding the square and using linearity
to move the sum outside of the expectation as in the unbiasedness derivation.
The first term in the third line follows from the equivalent term in the second line by again noting that
each $\hth_n$ has the same distribution.  The second term in the third line
follows from the assumption that the samples are drawn independently such that
\[
\E\left[ (f(\hth_n)-I)(f(\hth_m)-I)\right] = \E\left[ (f(\hth_n)-I)\right] \E\left[(f(\hth_m)-I)\right]=0.
\]
by unbiasedness of the estimator. The last line simply notes that $\E\left[ \left(f(\hth_1)-I\right)^2\right]$ is a constant,
namely the variance of $f(\theta)$.
  Our final result has a simple and intuitive form -- the mean squared error for
our estimator using $N$ samples is $1/N$ times the mean squared error of an estimator that only uses
a single sample, which is itself equal to the variance of $f(\theta)$.  As $N\rightarrow\infty$, we thus
have that our expected error goes to $0$.

A key upshot of this result is that the difference between our empirical estimate and the true value (i.e. $I_N-I$)
 should be of order $O(1/\sqrt{N})$.  In some way this is rather slow: deterministic numerical
integration schemes often have much faster theoretical convergence rates.  For example,
Simpson's rule has a convergence rate of $O(1/N^4)$ for one-dimensional functions~\citep[Chapter 7]{owen2013mc}.  
As such, \mc is often an inferior way of estimating integrals for smooth functions
in low dimensions.  However, these deterministic numerical integration schemes
require smoothness assumptions on $f$ and, more critically, their convergence rates diminish rapidly (typically
exponentially quickly) with the dimensionality.   
By comparison, the dimensionality only effects the \mc convergence rate through changes in the 
constant factor $\sigma_{\theta}$
and though this will typically increase with the dimensionality, this 
scaling will usually be substantially more graceful than deterministic numerical methods.
%generally do so substantially less aggressively
%then typical exponential decrease in converge rate of deterministic numerical integration schemes
%with dimension.  
%\mc therefore tends to dominate for moderate to high dimensional problems.

\subsection{Convergence of Random Variables}
\label{sec:inf:mc:conv}

To introduce the concept the LLN more precisely, we now consider some more formal notations
of convergence of random variables.  There will be times (e.g. Chapter~\ref{chp:nest})
when mathematical rigor will require us to distinguish between alternative notations of convergence.
However, those less interested in theoretical details may wish to
skip Sections~\ref{sec:inf:mc:conv:prob},~\ref{sec:inf:mc:conv:as}, and~\ref{sec:inf:mc:conv:dist}
on first reading, as the notion of $L^p$ convergence will be sufficient, for most practical purposes,
to guarantee that an estimator will return the correct answer if provided 
with sufficient samples. 

\subsubsection{$L^p$-Convergence}
\label{sec:inf:mc:conv:Lr}

We start by introducing the notion of $L^p$-convergence, also known as convergence in expectation,
as this is the type of convergence we have just alluded to in our informal proof of the LLN.
At a high level, $L^p$-convergence means that the expected value of the related error metric
tends to zero as $N\rightarrow \infty$.  More precisely, we first define the $L^p$-norm for
a random variable $X$ as
\begin{align}
\label{eq:inf:Lp-norm}
\norm{X}_p = \left(\E \left[\left|X\right|^{p}\right]\right)^{\frac{1}{p}}
\end{align}
where $\left|\cdot\right|$ denotes the absolute value.  For example, we can write the
mean squared error used in~\eqref{eq:inf:LLN-informal} as the squared $L^2$-norm:
$\E \left[(I_N-I)^2\right] = \norm{I_N-I}_2^2$.  We further define the notion of $L^p$-space
as being the space of random variables for which $\norm{X}_p < \infty$.  We can now
formally define $L^p$-convergence as follow.
\begin{definition}[$L^p$-convergence]
A sequence of random variables $X_N$ converges in its $L^p$-norm to 
$X$ (where $p\ge1$) if $X\in L^p$, each $X_N \in L^p$, and
\begin{align}
\lim\limits_{N\rightarrow\infty} \norm{X_N-X}_p=0. \label{eq:inf:Lp-conv-formal}
\end{align}
\end{definition}
\noindent A key point to note is that $\Vert X_N-X\rVert_p\ge0 \; \forall X_N, X$ by definition of the $L^p$-norm and so
rather than this simply being a statement of asymptotic unbiasedness, $L^p$-convergence says that the expected
\emph{magnitude} of the error tends to zero as $N\rightarrow\infty$.
Different values of $p$ correspond to different metrics for the error, with larger values of
$p$ constituting stronger converge guarantees, such that $L^{p_2}$-convergence implies
$L^{p_1}$-convergence whenever $p_2>p_1$.  Similarly, if a random variable satisfies
$X \in L^{p_2}$, then it follows that $X \in L^{p_1}$.

%
%\subsubsection{Convergence in Distribution}
%\label{sec:inf:mc:conv:dist}
%
%Convergence in distribution is a weaker form of convergence than is implied by all the
%other forms of convergence that we will discuss.

\subsubsection{Convergence in Probability}
\label{sec:inf:mc:conv:prob}

At a high level, convergence in probability between two random variables (or between a random variable and
a constant) means that they become arbitrarily close to one another.  More
formally we have the following definition.
\begin{definition}[Convergence in probability]
A sequence of random variables $X_N$ converges in probability to $X$ if, for every $\varepsilon>0$,
\begin{align}
\lim\limits_{N\rightarrow\infty} P(\left|X_N-X\right|\ge\varepsilon)=0.
\end{align}
\end{definition}
\noindent As $\varepsilon$ can be made arbitrarily small, this ensures that $X_N$ becomes arbitrarily
close to $X$ in the limit of large $N$.  Estimators are \emph{consistent} if they converge
in probability.

In~\eqref{eq:inf:LLN-informal} we demonstrated the $L^2$ convergence of the Monte Carlo
estimator as we have that 
\[
\lim\limits_{N\rightarrow\infty} \norm{I_N-I}_2=\lim\limits_{N\rightarrow\infty} \frac{\sigma_\theta}{\sqrt{N}} = 0.
\]
Convergence in probability is, in general, a weaker form of convergence than $L^p$ convergence
as $L^p$ convergence implies convergence in probability~\citep{williams1991probability}.
We therefore also have the Monte Carlo estimator convergences in
probability to its expectation.  This is known as the
\emph{weak law of large numbers}, which we can also prove more explicitly as follows
\begin{theorem}[Weak law of large numbers]
	\label{the:inf:weak-law}
If $I$ and $I_N$ are defined as per~\eqref{eq:inf:expt} and~\eqref{eq:inf:mc-est} respectively,
$I\in L^1$, each $I_N \in L^1$,
 and each $\hth_n$ in~\eqref{eq:inf:mc-est} is drawn independently, then $I_N$ converges to $I$
in probability:
	$\lim\limits_{N\rightarrow\infty} P(\left|I_N-I\right|\ge \varepsilon)=0 \quad \forall \varepsilon>0$.
\end{theorem}
\begin{proof}
In the interest of exposition, we prove the result in the case where the stronger 
assumptions that $I\in L^2$ and each $I_N \in L^2$ hold.  In practice this is not needed as
the theorem can be proved by other means, see for example~\cite[Theorem 2.2.7]{durrett2010probability}.

By~\eqref{eq:inf:unbiased} we have that $\E [I_N]=I$ and by~\eqref{eq:inf:LLN-informal} we have that 
$\norm{I_N-I}_2^2 = \frac{\sigma_{\theta}^2}{N}$
where $\sigma_{\theta}^2$ is an unknown, but finite, constant by the assumption that $I\in L^2$.
Chebyshev's inequality states that if $\E [I_N] = I$, then for any $k>0$
\[
P(\left|I_N-I\right|\ge\varepsilon)\le\frac{\var(I_N)}{\varepsilon^2}.
\]
By further noting that as $I_N$ is unbiased, we have that $\var(I_N) = \norm{I_N-I}_2^2$
and therefore
\[
\lim\limits_{N\rightarrow\infty} P(\left|I_N-I\right|\ge\varepsilon)
\le\lim\limits_{N\rightarrow\infty} \frac{\sigma_{\theta}^2}{\varepsilon^2 N}
=0 \quad \forall \varepsilon>0.
\]
\end{proof}
%\noindent As an interesting aside, we can prove that convergence in probability follows from
%$L^p$ convergence more generally in a similar fashion.  This can be done
%by using the more general form of Chebyshev's inequality given in, for example,
%Theorem 1.6.4 and Lemma 2.2.2 of~\cite{durrett2010probability} to show that
%\begin{align}
%\label{eq:inf:prob-from-lp}
%\begin{split}
%\lim\limits_{N\rightarrow\infty} P(\left|I_N-I\right|\ge\varepsilon)
%&\le \lim\limits_{N\rightarrow\infty} \frac{\E \left[\left|I_N-I\right|^{p}\right]}{\varepsilon^p} = 
%\left(\frac{1}{\varepsilon}\lim\limits_{N\rightarrow\infty} \norm{I_N-I}_p\right)^p.
%\end{split}
%\end{align}
%If $I_N \Lp I$ then $\lim\limits_{N\rightarrow\infty} \norm{I_N-I}_p = 0$ by definition
%and we have convergence in probability as required.

\subsubsection{Almost Sure Convergence}
\label{sec:inf:mc:conv:as}

Almost sure convergence, also known as strong convergence, is a similar, but stronger, form of convergence than convergence in
probability.  At a high level, the difference between
convergence in probability and almost sure converge is a difference in the tail behavior:
convergence in probability suggests the rate at which an event happens tends to zero; almost
sure convergence means that there is some point in time after which the event never happens
again.  More formally we have the following definition
\begin{definition}[Almost sure convergence]
A sequence of random variables $X_N$ converges almost surely to $X$ if
\begin{align}
	P\left(\lim\limits_{N\rightarrow\infty} X_N=X\right)=1.
\end{align}
\end{definition}
\noindent Almost sure convergence implies convergence in probability, but not vice-versa -- the rate
at which events occur might tend to zero without there ever being a point at which the
event never occurs again.  It does not imply, nor is it implied by, $L^p$ convergence.
The \emph{strong law of large numbers} is the dual for the weak law of large numbers as
follows.
\begin{theorem}[Strong law of large numbers]
	Assuming the setup of Theorem~\ref{the:inf:weak-law} then $I_N$ converges to $I$ almost surely
	\begin{align}
	P\left(\lim\limits_{N\rightarrow\infty} I_N=I\right)=1.
	\end{align}
\end{theorem}
The proof is somewhat more complicated than the weak law and so is not provided here, but can
be found in, for example,~\cite[Theorem 2.4.1]{durrett2010probability}.

\subsubsection{Convergence in Distribution}
\label{sec:inf:mc:conv:dist}

At a  high level, convergence in distribution, also known as weak convergence, states that a sequence of variables become
increasingly closely distributed to a target distribution.  Whereas our previous notions of
convergence ensure that our sequence of random variables converge to a particular value, 
convergence in distribution only implies that our sequence of random variables tends towards
having a particular distribution.  For example, a variable might converge in distribution to
having a unit normal distribution, whereas a different variable might converge in probability
to zero. More formally, we can define convergence in distribution as follows.
\begin{definition}[Convergence in distribution]
	A sequence of random variables $X_N$ converges in distribution $X$  if
	\begin{align}
		\lim\limits_{N\rightarrow\infty} P(X_N \le x) = P(X \le x)
	\end{align}
	for every $x$ at which $P(X\le x)$ is continuous, where $P(X_N \le x) $ and $P(X \le x)$
	are the cumulative distribution functions for $X_N$ and $X$ respectively.
\end{definition}
\noindent Because it only requires that a variable has a particular distribution, rather than an
 exact value, convergence in distribution is a weaker form of convergence than those previously
 discussed and is implied by any of the other forms of convergence.  
\vspace{5pt}

\subsection{The Central Limit Theorem}
\label{sec:inf:mc:clt}

The central limit theorem (CLT) is a core result in the study of \mc methods.  In its simplest form,
it states that the empirical mean of $N$ i.i.d. random variables tends towards a Gaussian in the limit $N\to\infty$.
While the LLN demonstrated the convergence of the average of i.i.d. random variables towards
the true mean, the CLT provides a means of constructing consistent confidence intervals of our estimate
$I_N$ by exploiting its asymptotic normality.  Furthermore, it has variants that do not
require that the variables are i.i.d., meaning we can use it do demonstrate convergence in scenarios
where samples are correlated, such as occurs when doing MCMC inference as shown in
Section~\ref{sec:inf:foundation:mcmc}.
In the i.i.d.~case, the CLT is as follows.
\begin{theorem}[Central Limit Theorem]
	\label{the:inf:clt}
Assume $X_1,\dots,X_N$ is a sequence of i.i.d. random variables with 
$\E [X_i] = I$ and $\E[X_i^2]=\sigma<\infty$ for all  $i \in \{1,\dots,N\}$. Let 
$I_N := \frac{1}{N} \sum_{n=1}^{N} X_n$ be the
sample average of this sequence.  Then $\sqrt{N}\left(I_N-I\right)/\sigma$ converges
in distribution to the unit normal:
\begin{align}
\lim\limits_{N\rightarrow\infty} P\left(\frac{\sqrt{N}\left(I_N-I\right)}{\sigma} \le z\right)
=\Phi(z)= \int_{-\infty}^{z} \frac{1}{2\pi} \exp\left(-\frac{\zeta^2}{2}\right) d\zeta \quad \forall z\in\real
\end{align}
where $\Phi(z)$ is the cumulative distribution function for the unit normal.
Furthermore, if $\sigma_N^2 = \frac{1}{N-1} \sum_{n=1}^{N} \left(X_N-I_N\right)^2$ is
the empirical estimate of the variance of $X_N$ (including Bessel's correction), then
$\sqrt{N}\left(I_N-I\right)/\sigma_N$ also converges
in distribution to the unit normal.
\end{theorem}
\begin{proof}
	See, for example,~\cite[Chapter 3]{durrett2010probability}.
\end{proof}
\noindent The second result here is especially key as it will allow us to calculate consistent confidence intervals
on our estimates without needing to know $\sigma$ as we can instead use our empirical estimate
\begin{align}
	\label{eq:inf:emprical-var}
	\sigma_N^2 = \frac{1}{N-1} \sum_{n=1}^{N} \left(X_N-I_N\right)^2 \quad \mathrm{where} 
	\quad I_N = \frac{1}{N}\sum_{n=1}^{N}X_N.
\end{align}
The form of~\eqref{eq:inf:emprical-var} might at first look at bit strange - why is the normalizing
term $1/(N-1)$ instead of $1/N$?  Although replacing the $1/(N-1)$ term with $1/N$ would still lead
to a consistent estimator, this estimator turns out to be biased whereas~\eqref{eq:inf:emprical-var}
is in fact unbiased.  The bias correction $N/(N-1)$ is known as Bessel's correction, while the proof
that~\eqref{eq:inf:emprical-var} is unbiased follows from straightforward algebraic manipulations.
Note though that the estimator of the standard deviation, $\sigma_N$, is still biased by Jensen's
inequality.
As a consequence of the unbiasedness, $\sigma_N^2 \asto \sigma^2$ by the strong LLN, 
from which the second result
in Theorem~\ref{the:inf:clt} follows from the first by Slutsky's theorem.

Imagine we want to construct a confidence interval on our estimate $I_N$ such that there is
a probability $0<\alpha<1$ that $I$ is in the range $[I_N-\varepsilon,I_N+\varepsilon]$.  Using
the CLT we can do this as follows
\begin{align}
	\label{eq:inf:conf-int}
	\alpha :&= P(I_N-\varepsilon \le I \le I_N+\varepsilon ) = P(-\varepsilon \le I_N-I \le \varepsilon) \nonumber\\
	&= P\left(-\frac{\sqrt{N}\varepsilon}{\sigma_N}\le \frac{\sqrt{N}\left(I_N-I\right)}{\sigma_N}
				\le\frac{\sqrt{N}\varepsilon}{\sigma_N}\right) \nonumber\\
	&\approx \Phi\left(\frac{\sqrt{N}\varepsilon}{\sigma_N}\right)-
					\Phi\left(-\frac{\sqrt{N}\varepsilon}{\sigma_N}\right) = 2\Phi\left(\frac{\sqrt{N}\varepsilon}{\sigma_N}\right)-1.
\end{align}
Rearranging for $\epsilon$ we have that 
$\varepsilon = \left(\sigma_N/\sqrt{N}\right)\Phi^{-1}\left(\frac{\alpha+1}{2}\right)$
and therefore the confidence interval
\begin{align}
P\left(I_N-\frac{\sigma_N\Phi^{-1}\left(\frac{\alpha+1}{2}\right)}{\sqrt{N}} 
	\le I \le I_N+\frac{\sigma_N\Phi^{-1}\left(\frac{\alpha+1}{2}\right)}{\sqrt{N}} \right)  = \alpha.
\end{align}
Thus for example, if we set $\alpha=0.99$ then $\Phi^{-1}\left(\frac{\alpha+1}{2}\right)\approx 2.576$
and our confidence interval $99\%$ confidence interval for $I$ is
 $I_N \pm \frac{2.576 \sigma_N}{\sqrt{N}}$.  We reiterate that these confidence intervals are exact
 in the limit $N\to\infty$, but approximate for finite $N$.  As one is often far from the asymptotic
 regime, care is often required in interpreting them.  Nonetheless, the ability to
 estimate realistic uncertainties for sufficiently large $N$ is an extremely powerful result.
 
Through most of this section, we have assumed that our random variables are i.i.d..
In fact, neither the assumption of being identically distributed nor that of independence is actually
necessary for the CLT to hold.  One can instead use the concept of \emph{strong mixing}, namely
that variables sufficiently far apart in the sequence are independent, to generalize beyond the i.i.d.
setting~\citep{jones2004markov}.  This is critical for the numerous \mc inference schemes, 
such as MCMC methods, that
do not produce independent samples but instead rely on the samples converging in distribution
to a target distribution.  In most MCMC settings, one can also prove a CLT using reversibility of
the Markov chain~\citep{kipnis1986central}.
It is beyond the scope of this thesis to go into these more general forms of the CLT
in depth, so we simply note their critical importance and refer the reader to~\cite{durrett2010probability}
for a comprehensive introduction.
% !TEX root = ../main.tex

\section{Foundational Monte Carlo Methods}
\label{sec:inf:foundation}

In this section we introduce the key inference methods that form
the basis upon which most Monte Carlo inference schemes are based.  The
key idea at the heart of all Monte Carlo inference methods is to use some form
of proposal distribution that we can easily sample from and then make
appropriate adjustments to achieve (typically approximate) samples from
the posterior.  All methods will require only an unnormalized distribution
\begin{align}
\label{eq:inf:unnorm-target}
\gamma(\theta) = \pi(\theta)Z
\end{align}
as a target where $Z = \int \gamma(\theta) d\theta$.  As such they will apply to
any situation where we desire to sample from an unnormalized 
(or in some cases normalized) distribution, for which
the Bayesian inference setting is a particular case where
$\gamma(\theta) = p(\mathcal{D}|\theta)p(\theta)$ and $Z = p(\mathcal{D})$.
They will, in general, vary only on how samples are proposal
and the subsequent adjustments that are made.  However, this will lead to a
plethora of different approaches, varying substantially in their motivation,
theoretical justification, algorithmic details, and the scenarios for which they
are effective.

\subsection{Rejection Sampling}
\label{sec:inf:foundation:rejection}

Rejection sampling~\citep{robert2004monte} is one of the simplest Monte Carlo 
inference methods and one of the only ones to produce exact samples from 
the target.  Before going into the method itself, we first consider an example to
demonstrate the underlying intuition.  Imagine we want to generate samples 
distributed uniformly over some arbitrary two dimensional shape.  One simple
way of doing this would be to sample uniformly from a box enclosing the
shape and then only taking the samples which fall with the shape.
An example of such sampling by rejection is shown in Figure~\ref{fig:inf:rej-butt}.
As all the samples within the space or distributed uniformly, they are also
uniformly distributed on any subset of the space.  Therefore if we sample
from a space incorporating the area we care about and then only take the samples
that fall within the desired shape, we will generated samples uniformly over
that shape. We can also use this method to estimate the area of the shape by using
the fact that the probability of any one sample falls within the shape is equal to
the ratio of the areas of the shape and the bounding box, namely
\begin{align}
A_{\text{shape}} &= A_{\text{box}}	P(\theta \in \text{shape}) \\
&\approx \frac{A_{\mathrm{box}}}{N} \sum_{n=1}^{N} \ind (\hat{\theta}_n \in \text{shape})
\quad \text{where} \quad \hat{\theta}_n \sim \textsc{Uniform}(\text{box})
\end{align}
where we have used a Monte Carlo estimate for $P(\theta \in \text{shape})$.
Note that the value of $P(\theta \in \text{shape})$ will
dictate the efficiency of our estimation as it represents the \emph{acceptance rate}
of our samples.  In other words, we need to generate on average $1/P(\theta \in \text{shape})$
samples from our proposal for each sample created in the target area.  As we
will show later in Section INSERT, $P(\theta \in \text{shape})$ typically becomes very
small as $\theta$ becomes high dimensional, so this approach will typically only
be effective in low dimensions.

\begin{figure}[t]
	\centering
	\includegraphics[width=0.6\textwidth]{butterfly}
	\caption{Example of sampling uniformly from an arbitrary shape by 
		rejection.  Here samples are proposed uniformly from the $[-1,1]$
		square.  Any sample falling within the black outline is accepted 
		(blue), otherwise it is rejected (red).  \label{fig:inf:rej-butt}}
\end{figure}

The underlying idea for rejection sampling is that we can sample from any distribution
by sampling uniformly from the hyper-volume under its unnormalized probability density function.
Though this is effectively axiomatic by the definition of a probability density
function with respect to the Lebesgue measure, we can get a non measure-theoretic
intuition for this by considering augmenting a target distribution with a new variable $u$
such that $p(u|\theta) = \textsc{Uniform}(0,\gamma(\theta))$.  Sampling 
$\hat{\theta} \sim \pi(\theta)$ and then $\hat{u}\sim p(u|\theta)$ corresponds to
sampling uniformly from hyper-volume under the probability density function, while we
clearly have that the marginal distribution on $\theta$ is $\pi(\theta)$.

\begin{figure}[t]
	\centering
	\includegraphics[width=0.7\textwidth]{reject_samp}
	\caption{Demonstration of rejection sampling for problem shown in~\eqref{eq:inf:example}.  
		We first sample $\hat{\theta}\sim q(\theta)$, correspond to the sampling for the distribution
		shown in blue,
		and then sample $\hat{u}\sim \textsc{Uniform}(0,q(\theta))$, corresponding to
		sampling a point uniformly along the black lines for the two shown example values of 
		$\hat{\theta}$.  The point is accepted 
		if $\hat{u} \le C p(\theta | y=5)$ (i.e. if it below the yellow curve) and is 
		otherwise rejected, where we have taken
		$C=0.52$ to ensure $C p(\theta | y=5)\le q(\theta)$ for all theta.
		Here the example sample pair $\{\hat{\theta}_1,\hat{u}_1\}$ is accepted, while
		$\{\hat{\theta}_2,\hat{u}_2\}$ is rejected.  The
		resulting accepted sample pairs will be uniformly sampled from the region under
		the unnormalized target distribution given by the yellow curve and therefore
		the accepted $\hat{\theta}$ will correspond to exact samples from the target.
		 \label{fig:inf:rej-samp}}
\end{figure}

Using this idea, we can sample from any unnormalized distribution by sampling from
an appropriate bounding as per our intuitive example and then accepting only samples
that fall within the hyper-volume of the probability density function. 
More specifically, we define a proposal
distribution $q(\theta)$ which completely envelopes a
scaled version of the unnormalized target distribution $C\gamma(\theta)$ such that 
$q(\theta)\ge C \gamma(\theta)$ for all values of $\theta$ and $C>0$.  We then sample a pair 
$\{\hat{\theta},\hat{u}\}$ by first sampling $\hat{\theta} \sim q(\theta)$ and then
$\hat{u} \sim \textsc{Uniform}(0,q(\theta))$.  The sample is accepted if
\begin{align}
	\label{eq:inf:rej-acc-criteria}
	\hat{u} \le C \gamma(\hat{\theta})
\end{align}
which occurs with an acceptance rate $CZ$ (note that $q(\theta)\ge C \gamma(\theta) \; \forall \theta$
ensures that $C \le 1/Z$).  Note that this can be used to estimate the normalization
constant, corresponding to the marginal likelihood for Bayesian models, by calculating
the empirical estimate of the acceptance rate and dividing this by $C$.
A graphical demonstration of the rejection sampling process is shown in 
Figure~\ref{fig:inf:rej-samp}.

Rejection sampling can be a highly effective sampling or inference method in low dimensions.
In particular, the fact that it generates exact samples from the target distribution can be very
useful.  For example, this characteristic is used to construct efficient samplers for many 
common distributions such as in the ziggurat algorithm~\citep{marsaglia2000ziggurat} often
used for generating Gaussian random variables.  However, its efficiency is critically dependent
on the value of $C$ because it is directly proportional to the acceptance rate.  By proxy, it
is also critically dependent on the proposal $q(\theta)$ as this dictates the minimum possible
value of $C$, namely $C_{\min} = \min_{\theta} q(\theta) Z / \pi(\theta)$.  Note that if 
$q(\theta) = \pi(\theta) \; \forall \theta$ then the acceptance rate will be $1$, while the
more different they are (in terms of $\min_{\theta} q(\theta) / \pi(\theta)$), the lower the
best possible acceptance rate becomes.  In low dimensions, adaptive rejection 
sampling~\citep{gilks1992adaptive} often forms an effective method for adaptively 
learning an effective proposal and corresponding value for $C$, leading to good acceptance
rates.  However, the approach will be very prone to the \emph{curse of dimensionality} as
we discuss in Section~\ref{sec:inf:foundation:curse}, meaning performance cannot be
maintained for higher dimensional problems.

\subsection{Importance Sampling}
\label{sec:inf:foundation:importance}

\subsection{The Curse of Dimensionality}
\label{sec:inf:foundation:curse}

\todo[inline]{Go back to the square-circle case to explain the curse of dimensionality.  Whole
	section on curse of dimensionality?}

\subsection{Markov Chain Monte Carlo}
\label{sec:inf:foundation:mcmc}

\subsection{Gibbs Sampling}
\label{sec:inf:foundation:gibbs}

\section{Alternatives to Monte Carlo Inference}
\label{sec:inf:alt}

Though our focus in this chapter has mostly been on Monte Carlo inference methods, we finish
by noting that these are far from the only viable approaches.  Two key advantages of Monte Carlo
methods are their ubiquitous nature, i.e. many can almost always be applied, 
and that most commonly used Monte Carlo methods are asymptotically exact, such that given
enough time, we can always achieve a required level of accuracy.  However, in some scenarios,
Monte Carlo methods can be problematically slow to converge and so alternative, asymptotically approximate,
methods can be preferable such as variational inference~\citep{blei2016variational} and
 messaging passing methods~\citep{lauritzen1988local}.  Of these, variational inference has become an increasingly
 popular approach.  Its key idea is to reformulate the inference problem to 
an optimization, by learning parameters of an approximation to the posterior.  Typically this involves
defining some family  of distributions within which the posterior approximation can live, e.g. an exponential
distribution family, and then optimizing an \emph{evidence lower bound} (ELBO) with respect to the parameters of
this approximation.  Doing this implicitly minimizes
the Kullback-Leiber divergence between the approximation the target.
Variational inference often forms a highly efficient means of calculating a posterior approximation, but,
in addition to the obvious bias from using a particular family of distributions for the approximation, it typically
requires strong structural assumptions to be made about the form of the posterior. Namely most methods make a
so-called \emph{mean-field} assumption that presumes that the posterior factorizes over all
latent variables.  Its effectiveness is thus critically dependent on the reasonableness of these assumptions.


% !TEX root = ../main.tex

\section{Foundational Monte Carlo Methods}
\label{sec:inf:foundation}

In this section we introduce the key inference methods that form
the basis upon which most Monte Carlo inference schemes are based.  The
key idea at the heart of all Monte Carlo inference methods is to use some form
of proposal distribution that we can easily sample from and then make
appropriate adjustments to achieve (typically approximate) samples from
the posterior.  All methods will require only an unnormalized distribution
\begin{align}
\label{eq:inf:unnorm-target}
\gamma(\theta) = \pi(\theta)Z
\end{align}
as a target where $Z = \int \gamma(\theta) d\theta$.  As such they will apply to
any situation where we desire to sample from an unnormalized 
(or in some cases normalized) distribution, for which
the Bayesian inference setting is a particular case where
$\gamma(\theta) = p(\mathcal{D}|\theta)p(\theta)$ and $Z = p(\mathcal{D})$.
They will, in general, vary only on how samples are proposal
and the subsequent adjustments that are made.  However, this will lead to a
plethora of different approaches, varying substantially in their motivation,
theoretical justification, algorithmic details, and the scenarios for which they
are effective.

\subsection{Rejection Sampling}
\label{sec:inf:foundation:rejection}

Rejection sampling~\citep{robert2004monte} is one of the simplest Monte Carlo 
inference methods and one of the only ones to produce exact samples from 
the target.  Before going into the method itself, we first consider an example to
demonstrate the underlying intuition.  Imagine we want to generate samples 
distributed uniformly over some arbitrary two dimensional shape.  One simple
way of doing this would be to sample uniformly from a box enclosing the
shape and then only taking the samples which fall with the shape.
An example of such sampling by rejection is shown in Figure~\ref{fig:inf:rej-butt}.
As all the samples within the space or distributed uniformly, they are also
uniformly distributed on any subset of the space.  Therefore if we sample
from a space incorporating the area we care about and then only take the samples
that fall within the desired shape, we will generated samples uniformly over
that shape. We can also use this method to estimate the area of the shape by using
the fact that the probability of any one sample falls within the shape is equal to
the ratio of the areas of the shape and the bounding box, namely
\begin{align}
A_{\text{shape}} &= A_{\text{box}}	P(\theta \in \text{shape}) \\
&\approx \frac{A_{\mathrm{box}}}{N} \sum_{n=1}^{N} \ind (\hat{\theta}_n \in \text{shape})
\quad \text{where} \quad \hat{\theta}_n \sim \textsc{Uniform}(\text{box})
\end{align}
where we have used a Monte Carlo estimate for $P(\theta \in \text{shape})$.
Note that the value of $P(\theta \in \text{shape})$ will
dictate the efficiency of our estimation as it represents the \emph{acceptance rate}
of our samples.  In other words, we need to generate on average $1/P(\theta \in \text{shape})$
samples from our proposal for each sample created in the target area.  As we
will show later in Section INSERT, $P(\theta \in \text{shape})$ typically becomes very
small as $\theta$ becomes high dimensional, so this approach will typically only
be effective in low dimensions.

\begin{figure}[t]
	\centering
	\includegraphics[width=0.6\textwidth]{butterfly}
	\caption{Example of sampling uniformly from an arbitrary shape by 
		rejection.  Here samples are proposed uniformly from the $[-1,1]$
		square.  Any sample falling within the black outline is accepted 
		(blue), otherwise it is rejected (red).  \label{fig:inf:rej-butt}}
\end{figure}

The underlying idea for rejection sampling is that we can sample from any distribution
by sampling uniformly from the hyper-volume under its unnormalized probability density function.
Though this is effectively axiomatic by the definition of a probability density
function with respect to the Lebesgue measure, we can get a non measure-theoretic
intuition for this by considering augmenting a target distribution with a new variable $u$
such that $p(u|\theta) = \textsc{Uniform}(0,\gamma(\theta))$.  Sampling 
$\hat{\theta} \sim \pi(\theta)$ and then $\hat{u}\sim p(u|\theta)$ corresponds to
sampling uniformly from hyper-volume under the probability density function, while we
clearly have that the marginal distribution on $\theta$ is $\pi(\theta)$.

\begin{figure}[t]
	\centering
	\includegraphics[width=0.7\textwidth]{reject_samp}
	\caption{Demonstration of rejection sampling for problem shown in~\eqref{eq:inf:example}.  
		We first sample $\hat{\theta}\sim q(\theta)$, correspond to the sampling for the distribution
		shown in blue,
		and then sample $\hat{u}\sim \textsc{Uniform}(0,q(\theta))$, corresponding to
		sampling a point uniformly along the black lines for the two shown example values of 
		$\hat{\theta}$.  The point is accepted 
		if $\hat{u} \le C p(\theta | y=5)$ (i.e. if it below the yellow curve) and is 
		otherwise rejected, where we have taken
		$C=0.52$ to ensure $C p(\theta | y=5)\le q(\theta)$ for all theta.
		Here the example sample pair $\{\hat{\theta}_1,\hat{u}_1\}$ is accepted, while
		$\{\hat{\theta}_2,\hat{u}_2\}$ is rejected.  The
		resulting accepted sample pairs will be uniformly sampled from the region under
		the unnormalized target distribution given by the yellow curve and therefore
		the accepted $\hat{\theta}$ will correspond to exact samples from the target.
		 \label{fig:inf:rej-samp}}
\end{figure}

Using this idea, we can sample from any unnormalized distribution by sampling from
an appropriate bounding as per our intuitive example and then accepting only samples
that fall within the hyper-volume of the probability density function. 
More specifically, we define a proposal
distribution $q(\theta)$ which completely envelopes a
scaled version of the unnormalized target distribution $C\gamma(\theta)$ such that 
$q(\theta)\ge C \gamma(\theta)$ for all values of $\theta$ and $C>0$.  We then sample a pair 
$\{\hat{\theta},\hat{u}\}$ by first sampling $\hat{\theta} \sim q(\theta)$ and then
$\hat{u} \sim \textsc{Uniform}(0,q(\theta))$.  The sample is accepted if
\begin{align}
	\label{eq:inf:rej-acc-criteria}
	\hat{u} \le C \gamma(\hat{\theta})
\end{align}
which occurs with an acceptance rate $CZ$ (note that $q(\theta)\ge C \gamma(\theta) \; \forall \theta$
ensures that $C \le 1/Z$).  Note that this can be used to estimate the normalization
constant, corresponding to the marginal likelihood for Bayesian models, by calculating
the empirical estimate of the acceptance rate and dividing this by $C$.
A graphical demonstration of the rejection sampling process is shown in 
Figure~\ref{fig:inf:rej-samp}.

Rejection sampling can be a highly effective sampling or inference method in low dimensions.
In particular, the fact that it generates exact samples from the target distribution can be very
useful.  For example, this characteristic is used to construct efficient samplers for many 
common distributions such as in the ziggurat algorithm~\citep{marsaglia2000ziggurat} often
used for generating Gaussian random variables.  However, its efficiency is critically dependent
on the value of $C$ because it is directly proportional to the acceptance rate.  By proxy, it
is also critically dependent on the proposal $q(\theta)$ as this dictates the minimum possible
value of $C$, namely $C_{\min} = \min_{\theta} q(\theta) Z / \pi(\theta)$.  Note that if 
$q(\theta) = \pi(\theta) \; \forall \theta$ then the acceptance rate will be $1$, while the
more different they are (in terms of $\min_{\theta} q(\theta) / \pi(\theta)$), the lower the
best possible acceptance rate becomes.  In low dimensions, adaptive rejection 
sampling~\citep{gilks1992adaptive} often forms an effective method for adaptively 
learning an effective proposal and corresponding value for $C$, leading to good acceptance
rates.  However, the approach will be very prone to the \emph{curse of dimensionality} as
we discuss in Section~\ref{sec:inf:foundation:curse}, meaning performance cannot be
maintained for higher dimensional problems.

\subsection{Importance Sampling}
\label{sec:inf:foundation:importance}

\subsection{The Curse of Dimensionality}
\label{sec:inf:foundation:curse}

\todo[inline]{Go back to the square-circle case to explain the curse of dimensionality.  Whole
	section on curse of dimensionality?}

\subsection{Markov Chain Monte Carlo}
\label{sec:inf:foundation:mcmc}

\subsection{Gibbs Sampling}
\label{sec:inf:foundation:gibbs}
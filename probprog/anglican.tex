% !TEX root = ../main.tex

\section{The Anglican Probabilistic Programming Language}
\label{sec:probprog:anglican}

To allow for a more precise consideration of issues associated with designing and using a universal
PPS, we now introduce the particular language \emph{Anglican}~\citep{wood2014new,tolpin2016design},
which we will use for reference throughout the rest of the thesis.  
Anglican is a universal PPL integrated into \emph{Clojure}~\citep{hickey2008clojure}, a dynamically-typed, general-purpose, functional
programming language (specifically a dialect of Lisp) that runs on the Java virtual machine and uses just-in-time compilation.
Anglican inherits most of the syntax of Clojure, but extends it with the key
special forms \sample and \observe \citep{tolpin2015probabilistic,tolpin2016design}, defined in the same way as
our example language setup in the previous section, along with a couple of others which aid in defining queries
such as \mem, \store, and \retrieve.  Despite using predominantly the same syntax, Anglican has different
\emph{probabilistic} semantics to Clojure, i.e. code is written in the same way, but is interpreted differently.

Anglican was the first PPL to introduce particle based 
inference schemes such as SMC and particle MCMC (see Chapters~\ref{chp:part} and \ref{chp:proginf}), which was a
key advancement for universal PPSs because it allows the structure of the query to be exploited, often providing
substantially more efficient inference then previous approaches.  Though these methods still form the core
of the inference in Anglican, there have been a number of advancements and alternative inference approaches introduced since
the original work~\citep{paige2014asynchronous,tolpin-socs-2015,tolpin2015output,vandemeent_aistats_2015,
	rainforth2016bayesian,rainforth2016interacting,le2017inference}.  Another notable feature of Anglican is its
support for Bayesian non-parametric modeling, providing constructs for general processes along with
primitives for particular models such as Dirichlet processes.  Inevitably we can only provide a limited 
introduction here and so we also refer the interested reader to~\cite{tolpin2016design} and to the Anglican website {\small\url{http://www.robots.ox.ac.uk/~fwood/anglican/}} for more information.

\subsection{Clojure}
\label{sec:probprog:anglican:clojure} 

Before getting into the details of Anglican, we first give a very brief introduction to Clojure~\citep{hickey2008clojure},
because its syntax may be quite unfamiliar to readers without experience in Lisp-based notation or functional programming
more generally.  There are two key things to get your head around for reading and using Clojure (and Lisp languages more
generally): almost everything is a function and parentheses evaluate functions.  For example, to code $a+b$ in Clojure one
would write {\small \lsi{(+ a b)}} where \clj{+} is a function taking two arguments (here \clj{a} and \clj{b}) and the parentheses cause the
function to evaluate.  More generally, Clojure uses prefix notation such that 
the first term in a parenthesis is always a function, with the other terms the arguments.
Thus {\small \lsi{((+ a b))}} would be invalid syntax as the result of {\small \lsi{(+ a b)}}  is not a function so cannot be evaluated.
Expressions are nested in a tree structure so to code $2(a+b)/c$ one would write {\small \lsi{(/ (* 2 (+ a b)) c)}}.  One can
thus think about going outwards in terms of execution order -- a function first evaluates its arguments before
itself.  Functions in Clojure are first class (i.e. they can be variables) and can be declared either anonymously using
{\small \lsi{(fn [args] body)}}, for example {\small \lsi{(fn [a b] (/ a b))}}, or using the
macro \defn to declare a named function in the namespace, for example {\small \lsi{(defn divide [a b] (/ a b))}}.
Local bindings in Clojure are defined using \cllet blocks which denote a number of name-value pairs for
binding variables, followed by a series of evaluations to  carry out. Their return value
is the output of the last of these evaluations.  Thus for example
{\small \lsi{(let [a 2 b 3] (print a) (+ a b))}} says let $a$ be equal to $2$, $b$ be equal to $3$, print $a$,
and then evaluate $a+b$, thus returning $5$.  Note that $a$ and $b$ remain undefined outside of this let block.  

Clojure allows various types of compound literals such as vectors, lists, hash maps, and sets.
In general, elements in these compound literals are not restricted in
their type so it valid to write for example {\small \lsi{[1 (fn [x] (inc x)) "2"]}} to construct a vector 
whose component terms are different types.

Clojure does not generally use
for loops, instead relying upon the constructs \map and \reduce, or the \clj{loop}-\clj{recur} pairing. Here
\map applies a function to every value in a list or vector so for example {\small \lsi{(map (fn [a] (* a 2)) [2.1 3])}}
doubles each value in the vector of inputs returning a list {\small \lsi{(4.2 6)}}, where the parentheses in the output
now just represent a list rather than a function evaluation, just to be confusing.  Meanwhile, \reduce recursively applies
a function and so for example {\small \lsi{(reduce + 0 [1.2 3.4 2])}} sums the vector of elements,
returning {\small \lsi{6.6}}.  The \clj{loop}-\clj{recur} pairing forms
a syntactic sugar for doing looping through tail recursion.  In essence, \clj{loop} defines a function and provides
initial bindings for the input arguments.  Inside the \clj{loop} function block, \clj{recur} recursively calls the
function defined by \clj{loop} with new bindings to the inputs.  In general, the \clj{recur} function should
only be called depending on a condition holding true to prevent infinite recursion.  To give a concrete example,
 \begin{lstlisting}[basicstyle=\ttfamily\small,frame=none]
 (loop [x 1] 
  (if (> x 10) 
   nil 
   (do (print x) (recur (+ x 1)))))
 \end{lstlisting}\vspace{-5pt}
will print out $1$ to $10$ before returning \clj{nil}.  
Here the syntax of \clj{if} is \clj{(if test then else)} so when \clj{x}$>10$
it returns \clj{nil}.  Otherwise, it calls the \clj{do} statement which evaluates each of its arguments in turn, in
this case printing out \clj{x} before recalling the body of the \clj{loop} block with \clj{x}  reset to \clj{(+ x 1)}.

An important feature of Clojure, particularly with regards to Anglican, is its support for (infinite) lazy sequences.
Lazy sequences are sequences of terms that are only evaluated as and when they are required.  As such, it is 
valid for them to be infinitely long (typically through recursive definition), but with only a finite number
of values from which are ever evaluated in practice.    For example one can define the function
 \begin{lstlisting}[basicstyle=\ttfamily\small,frame=none]
 (defn ints [n] (lazy-seq (cons n (ints (inc n)))))
 \end{lstlisting}\vspace{-5pt}
 and then call {\small \lsi{(ints 1)}} to produce a
lazy sequence comprising of all the positive integers.  We can evaluate the sequence by explicitly requesting
terms with the sequence.  For example, we can use {\small \lsi{(take 5 (ints 1))}} to return the first $5$ elements
of our lazy sequence, thus returning {\small \lsi{(1 2 3 4 5)}}.  We could also ask for a particular term in the
sequence using, for example, {\small \lsi{(nth 4 (ints 1))}} to return $5$.
Lazy sequences are conceptually useful, amongst other things, for specifying things that
are required to be infinitely long to ensure theoretical guarantees (e.g. an MCMC sampler needs to be run
infinitely long to converge), but which are restricted by computational budgets.

\subsection{Writing Models in Anglican}
\label{sec:probprog:anglican:models} 

Anglican queries are written using the macro \defquery.  This allows users to define a model using a mixture
of \sample and \observe statements and deterministic code in the manner synonymous to that explained in
Section~\ref{sec:probprog:models}, and bind that model to a variable.  
As before, Anglican makes use of distribution objects for providing inputs to \sample and \observe, for which
it provides a number of common constructors corresponding to common
elementary random procedures such
as \gammaa, \normal, and \betaa.  A distribution object is generated by calling the class constructor
with the required parameters, e.g. {\small \lsi{(normal 0 1)}} for the unit Gaussian.
Anglican also allows custom distribution classes to be defined using
the \defdist macro, which requires the user to provide parameterized code for sampling from that distribution
class and evaluating the log density at given output.

\begin{wrapfigure}{r}{0.45\textwidth}
	\centering 
	\vspace{-10pt}
	\begin{lstlisting}[basicstyle=\ttfamily\small]
(defquery my-query [data]
  (let [mu (sample (normal 0 1))
        sig (sample (gamma 2 2))
        lik (normal mu sig)]
   (map (fn [obs]
          (observe lik obs))
     data)
   [mu sig]))
	\end{lstlisting}	
	\vspace{-5pt}
	\caption{A simple Anglican query.\label{fig:probprog:simple-ang}}
	\vspace{-10pt}
\end{wrapfigure}
A simple example of an Anglican query is shown in Figure~\ref{fig:probprog:simple-ang},
corresponding to a model where we are trying to infer the mean and standard deviation
of a Gaussian given some data.  The syntax of \defquery is {\small \lsi{(defquery name [args] body)}}
so in Figure~\ref{fig:probprog:simple-ang} our query is named {\small \lsi{my-query}} and takes
in arguments {\small \lsi{data}}.  The query starts by sampling {\small \lsi{mu}}$\sim\mathcal{N}(0,1)$
and {\small \lsi{sig}}$\sim\textsc{Gamma}(2,2)$ and constructing a distribution object {\small \lsi{lik}}
to use for the observations.  It then maps over each datapoint and observes it under the distribution
{\small \lsi{lik}}, remembering that {\small \lsi{fn}} defines a function and \map applies a function
to every element of a list or vector.  
Note that \observe simply returns {\small \lsi{nil}} and so we
are not interested in these outputs -- they affect the probability of a program execution trace, but
not the calculation of a trace itself.  After the observations are made, {\small \lsi{mu}} and {\small \lsi{sig}}
are returned from the \cllet block and then by proxy the query itself.
Because the data terms only get used as observations and we do
not observe any internal random variables, this simple query corresponds exactly to a Bayesian model where
the \sample terms form the prior and the \observe terms form the likelihood as we explained in
Section~\ref{sec:probprog:models:general}.  The query thus defines a correctly normalized joint
distribution given by
\begin{align}
p(\mu,\sigma,y_{1:S}) = \mathcal{N}(\mu ; 0,1) \; \textsc{Gamma}(\sigma ; 2, 2) \prod_{s=1}^{S} \mathcal{N}(y_s ; \mu, \sigma)
\end{align}
where we have defined $\mu:=${\small \lsi{mu}}, $\sigma:=${\small \lsi{sig}}, and $y_{1:S}:=${\small \lsi{data}}.
Other more complicated example Anglican models are given in Figures~\ref{fig:probprog:example-ang}
and~\ref{fig:probprog:schell}.

\begin{figure}[t]
	\centering
	\begin{subfigure}[t]{0.45\textwidth}
			\centering	
\begin{lstlisting}[basicstyle=\ttfamily\footnotesize]
(defquery lin-reg [xs ys]
 (let [m (sample (normal 0 10))  
       c (sample (normal 0 3))
       s (sample (gamma 1 1))
       f (fn [x] (+ (* m x) c))]
  (map (fn [x y]
    (observe 
      (normal (f x) s) y))
   xs ys)
  [m c s]))
\end{lstlisting}				
			\caption{Linear regression model with unknown slope, intercept, and
				observation variance.  Here \clj{xs} and
	 \clj{ys} are the data comprising of the inputs and corresponding
	 outputs of the regression.  The query returns estimates
	 for the slope \clj{m}, the intercept \clj{c}, and the observation
	 standard deviation (i.e. noise) \clj{s}. \label{fig:probprog:lregang}
		}
	\end{subfigure}
~~
	\begin{subfigure}[t]{0.52\textwidth}
		\centering	
\begin{lstlisting}[basicstyle=\ttfamily\footnotesize]
(defquery discrete-hmm
 [ys x0 trans obs]
 (loop [xs x0 t 0]
  (let 
   [x (sample (nth trans (last xs)))]
   (observe (nth obs x) 
     (nth ys t))
   (if (= (inc t) (count ys))
    (conj xs x)
    (recur (conj xs x) (inc t))))))
\end{lstlisting}	
		\caption{Hidden Markov model with discrete states.
	Here \clj{ys} is the data, \clj{x0} is a starting state (which
	has no associated observation), \clj{trans} is the collection of 
	transition distribution objects for each of the $K$ possible states
	(indexed by the value of $x_t$), and \clj{obs} are the matching
	emission distributions.	Query returns
	estimates for the latent states \clj{xs}.\label{fig:probprog:hmm}
		 }
	\end{subfigure}
	\vspace{5pt}
		\caption{Example Anglican queries for simple models. Here \defquery
			binds a query to a variable, for which inference can be run using
			{\footnotesize \lsi{(doquery inf-alg model inputs & options)}}. 
%			with syntax
%			{\footnotesize \lsi{(doquery inf-alg model inputs & options)}}. 
%			For example, we could call {\footnotesize \lsi{(doquery :ipmcmc
%					lin-reg [xs ys] :number-of-particles 1000)}}  for some
%			predefined \lsi{xs} and \lsi{ys} to run iPMCMC inference (see Section~\ref{sec:part:ipmcmc}))
%			on the \lsi{lin-reg} model with with 1000 particles.
			\label{fig:probprog:example-ang}}
\end{figure}

Compilation of an Anglican program is performed by the macro \query.  Calling \query
invokes a source-to-source compilation from an Anglican query to a continuation-passing 
style (CPS) Clojure function that can be executed using a particular inference algorithm.  We
leave discussing in depth what this means to Chapter~\ref{chp:proginf}, noting only that
the inference algorithms associated with Anglican are written in pure Clojure (except for the odd
bit of Java code) and the aim of the compilation is to produce a model artifact which can be
interpreted by that inference code.
When \defquery is called, Anglican internally calls \query and assigns the output to
the symbol provided by the first argument of \defquery.  Therefore {\small \lsi{my-query}} in
our example is Clojure function that can be read by the Anglican inference engines to produce
approximate samples from the conditional distribution.  Note that the call structure of this Clojure
function is not as per the original Anglican query, but is instead a compilation artifact with
call syntax as per what is required by the Anglican inference engine.
%, as a Clojure function, the
%inputs and body of {\small \lsi{my-query}} and not as per lexical definition of the Anglican program
%(e.g. in now takes multiple inputs).  It is instead a function that encodes all the information 
%required by the inference engine to perform inference and has the corresponding required call structure
%for the input required by the engine.

\begin{figure}[p]
		\centering	
		\vspace{-20pt}		
\rule{\linewidth}{0.4pt}
\vspace{-28pt}		
		\begin{lstlisting}[basicstyle=\ttfamily\footnotesize,multicols=2,frame=none]
(defdist strike [volatility] []
 (sample* [this] 
   (if (sample* (flip volatility)) 
       :war :peace))
 (observe* [this value] 
  (observe* (flip volatility) 
    (= value :war))))
  
(declare A B)

(with-primitive-procedures [strike]
 (defm A [depth]
  (let [B-strike (B (dec depth))
        A-strike (strike 0.1)]
   (observe A-strike B-strike)
   B-strike))
   
 (defm B [depth]
  (let [B-strike (strike 0.2)]
   (if (> depth 0)
    (let [A-strike (A depth)]
      (observe B-strike A-strike)
      A-strike)
    (sample B-strike))))

 (defquery war [depth]
   (or (= (A depth) :war)
       (= (B depth) :war))))
\end{lstlisting}	
\vspace{-20pt}		
\rule{\linewidth}{0.4pt}
%\vspace{5pt}
		\caption{Anglican code for an example Schelling co-ordination game~\citep{schelling1980strategy,stuhlmuller2014reasoning}
		modeling if a peace will hold between two volatile countries which we refer to as A and B.
		This example is adapted from an online Anglican example developed by Brooks Paige and Frank Wood
		which can be found at~\url{http://www.robots.ox.ac.uk/~fwood/anglican/examples/}, providing a more
		detailed consideration along with other examples.
		Our code first defines a custom distribution called \lsi{strike} for modeling whether one country would
		attack the other if it did not reason about the other's actions.  
It then constructs an Anglican function for each country using \defm.  
Each country has their own volatility and the ability to reason about whether they think the other country will
initiate a strike against them. We presume they have access to the exact model of the other country.  A's aim is to match
the action it thinks B will do -- it would prefer there to be no war, but if there is, it wants to strike
first.  Its model first samples a draw for B's action by simulating from B's model and then observes this action
under its own distribution on whether to strike; this is equivalent to sampling both in isolation and constraining
them to be the same.  However, B can also reason about A and
A knows it.  It, therefore, provides the simulation of B's model with a \emph{meta-reasoning depth}.  If this depth
is zero, B makes its decision in isolation, simply sampling from \lsi{B-strike}.  If the depth is greater 
than zero though, it uses the same approach as A, simulating from A's model (with a depreciated meta-reasoning 
depth) and then observing this 
outcome under \lsi{B-strike}.  This creates a mutually recursive cycle of reasoning.  As an observer, we are
interested in establishing the probability that one of the two will strike
and so we define a query which outputs whether the two are the same.
The expectation of the output from this query corresponds to the probability of war.
Invoking inference using \lsi{(doquery :lmh war [depth])} we see that for a depth of $0$, there
is roughly a $0.222$ chance of war, while depths of $1$, $2$, and $10$ give rough respective chances of $0.052$, $0.0021$,
and $3\times10^{-6}$.  Thus the chance of war goes down the more each country thinks about the reasoning
of the other.  It is interesting to note that if the volatilities are set to the same value then
the critical tipping point, above which things escalate towards war with increased depth, is around $0.38$.
The mutual recursion involved in the model and the fact that variables are both sampled and 
observed means that it would be difficult to encode it using a graphical model.  It is not even clear how
one would go about writing down a well-defined joint distribution for the model, without having to empirically
calculate a normalization constant.
\label{fig:probprog:schell}
		}
\end{figure}

Inference on the model is performed using the macro \doquery, which produces a lazy infinite sequence of 
approximate samples from the conditional distribution and, for appropriate inference algorithms,
an estimate of the partition function.
The syntax of \doquery is {\small \lsi{(doquery inf-alg model inputs & options)}} where {\small \lsi{inf-alg}}
specifies an inference algorithm, {\small \lsi{model}} is our query, {\small \lsi{inputs}}
is a vector of inputs to the query, the \& represents that there might be multiple additional inputs, and {\small \lsi{options}}
is a series of option-value pairs for the inference engine.  For example, to produce 100 samples from
our {\small \lsi{my-query}} model with data {\small \lsi{[2.1 5.2 1.1]}} using the LMH inference algorithm with no
additional options, we can write {\small \lsi{(take 100 (doquery :lmh my-query [2.1 5.2 1.1]))}}.

Although Anglican mostly inherits Clojure syntax, some complications arise from the compilation to
CPS-style Clojure functions.  It is thus not possible to na\"{i}vely use all Clojure functions inside Anglican
without providing appropriate information to the compiler to perform the required transformation.
The transformation for many core Clojure functions has been coded manually, such that they form
part of the Anglican syntax and can be used directly.  Anglican further allows users to write functions externally
to \defquery using the macro \defm, which is analogous to \defn in Clojure and has the same call syntax,\footnote{A small
	exception to this is that it is restricted to have a single input.  As this itself can be a vector of inputs, one
	can still write arbitrary functions.}
which can then be called from inside the query without restrictions.  Other deterministic Clojure functions can also
be called from inside queries but must be provided with appropriate wrapping to assist the compiler.  Thankfully
this is done automatically (except when the function itself is higher order) using the macro
{\small \lsi{with-primitive-procedures}} that takes as inputs Clojure functions and creates an environment where
these functions have been converted to their appropriate CPS forms.

We finish our introduction by noting the ability of Anglican to \emph{nest} queries within one another.  This
is somewhat experimental and comes with a number of health warnings because, as we will explain in Chapter~\ref{chp:nest},
it allows users to write so-called \emph{nested estimation} problems that fall beyond the scope of the standard
proofs of convergence for stand Monte Carlo estimation schemes. Nonetheless, there are problems that cannot
be encoded without this ability, as we will discuss in Section~\ref{sec:nest:imp}.
Though \doquery is not allowed directly within a \defquery block, one can still nest queries by 
either using a \doquery in a Clojure function that is then passed to another query using {\small \lsi{with-primitive-procedures}},
using a \doquery within a custom distribution defined by \defdist, or using the special form \conditional which
takes a query and returns a distribution object constructor corresponding to that query, 
for which the inputs to the query become the parameters.
We will return to consider the statistical implications of this at length in Section~\ref{sec:nest:imp}.
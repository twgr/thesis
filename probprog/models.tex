% !TEX root = ../main.tex

\section{Bayesian Models as Program Code}
\label{sec:probprog:models}

In Section~\ref{sec:probprog:inv} we showed how one can think of PPS as inverting simulators, 
predicting internal variables given the outputs.  In this section we will take a 
different perspective and
show how we can translate Bayesian modeling into the framework of program code.   
In Chapter~\ref{chp:bayes} we showed how a Bayesian model is defined by a prior over
parameters and a likelihood function for those parameters given the data.  This viewpoint will
mostly translate into the probabilistic programming setting by equating between the prior
and sampling statements and between the likelihood and conditioning statements, though as
we explain in Section~\ref{sec:probprog:models:general}, this is not always true.
%At the end
%of the section we will explain why this is actually a slight approximation (in short because we
%might condition on internally sampled variables) but for most purposes this viewpoint will suffice.
%We will keep things predominantly high-level for now, giving a more detailed look in
%Section~\ref{sec:probprog:anglican} by introducing a particular PPS, namely Anglican, in detail.

A key point to note throughout this section is that probabilistic programs define
models rather than procedures.  In a standard programming language, functions take in inputs
and then run through a series of commands in order until termination is reached.\footnote{In functional
	programming languages operations are not necessary carried out in the order they are defined,
	but it still holds that the function takes inputs carries out a series of actions until the desired output
	is (perhaps lazily) calculated.}  Likewise, random sampling statements like \texttt{rand}, \texttt{randn}
etc, make a single draw from the same distribution each time they appear in the execution trace.
Neither is the case for a probabilistic program.  Instead a probabilistic program defines a model
which is interpreted or compiled to a form that can be interpreted by an inference engine which then
outputs some characterization of the posterior such as a series of samples.  Perhaps the easiest way to think
about how a probabilistic program runs (though not necessarily what happens for all systems) is that
the program is, or sometimes parts of the program are, run many times and the exact behavior
of this running is control by the inference engine.  For example, the inference engine 
might sample from a different distribution then specified by the original sampling statement or it
might run lots of instances of the program in parallel and then terminate some of the instances
performing poorly, while duplicating instances doing well.

\subsection{A Simplified Probabilistic Programming Setup}
\label{sec:probprog:models:first}

We first consider the case of constructing a restricted example PPS.  We emphasize that
this is by no means the only setup one can use, with design choices made in the interest of exposition.
We will presume that our
language has no branching (i.e. there are no \texttt{if} statements or equivalent), is  first order
(i.e. variables cannot be functions), that there is no recursion, and that it does not allow 
any conditioning on internally sampled variables.  
We will give our language
two special constructs, \sample and \observe, between which the distribution of the
program is defined.  We will presume here and throughout that, other than the effects of \sample and \observe,
functions in our PPS are \emph{pure} such that they always provide the same outputs when
called with the same inputs.  This restriction naturally means that programs should not have
any random components other than dictated by \sample and \observe, but also suggests that
they should be free from side effects such as modifications of global variables.
Informally, \sample will be used to specify terms in the prior and \observe terms in the
likelihood.  More precisely, \sample will be used to make random draws $x_t \sim f_t(x_t | \Xi_t)$,
where $\Xi_t$ is a subset the other variables in scope at the point of sampling, and \observe will use to condition on
data $g_s(y_s|\Lambda_s)$ with $\Lambda_s$ defined in the same way as $\Xi_t$.  
For our inference, it will be necessary to control the sampling and so we define
the syntax of \sample to take a \emph{distribution object} as its only input and for \observe
to take a distribution object and an observation as input.  We further define each distribution
object as containing a sampling procedure and a density function that can be evaluated
exactly.\footnote{It will actually often be possible to use incomplete distribution objects for
	both \sample and \observe that only contain the sampler and the density function respectively.
	This former can be permitted by using inference algorithms that use the prior as the proposal, such
	that only the sampler will be required.  The latter is sufficient in the vast majority of scenarios as, unless
	one is carrying out amortization~\citep{paige2016inference,le2017inference}, the data is fixed and does
	not need to be sampled.}  Our language will be provided with a number of \emph{random primitives} in
the form of distribution classes for common sampling distributions such as the normal and Poisson
distributions, but will also provide the ability for users to define their own distribution classes.   These
classes allow a distribution object to be constructed when provided with the required parameters, such that
the distribution is fully defined before being passed to a \sample or \observe.
We complete our syntactic definition of \sample and \observe by defining them to return a sample and \texttt{nil}
respectively.

\begin{figure}[p]
	\centering
	\begin{subfigure}[t]{\textwidth}
		\centering	
		\begin{minipage}[t]{0.48\textwidth}
		\begin{algorithmic}[1]
			\renewcommand{\algorithmicrequire}{\textbf{Inputs:}}
			\renewcommand{\algorithmicensure}{\textbf{Outputs:}}			 
			\Require Student-t degrees of freedom $\nu$, error
			scale $\sigma$, data $y_{1:S} = \{u_s,v_s\}_{s=1}^S$
			%\Ensure weighted samples $\left\{x_{1:T}^i,w_T^i\right\}_{i=1}^N$
%			\State \texttt{m\_dist} $\leftarrow$ 
%			\State \texttt{c\_dist} $\leftarrow$ \normal(0,1)
			\State $m\leftarrow$\sample(\normal(0,1))
			\State $c\leftarrow$\sample(\normal(0,1))
			\State \lstinline$obs-dist$\xspace $\leftarrow$ \studentt $(\nu)$
			\For{$s=1,\dots,S$}	
			\State $d \leftarrow (v_s-m u_s - c)/\sigma$
			\State \observe(\lstinline$obs-dist$\xspace,~$d$)
			\EndFor
			\State \Return $m, \; c$
		\end{algorithmic}
		\end{minipage}
		~~
		\begin{minipage}[t]{0.48\textwidth}
			\vfill
			\resizebox{\textwidth}{!}{
				% !TEX root = ../../main.tex

\begin{tikzpicture}

\node[latent, minimum size=27pt] (m) {$m$};
\node[latent, right=1.4cm of x1, minimum size=27pt] (c) {$c$};

\node[obs, below=of m, minimum size=27pt, xshift=-2.7cm] (y1) {$y_1$};
\node[obs, right=1cm of y1, minimum size=27pt] (y2) {$y_2$};
\node[right=1cm of y2, minimum size=27pt] (y3) {{\tiny $\bullet \; \bullet \; \bullet$}};
\node[obs, right=1cm of y3, minimum size=27pt] (y4) {$y_{S\text{-}1}$};
\node[obs, right=1cm of y4, minimum size=27pt] (y5) {$y_S$};

\edge {m,c} {y1,y2,y4,y5};

\end{tikzpicture}}
			{\small
			\begin{align*}
			p(m,c&, y_{1:S} | \theta)= \mathcal{N}(m;0,1) \mathcal{N}(c;0,1) \\
										&\prod_{s=1}^{S} \textsc{Student-T} \left(\frac{v_s-mu_s-c}{\sigma} ; \nu \right)
			\end{align*}}
		\end{minipage}
		\caption{Bayesian linear regression model with student-t likelihood, namely
			$v_s = m u_s + c + \sigma \epsilon_s$ where $\epsilon_t \sim \textsc{Student-T}(\nu)$.
			We presume that the scaling of the error $\sigma$ and the number of degrees of freedom $\nu$
			are fixed input parameters (i.e. $\theta=\{\nu,\sigma\}$), that our fixed data is $y_{1:S} = \{u_s,v_s\}_{t=1}^S$,
		 and that we are trying to infer the slope $m$ and intercept $c$ (we thus have $x_1=m$, $x_2=c$ in our
		 general notation), both of which are assigned a unit Gaussian as a prior.  Our probabilistic program 
			first samples $m$ and $c$ (note that \normal(0,1) generates a unit Gaussian distribution object)
			and constructs a student-t distribution object \lstinline$obs-dist$.  It then cycles
		  over each datapoint and observes $(v_t-m u_t -c)/\sigma$ using \lstinline$obs-dist$,
		  before finally returning $m$ and $c$ as outputs.  
%		  The program corresponds to the graphical model
%		  shown in the top right and induces the joint distribution shown bottom right.  
		  Note that if we instead
		  wished to directly predict the outputs at some untested inputs points $u_{S+1:S+n}$ then we could
		  predict these anywhere in the program (after $m$ and $c$ have been defined) and return these
		  as outputs along with, or instead of, $m$ and $c$.
		  \label{fig:probprog:linear-reg}
		  }
	\end{subfigure}
		\begin{subfigure}[t]{\textwidth}
			\vspace{10pt}
\centering	
\begin{minipage}[t]{0.45\textwidth}
	\begin{algorithmic}[1]
		\renewcommand{\algorithmicrequire}{\textbf{Inputs:}}
		\renewcommand{\algorithmicensure}{\textbf{Outputs:}}			 
		\Require Transition std-dev $\sigma$, output shape $\alpha$,
		output rate $\beta$, data $y_{1:T}$
		%\Ensure weighted samples $\left\{x_{1:T}^i,w_T^i\right\}_{i=1}^N$
		%			\State \texttt{m\_dist} $\leftarrow$ 
		%			\State \texttt{c\_dist} $\leftarrow$ \normal(0,1)
		\State $x_0\leftarrow0$
		\State \lstinline$tr-dist$\xspace $\leftarrow$ \normal $(0,\sigma)$
		\State \lstinline$obs-dist$\xspace $\leftarrow$ \lstinline$gamma$\xspace $(\alpha,\beta)$
		\For{$t=1,\dots,T$}	
		\State $x_t \leftarrow x_{t\text{-}1}+$\sample(\lstinline$tr-dist$\xspace)
		\State \observe(\lstinline$obs-dist$\xspace,~$y_t-x_t$)
		\State $z_t \leftarrow \mathbb{I}(x_t>4)$
		\EndFor
		\State \Return $z_{1:T}$
	\end{algorithmic}
\end{minipage}
~~
\begin{minipage}[t]{0.52\textwidth}
	\vfill
	~~~\resizebox{0.96\textwidth}{!}{
		% !TEX root = ../../main.tex

\begin{tikzpicture}

\node[latent, minimum size=27pt] (x1) {$x_1$};
\node[latent, right=1cm of x1, minimum size=27pt] (x2) {$x_2$};
\node[right=1cm of x2] (x3) {{\tiny $\bullet \; \bullet \; \bullet$}};
\node[latent, right=1cm of x3, minimum size=27pt] (x4) {$x_{T\text{-}1}$};
\node[latent, right=1cm of x4, minimum size=27pt] (xT) {$x_T$};

\node[obs, below=1cm of x1, minimum size=27pt] (y1) {$y_1$};
\node[obs, below=1cm of x2, minimum size=27pt] (y2) {$y_2$};
%\node[obs, below=of x3] (y3) {$y_3$};
\node[obs, below=1cm of x4, minimum size=27pt] (y4) {$y_{T\text{-}1}$};
\node[obs, below=1cm of xT, minimum size=27pt] (yT) {$y_T$};

%\node[latent, above=of x1, xshift=-1.2cm, minimum size=27pt] (t) {$\theta$};

\edge {x1} {x2,y1} ; %
\edge {x2} {x3,y2} ; %
\edge {x3} {x4} ; %
\edge {x4} {xT,y4} ; %
\edge {xT} {yT} ; %

%\edge {t} {x1,x2};
%\edge[bend left=0] {t} {x4};
%\edge[bend left=10] {t} {xT};

\end{tikzpicture}}
	{\small
	\begin{align*}
	p(x_{1:T}&, y_{1:T} | \theta)= \mathcal{N}(x_1;0,\sigma^2) \textsc{Gam}(y_1\text{-}x_1;\alpha,\beta)\\
	&\prod_{t=2}^{T} \mathcal{N}(x_t\text{-}x_{t-1};0,\sigma^2)\textsc{Gam} (y_t\text{-}x_t;\alpha,\beta)
	\end{align*}}
\end{minipage}
			\caption{Linear state-space model with Gaussian transition and Gamma emission distributions.
				This is a continuous form of the HMM model given in~\eqref{eq:bayes:hmm} with
				$p(x_1)=\mathcal{N}(x_1;0,\sigma^2)$, $p(x_t | x_{t-1}) = \mathcal{N}(x_t-x_{t-1}; 0,\sigma^2)$,
				and Gamma likelihood model $p(y_t | x_t) = \textsc{Gamma}(y_t-x_t ; \alpha, \beta)$ with
				shape parameter $\alpha$ and scale parameter $\beta$.  We assume that the input
				parameters $\theta = \{\sigma,\alpha,\beta\}$ are fixed and we want
				to sample from $p(z_{1:T} | y_{1:T}, \theta)$ given data $y_{1:T}$ where each 
				$z_t$ is an indicator for if $x_{t}$ exceeds a threshold of $4$. Our program, exploiting the
				equivalence between $p(x_1)$ and $p(x_t|x_{t-1}=0)$, first initializes $x_0=0$ and creates
				distribution objects for the transitions \lstinline$tr-dist$\xspace and emissions
				\lstinline$obs-dist$\xspace.  It then loops over time steps, sampling each $x_t$ given
				$x_{t-1}$, observing $y_t$ given $x_t$, and deterministically calculating $z_t$.  Finally
				the $z_{1:T}$ are returned as the desired output.
				\label{fig:probprog:LGSSM}
				}
		\end{subfigure}
\caption{Example pseudo-programs for our simplified probabilistic programming setup with
	corresponding graphical models and joint distributions.\label{fig:probprog:example-progs}}
\end{figure}

We distinguish between two types of inputs to our programs: external parameters $\theta$ and data $y_{1:S}$.
The external parameters are defined as the inputs that are 
not ``observed'' at any point but can effect the conditioning through $\Xi_t$ and $\Lambda_s$.
We presume for our simplified setup that the data terms, defined as the inputs we are observed, 
appear in neither $\Xi_t$ or $\Lambda_s$.
We now define both \sample and \observe from the probability model perspective
as adding a factor to the joint distribution which is therefore given by
\begin{align}
\label{eq:probprog:simple-joint}
p(x_{1:T},y_{1:S} | \theta) = \prod_{t=1}^{T} f_t(x_t | \Xi_t) \prod_{s=1}^{S} g_s(y_s|\Lambda_s).
\end{align}
The two vary in whether they define a new random variable or effect the probability of the
program given particular instances of the other random variables.
Our presumptions for this simplified setup that no $y_{s}$ terms are present in the $\Xi_t$ or $\Lambda_s$
and that we do not condition on internally sampled variables, means that the product of the \sample terms
correspond exactly to our prior $\prod_{t=1}^{T} g_t(x_t | \Xi_t) =: p(x_{1:T} | \theta)$ and that the
product of the \observe terms corresponds exactly to our likelihood
$\prod_{s=1}^{S} g_s(y_s|\Lambda_s) =: p(y_{1:S} | x_{1:T}, \theta)$.  Consequently, for our simplified setup,
each program defines a finite directed acyclic graphical model (see Section~\ref{sec:bayes:paradigm:graph})
where the conditional relationships are defined through the definitions of $f_t$ and $g_s$.
This breakdown into a prior and likelihood and the equivalence to graphical models
will not hold in the more general cases we consider later.  Our aim will be to perform inference
to provide a characterization of $p(x_{1:T} | y_{1:S}, \theta)$ (or the posterior for some deterministic
mapping of $x_{1:T}$), typically in the form of (approximate)
samples.  Figure~\ref{fig:probprog:example-progs} shows two example programs in our example
language along with the corresponding graphical models and joint distributions they define.

Other than \sample and \observe statements, the rest of our program is by construction totally deterministic.  Therefore,
though it may contain random variables other than $x_{1:T}$, these random variables are deterministic
functions of the ``raw'' random draws $x_{1:T}$ and inputs $\theta$ and $y_{1:S}$.  We can therefore 
define the outputs of our program as $z := h(x_{1:T},y_{1:S},\theta)$ for some deterministic function $h$.
As we explained in Section~\ref{sec:prob:measure}, the change of variables means that the density function on $z$,
$p(z | y_{1:S}, \theta) $
can have a different form to the posterior implied by our program, namely $p(x_{1:T} | y_{1:S}, \theta)$.
  Though this is a serious complication in the
context of optimization (we may not in general be able to find $\argmax_z p(z|y_{1:S},\theta)$
or even evaluate $p(z | y_{1:S}, \theta)$ exactly), it
is perfectly acceptable in the context of calculating expectations as
\begin{align}
\int f(z) p(z | y_{1:S}, \theta) dz = \int f(h(x_{1:T}, y_{1:S}, \theta)) p(x_{1:T} | y_{1:S}, \theta) dx_{1:T}
\end{align}
for implicitly defined measures $dz$ and $dx_{1:T}$.  One consequence of this is that
we can express any expectations calculated by our program as expectations over $p(x_{1:T} | y_{1:S}, \theta)$
which was fully defined by the joint~\eqref{eq:probprog:simple-joint}.  Another is that, provided we are not worried
about carrying out optimizations, we do not need to explicitly worry about the implicit measures defined
by the definition of our program, other than any potential effects on the inference scheme.  In particular,
if our aim is to generate samples from $p(z | y_{1:S}, \theta)$ then we can simply generate samples then
we can simply generate samples from $p(x_{1:T} | y_{1:S}, \theta)$ and then deterministically convert each
sample to the space of $z$.  In other words, our inference engine does not need to worry about
the consequences of changes of variables if our intended output is just a sequence of samples.

An important point to note is that~\eqref{eq:probprog:simple-joint} shows that all of our \sample
and \observe statements are exchangeable, in the sense that their order can be moved around and
still define the same joint distribution, up to restrictions about all the required variables (namely the
distribution object and, if required, observation in our current setup) existing and being in scope.  
For example, if all variables remain in scope
and are not redefined, we can generally move all our \observe statements to the end of the program
without changing the joint distribution.  Therefore the program given in Figure~\ref{fig:probprog:LGSSM}
would define the same model if all the $x_t$ were sampled upfront before making any observations.
Nonetheless, the position of the \observe statements
can often be important from the perspective of the performance of the inference engine.  This exchangeability
result will carry over to the non-simplified cases.

%
%Because of the assumptions we have made for our language, the latent
%variables we wish to do inference for are statically determined as $x_{1:T} = x_1,\dots,x_T$ 
%such that the posterior of interest is $p_{\theta} (x_{1:T} | y_{1:S})$ (or some marginal of this for
%which we can still using Monte Carlo inference on the joint).
%
%Our implied target posterior
%is proportional to this joint in the standard way $p_{\theta}(x_{1:T}|y_{1:S}) \propto p_{\theta}(x_{1:T},y_{1:S})$.

\subsection{A General Probabilistic Programming Setup}
\label{sec:probprog:models:general}

Note that as $y_s$ terms can appear in the $\Xi_t$ terms, it can be the case that 
$p(x_{1:T} | \theta) \neq \prod_{t=1}^{T} p(x_t | \Xi_t)$ such that the latter does not
ex
can also enter \sample or \observe terms through $\Xi_t$ and $\Xi_s$ respectively.

\todo[inline]{Conditioning on internally sampled variables
	
	Ordering of decelerations matters
	
	Memoization
	
	Complications of maximization}
% !TEX root = ../main.tex

\section{Differing Approaches}
\label{sec:probprog:two}

Rather than being a clearly defined particular algorithm,
probabilistic programming is more of an umbrella term that covers a spectrum of 
different approaches, varying from inference toolboxes through to languages that allow
arbitrary probabilistic code to be written, even that which might not correspond to a valid
model.  Often there is a trade-off between efficiency and expressivity: the more restricted
one makes the language, the more those restrictions can be exploited to improve the efficiency
of the inference.  This leads itself two distinct philosophies when developing a system. 
Firstly one can start with a particular inference algorithm and then design a language around making it as
easy as possible to write models for which that inference algorithm is suitable.  Secondly one can start
with  a general purpose language that allows as many model as possible to be written and then try to construct
inference engines that are capable of work in such a general framework.  Both approaches 
have their merits and drawbacks, with the distinction typically coming down to the intended use.
We will now elucidate each approach more precisely.  

\subsection{Inference Driven Systems}
\label{sec:probprog:two:inf}

Though there is a plethora of bespoke inference algorithms designed for particular models, the vast majority of these are based around
a relatively small number of foundational methods such as importance sampling, sequential Monte Carlo,
Metropolis-Hastings, Gibbs sampling, message passing, and variational inference (see Chapter~\ref{chp:inf}
for an introduction to Bayesian inference).
The extensive use of these core inference approaches throughout Bayesian statistics and machine
learning means that it makes clear sense to write packages for automating them and which
make it easy for the user to define an appropriate graphical models for which the inference can be automated.
This both improves efficiency of modeling and reduces barriers to effective Bayesian modeling by reducing the
required inference expertise for users.  This inference-first philosophy is taken by a number of successful PPSs
and inference toolboxes (the distinguishing line between which can be a little blurry) such as, amongst many others,
\begin{itemize}
	\item BUGS (Bayesian inference Using Gibbs Sampling) \citep{spiegelhalter1996bugs} and its 
	extensions~\citep{lunn2000winbugs,plummer2003jags,todeschini2014biips}
	allows finite DAGs to be specified using declarative code or pictorially using a graphical user
	interface.  These are converted to a form that is suitable for inference, the exact nature of which
	depends on the implementation.
	\item Infer.Net \citep{minka_software_2010} is modeling language for defining, and automating approximate inference in,
	both DAGs and Markov random fields, using predominantly message-passing algorithms. Distributions
	are predominantly, though not exclusively, restricted to be exponential families.  Branching (i.e. \texttt{if} statements) 
	is allowed, but requires enumeration of all possible paths.
	\item LibBi \citep{murray2013bayesian} is a package for doing Bayesian inference for state-space models,
	using particle-based inference methods (see Chapter~\ref{chp:part}).  There is a strong focus on scalable
	computation, providing support for multi-core architectures and graphics processing units.
	\item PyMC3 \citep{salvatier2016probabilistic} is a python framework for carrying out MCMC and variational
	inference, using Theano~\citep{bergstra2010theano} to calculate the gradients required by some inference methods.
	\item Stan \citep{carpenter2015stan} in a PPS with interfaces to many difference languages with a
	focus on performing Hamiltonian Monte Carlo inference~\citep{duane1987hybrid,hoffman2014no}, though
	other inference methods such as variational inference are provided as well~\citep{kucukelbir2015automatic}.
	As with PyMC3, automatic differentiation is used to calculate required gradients.  The need to take
	derivatives, means that there is limited support for discrete variables or branching.
\end{itemize}
These systems do not allow users to write models that would be difficult (at least for
an expert) to code without a PPS -- in general they all can be thought of as defining a graphical model
or sometimes factor graph -- but they offer substantial utility through ease of model exposition and
automating inference.

\subsection{Universal Probabilistic Programming}
\label{sec:probprog:two:general}

As useful as these inference-driven systems are, they do not fit very well with the notion of
inverting simulators we introduced in Section~\ref{sec:probprog:inv}.  They are still closely tied
to graphical models and are more toolboxes for streamlining the Bayesian modeling process than
a means of writing models that would be problematic to define by conventional means.  Achieving
our long term ambitious aim of making general purpose systems for conducting inference of
arbitrary simulators will require us to take a somewhat different approach that instead starts
with a general-purpose language and then attempts to design inference algorithms capable of
working on arbitrary models and code.  It will be necessary for such systems to
support models where the set of random variables is dynamically typed, such that it is possible 
to write programs in which this set, and thus potentially the number of random variables, differs 
from execution to execution.  To avoid hindering the user or restricting the models which can be
defined, it will important to allow 
things such as branching, recursion, higher-order functions,
conditional existence of variables, and arbitrary black-box
deterministic functions.  Ideally, we would like to provide no restrictions on the code that the user
can write, except for eliminating programs do not define valid probability distributions, such as
those that have a non-zero probability of never terminating.  In practice catching such cases can
be hard or even impossible and so many systems actually adopt a philosophy of applying no restrictions,
such that it is perfectly possible to define invalid models.  General purpose PPSs actually bring up new
theoretical questions about what constitutes a valid probability model CITE HONGSEOK/SAM, while
even the set of valid definable models is a strict super-set of the those definable by graphical models 
for many systems CITE SOMETHING.

In the rest of this thesis, we will predominantly focus on these \emph{universal} PPSs, 
so-called because they are based on \emph{Turing complete} languages that can specify any
computable prior , 
We will regularly using the PPS Anglican~\citep{wood2014new} as a reference, an introduction
to which is provided in Section~\ref{sec:probprog:anglican}. Other prominent
universal PPSs include
\begin{itemize}
\item Church \citep{goodman_uai_2008}
\item Venture \citep{mansinghka2014venture}
\item WebPPL \citep{goodman_book_2014}
\end{itemize}

The price for the expressivity of these general purpose systems is a substantial extra 
burden on the inference engine as we will
discuss in Chapter~\ref{chp:proginf}.  In general, inference methods for such systems 
must be formulated in such a manner that they are applicable to models where the 
density function is intractable and can only be evaluated during forwards simulation of the program. 
For example, it may not be possible to know if a variable is continuous or discrete except by
running the program, while some variables will only exist conditioned on the values of others.
This required generality of the inference engine will naturally lead to a drop in performance compared to
custom written inference code, but this is often a price worth paying for generality, particularly
when considering models that would be challenging to express, let alone do inference in, using more
conventional frameworks.
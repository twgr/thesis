% !TEX root = ../main.tex

\section{Compiling Queries}
\label{sec:proginf:comp}

Non-probabilistic languages can be either compiled, whereby high-level source code
is converted to a lower level language (e.g. byte-code) before being evaluated, or interpreted,
whereby an interpreter reads the program and directly evaluates it based on the language
semantics.  Other than some earlier systems that were based around guess-and-check 
type strategies, PPLs by construction tend to require some form compilation due to the fact that
the original code will typically run many times and because they tend to be integrated
with an existing language, such that they generally require compilation to the host language
before anything can be run.  Rather than just being a technical hurdle, compilation is also 
often an important tool in the ability of probabilistic programming systems to perform
effective inference as it can often be used to establish helpful salient features in the model,
such as dependency structures, variable types, and features of the model that can guide
which inference algorithm is likely to be most successful or even elements of the model
that can be calculated analytically.  Hakaru~\citep{narayanan2016probabilistic} is a good
example of what can be achieved in probabilistic programming through compilation alone,
as it, for example, uses the computer algebra system Maple to automatically simplify models
when possible.

\subsection{What do we want to Achieve through Compilation?}
\label{sec:proginf:comp:want}

So what are the aims of our PPL compiler?  First and foremost, we need to
perform a source-to-source transformation of the program to
produce some representation in the conventional programming language in which
the inference engine is written so that execution of program is possible and inference
can be carried out.  As a simple example, consider the case of an Anglican program where
we wish only to run importance sampling using the generative model as a proposal 
(i.e. the query with all the\observe statements removed).  Here we need only produce a Clojure program
that samples directly from the generative model and accumulates weights from the
\observe terms as a side-effect.  We can then rerun this program arbitrarily many
times, each returning a sampled output and accompanying weight, to produce a sequence
of importance samples which can then be used to make consistent Monte Carlo estimations.
Such a strategy would be na\"{i}ve though as we would be able to use our Clojure program
for little else than importance sampling with this particular proposal.  Clearly we want
to compile to a more general purpose representation that permits a wider array inference algorithms
and proposals, and which ideally allows features of the model, such as its dependency
structure, to be exploited.

One desirable feature of our compiled representation is an ability to make partial
program evaluations.  This will allow us to re-evaluate elements of a the program in
isolation in a Gibbs sampling style fashion and it will allow us to interrupt the execution
of the program, giving us the ability to add in things such as the resampling step in SMC.
Supporting partial evaluation can also allow us to avoid gratituitious re-executions of
elements of the program whose result is already known by using databases to store and
recall the effect of previous executions.

Another desirable feature is to avoid being restricted to only directly sampling from \sample
statements.  We may wish to sample from a different distribution as a proposal or to evaluate
the probability density of the \sample statement producing a particular output, e.g. as part
of an acceptance ratio calculation for an MCMC scheme.  The key requirement for both of these
is the knowledge of density function associated with the \sample statement, rather than
just access to a black-box sampling scheme.  This is the motivation behind why we defined
the syntax of \sample as taking as input a distribution object which encodes both the density
function and means of sampling from that density.

It will also often be helpful for our compiler to delineate between probabilistic and deterministic
elements of the code.  Other than potentially trying to make efficiency gain by avoiding
repeated computation, we know that any deterministic parts of our code (namely code in between
consecutive \sample and/or \observe statements) can be safely run without the need
to worry about the implications this has on the inference.  The execution of these segments of
the program in isolation can thus be per the language being compiled to, without needing to
worry about the probabilistic semantics.  On the other hand, we will generally desire the identification
of \emph{checkpoints} for positions in the program where \sample or \observe statements are
made so that the behavior of the program at these points can be handed over to the inference
engine.

The compiler will in general also play an important role in the interaction between the PPL
as its integrated language on the behalf of the user.  Any PPL that does not provide produce
outputs in a helpful form that can be manipulated by the user will be somewhat impractical,
while for general purpose systems it is essential to allow the user to link in external 
deterministic code packages specific their particular problem.

This example list of desirable features for our compiler is far from exhaustive.
In particular, there will be many inference specific features required for some
systems, such as the need to have a representation of the derivations, e.g. calculated
through automatic differentiation~\cite{baydin2015automatic}, to carry out 
Hamiltonian Monte Carlo inference~\cite{carpenter2015stan}
or common variational inference methods~\citep{kucukelbir2015automatic}.  As we
said earlier, we will often also want our compilation to, when possible,
pick out salient features of the model, carry out simplifications, and even automatically 
establish the most suitable inference algorithm for a particular problem.  The compiler
is an integral part of any PPL and there is no one best approach for all situations. 
One of the key distinguishing features between different PPLs is how they approach
this compilation problem, and different design choices inevitably lead to systems
geared towards different models or inference algorithms. 

\subsection{Compilation of Anglican Queries}
\label{sec:proginf:comp:ang}

As it is inevitably infeasible to detail the inner workings of a PPL compiler in a general
manner, we now provide a more in depth introduction to compilation method
employed by Anglican.  As we explain in Section~\ref{sec:probprog:anglican:models},
Anglican programs, or queries, are compiled use the macro \query which provides a
Clojure function that can be passed to one of the provided inference algorithms.
The key element of this compilation for providing the desirable properties discussed
in the last section and a convenient interface for the inference algorithms, is that
Anglican compiles queries to \emph{continuation passing style} (CPS)~\citep{appel1989continuation}
Clojure functions.  At a high level, a continuation is a function that represents the rest of the
program.  CPS is a style of functional programming that uses a series of continuations
to represent the program through a series of nested function calls, where the program
is run by evaluating each function and then passing the output to the current continuation
which invokes the rest of the program to be run.  This is perhaps easiest to see through
example.  Consider the simple function \clj{+}, which in Clojure has syntax \clj{(+ a b)}.  The
CPS transformed version of \clj{+}, which we will call \clj{+&} takes an extra input
of the continuation \clj{C} and invokes it after evaluation such that we have
\clj{(defn +& [a b C] (C (+ a b)))}.  More generally, for any simple function \clj{f}, we have
that its the CPS transformation is \clj{(defn f& [args C] (C (f args)))}.  Things can get a little
tricky for not standard functions, e.g. \clj{let} blocks or loops, but the high level idea is the same.
In CPS style code, functions never return (until the tail call) and every function takes an
extra input corresponding to the continuation.  This would be a somewhat awkward method for
writing programs, as the whole program must be written as a single nested function, but
it can be a very useful form to compile to as the execution of the program becomes
exceptionally simple and just involves evaluating functions and passing the output to the
next continuation -- it explicitly linearizes the computation.  For our purposes, the fact
that the computation left to do at the any point in the program is explicitly encoded by
continuation will be particularly useful as it will make the problem of partial program
evaluation relatively simple.  We will also be able to explicitly delineate the probabilistic
components of our program, as they simple correspond to the points in our nested call
structure that invoke a \sample or \observe statements -- the rest of the program is
deterministic and we can run it as normal

We note that WebPPL also does a CPS style transformation, namely to a purely functional
subset of Javascript.
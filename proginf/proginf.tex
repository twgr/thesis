% !TEX root = ../main.tex

\chapter{General Purpose Inference for Probabilistic Programs}
\label{chp:proginf}

In Chapter~\ref{chp:probprog} we showed how probabilistic programming systems (PPSs) provide
an expressive framework for specifying probabilistic models.  We now consider the other major component
for PPSs: automating inference for any model the user is allowed to specify using general-purpose
inference engines.  For most PPSs this requires two things --  the inference engine itself and
either an interpreter controlling the probabilistic semantics or a compiler to convert the \emph{query} to a
suitable form for input to the inference engine.   Our focus will be on the compiled case. 

For the inference
driven systems discussed in Section~\ref{sec:probprog:two:inf}, the inference engine typically comprises of
a standard Bayesian inference method for graphical models such as those discussed in Chapters~\ref{chp:bayes}
and~\ref{chp:part}.  Developing these in a way to robustly work for a wide range of problems typically
requires careful engineering and algorithmic innovation -- e.g. because many inference methods require the definition of
a proposal, upon which performance can critically depend -- but does not generally require the development 
of approaches distinct to those used outside a probabilistic programming context.  In these 
systems, the inference algorithm(s) is/are usually chosen first, with the language and its restrictions built around it.
Therefore the challenges of the designing the system are generally rooted in generalizing and increasing the robustness of the 
specific inference method(s) used.  Similarly, the language itself and associated compiler is generally built
around providing the easiest representation to work with the target model class, while the fact that most of
these such systems to not support higher order functions usually substantially simplifies the compilation
process.

Because the design of these systems is very much driven by the particular inference algorithm used, it
beyond the scope of this thesis to do the associated literature justice.  Our focus will instead mostly be on 
conducting inference for universal PPSs, though some of what we introduce will apply to both.
Unfortunately, we will find that there are very few (known) inference
methods which can actually cope with the most general possible models as we introduced in
Section~\ref{sec:probprog:models:general}, all of which suffer particularly badly from the curse of
dimensionality.  Consequently, it will be necessary to make certain (mostly very small) concessions in 
generality to achieve any reasonable performance on non-toy models.  We will focus on conducting
inference in Anglican~\citep{wood2014new,tolpin2016design} to give us a basis for explanation, but much of
what we discuss will still be relevant to other universal systems, in particular, those which
are also built around \sample-\observe syntaxes, such as VentureScript~\citep{mansinghka2014venture}, 
WebPPL~\citep{goodman_book_2014}, and Probabilistic C~\citep{paige2014compilation}.

% !TEX root = ../main.tex

\section{A High-Level Introduction to General Purpose Inference}
\label{sec:proginf:high}

Before getting into the nitty-gritty of designing a compiler and inference engine for a universal PPL, we
first consider at a high-level how we might hope to do inference in such systems.  The simplest
thing we could do is self normalized importance sampling (SNIS) using the generative model specified by the \sample
statements as a proposal.  This strategy is sometimes known as \emph{likelihood weighting} in the probabilistic
programming literature and simply involves directly sampling from the forward model an accumulating
weights from the \observe conditioning statements in a sequential importance sampling fashion (see 
Section~\ref{sec:part:smc:sis}).
In Section~\ref{sec:probprog:models:general}, we specified with~\eqref{eq:probprog:universal-cond}
the unnormalized conditional distribution of the program using execution traces generated
by the program.  If we sample from the generative model then it is clear that all samples will satisfy
$\mB(x_{1:n_x},\lambda)=1$ while  the support of our proposal will clearly cover that of
the posterior provided the conditioning density terms are always finite.  This is also a proposal
we can always sample from because our \observe terms will only affect the probability of a trace, not
the variables it generates.  Using the notation from Section~\ref{sec:probprog:models:general} 
this approach leads to the following SNIS characterization of the posterior
\begin{align}
p(x_{1:n_x} | \lambda) &\approx \hat{p}(x_{1:n_x} | \lambda) := \sum_{n=1}^{N} \nw_n \delta_{\hxnx^n} (x_{1:n_x})
\quad \text{where}  \\
\hxnx^n &\sim \prod_{j=1}^{n_x} f_{a_j}(x_j | \eta_j); \quad \nw_n = \frac{w_n}{\sum_{n=1}^{N} w_n}; \quad
w_n = \prod_{k=1}^{\hat{n}_{y}^n} g_{\hat{b}_{k}^n}(\hat{y}_{k}^n | \hat{\psi}_{k}^n);
\end{align}
$\delta_{\hxnx^n} (\cdot)$ is a delta function centered at $\hxnx^n$; $\eta_j$ and $a_j$ are deterministic functions of $\hxnx^n$ 
and $\lambda$; and $\hat{n}_{y}^n$, $\hat{b}_{k}^n$, $\hat{y}_{k}^n$, and $\hat{\psi}_{k}^n$ are random variables
that are deterministic functions of $\hxnx^n$ and $\lambda$.  Here we are sampling $\hxnx^n$ from
the generative model defined by the program, which is equivalent to a running the forward program
ignoring all of the \observe statements.  Our unnormalized likelihood weights $w_n$ correspond to
product of all the probabilities accumulated from the \observe terms.  We remind the reader that each 
$x_j$ corresponds to the direct output of a sample statement, not an explicit variable in the program, 
with the program variables and outputs being deterministic functions of $x_{1:n_x}$ (see 
Section~\ref{sec:probprog:anglican:models}). In addition to providing a 
characterization of the conditional distribution through $\hat{p}(x_{1:n_x} | \lambda)$, we can also
use our samples to construct consistent estimates from the outputs of program $\Omega$, remembering
that these are a deterministic function of $\xnx$ and so an expectation over $\Omega$ can be
represented as an expectation over $\xnx$.
This convergence follows in the standard way for SNIS provided our program terminates with probability $1$.

In practice, our importance sampling inference engine will not sample a full $\hxnx^n$ before 
evaluating its weight in one go -- it will use
sequential importance sampling strategy whereby we sample directly from any \sample statement as we encounter it
and accumulate weights from \observe statements as we go (see Figure~\ref{fig:probprog:poisson} in
Section~\ref{sec:probprog:models:general}).
Another way of viewing this is that we will repeatedly run our query as if it were an ordinary
program, except for the fact that we accumulate a weight as a side effect that is then returned with
the produced sample.  Such a guess an check strategy will obviously be ineffective except in very low
dimensions.  However, it will both be a go-to strategy for the most extreme possible problems that
fail to satisfy any of the required assumptions for more advanced methods and a basis for more complex 
inference strategies.  For example, as we will shown in Section~\ref{sec:proginf:str}, we can
convert this to an SMC strategy, provided $n_y$ is fixed, by running a number of separate executions
in parallel and performing resampling at each of the \observe statements.  Alternatively, we can
develop component-wise MCMC strategies by proposing changes to individual $x_j$ and then
re-evaluating the weights.  However, doing so will require our inference strategy to have more
control that simply running the program forwards in its entirety and this will require both a compilation
and an interface for inference, which will be the focus of the next two sections.
% !TEX root = ../main.tex

\section{Compiling Queries}
\label{sec:proginf:comp}

Non-probabilistic languages can be either compiled, whereby high-level source code
is converted to a lower level language (e.g. byte-code) before being evaluated, or interpreted,
whereby an interpreter reads the program and directly evaluates it based on the language
semantics.  Other than some earlier systems that were based around guess-and-check 
type strategies, PPLs by construction tend to require some form compilation due to the fact that
the original code will typically run many times and because they tend to be integrated
with an existing language, such that they generally require compilation to the host language
before anything can be run.  Rather than just being a technical hurdle, compilation is also 
often an important tool in the ability of probabilistic programming systems to perform
effective inference as it can often be used to establish helpful salient features in the model,
such as dependency structures, variable types, and features of the model that can guide
which inference algorithm is likely to be most successful or even elements of the model
that can be calculated analytically.  Hakaru~\citep{narayanan2016probabilistic} is a good
example of what can be achieved in probabilistic programming through compilation alone,
as it, for example, uses the computer algebra system Maple to automatically simplify models
when possible.

\subsection{What do we want to Achieve through Compilation?}
\label{sec:proginf:comp:want}

So what are the aims of our PPL compiler?  First and foremost, we need to
perform a source-to-source transformation of the program to
produce some representation in the conventional programming language in which
the inference engine is written so that execution of the program is possible and inference
can be carried out.  As a simple example, consider the case of an Anglican program where
we wish only to run importance sampling using the generative model as a proposal 
(i.e. the query with all the\observe statements removed).  Here we need only produce a Clojure program
that samples directly from the generative model and accumulates weights from the
\observe terms as a side-effect.  We can then rerun this program arbitrarily many
times, each returning a sampled output and accompanying weight, to produce a sequence
of importance samples which can then be used to make consistent Monte Carlo estimations.
Such a strategy would be na\"{i}ve though as we would be able to use our Clojure program
for little else than importance sampling with this particular proposal.  Clearly, we want
to compile to a more general purpose representation that permits a wider array inference algorithms
and proposals, and which ideally allows features of the model, such as its dependency
structure, to be exploited.

One desirable feature of our compiled representation is an ability to make partial
program evaluations.  This will allow us to re-evaluate elements of the program in
isolation in a Gibbs sampling style fashion and it will allow us to interrupt the execution
of the program, giving us the ability to add in things such as the resampling step in SMC.
Supporting partial evaluation can also allow us to avoid gratuitous re-executions of
elements of the program whose result is already known by using databases to store and
recall the effect of previous executions.

Another desirable feature is to avoid being restricted to only directly sampling from \sample
statements.  We may wish to sample from a different distribution as a proposal or to evaluate
the probability density of the \sample statement producing a particular output, e.g. as part
of an acceptance ratio calculation for an MCMC scheme.  The key requirement for both of these
is the knowledge of density function associated with the \sample statement, rather than
just access to a black-box sampling scheme.  This is the motivation behind why we defined
the syntax of \sample as taking as input a distribution object which encodes both the density
function and means of sampling from that density.

It will also often be helpful for our compiler to delineate between probabilistic and deterministic
elements of the code.  Other than potentially trying to make efficiency gain by avoiding
repeated computation, we know that any deterministic parts of our code (namely code in between
consecutive \sample and/or \observe statements) can be safely run without the need
to worry about the implications this has on the inference.  The execution of these segments of
the program in isolation can thus be per the language being compiled to, without needing to
worry about the probabilistic semantics.  On the other hand, we will generally desire the identification
of \emph{checkpoints} for positions in the program where \sample or \observe statements are
made so that the behavior of the program at these points can be handed over to the inference
engine.

The compiler will in general also play an important role in the interaction between the PPL
as its integrated language on the behalf of the user.  Any PPL that does not provide produce
outputs in a helpful form that can be manipulated by the user will be somewhat impractical,
while for general purpose systems it is essential to allow the user to link in external 
deterministic code packages specific their particular problem.

This example list of desirable features for our compiler is far from exhaustive.
In particular, there will be many inference specific features required for some
systems, such as the need to have a representation of the derivations, e.g. calculated
through automatic differentiation~\cite{baydin2015automatic}, to carry out 
Hamiltonian Monte Carlo inference~\cite{carpenter2015stan}
or common variational inference methods~\citep{kucukelbir2015automatic}.  As we
said earlier, we will often also want our compilation to, when possible,
pick out salient features of the model, carry out simplifications, and even automatically 
establish the most suitable inference algorithm for a particular problem.  The compiler
is an integral part of any PPL and there is no one best approach for all situations. 
One of the key distinguishing features between different PPLs is how they approach
this compilation problem, with different design choices inevitably lead to systems
geared towards different models or inference algorithms. 

\subsection{Compilation of Anglican Queries}
\label{sec:proginf:comp:ang}

As it is inevitably infeasible to detail the inner workings of a PPL compiler in a general
manner, we now provide a more in depth introduction to compilation method
employed by Anglican.  Our introduction is inevitably not exhaustive, focusing more on
intuition than being exactly true to the implementation details; we refer the reader
to~\citep{tolpin2016design} for a more complete and rigorous introduction.

As we explain in Section~\ref{sec:probprog:anglican:models},
Anglican programs, or queries, are compiled using the macro \query which provides a
Clojure function that can be passed to one of the provided inference algorithms.
The key element of this compilation for providing the desirable properties discussed
in the last section and a convenient interface for the inference algorithms is that
Anglican compiles queries to \emph{continuation passing style} (CPS)~\citep{appel1989continuation}
Clojure functions.\footnote{WebPPL also does a CPS style transformation, 
	namely to a purely functional subset of Javascript.}
At a high level, a continuation is a function that represents the rest of the
program.  CPS is a style of functional programming that uses a series of continuations
to represent the program through a series of nested function calls, where the program
is run by evaluating each function and then passing the output to the continuation
which invokes the rest of the program.  This is perhaps easiest to see through
example.  Consider the simple function \clj{+}, which in Clojure has syntax \clj{(+ a b)}.  The
CPS transformed version of \clj{+}, which we will call \clj{+&} takes an extra input
of the continuation $\mP$ and invokes it after evaluation such that we have
\clj{(defn +& [a b } ~$\mP$\clj{] (}$\mP$ \clj{(+ a b)))}.  More generally, for any simple function \clj{f}, we have
that its the CPS transformation is \clj{(defn f& [args} ~$\mP$\clj{] (}$\mP$ \clj{(f args)))}.\footnote{Note that
	in practice, the continuation is usually set to be the first argument in order to provide support
	for functions with a variable number of inputs.}
  We 
will use this notation of adding an \clj{&} to an expression name to denote its CPS transformation throughout.
To give a more detailed example, the CPS transform of the program \clj{(max 6 (* 4 (+ 2 3)))} would be
\begin{lstlisting}[basicstyle=\ttfamily\small,frame=none]
  (fn [$\mP$] (+& 2 3 (fn [x] (*& 4 x (fn [y] (max& 6 y $\mP$)))))
\end{lstlisting}\vspace{-8pt}
where \clj{*&} and
\clj{max&} are analogous to \clj{+&} defined as before.
Here our first continuation is \clj{(fn [x] (*& 4 x (fn [y] (max& 6 y))))} and our second continuation 
is \clj{(fn [y] (max& 6 y))}.  Note that the CPS transformed code is itself a function because
it itself takes a continuation.  

Things are a little
trickier for general expression that are neither literals nor simple first order functions, for example,
binding forms like \clj{let}, Anglican special forms like \sample, and branching statements like \clj{if}.
For these expressions, the high-level idea is the same, but the CPS transformation is, unfortunately, 
expression specific and must, in general, be implemented on a case-by-case basis.  
For example, one can CPS transform \clj{let} by going from
\clj{(let [x (foo1 1) y (foo2 2)] (foo3 x y))} to
 \begin{lstlisting}[basicstyle=\ttfamily\small,frame=none]
 (fn [$\mP$] (foo1& 1 (fn [x] (foo2& 2 (fn [y] (foo3& x y $\mP$)))))).
 \end{lstlisting}\vspace{-8pt}
noting that within the \clj{(fn [x] .)} closure, \clj{x} is bound to the input of the 
function, giving behavior synonymous to the original
\clj{let} block.  
Another special case of particular note is \clj{loop}-\clj{recur} blocks.  These can be CPS
transformed by explicitly redefining them as a self-recursive function which can then
be transformed in the normal way.  For example,
\clj{(loop [x 10] (if (> x 1) (recur (- x 2)) x))}
becomes
\begin{lstlisting}[basicstyle=\ttfamily\small,frame=none]
 (fn [$\mP$] ((fn foo [x] (if (> x 1) (foo (- x 2)) ($\mP$ x))) 10))
 \end{lstlisting}\vspace{-8pt}
where we have exploited the fact that \clj{fn} allows the function to be named (in this case to
\clj{foo}) so that it can call itself.
 
In CPS style code, functions never return (until the final tail call) and every function takes an
extra input corresponding to the continuation.  This would be a somewhat awkward method for
writing programs, as the whole program must be written as a single nested function.  However,
it can be a very useful form to compile to as the execution of the program becomes
exceptionally simple and just involves evaluating functions and passing the output to the
next continuation -- it explicitly linearizes the computation.  For our purposes, having access
to functions representing the rest of the program in the form of continuations will be
particularly useful as it will allow for partial program evaluations.  It will also be convenient
for adding checkpoints at particular points in the program -- namely at the \sample and
\observe calls -- where control is handed over to the inference algorithm.

Compilation of an Anglican query, triggered through the \query macro, is done in recursively
in a top down manner.  The key function in doing this, called \clj{cps-of-expression},
dispatches to the individual CPS transformations by matching types of expression, or, if necessary,
directly by key-word.  Individual CPS transformations then recursively call \clj{cps-of-expression}
or themselves if necessary, e.g. if the top level call is a \clj{let} block, until the full query is
is transformed.  

The individual CPS transformations in Anglican are marginally more complicated than the
framework we have laid out thus far.  This is because it is necessary to not just run the
program, but also track its state from a probabilistic perspective, storing things such
as sample weights and other probabilistic side-effects.  To deal with this, the transformed Anglican
continuations take as input an \emph{internal state}, which we will denote as \angstate,
in additional to the computed value.
Thus, for example, the transformation for a simple function becomes\footnote{As before, the original arguments
	are actually last in the argument list for the true implementation.}
\begin{lstlisting}[basicstyle=\ttfamily\small,frame=none]
  (defn f& [args $\mP~\dollar$state] ($\mP$ (f args) $\dollar$state)).
\end{lstlisting}\vspace{-8pt}
More generally, we will simply pass on \angstate unchanged except for the Anglican
special forms which can use and/or manipulate the internal state.  \angstate itself
is defined to be a hash map, initialized as follows
\begin{lstlisting}[basicstyle=\ttfamily\small,frame=none]
  (def initial-state {:log-weight 0.0 :result nil ...})
\end{lstlisting}\vspace{-8pt}
where \clj{...} includes some fields we will not directly consider at the moment and some algorithm
specific fields (e.g. information required to the retained particle in PMCMC methods).
Here the field \clj{:log-weight} allows the program to be assigned a
(relative) weight as accumulated by, for example, \observe statements or importance weights
when sampling from a different distribution than that provided to the \sample statement.
The return for each sample in our inference will be the \angstate, hence the inclusion
of the \clj{:result} field which is set to the output of the query at its tail call.

One of the key components of the Anglican compilation is the transformation applied to the
the special forms and in particular the probabilistic forms \sample and \observe.  These
are respective transformed to the Clojure record constructors \samplecps and \observecps 
which have the call syntaxes of
\clj{(->sample id dist} ~$\mP$ ~\angstate\clj{)} and \clj{(->observe id dist value} ~$\mP$ ~\angstate\clj{)},
where \clj{id} is a unique checkpoint identifier (e.g. the $\{1,\dots,n_s\}$ and $\{1,\dots,n_o\}$ identifiers
we used in~\ref{sec:probprog:models:general}) set at compilation time, and  \clj{dist} and \clj{value}
are the original inputs to the \sample and \observe statements.  At a high level, one can
think of a Clojure record as defining a new class type (here {\small \texttt{trap.sample}} and {\small \texttt{trap.observe}}
respectively) with given fields (here \clj{:id}, \clj{:dist} etc).  Our constructors thus create
an object of the given type with appropriately set fields.  The significance of this is that
it allows definition of a multimethod, which we call \checkpoint, to provide a runtime polymorphism
(i.e. single interface) for dispatching depending on the checkpoint
type and the inference algorithm.  In other words, we will have one function, \checkpoint, whose
behavior can be redefined for different checkpoint types and inference types.  There are many
consequences of this.  Firstly, our compiler does not need to be inference algorithm specific
because we can use \checkpoint to distribute the behavior to the required inference algorithm
at run time.  Secondly, it creates an abstraction barrier for writing inference algorithms -- implementing
an inference algorithm now only requires us to implement, along with a top level function \anginfer that
we will discuss later, new methods describing the behavior of \checkpoint at \sample and \observe
checkpoints.  Furthermore, these can be inherited from other inference algorithms, for example,
the \observe checkpoints for particle Gibbs are inherited from SMC in Anglican's implementation.

A slightly more subtle consequence of compiling to a constructor with typed output is that
we will use this to catch the checkpoints themselves.  Once compiled, individual instances of
a program in the Anglican inference engines are run using the \clj{exec} function.  The role of
\clj{exec} is to run the program until it reaches a checkpoint that requires control to be transferred
to the top level inference function of a particular method, i.e. the \anginfer function, such as when
a particular trace has finished running or when interaction is required between different samples, e.g. 
the resampling step in SMC.  In Anglican this is achieved using \emph{trampolining}.  In functional
programming languages, trampolining is a process of looping execution where if an iteration of
a loop returns a function, that function is immediately evaluated without passing forward any arguments,
with this process continuing until a non-functional output is returned.  The primary use of trampolining
is for managing stack sizes by constructing a nested call structure of \emph{thunks} (i.e. functions which
require no input arguments) that when called by the trampolining function (called \clj{trampoline} in Clojure)
invoked the full set of nested calls without requiring a stack to be constructed or stored, as would be necessary
for an ordinary nested function structure.  In the context of our CPS transformations, we can do this by simply
wrapping each call with an anonymous function, namely converting \clj{(foo args }~$\mP$ ~\angstate\clj{)}
to \clj{(fn [] (foo args }~$\mP$ ~\angstate\clj{))}, which will be carried out for all continuations
except the checkpoints.  Though a large part of the motivation for doing this in
Anglican is similarly to maintain the stack size, it is also highly useful for using the checkpoints to trigger
control to be transferred to the \anginfer function.  To be precise, the \clj{exec} function is defined as
\begin{lstlisting}[basicstyle=\ttfamily\small,frame=none]
  (defn exec [algorithm prog value $\dollar$state]
    (loop [step (trampoline prog value $\dollar$state)]
      (let [next (checkpoint algorithm step)]
       (if (fn? next) (recur (trampoline next)) next))))
\end{lstlisting}\vspace{-8pt}
where \clj{algorithm} specifies the inference algorithm; \clj{prog} is the program to call, comprising of either the full
CPS compiled program output by \query or a continuation representing the rest of the program; 
\clj{value} is the input required by \clj{prog}, comprising of either the original program inputs or the value
passed to the continuation; and \angstate is the program state as before.  Here we see that \clj{exec}
first creates the variable \clj{step} by making a trampolined call to \clj{prog}.  As all continuations are now
represented by anonymous functions with no inputs except for our checkpoints, this will run until it
hits one of those checkpoints, returning the corresponding constructed checkpoint object.  In other words,
this causes the program to run until it reaches a \sample or \observe statements, or it reaches the end of the
program.  This is a highly desirable behavior, as it means the deterministic elements of our program
are run as normal.  Once returned, our multimethod \checkpoint is called using \clj{algorithm} and
\clj{step}, which distributes to the appropriate \checkpoint implementation based on the value of
\clj{algorithm} and the type of \clj{step} (e.g. calling to the \sample checkpoint method of an algorithm 
if it is of type  {\small \texttt{trap.sample}}).  Note that individual \checkpoint implementations
are based on using \clj{step} as the only input of note, which, as we remember, is a Clojure record
containing fields providing information on the distribution object, the state, etc.
Invoking the appropriate checkpoint method will cause
inference algorithm specific behavior, e.g. updating a weight in \angstate, and provide a return value
\clj{next}.  If \clj{next} is a function, the loop
is recurred, with \clj{step} now replaced by the output of a trampolined call of \clj{next}.  If \clj{next} is
not a function, the \clj{exec} function terminates, returning \clj{next}.  The reason for this looping is
that it allows different inference algorithms to specify different occurrences for when control needs to
be passed back to the \anginfer function in an algorithm specific way, by defining the checkpoint methods
to either return a checkpoint object (for which the \clj{exec} call will terminate) or the continuation
wrapped in a thunk (for which execution will continue).  For example, when running SMC,
action distinct to running the program forwards only needs to be taken at the \observe statement 
checkpoints.  As different traces may have a different number of \sample statements between observations,
it is therefore convenient to a have a function that does not terminate at the \sample checkpoints, returning
only once an \observe or the end of the program is reached.  On the other hand, for MCMC samplers, we
will need to externally control the sampling behavior and so it may be necessary to return control at each
\sample statement.

We have now demonstrated how an Anglican query is compiled, a summarizing example for which
is provided in Figure XX\todo{Add compilation example}.  We have also laid out a framework for
writing inference algorithms in Anglican: we need to implement methods of the \checkpoint multimethod
and a top level inference function \anginfer.  For the former, there are three possible \checkpoint methods
we might have to reimplement for a particular inference algorithm, namely ones to dictate the behavior
of \sample and \observe, and a ``\clj{result}'' checkpoint for controlling behavior at the termination of
the query (for most methods this will simply provide the return value).  In the rest of this Chapter we will
explain how this abstraction allows us to write inference algorithms suitable for arbitrary programs.
% !TEX root = ../main.tex

\section{Writing Inference Algorithms in Anglican}
\label{sec:proginf:inf}

Taking stock, what has our compilation
given access to and what do we not have access to?  The key upshots of our compilation are
that we have access to a continuation representing the rest of the program at any point
duing the execution and 
the interactions between the inference back-end and the query code happens only through
the \sample and \observe statements and the final returned \angstate of each execution.
Consequently, we have means of running the program
forwards and controlling the behavior at each \sample and \observe statement.  We can
stop or restart the program at any one of our checkpoints and chose whether to do so
adaptively, for example, by killing of certain evaluations or duplicating others.  Once
a program is run, we can test the effect of changing one or more of our sampled variables
in an MCMC fashion, though it may be difficult to statically determine an appropriate global proposal
as we can only establish the structure of the target through evaluation.  Although we might be
able to do so with other code analysis, we do not directly have access to any information
about independence relationships or even knowledge under what conditions a certain
variable will exist.  
%We can transfer information from one sampling iteration to the next
%using a database, giving us the potential to avoid repeated computation or adaptively learn proposals.
In theory, we can query the unnormalized density $\gamma(x_{1:n_x},\lambda)$ of any fixed set of 
parameter values $x_{1:n_x}$, given fixed inputs $\lambda$, by effectively making
\sample operate as an \observe statement.  However, it may be difficult, or even impossible, to find
values of $x_{1:n_x}$ that have non-zero probability unless they are generated by running
the program forwards.  In particular, it might be arbitrarily hard to find a trajectory that satisfies
$\mB(x_{1:n_x},\lambda)=1$, or construct an external proposal more generally,
 without forward sampling from the program.  
%It is liable
%to be even more difficult to externally construct a valid proposal to generate consistent estimates
%through, for example, importance sampling based schemes or that will lead to ergodic Markov chains,
%particularly if the program contains a potentially unbounded number of variables.

These availabilities and restrictions suggest that Anglican is, and universal PPSs more generally are,
best suited to \emph{evaluation based} inference methods.  At a high level, we can think of such methods
as acting like a controller wrapping around the standard program
execution (or potentially multiple simultaneous executions).  When the program hits a checkpoint,
control is transferred to the inference engine, providing information about the current state through
\angstate and about the current checkpoint command (i.e. checkpoint type, distribution object, etc).
The inference engine then decides exactly how the checkpoint command should be executed (e.g. how to
sample the new variable for a \sample statement), updates the \angstate appropriately, and decides if
and how the execution should continue.  Once an execution is complete (i.e. the return checkpoint
is reached), the inference controller decides what to run next, which could, for example, be an entirely new independent
execution or an MCMC update to the current execution.  Two important things the inference controller
does not do is alter the deterministic elements of the program (which we can think of as being single
deterministic functions from one checkpoint to the next)
or do further analysis on the program source code itself.  

There are four functions required to define an Anglican inference algorithm more generally: a top level 
\anginfer function and three \checkpoint implementations
for types \smalltt{trap.sample}, \smalltt{trap.observe}, and \smalltt{trap.return} .  The \anginfer
implementation must always be written anew, but the \checkpoint implementations can be inherited from
another inference algorithm of default behavior.  Other than some top level book keeping, the user
inference command \doquery directly calls the \anginfer function of the specified inference algorithm, providing
as input the CPS-transformed query, the fixed query inputs, and inference algorithm specific options.
The \anginfer function is required to then return a lazy infinite sequence of samples, each in the form of
a \angstate.  To carry out inference, \anginfer uses the \clj{exec} function introduced earlier
to run (partial) executions of the program, which will run until they hit a checkpoint that returns something
other than a thunk (see below).  

All of the \checkpoint implementations
take as their only input of note an instance of the relevant Clojure record, i.e. a
 \smalltt{trap.sample}, \smalltt{trap.observe}, or \smalltt{trap.result}.  
 Here \smalltt{trap.sample} has
fields \clj{:id} which is a unique identifier for the corresponding lexical \sample statement, \clj{:dist}
which is the distribution object input, \clj{:cont} which is the continuation function $\mP$, and \clj{:state}
which is the current instance of \angstate.  Meanwhile, \smalltt{trap.observe} has the same fields and additionally
a \clj{:value} field giving the observation and \smalltt{trap.result} has only the \clj{:state} field.  Each
\checkpoint method should either return a thunk, in which case the \clj{exec} function will resume running,
or an instance of the relevant Clojure record, in which case \clj{exec} terminates and returns control to the \anginfer
function (or some intermediate function used by the \anginfer function).  For example, in SMC methods
then we will want to return control to \anginfer at an \observe point to do the resampling, while all methods
return control for a \clj{return} checkpoint.
Default behaviors are provided for each checkpoint that dictate their behavior unless overwritten by
a particular inference algorithm.  The default behavior of \sample is to sample from the distribution
object and return a thunk that calls the continuation with this sample and \angstate as input arguments.
The default behavior of \observe is to update the \clj{:log-weight} field in \angstate with the current
observation term and then returns a thunk which calls the continuation with inputs \clj{nil} and \angstate.  The default behavior
of \clj{return} simply returns the \smalltt{trap.result} object.
% !TEX root = ../main.tex

\section{Inference Strategies}
\label{sec:proginf:str}

%Before jumping into describing some specific inference algorithms for
%Anglican, we first link our compilation back to the execution-based 
%definition of the conditional distribution specified by a program given in 
%Section~\ref{sec:probprog:models:general} and consider what classes of inference
%algorithms might be able to operate in such settings. 

As we previously alluded to, the simplest inference strategy we can carry out is importance
sampling.  This is in fact so simple in our current framework that it uses the default behavior
of all the checkpoints, while the \anginfer function only involves constructing a lazy infinite
sequence of the output from independent calls of \clj{exec} on the full program.
Keeping the default behavior for the \sample implicitly means that our
inference will use a bootstrap proposal (i.e. the generative model is taken as the proposal).
Though not technically required, this is still a highly convenient choice of proposal as
amongst other things, this ensures that we can always sample from the proposal and
that the proposal is valid in terms of its tail behavior (presuming the conditional probabilities
are bounded).

\subsection{MCMC Strategies}
\label{sec:proginf:str:lmh}

We have already explained why it might be difficult to construct a global MH proposal for
our program due to the difficulties in varying dimensionality and unknown variable supports.
One solution to this is to try and use compilation to establish this support so that
a valid proposal can be specified.  However, doing this in a general manner is somewhat
challenging and remains an open problem in the field.  Another more immediately viable
approach is to use a proposal that looks to update particular $x_j$ in the trace in a
component-wise MH manner (see Section~\ref{sec:inf:foundation:gibbs}), potentially rerunning
the rest of the program after that choice if necessary.  In the context of PPSs such
approaches are generally known as either single-site MH, random database sampling, or lightweight MH (LMH), and were
originally suggested by~\citep{wingate2011lightweight}.\footnote{The MH acceptance ratio in the original version of this work
	is incorrect so we point the reader to the following updated version
	\url{https://stuhlmueller.org/papers/lightweight-mcmc-aistats2011.pdf}.} There are two key factors
that make such approaches viable.  Firstly, if we update a term $x_j$ in our trace, there is
no need to update any of the factors in our trace probability that occur before the sampling
$x_j$, while we may be able to avoid evaluating many of those after as well by establishing
conditional independence.  Secondly, the support for a particular $x_j$ given
$x_{1:j-1}$ is typically known (as 
we have access to the appropriate distribution object) and so it is generally possible to specify
an appropriate proposal for an individual $x_j$.  Even if that transpired not to be possible, 
simply using the prior as the proposal is still often reasonable as individual $x_j$ terms will typically
be low dimensional.  However, a key downside of this approach is that changing a particular $x_j$
might lead to an invalid trace and checking for this might require revaluation of the entire rest of the
trace, making it an $O(n_x+n_y)$ operation.  Perhaps even more problematically, if $n_x$ is not fixed
then, unless we make adjustments to the algorithm, the approach will not produce a 
valid Markov chain.  Similarly, as we showed in Section~\ref{sec:inf:foundation:gibbs},
there are models for which component-wise MH approaches lead to reducible Markov chains
that no longer admit the correct target.  The emphasis on branching in universal PPSs further
means that the chance of falling into this category of invalid models is relatively high.  As
noted by~\citep{kiselyov2016problems}), this is a rather serious issue missed (or at least not
acknowledged) by~\cite{wingate2011lightweight} and various follow-up implementations.
See also issues highlighted by~\citep{hur2015provably} in other implementations.
The problem is not shared by all implementations though, with the Anglican LMH method, amongst
others, not suffering from the issue.
More generally, there are means of potentially overcoming both the reducibility and computational
issues of LMH as we will discuss later, but
they still represent noticeable practical and theoretical hurdles.

As a simple illustrative approach that avoids some of the theoretical pitfalls associated
with LMH approaches, consider an MH sampler whose proposal simply chooses one
of the \sample statements, $m$, uniformly at random from $\{1,\dots,n_x\}$, proposes
 a new $x_m'$ using a local reversible
MH kernel for that \sample statement $\kappa(x_m' | x_{m})$ (which can always be independently sampling
from $f_m(x_m|\eta_m)$ if necessary) and then reruns the entire rest of the program
from that point~\citep{wood2014new}.  This equates to using the 
proposal\footnote{We omit our trace validity term because it always holds when sampling from the generative model.}
\begin{align}
q(x_{1:n_x'}' | x_{1:n_x}) &= q(m|x_{1:n_x}) q(x_m' | x_{m}, m) q(x_{m+1:n_x}' | x_{1:m-1}, m, x_m') \mathbb{I}(x_{1:m-1}'=x_{1:m-1}) \nonumber \\
&=\frac{1}{n_x} \kappa(x_m' | x_{m}) \mathbb{I}(x_{1:m-1}'=x_{1:m-1})  \prod_{j=m+1}^{n_x'} f_{a_{j}'} (x_{j}' | \eta_{j}')
\end{align}
which in turn gives an acceptance probability of
\begin{align}
P(\text{Accept}) &= \min\left(1, \frac{\gamma(x_{1:n_x'}',\lambda) q(x_{1:n_x} | x_{1:n_x'}') }
{\gamma(x_{1:n_x},\lambda) q(x_{1:n_x'}' | x_{1:n_x}) }\right) \nonumber\\
&= \min\left(1, \frac{\kappa(x_m | x_{m}') f_{a_m} (x_m' | \eta_m)} {\kappa(x_m' | x_{m}) f_{a_m} (x_m | \eta_m)} \;
\frac{n_x}{n_x'} \; \frac{\prod_{k=k_0(x_{1:m})+1}^{n_y'} g_{b_{k}'} (y_{k}' | \psi_{k}')}
{\prod_{k=k_0(x_{1:m})+1}^{n_y} g_{b_{k}} (y_{k} | \psi_{k}) }\right) \label{eq:proginf:lmh-A}
\end{align}
where $k_0(x_{1:m})$ is equal to the number of \observe statements encountered by the partial
trace $x_{1:m}$.  Here we have used the fact that the trace probability factors for terms 
before $x_m$ cancel
exactly, while the \sample terms for $m+1$ onwards cancel between the target and the proposal.
We can intuitively break down the terms in~\eqref{eq:proginf:lmh-A} by first noting that the first
ratio of terms, $\frac{\kappa(x_m | x_{m}') f_{a_m} (x_m' | \eta_m)} {\kappa(x_m' | x_{m}) f_{a_m} (x_m' | \eta_m)}$, is what we would get from
doing MH targeting $f_{a_m}(x_m|\eta_m)$.  The next ratio of terms,
$n_x / n_x'$ reflects the fact that if our new trace is longer it is less likely that we would have chosen the
point $m$ to resample and vice versa.  The final ratio of terms reflects the ratio of the likelihood weight
for the new generated section of the trace over the existing section of the trace.

This approach avoids the reducibility issues because it
includes as a possible step generating a completely new trace from the generative model.  However,
it will clearly be heavily limited in practical performance because each iteration
is effectively using importance sampling in $n_x-m$ dimensions such that, once the sampler has burnt in,
its acceptance rate will typically decrease dramatically with dimension.  Each iteration is also $O(n_x+n_y)$ as
the complete new trace needs to be proposed each time.  To get around these issues, we would like
to have some concept of whether we can update $x_m$ without invalidating the trace.  This can sometimes
be done using code analysis or compilation to establish the Markov blanket of $x_m$
\citep{yang2014generating,mansinghka2014venture,ritchie2016c3}, thereby providing a representation of which
terms will be unaffected by an update.  One needs to be
careful, however, not to reintroduce reducibility issues and, in general, correctness of LMH schemes is
far from trivial and perhaps requires further consideration in the literature.

Even if we do not have a convenient means of extracting Markov blankets, it will still generally be desirable
to reuse $x_{m+1:n_x}$ if it produces a valid trace, i.e. if $\mB ([x_{1:m-1},x_m',x_{m+1:n_x}],\lambda)=1$, and reducibility
issues can be guarded against, in order to avoid the terrible scaling in the acceptance rate 
of~\eqref{eq:proginf:lmh-A}  with $n_x-m$.  One way to do this would be to propose
a new $x_m'$ in the same way but then deterministically run the program forward with the old
$x_{m+1:n_x}$ to evaluate whether the trace is valid, noting that if it is, it must also be the case that any 
extension or reduction of the trace, i.e. increasing or decreasing
$n_x$, is invalid as the program is deterministic given $\xnx$.  If $x_m \rightarrow x_m'$ gives
a valid trace, then $\gamma([x_{1:m-1},x_m',x_{m+1:n_x}],\lambda)$ will be well defined (though not necessarily
non-zero) and we can use the update as a proposal without needing to regenerate $x_{m+1:n_x}$.  Note though
that as the $a_j$, $b_k$, $\eta_j$, and $\psi_k$ terms downstream of $m$ can change, the 
probability of the new trace still needs recalculating.  In this scenario, our acceptance ratio becomes
\begin{align}
  P(\text{Accept}) &= \min\left(1, \frac{\gamma(x_{1:n_x}',\lambda) \kappa(x_m | x_{m}')  }
  {\gamma(x_{1:n_x},\lambda) \kappa(x_m' | x_{m})  }\right)
\end{align}
where all terms in $\gamma(x_{1:n_x}',\lambda)$ (i.e. both \sample and \observe) will need
to be evaluated for $j\ge m$, as will $\gamma(x_{1:n_x},\lambda)$ if these are not already known. 

If, on the other hand, our trace becomes invalid
at any point (noting that we can evaluate the validity of sub-traces as we rerun them), we can resort to
resampling the trace anew from the required point onwards.  Note that we can do this validity evaluation
and regeneration during the same single forward pass through trace and that whether we do regeneration
is deterministic for given $\{x_m,x_m'\}$ pair (and thus does not affect the MH acceptance ratio by symmetry).
If we regenerate from \sample $\ell$ onwards then we now have
\begin{align}
P(\text{Accept}) &= \min\left(1, \frac{\gamma(x_{1:n_x}',\lambda) \kappa(x_m | x_{m}') n_x \prod_{j=\ell}^{n_x} f_{a_{j}} (x_j | \eta_j)}
{\gamma(x_{1:n_x},\lambda) \kappa(x_m' | x_{m})  n_x' \prod_{j=\ell}^{n_x'} f_{a_{j}'} (x_j' | \eta_j')}\right).
\end{align}
We can further overcome reducibility issues with such a scheme by forcing a fresh trace generation
not only when we realize the trace is invalid, but also when we find a term in our trace with probability zero.
Once this occurs, it is clear that the full trace probability must be zero, even if it produces a valid path.  We
could just immediately reject the trace, but we would not solve the reducibility issues.  By instead
regenerating the rest of the trace in this scenario, at each iteration we can propose any value of $x_1$ that has non-zero marginal
mass under $\gamma(x_{1:n_x},\lambda)$ regardless of our current state, so we clearly have mixing on $x_1$.
For a given value of $x_1$, we can similarly propose any value of $x_2$ with non-zero density under the
marginal conditional distribution of $x_2|x_1$.  As we have mixing of $x_1$, this now implies we have
mixing on $x_{1:2}$ as well.  By induction, our method now leads to mixing on all of $x_{1:n_x}$, which
coupled with detailed balance provides an informal demonstration of the consistency of the method.

We can refine this process further by using the concept of a database.  To do this we mark each \sample
statement in the trace with a unique identifier that is common to all traces that evaluate the same \sample
statement at the same point, i.e. points in the trace for which both the \sample number $j$ and 
the lexical \sample identifier $a_j$ are the same.  Our database can be used to store previous samples and
deterministically return them when revisiting a \sample statement with the same identifier if this old
value still constitutes a valid sample with non-zero probability, regenerating it if not.  
If we use $\mathbb{D}(j)=0$ to denote terms taken from
the database and $\mathbb{D}(j)=1$ to indicate terms that are redrawn, then our acceptance ratio now becomes
\begin{align}
P(\text{Accept}) &= \min\left(1, \frac{\gamma(x_{1:n_x}',\lambda) \kappa(x_m | x_{m}') n_x \prod_{j=m+1}^{n_x} 
	\left(f_{a_{j}} (x_j | \eta_j)\right)^{\mathbb{D}(j)}}
{\gamma(x_{1:n_x},\lambda) \kappa(x_m' | x_{m})  n_x' \prod_{j=m+1}^{n_x'} \left(f_{a_{j}'} (x_j' | \eta_j')\right)^{\mathbb{D}(j)}}\right).
\end{align}
Note the importance of points in the database being defined by both $j$ and $a_j$ -- if we hit the same
lexical sample statement at a different point in the program, we always need to redraw it.
Care is also needed to ensure superfluous terms are removed from the database at each iteration -- it should
contain only terms from the current trace.  

Note, however, that our identification scheme for points in the database, namely points for which both $j$ and $a_j$ are the same
have the same point in the database, is
potentially stronger than necessary and there may be more useful addressing schemes.  Remembering back
to Section~\ref{sec:probprog:models:general}, each of the \sample and \observe statements are exchangeable up to
the required inputs being in scope.  Now consider the case where the update $x_m'\leftarrow x_m$ triggers an
inconsequential extra \sample to be invoked between points $j$ and $j+1$, with everything else staying the same.
Under our current system then all points after $j$ would need to be resampled.  However, our program would
define the same distribution if this superfluous \sample statement came at the end of the program, in which case
our method would mean that $x_{j+1:n_x}$ no longer necessarily needs updating.  From a more practical perspective,
we can consider using more useful naming strategies for our database entries that exploit this exchangeability such
that $x_{j+2}'$ could, for example, inherit from $x_{j+1}$~\citep{wingate2011lightweight}.  
%Though this should
%intuitively not effect the convergence of estimates based on the program output, the formal proof is not immediately
%trivial and is perhaps missing from the literature.

We finish the section by briefly discussing some choices in the proposal.  The simplest choice for
$\kappa(x_m'|x_m)$ is just to redraw a new value from the prior.  In this case then all the 
$\frac{\kappa(x_m | x_{m}') f_{a_m} (x_m' | \eta_m)} {\kappa(x_m' | x_{m}) f_{a_m} (x_m' | \eta_m)}$ terms
will cancel in our acceptance ratios and giving a trivially valid proposal, other than the aforementioned 
reducibility issues.  We will refer to this strategy simply
as LMH in the rest of the thesis.  For models with significant prior-posterior mismatch, such an approach could be
slow to mix as it does not allow for any locality in the moves (i.e. $\kappa(x_m'|x_m)$ is independent of $x_m$).
As suggested by, for example,~\cite{le2015rmh}, one can sometimes achieve improvements by instead using the type of the 
distribution object associated with $x_m$ to automatically construct a valid random walk proposal that allows for 
improved hill climbing
behavior on the individual updates.  We will refer to this method as RMH elsewhere in the thesis.\footnote{Note that
	the Anglican RMH implementation uses a mixture proposal where it samples from the prior, as per LMH, half the time
	and from the random walk proposal the other half.}  Thus far, we have
presumed that which \sample statement to update at each iteration is selected uniformly at random.  This is not
actually a necessary assumption, with~\cite{tolpin2015output} demonstrating that one can construct an adaptive LMH (ALMH)
that adaptively updates proposal probabilities for which term in the trace to update.

\subsection{Particle-Based Inference Strategies}
\label{sec:proginf:str:part}

\subsubsection{Sequential Monte Carlo}
\label{sec:proginf:str:part:smc}
\vspace{-5pt}
Going from importance sampling to SMC in our framework is remarkably simple
from an implementation perspective~\citep{wood2014new,paige2014compilation}.  The behavior of the \sample and \clj{result} 
checkpoints are kept as per the default.  
The \observe checkpoints are redefined
to carry out the same operations, but return a record rather
than a thunk, returning control to the \anginfer function.  This means that 
calling \clj{exec} for the SMC checkpoint setup will run the program up to and including
the next \observe statement.  Consequently, if we run multiple threads of \clj{exec} at
once, each corresponding to a separate particle, these will all stop exactly when
the next resampling point is required for SMC.  Thus all the \anginfer function needs
to do for SMC, other than some bookkeeping,
is alternate between mapping an \clj{exec} call across all of the particles and
performing resampling steps (remembering to reset the internal weights for the traces to
be the same).  The marginal likelihood estimate can also be calculated in
the standard way, so the required lazy infinite sequence
of output samples can be produced by running independent SMC sweeps and setting the weights
to the product of the sweep marginal likelihood and local sample weight.\footnote{In practice, Anglican
	resamples after the last observation, so the
	local sample weights are all actually the same.}

From a theoretical perspective, running SMC in Anglican
requires us to make one small model assumption -- that the number of observations $n_y$ is fixed.  In
practice, this assumption is usually satisfied, particularly if there are no observations of
internally sampled variables.  Violations are caught at run time.  Given
a fixed $n_y$, we can define the series of targets for SMC as being the distributions
induced by running the program up to the $t^{\text{th}}$ \observe statement, namely
\begin{align}
\label{eq:proginf:smc-targ}
\gamma_t(x_{1:n_x}, \lambda) = \begin{cases}
\prod_{j=1}^{n_x} 
f_{a_j}(x_j | \eta_j)
\prod_{k=1}^{t}
g_{b_k}(y_k | \psi_k) \;\; \text{if} \;\; \mathcal{B}_t(x_{1:n_x},\lambda)=1 \\
0 \quad \text{otherwise}
\end{cases}
\end{align}
where $\mathcal{B}_t(x_{1:n_x},\lambda)$ is a function establishing the validity of the 
partial program trace.  More formally, we can define $\mathcal{B}_t(x_{1:n_x},\lambda)$ as
being a function indicating validity of a trace for transformation of the original program
that terminates after making its $t^{\text{th}}$ observe.
It may be that executions corresponding to different particles have not gone through the 
same \sample and \observe statements at any particular point, but this not a problem, from a theoretical perspective,
as provided that $n_y$ is fixed,~\eqref{eq:proginf:smc-targ} still defines an appropriate
series of targets for SMC inference.
Although changing the position of the \observe statements in
our program does not change the final distribution targeted by running SMC, we note that it can change
the intermediate target distributions, by adjusting at what point during the series of targets
the \sample statements are introduced.  Consequently, changing the position of the \observe
statements can have a dramatic effect on the practical performance of the inference, e.g.
placing all the \observe statements just before the program returns will cause the algorithm
to reduce to basic importance sampling.  The earlier the \observe statements are in the program,
or more precisely the later variables are sampled relative to the \observe statements, the
better inference will perform as less information is lost in
the resampling.  Tricks such as lazily sampling variables (such that \sample statements
are only invoked when needed) can, therefore, lead to substantial
performance gains.

\subsubsection{Particle Gibbs}
\label{sec:proginf:str:part:pgibbs}

Provided one does not try to support the special treatment of global variables that particle
Gibbs allows (i.e. restricting to the iterated CSMC case), extending SMC to particle Gibbs
is relatively straightforward in our framework.  From a theoretical perspective, the algorithm
extends from the SMC case in the same way as outside of the probabilistic programming framework.
From a practical perspective, there are two distinct challenges. Firstly, resampling for CSMC
sweeps does not maintain the target distribution in the same way as SMC~\citep{holenstein2009particle} and so one has
to be careful that there is no possibility for gratuitous resampling or missing a required resampling step (e.g. 
we cannot use the adaptive resampling discussed in Section~\ref{sec:part:smc:prat:ad-re}).
At first this would make it
seem like one would need to take care not to resample after the $n_y^{\mathrm{th}}$ observation.
However, resampling and choosing the retained particle uniformly at random from the
present particles turns out to be identical to sampling the retained particle in proportion to weight and
so this is, in fact, not a problem.  Secondly, we need a method of storing and retrieving the
state of the retained particle in a manner that allows other particles to inherit from it.  Given our
inference methods do not use a stack, this is done by storing the raw $x_{1:n_x}$ samples within
\angstate for each particle and then retrieving them for the retained particle.  The retained
particle is thus re-run in a deterministic fashion from a stored $x_{1:n_x}$, regenerating
all the variables in the program and the continuation.  This is achieved by
editing the \sample checkpoint so that it deterministically retrieves the $x_j$ for the 
retained particle, but samples normally for other particles.  For all particles it also stores the current
$x_j$ in case this particle becomes the retained particle at the next sweep.
Provided one is running a reasonably large number of particles the extra computational cost for
re-running the retained particle is negligible, but the need to store all the $\{\hxnx^n\}_{n=1}^{N}$
can noticeably increase the memory requirements.  Other than these minor complications, the
extension from SMC to particles Gibbs in the probabilistic programming setting is identical to
that for conventional settings as discussed in Section~\ref{sec:part:pmcmc:pgibbs}.

\subsubsection{Interacting PMCMC}
\label{sec:proginf:str:part:ipmcmc}

Given the particle Gibbs and SMC implementations, iPMCMC can be implemented
relatively simply by using those implementations for the CSMC and SMC sweeps respectively.
All checkpoint implementations are inherited from particle Gibbs and the extension does
not have any distinct challenges unique to probabilistic programming.  We note though that
the Anglican implementation of iPMCMC exploits the algorithmic ability for parallelization 
by creating a pool of threads to distribute computation of the different nodes.   Consequently,
the implementation has substantial computational benefits over that of say particle Gibbs, in addition
to the improved per-sample performance demonstrated in Section~\ref{sec:part:ipmcmc}.

\subsection{Other Methods}
\label{sec:proginf:str:part:other}

Though it would be impractical to do justice to all of the methods one can use for general
purpose in universal PPSs, we briefly lay out some other algorithms of particular note.

The particle cascade~\citep{paige2014asynchronous} is an asynchronous and anytime
	variant of SMC that can provide improved in-sweep parallelization compared to SMC
	and an alternative to PMCMC methods for overcoming memory restrictions in
	SMC.  It has the crucial advantage over PMCMC methods of maintaining an unbiased estimate
	of the marginal likelihood, but it can suffer instability issues in the number of particles produced
	by the schedule strategy during a sweep. It also does not permit any distinct treatment of global
	variables.

Particle Gibbs with ancestor sampling (PGAS)~\citep{lindstenJS2014,vandemeent_aistats_2015}
	looks to alleviate degeneracy for the particle Gibbs algorithm by performing ancestor sampling
	for the retained particle.  This can substantially improve mixing for the early samples, but can
	result in very large computational overheads, sometimes requiring $O(Nn_y^2)$ computation
	(compared to $O(Nn_y)$).  This can be partly 
	mitigated by using database techniques to avoid unnecessary
	repeated computation~\citep{vandemeent_aistats_2015}, but is still typically substantially more computationally intensive than
	standard particle Gibbs.  
	%Nonetheless, this extra computation is sometimes justified for challenging
	%problems with severe degeneracy.

Some black-box variational methods~\citep{ranganath_aistats_2014} 
	have been developed for general purpose PPSs~\citep{kucukelbir2015automatic} including Anglican~\citep{vandemeent2016black,paige2017thesis}.  Though these can suffer from difficulties
	in finding effective addressing schemes and or high variance of the ELBO gradients, they can still
	form an effective approximate approach for certain problems.

\subsection{Which Anglican Inference Algorithm Should I Use?}
\label{sec:proginf:str:which}

The question of which inference algorithm one should use is a critical, but somewhat subjective and
problem dependent question.  We now provide some (inevitably biased) practical recommendations for which
inference algorithms to use in Anglican in its current form.  These
should be taken as a rough starting point rather than clear-cut truth.

Firstly, except for very low dimensional problems, importance sampling should
be avoided as it is almost universally worse than, e.g. SMC.  Similarly, there is very little reason to use PIMH
instead of SMC or other PMCMC methods, while RMH or ALMH should generally be preferable to vanilla LMH.
The relative performance of the MCMC
based and particle-based (including PMCMC) methods will depend on the amount of structure that can
be exploited in the system by the respective approaches.  The particle-based approaches will generally be substantially
preferable when the \observe statements are interlaced with the \sample statements so that the intermediate
information can be exploited.  However, if all the variables need to be sampled before making observations, or if
all the variables are completely independent of one another given the data, SMC will reduce to importance sampling,
while the MCMC methods can still take advantage of hill-climbing effects.

If using SMC or PMCMC based methods, then, as per Section~\ref{sec:part:smc:prat:part}, the most important
thing is not the exact algorithm, but ensuring that sufficient particles are used.  Because PMCMC methods
in Anglican do not use separate updates for global parameters, using enough particles
can be even more important than in conventional settings.  Presume that your computational budget is $M$
particles and your memory budget is $N$ particles.  Our recommendation is that if $M<N$, you should look
to run a single SMC sweep with $M$ particles -- all the more advanced algorithms are mostly setup to avoid
issues that are easily solved by running more particles when you can.  For larger values of $M$, then you should
keep the number of particles as high as possible (ideally $N$).  Our recommendation here is to use iPMCMC
as the go-to algorithm as this will rarely be noticeably worse than the other particle-based approaches and sometimes
substantially better, both in terms of per-sample performance and its support for parallelization.
Two possible exceptions to this are that PGAS and the particle cascade can sometimes still be better
when $M\gg N$, with the latter also supporting effective parallelization of computation.  
For example, PGAS can be very effective in models where there are extreme restrictions on
the number of particles that can be run, e.g. due to calling an expensive external simulator.

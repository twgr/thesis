% !TEX root = ../main.tex

\chapter{General Purpose Inference for Probabilistic Programs}
\label{chp:proginf}

In Chapter~\ref{chp:probprog} we showed how probabilistic programming systems (PPSs) provide
and expressive for specifying probabilistic models.  We now consider the other major component
for PPSs: automating inference for any model the user is allowed to specify using general purpose
inference engines.  For most PPSs this requires two things --  a compiler to convert the \emph{query} to an
suitable form to provide as input the inference engine and the inference engine itself.  

For the inference
driven systems discussed in Section~\ref{sec:probprog:two:inf}, the inference engine typically comprises of
a standard Bayesian inference method for graphical models such as those discussed in Chapters~\ref{chp:bayes}
and~\ref{chp:part}.  Developing these in a way to robustly work for a wide range of problems typically
requires careful engineering and algorithmic innovation -- e.g. because many inference methods require the definition of
a proposal, upon which performance can critically depend -- but does not generally require development 
of approaches distinct to those used outside a probabilistic programming context.  In these 
systems the inference algorithm(s) is/are usually chosen first, with the language and its restrictions built around it.
Therefore the challenges of the designing the system are generally rooted in generalizing and increasing the robustness of the 
specific inference method(s) used.  Similarly, the language itself and associated compiler is generally built
around providing the easiest representation to work with the target models, while the fact that most of
these such systems to not support higher order functions usually substantially simplifies the compilation
process.

Because the design of these systems is very much driven by the particular inference algorithm used, it
beyond the scope of this thesis to do the associated literature justice.  Our focus will instead be on 
conducting inference for universal PPSs.  Nonetheless, we will find that there are very few (known) inference
methods which can actually cope with the most general possible models as we introduced in
Section~\ref{sec:probprog:models:general}, all of which suffer particularly badly from the curse of
dimensionality.

\section{Inference Toolboxes for Graphical Models}
\label{sec:proginf:tool}

\section{Inference for Arbitrary Programs}
\label{sec:proginf:inf}

\section{Particle Based Inference for Probabilistic Programs}
\label{sec:proginf:probprog}
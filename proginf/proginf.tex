% !TEX root = ../main.tex

\chapter{General Purpose Inference for Probabilistic Programs}
\label{chp:proginf}

In Chapter~\ref{chp:probprog} we showed how probabilistic programming systems (PPSs) provide
an expressive framework for specifying probabilistic models.  We now consider the other major component
for PPSs: automating inference for any model the user is allowed to specify using general-purpose
inference engines.  For most PPSs this requires two things --  a compiler to convert the \emph{query} to a
suitable form for input to the inference engine and the inference engine itself.  

For the inference
driven systems discussed in Section~\ref{sec:probprog:two:inf}, the inference engine typically comprises of
a standard Bayesian inference method for graphical models such as those discussed in Chapters~\ref{chp:bayes}
and~\ref{chp:part}.  Developing these in a way to robustly work for a wide range of problems typically
requires careful engineering and algorithmic innovation -- e.g. because many inference methods require the definition of
a proposal, upon which performance can critically depend -- but does not generally require the development 
of approaches distinct to those used outside a probabilistic programming context.  In these 
systems, the inference algorithm(s) is/are usually chosen first, with the language and its restrictions built around it.
Therefore the challenges of the designing the system are generally rooted in generalizing and increasing the robustness of the 
specific inference method(s) used.  Similarly, the language itself and associated compiler is generally built
around providing the easiest representation to work with the target model class, while the fact that most of
these such systems to not support higher order functions usually substantially simplifies the compilation
process.

Because the design of these systems is very much driven by the particular inference algorithm used, it
beyond the scope of this thesis to do the associated literature justice.  Our focus will instead mostly be on 
conducting inference for universal PPSs, though some of what we introduce will apply to both.
Unfortunately, we will find that there are very few (known) inference
methods which can actually cope with the most general possible models as we introduced in
Section~\ref{sec:probprog:models:general}, all of which suffer particularly badly from the curse of
dimensionality.  Consequently, it will be necessary to make certain (mostly very small) concessions in 
generality to achieve any reasonable performance on non-toy models.  We will focus on conducting
inference in Anglican~\citep{wood2014new,tolpin2016design} to give us a basis for explanation, but much of
what we discuss will still be relevant to other universal systems, in particular, those which
are also built around \sample-\observe syntaxes, such as VentureScript~\citep{mansinghka2014venture}, 
WebPPL~\citep{goodman_book_2014}, and Probabilistic C~\citep{paige2014compilation}.

% !TEX root = ../main.tex

\section{Compiling Queries}
\label{sec:proginf:comp}

Non-probabilistic languages can be either compiled, whereby high-level source code
is converted to a lower level language (e.g. byte-code) before being evaluated, or interpreted,
whereby an interpreter reads the program and directly evaluates it based on the language
semantics.  Other than some earlier systems that were based around guess-and-check 
type strategies, PPLs by construction tend to require some form compilation due to the fact that
the original code will typically run many times and because they tend to be integrated
with an existing language, such that they generally require compilation to the host language
before anything can be run.  Rather than just being a technical hurdle, compilation is also 
often an important tool in the ability of probabilistic programming systems to perform
effective inference as it can often be used to establish helpful salient features in the model,
such as dependency structures, variable types, and features of the model that can guide
which inference algorithm is likely to be most successful or even elements of the model
that can be calculated analytically.  Hakaru~\citep{narayanan2016probabilistic} is a good
example of what can be achieved in probabilistic programming through compilation alone,
as it, for example, uses the computer algebra system Maple to automatically simplify models
when possible.

\subsection{What do we want to Achieve through Compilation?}
\label{sec:proginf:comp:want}

So what are the aims of our PPL compiler?  First and foremost, we need to
perform a source-to-source transformation of the program to
produce some representation in the conventional programming language in which
the inference engine is written so that execution of the program is possible and inference
can be carried out.  As a simple example, consider the case of an Anglican program where
we wish only to run importance sampling using the generative model as a proposal 
(i.e. the query with all the\observe statements removed).  Here we need only produce a Clojure program
that samples directly from the generative model and accumulates weights from the
\observe terms as a side-effect.  We can then rerun this program arbitrarily many
times, each returning a sampled output and accompanying weight, to produce a sequence
of importance samples which can then be used to make consistent Monte Carlo estimations.
Such a strategy would be na\"{i}ve though as we would be able to use our Clojure program
for little else than importance sampling with this particular proposal.  Clearly, we want
to compile to a more general purpose representation that permits a wider array inference algorithms
and proposals, and which ideally allows features of the model, such as its dependency
structure, to be exploited.

One desirable feature of our compiled representation is an ability to make partial
program evaluations.  This will allow us to re-evaluate elements of the program in
isolation in a Gibbs sampling style fashion and it will allow us to interrupt the execution
of the program, giving us the ability to add in things such as the resampling step in SMC.
Supporting partial evaluation can also allow us to avoid gratuitous re-executions of
elements of the program whose result is already known by using databases to store and
recall the effect of previous executions.

Another desirable feature is to avoid being restricted to only directly sampling from \sample
statements.  We may wish to sample from a different distribution as a proposal or to evaluate
the probability density of the \sample statement producing a particular output, e.g. as part
of an acceptance ratio calculation for an MCMC scheme.  The key requirement for both of these
is the knowledge of density function associated with the \sample statement, rather than
just access to a black-box sampling scheme.  This is the motivation behind why we defined
the syntax of \sample as taking as input a distribution object which encodes both the density
function and means of sampling from that density.

It will also often be helpful for our compiler to delineate between probabilistic and deterministic
elements of the code.  Other than potentially trying to make efficiency gain by avoiding
repeated computation, we know that any deterministic parts of our code (namely code in between
consecutive \sample and/or \observe statements) can be safely run without the need
to worry about the implications this has on the inference.  The execution of these segments of
the program in isolation can thus be per the language being compiled to, without needing to
worry about the probabilistic semantics.  On the other hand, we will generally desire the identification
of \emph{checkpoints} for positions in the program where \sample or \observe statements are
made so that the behavior of the program at these points can be handed over to the inference
engine.

The compiler will in general also play an important role in the interaction between the PPL
as its integrated language on the behalf of the user.  Any PPL that does not provide produce
outputs in a helpful form that can be manipulated by the user will be somewhat impractical,
while for general purpose systems it is essential to allow the user to link in external 
deterministic code packages specific their particular problem.

This example list of desirable features for our compiler is far from exhaustive.
In particular, there will be many inference specific features required for some
systems, such as the need to have a representation of the derivations, e.g. calculated
through automatic differentiation~\cite{baydin2015automatic}, to carry out 
Hamiltonian Monte Carlo inference~\cite{carpenter2015stan}
or common variational inference methods~\citep{kucukelbir2015automatic}.  As we
said earlier, we will often also want our compilation to, when possible,
pick out salient features of the model, carry out simplifications, and even automatically 
establish the most suitable inference algorithm for a particular problem.  The compiler
is an integral part of any PPL and there is no one best approach for all situations. 
One of the key distinguishing features between different PPLs is how they approach
this compilation problem, with different design choices inevitably lead to systems
geared towards different models or inference algorithms. 

\subsection{Compilation of Anglican Queries}
\label{sec:proginf:comp:ang}

As it is inevitably infeasible to detail the inner workings of a PPL compiler in a general
manner, we now provide a more in depth introduction to compilation method
employed by Anglican.  Our introduction is inevitably not exhaustive, focusing more on
intuition than being exactly true to the implementation details; we refer the reader
to~\citep{tolpin2016design} for a more complete and rigorous introduction.

As we explain in Section~\ref{sec:probprog:anglican:models},
Anglican programs, or queries, are compiled using the macro \query which provides a
Clojure function that can be passed to one of the provided inference algorithms.
The key element of this compilation for providing the desirable properties discussed
in the last section and a convenient interface for the inference algorithms is that
Anglican compiles queries to \emph{continuation passing style} (CPS)~\citep{appel1989continuation}
Clojure functions.\footnote{WebPPL also does a CPS style transformation, 
	namely to a purely functional subset of Javascript.}
At a high level, a continuation is a function that represents the rest of the
program.  CPS is a style of functional programming that uses a series of continuations
to represent the program through a series of nested function calls, where the program
is run by evaluating each function and then passing the output to the continuation
which invokes the rest of the program.  This is perhaps easiest to see through
example.  Consider the simple function \clj{+}, which in Clojure has syntax \clj{(+ a b)}.  The
CPS transformed version of \clj{+}, which we will call \clj{+&} takes an extra input
of the continuation $\mP$ and invokes it after evaluation such that we have
\clj{(defn +& [a b } ~$\mP$\clj{] (}$\mP$ \clj{(+ a b)))}.  More generally, for any simple function \clj{f}, we have
that its the CPS transformation is \clj{(defn f& [args} ~$\mP$\clj{] (}$\mP$ \clj{(f args)))}.\footnote{Note that
	in practice, the continuation is usually set to be the first argument in order to provide support
	for functions with a variable number of inputs.}
  We 
will use this notation of adding an \clj{&} to an expression name to denote its CPS transformation throughout.
To give a more detailed example, the CPS transform of the program \clj{(max 6 (* 4 (+ 2 3)))} would be
\begin{lstlisting}[basicstyle=\ttfamily\small,frame=none]
  (fn [$\mP$] (+& 2 3 (fn [x] (*& 4 x (fn [y] (max& 6 y $\mP$)))))
\end{lstlisting}\vspace{-8pt}
where \clj{*&} and
\clj{max&} are analogous to \clj{+&} defined as before.
Here our first continuation is \clj{(fn [x] (*& 4 x (fn [y] (max& 6 y))))} and our second continuation 
is \clj{(fn [y] (max& 6 y))}.  Note that the CPS transformed code is itself a function because
it itself takes a continuation.  

Things are a little
trickier for general expression that are neither literals nor simple first order functions, for example,
binding forms like \clj{let}, Anglican special forms like \sample, and branching statements like \clj{if}.
For these expressions, the high-level idea is the same, but the CPS transformation is, unfortunately, 
expression specific and must, in general, be implemented on a case-by-case basis.  
For example, one can CPS transform \clj{let} by going from
\clj{(let [x (foo1 1) y (foo2 2)] (foo3 x y))} to
 \begin{lstlisting}[basicstyle=\ttfamily\small,frame=none]
 (fn [$\mP$] (foo1& 1 (fn [x] (foo2& 2 (fn [y] (foo3& x y $\mP$)))))).
 \end{lstlisting}\vspace{-8pt}
noting that within the \clj{(fn [x] .)} closure, \clj{x} is bound to the input of the 
function, giving behavior synonymous to the original
\clj{let} block.  
Another special case of particular note is \clj{loop}-\clj{recur} blocks.  These can be CPS
transformed by explicitly redefining them as a self-recursive function which can then
be transformed in the normal way.  For example,
\clj{(loop [x 10] (if (> x 1) (recur (- x 2)) x))}
becomes
\begin{lstlisting}[basicstyle=\ttfamily\small,frame=none]
 (fn [$\mP$] ((fn foo [x] (if (> x 1) (foo (- x 2)) ($\mP$ x))) 10))
 \end{lstlisting}\vspace{-8pt}
where we have exploited the fact that \clj{fn} allows the function to be named (in this case to
\clj{foo}) so that it can call itself.
 
In CPS style code, functions never return (until the final tail call) and every function takes an
extra input corresponding to the continuation.  This would be a somewhat awkward method for
writing programs, as the whole program must be written as a single nested function.  However,
it can be a very useful form to compile to as the execution of the program becomes
exceptionally simple and just involves evaluating functions and passing the output to the
next continuation -- it explicitly linearizes the computation.  For our purposes, having access
to functions representing the rest of the program in the form of continuations will be
particularly useful as it will allow for partial program evaluations.  It will also be convenient
for adding checkpoints at particular points in the program -- namely at the \sample and
\observe calls -- where control is handed over to the inference algorithm.

Compilation of an Anglican query, triggered through the \query macro, is done in recursively
in a top down manner.  The key function in doing this, called \clj{cps-of-expression},
dispatches to the individual CPS transformations by matching types of expression, or, if necessary,
directly by key-word.  Individual CPS transformations then recursively call \clj{cps-of-expression}
or themselves if necessary, e.g. if the top level call is a \clj{let} block, until the full query is
is transformed.  

The individual CPS transformations in Anglican are marginally more complicated than the
framework we have laid out thus far.  This is because it is necessary to not just run the
program, but also track its state from a probabilistic perspective, storing things such
as sample weights and other probabilistic side-effects.  To deal with this, the transformed Anglican
continuations take as input an \emph{internal state}, which we will denote as \angstate,
in additional to the computed value.
Thus, for example, the transformation for a simple function becomes\footnote{As before, the original arguments
	are actually last in the argument list for the true implementation.}
\begin{lstlisting}[basicstyle=\ttfamily\small,frame=none]
  (defn f& [args $\mP~\dollar$state] ($\mP$ (f args) $\dollar$state)).
\end{lstlisting}\vspace{-8pt}
More generally, we will simply pass on \angstate unchanged except for the Anglican
special forms which can use and/or manipulate the internal state.  \angstate itself
is defined to be a hash map, initialized as follows
\begin{lstlisting}[basicstyle=\ttfamily\small,frame=none]
  (def initial-state {:log-weight 0.0 :result nil ...})
\end{lstlisting}\vspace{-8pt}
where \clj{...} includes some fields we will not directly consider at the moment and some algorithm
specific fields (e.g. information required to the retained particle in PMCMC methods).
Here the field \clj{:log-weight} allows the program to be assigned a
(relative) weight as accumulated by, for example, \observe statements or importance weights
when sampling from a different distribution than that provided to the \sample statement.
The return for each sample in our inference will be the \angstate, hence the inclusion
of the \clj{:result} field which is set to the output of the query at its tail call.

One of the key components of the Anglican compilation is the transformation applied to the
the special forms and in particular the probabilistic forms \sample and \observe.  These
are respective transformed to the Clojure record constructors \samplecps and \observecps 
which have the call syntaxes of
\clj{(->sample id dist} ~$\mP$ ~\angstate\clj{)} and \clj{(->observe id dist value} ~$\mP$ ~\angstate\clj{)},
where \clj{id} is a unique checkpoint identifier (e.g. the $\{1,\dots,n_s\}$ and $\{1,\dots,n_o\}$ identifiers
we used in~\ref{sec:probprog:models:general}) set at compilation time, and  \clj{dist} and \clj{value}
are the original inputs to the \sample and \observe statements.  At a high level, one can
think of a Clojure record as defining a new class type (here {\small \texttt{trap.sample}} and {\small \texttt{trap.observe}}
respectively) with given fields (here \clj{:id}, \clj{:dist} etc).  Our constructors thus create
an object of the given type with appropriately set fields.  The significance of this is that
it allows definition of a multimethod, which we call \checkpoint, to provide a runtime polymorphism
(i.e. single interface) for dispatching depending on the checkpoint
type and the inference algorithm.  In other words, we will have one function, \checkpoint, whose
behavior can be redefined for different checkpoint types and inference types.  There are many
consequences of this.  Firstly, our compiler does not need to be inference algorithm specific
because we can use \checkpoint to distribute the behavior to the required inference algorithm
at run time.  Secondly, it creates an abstraction barrier for writing inference algorithms -- implementing
an inference algorithm now only requires us to implement, along with a top level function \anginfer that
we will discuss later, new methods describing the behavior of \checkpoint at \sample and \observe
checkpoints.  Furthermore, these can be inherited from other inference algorithms, for example,
the \observe checkpoints for particle Gibbs are inherited from SMC in Anglican's implementation.

A slightly more subtle consequence of compiling to a constructor with typed output is that
we will use this to catch the checkpoints themselves.  Once compiled, individual instances of
a program in the Anglican inference engines are run using the \clj{exec} function.  The role of
\clj{exec} is to run the program until it reaches a checkpoint that requires control to be transferred
to the top level inference function of a particular method, i.e. the \anginfer function, such as when
a particular trace has finished running or when interaction is required between different samples, e.g. 
the resampling step in SMC.  In Anglican this is achieved using \emph{trampolining}.  In functional
programming languages, trampolining is a process of looping execution where if an iteration of
a loop returns a function, that function is immediately evaluated without passing forward any arguments,
with this process continuing until a non-functional output is returned.  The primary use of trampolining
is for managing stack sizes by constructing a nested call structure of \emph{thunks} (i.e. functions which
require no input arguments) that when called by the trampolining function (called \clj{trampoline} in Clojure)
invoked the full set of nested calls without requiring a stack to be constructed or stored, as would be necessary
for an ordinary nested function structure.  In the context of our CPS transformations, we can do this by simply
wrapping each call with an anonymous function, namely converting \clj{(foo args }~$\mP$ ~\angstate\clj{)}
to \clj{(fn [] (foo args }~$\mP$ ~\angstate\clj{))}, which will be carried out for all continuations
except the checkpoints.  Though a large part of the motivation for doing this in
Anglican is similarly to maintain the stack size, it is also highly useful for using the checkpoints to trigger
control to be transferred to the \anginfer function.  To be precise, the \clj{exec} function is defined as
\begin{lstlisting}[basicstyle=\ttfamily\small,frame=none]
  (defn exec [algorithm prog value $\dollar$state]
    (loop [step (trampoline prog value $\dollar$state)]
      (let [next (checkpoint algorithm step)]
       (if (fn? next) (recur (trampoline next)) next))))
\end{lstlisting}\vspace{-8pt}
where \clj{algorithm} specifies the inference algorithm; \clj{prog} is the program to call, comprising of either the full
CPS compiled program output by \query or a continuation representing the rest of the program; 
\clj{value} is the input required by \clj{prog}, comprising of either the original program inputs or the value
passed to the continuation; and \angstate is the program state as before.  Here we see that \clj{exec}
first creates the variable \clj{step} by making a trampolined call to \clj{prog}.  As all continuations are now
represented by anonymous functions with no inputs except for our checkpoints, this will run until it
hits one of those checkpoints, returning the corresponding constructed checkpoint object.  In other words,
this causes the program to run until it reaches a \sample or \observe statements, or it reaches the end of the
program.  This is a highly desirable behavior, as it means the deterministic elements of our program
are run as normal.  Once returned, our multimethod \checkpoint is called using \clj{algorithm} and
\clj{step}, which distributes to the appropriate \checkpoint implementation based on the value of
\clj{algorithm} and the type of \clj{step} (e.g. calling to the \sample checkpoint method of an algorithm 
if it is of type  {\small \texttt{trap.sample}}).  Note that individual \checkpoint implementations
are based on using \clj{step} as the only input of note, which, as we remember, is a Clojure record
containing fields providing information on the distribution object, the state, etc.
Invoking the appropriate checkpoint method will cause
inference algorithm specific behavior, e.g. updating a weight in \angstate, and provide a return value
\clj{next}.  If \clj{next} is a function, the loop
is recurred, with \clj{step} now replaced by the output of a trampolined call of \clj{next}.  If \clj{next} is
not a function, the \clj{exec} function terminates, returning \clj{next}.  The reason for this looping is
that it allows different inference algorithms to specify different occurrences for when control needs to
be passed back to the \anginfer function in an algorithm specific way, by defining the checkpoint methods
to either return a checkpoint object (for which the \clj{exec} call will terminate) or the continuation
wrapped in a thunk (for which execution will continue).  For example, when running SMC,
action distinct to running the program forwards only needs to be taken at the \observe statement 
checkpoints.  As different traces may have a different number of \sample statements between observations,
it is therefore convenient to a have a function that does not terminate at the \sample checkpoints, returning
only once an \observe or the end of the program is reached.  On the other hand, for MCMC samplers, we
will need to externally control the sampling behavior and so it may be necessary to return control at each
\sample statement.

We have now demonstrated how an Anglican query is compiled, a summarizing example for which
is provided in Figure XX\todo{Add compilation example}.  We have also laid out a framework for
writing inference algorithms in Anglican: we need to implement methods of the \checkpoint multimethod
and a top level inference function \anginfer.  For the former, there are three possible \checkpoint methods
we might have to reimplement for a particular inference algorithm, namely ones to dictate the behavior
of \sample and \observe, and a ``\clj{result}'' checkpoint for controlling behavior at the termination of
the query (for most methods this will simply provide the return value).  In the rest of this Chapter we will
explain how this abstraction allows us to write inference algorithms suitable for arbitrary programs.
% !TEX root = ../main.tex

\section{Writing Inference Algorithms in Anglican}
\label{sec:proginf:inf}

Taking stock, what has our compilation
given access to and what do we not have access to?  The key upshots of our compilation are
that we have access to a continuation representing the rest of the program at any point
duing the execution and 
the interactions between the inference back-end and the query code happens only through
the \sample and \observe statements and the final returned \angstate of each execution.
Consequently, we have means of running the program
forwards and controlling the behavior at each \sample and \observe statement.  We can
stop or restart the program at any one of our checkpoints and chose whether to do so
adaptively, for example, by killing of certain evaluations or duplicating others.  Once
a program is run, we can test the effect of changing one or more of our sampled variables
in an MCMC fashion, though it may be difficult to statically determine an appropriate global proposal
as we can only establish the structure of the target through evaluation.  Although we might be
able to do so with other code analysis, we do not directly have access to any information
about independence relationships or even knowledge under what conditions a certain
variable will exist.  
%We can transfer information from one sampling iteration to the next
%using a database, giving us the potential to avoid repeated computation or adaptively learn proposals.
In theory, we can query the unnormalized density $\gamma(x_{1:n_x},\lambda)$ of any fixed set of 
parameter values $x_{1:n_x}$, given fixed inputs $\lambda$, by effectively making
\sample operate as an \observe statement.  However, it may be difficult, or even impossible, to find
values of $x_{1:n_x}$ that have non-zero probability unless they are generated by running
the program forwards.  In particular, it might be arbitrarily hard to find a trajectory that satisfies
$\mB(x_{1:n_x},\lambda)=1$, or construct an external proposal more generally,
 without forward sampling from the program.  
%It is liable
%to be even more difficult to externally construct a valid proposal to generate consistent estimates
%through, for example, importance sampling based schemes or that will lead to ergodic Markov chains,
%particularly if the program contains a potentially unbounded number of variables.

These availabilities and restrictions suggest that Anglican is, and universal PPSs more generally are,
best suited to \emph{evaluation based} inference methods.  At a high level, we can think of such methods
as acting like a controller wrapping around the standard program
execution (or potentially multiple simultaneous executions).  When the program hits a checkpoint,
control is transferred to the inference engine, providing information about the current state through
\angstate and about the current checkpoint command (i.e. checkpoint type, distribution object, etc).
The inference engine then decides exactly how the checkpoint command should be executed (e.g. how to
sample the new variable for a \sample statement), updates the \angstate appropriately, and decides if
and how the execution should continue.  Once an execution is complete (i.e. the return checkpoint
is reached), the inference controller decides what to run next, which could, for example, be an entirely new independent
execution or an MCMC update to the current execution.  Two important things the inference controller
does not do is alter the deterministic elements of the program (which we can think of as being single
deterministic functions from one checkpoint to the next)
or do further analysis on the program source code itself.  

There are four functions required to define an Anglican inference algorithm more generally: a top level 
\anginfer function and three \checkpoint implementations
for types \smalltt{trap.sample}, \smalltt{trap.observe}, and \smalltt{trap.return} .  The \anginfer
implementation must always be written anew, but the \checkpoint implementations can be inherited from
another inference algorithm of default behavior.  Other than some top level book keeping, the user
inference command \doquery directly calls the \anginfer function of the specified inference algorithm, providing
as input the CPS-transformed query, the fixed query inputs, and inference algorithm specific options.
The \anginfer function is required to then return a lazy infinite sequence of samples, each in the form of
a \angstate.  To carry out inference, \anginfer uses the \clj{exec} function introduced earlier
to run (partial) executions of the program, which will run until they hit a checkpoint that returns something
other than a thunk (see below).  

All of the \checkpoint implementations
take as their only input of note an instance of the relevant Clojure record, i.e. a
 \smalltt{trap.sample}, \smalltt{trap.observe}, or \smalltt{trap.result}.  
 Here \smalltt{trap.sample} has
fields \clj{:id} which is a unique identifier for the corresponding lexical \sample statement, \clj{:dist}
which is the distribution object input, \clj{:cont} which is the continuation function $\mP$, and \clj{:state}
which is the current instance of \angstate.  Meanwhile, \smalltt{trap.observe} has the same fields and additionally
a \clj{:value} field giving the observation and \smalltt{trap.result} has only the \clj{:state} field.  Each
\checkpoint method should either return a thunk, in which case the \clj{exec} function will resume running,
or an instance of the relevant Clojure record, in which case \clj{exec} terminates and returns control to the \anginfer
function (or some intermediate function used by the \anginfer function).  For example, in SMC methods
then we will want to return control to \anginfer at an \observe point to do the resampling, while all methods
return control for a \clj{return} checkpoint.
Default behaviors are provided for each checkpoint that dictate their behavior unless overwritten by
a particular inference algorithm.  The default behavior of \sample is to sample from the distribution
object and return a thunk that calls the continuation with this sample and \angstate as input arguments.
The default behavior of \observe is to update the \clj{:log-weight} field in \angstate with the current
observation term and then returns a thunk which calls the continuation with inputs \clj{nil} and \angstate.  The default behavior
of \clj{return} simply returns the \smalltt{trap.result} object.

\section{Particle Based Inference for Probabilistic Programs}
\label{sec:proginf:part}
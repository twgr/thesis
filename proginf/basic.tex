% !TEX root = ../main.tex

\section{Writing Inference Algorithms in Anglican}
\label{sec:proginf:inf}

Taking stock, what has our compilation
given access to and what do we not have access to?  The key upshots of our compilation are
that we have access to a continuation representing the rest of the program at any point
duing the execution and 
the interactions between the inference back-end and the query code happens only through
the \sample and \observe statements and the final returned \angstate of each execution.
Consequently, we have means of running the program
forwards and controlling the behavior at each \sample and \observe statement.  We can
stop or restart the program at any one of our checkpoints and chose whether to do so
adaptively, for example, by killing of certain evaluations or duplicating others.  Once
a program is run, we can test the effect of changing one or more of our sampled variables
in an MCMC fashion, though it may be difficult to statically determine an appropriate global proposal
as we can only establish the structure of the target through evaluation.  Although we might be
able to do so with other code analysis, we do not directly have access to any information
about independence relationships or even knowledge under what conditions a certain
variable will exist.  
%We can transfer information from one sampling iteration to the next
%using a database, giving us the potential to avoid repeated computation or adaptively learn proposals.
In theory, we can query the unnormalized density $\gamma(x_{1:n_x},\lambda)$ of any fixed set of 
parameter values $x_{1:n_x}$, given fixed inputs $\lambda$, by effectively making
\sample operate as an \observe statement.  However, it may be difficult, or even impossible, to find
values of $x_{1:n_x}$ that have non-zero probability unless they are generated by running
the program forwards.  In particular, it might be arbitrarily hard to find a trajectory that satisfies
$\mB(x_{1:n_x},\lambda)=1$, or construct an external proposal more generally,
 without forward sampling from the program.  
%It is liable
%to be even more difficult to externally construct a valid proposal to generate consistent estimates
%through, for example, importance sampling based schemes or that will lead to ergodic Markov chains,
%particularly if the program contains a potentially unbounded number of variables.

These availabilities and restrictions suggest that Anglican is, and universal PPSs more generally are,
best suited to \emph{evaluation based} inference methods.  At a high level, we can think of such methods
as acting like a controller wrapping around the standard program
execution (or potentially multiple simultaneous executions).  When the program hits a checkpoint,
control is transferred to the inference engine, providing information about the current state through
\angstate and about the current checkpoint command (i.e. checkpoint type, distribution object, etc).
The inference engine then decides exactly how the checkpoint command should be executed (e.g. how to
sample the new variable for a \sample statement), updates the \angstate appropriately, and decides if
and how the execution should continue.  Once an execution is complete (i.e. the return checkpoint
is reached), the inference controller decides what to run next, which could, for example, be an entirely new independent
execution or an MCMC update to the current execution.  Two important things the inference controller
does not do is alter the deterministic elements of the program (which we can think of as being single
deterministic functions from one checkpoint to the next)
or do further analysis on the program source code itself.  

There are four functions required to define an Anglican inference algorithm more generally: a top level 
\anginfer function and three \checkpoint implementations
for types \smalltt{trap.sample}, \smalltt{trap.observe}, and \smalltt{trap.return} .  The \anginfer
implementation must always be written anew, but the \checkpoint implementations can be inherited from
another inference algorithm of default behavior.  Other than some top level book keeping, the user
inference command \doquery directly calls the \anginfer function of the specified inference algorithm, providing
as input the CPS-transformed query, the fixed query inputs, and inference algorithm specific options.
The \anginfer function is required to then return a lazy infinite sequence of samples, each in the form of
a \angstate.  To carry out inference, \anginfer uses the \clj{exec} function introduced earlier
to run (partial) executions of the program, which will run until they hit a checkpoint that returns something
other than a thunk (see below).  

All of the \checkpoint implementations
take as their only input of note an instance of the relevant Clojure record, i.e. a
 \smalltt{trap.sample}, \smalltt{trap.observe}, or \smalltt{trap.result}.  
 Here \smalltt{trap.sample} has
fields \clj{:id} which is a unique identifier for the corresponding lexical \sample statement, \clj{:dist}
which is the distribution object input, \clj{:cont} which is the continuation function $\mP$, and \clj{:state}
which is the current instance of \angstate.  Meanwhile, \smalltt{trap.observe} has the same fields and additionally
a \clj{:value} field giving the observation and \smalltt{trap.result} has only the \clj{:state} field.  Each
\checkpoint method should either return a thunk, in which case the \clj{exec} function will resume running,
or an instance of the relevant Clojure record, in which case \clj{exec} terminates and returns control to the \anginfer
function (or some intermediate function used by the \anginfer function).  For example, in SMC methods
then we will want to return control to \anginfer at an \observe point to do the resampling, while all methods
return control for a \clj{return} checkpoint.
Default behaviors are provided for each checkpoint that dictate their behavior unless overwritten by
a particular inference algorithm.  The default behavior of \sample is to sample from the distribution
object and return a thunk that calls the continuation with this sample and \angstate as input arguments.
The default behavior of \observe is to update the \clj{:log-weight} field in \angstate with the current
observation term and then returns a thunk which calls the continuation with inputs \clj{nil} and \angstate.  The default behavior
of \clj{return} simply returns the \smalltt{trap.result} object.
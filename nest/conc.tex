% !TEX root =  main.tex

\section{Implications For Nesting Probabilistic Programs}
\label{sec:design:imp}

Given our results, it is now natural to ask the question what the implications
for nesting probabilistic programming queries?  In particular, when is doing
this valid and are there any precautions we can take to avoid problems?  Before
we get to these question though, we first need to explicitly define what we
mean by a nested query.  To this end, we distinguish between nested calling
structures (such as the example given in Figure~\ref{fig:probprog:schell}) and
queries that do not represent a single estimation, which we will refer
to as nested queries.  Both of these cases have at times been referred to as
nested queries in the literature.  Our motivation for defining only the later
as nested is that the former can always be represented as a single
query because they still represent a single expectation and define models
for which the unnormalized target distribution can be evaluated exactly
~\eqref{eq:probprog:universal-cond}.  Therefore, although these models are
of clear importance, they are not a fundamentally different problem class
to standard Bayesian inference problems and so we assert that existing MC
converge results directly apply.  However, the distinction between models that require
query nesting and can be written using just nested calling
can be surprisingly difficult to establish.
For example, the Schelling coordination example we gave in Figure~\ref{fig:probprog:schell}
is clearly an example of a nested calling structure as it is explicitly a single query.
On the other hand, the original version of this problem in~\cite[Figure 6]{stuhlmuller2014reasoning}
is actually an (equivalent!) example of a true nested query.  It turns out that this original
version constitutes one of our special cases that can be reformulated to a single estimate.
In the rest of this section we will delineate some possible ways one might want
to nested probabilistic program queries and
assert their respective correctness (or lack thereof if conditions are not met).
Namely, we will consider \emph{sampling} from the conditional distribution
of another query, \emph{observing} another query, and using \emph{estimates as variables}
in another query.  We will further identify special cases that will either provide Monte 
Carlo convergence rates when expressed as nested queries, or which can be rearranged to single estimators.

\subsection{Nested Sampling}
\label{sec:design:imp:sampling}

One of the clearest ways one might want to nest queries is by sampling from the conditional
distribution of one query inside another query.  Due to the ``sample then check'' setup of
Church, all of the examples given in~\cite{stuhlmuller2014reasoning} are by proxy this
type of query nesting,\footnote{Even though the nested calls are made inside the conditioning
	predicate for these examples, the semantics of the language is to sample these
	and then check if the predicate holds as a hard constraint.  This is equivalently to doing a nested
	sampling in the program and then evaluating a deterministic predicate using our PPL notation.}
as is any usage of the \conditional predicate in Anglican (as this is only currently supported for use
with \sample statements and inference algorithms that do not required the density to be evaluated).
Such problems fall under a more general framework of \emph{nested inference}~\cite{mantadelis2011nesting} or equivalently so-called
\emph{doubly intractable} (or multiply intractable for repeated nesting~\citep{stuhlmuller2014reasoning}) 
inference~\citep{murray2006mcmc}.  The key feature of these problems is that they include
terms with unknown, \emph{parameter dependent}, normalization constants.  For nested probabilistic programming
queries, this is manifested in \emph{conditional normalization} within another query.
Consider as an example the following model using a nested calling structure
\vspace{-15pt}
\begin{lstlisting}[basicstyle=\ttfamily\footnotesize,multicols=2,frame=none]
(defm inner [y D]
 (let [z (sample (gamma y 1))]
  (observe (normal z y) D)
  z))

(defquery outer [D]
 (let [y (sample (beta 2 3))
       z (inner y D)]
  (* y z)))
\end{lstlisting}
\vspace{-25pt}
compared to a following nested query model
\vspace{-15pt}
\begin{lstlisting}[basicstyle=\ttfamily\footnotesize,multicols=2,frame=none]
(defquery inner [y D]
 (let [z (sample (gamma y 1))]
  (observe (normal z y) D)
  z))
  
(defquery outer [D]
 (let [y (sample (beta 2 3))
       dist (conditional inner)
       z (sample (dist y D))]
  (* y z)))
\end{lstlisting}
\vspace{-15pt}
where we have replaced the \defm with a nested query.  The unnormalized distribution on traces for the first model
is straightforwardly given by
\[
\gamma_1(z,y,D) = p(y)p(z|y)p(D|y,z) = \textsc{Beta}(y;2,3) \textsc{Gamma}(z;y,1) \mathcal{N}(D;z,y^2)
\]
for which we can clearly use conventional Monte Carlo inference schemes.  The second model
is only subtly, but critically different in that $p(z|y)p(D|y,z)$ is conditionally normalized 
before being used in the outer query
\[
\gamma_2(z,y,D) = p(y)p(z|y,D) = p(y)\frac{p(z|y)p(D|y,z)}{\int p(z|y)p(D|y,z)dz} = p(y)\frac{p(z|y)p(D|y,z)}{p(D|y)} \neq \gamma_1(z,y,D).
\]
The key point here is that the partial normalization constant $p(D|y)$ depends on $y$ and so 
$\gamma_2(z,y,D)$ is doubly intractable because we cannot evaluate our unnormalized target
distribution exactly.  By normalizing $p(z|y)p(D|y,z)$ within the outer query, 
 \conditional has changed the distribution defined by the program.  Another interesting way of looking at this
 is that wrapping \clj{inner} in \conditional has ``protected'' $y$ from the conditioning -- noting
 that $\gamma_1 (z,y,D) \propto p(y|D)p(z|y,D)$ -- such that the \observe statement only effects the probability
 of $z$ given $y$ and not the probability of $y$.
 
 Unfortunately the inverse function $f(x) = 1/x$ constitutes
a nonlinear mapping and so expectations with respected the nested queries defined in this way using
\conditional will generally require using nested estimation schemes unless we can find a means of
sampling from $p(z|y,D)$ exactly with finite computational budget.
The obvious question is therefore what behavior do we need for \conditional, or query nesting through sampling
more generally, in order to ensure consistency?  

One possible approach would be to attempt to
use an exact sampling method for sampling from \conditional. Presuming we cannot
do this analytically~\citep{cornish2017efficient}, the most obvious way to generate such exact
samples is using rejection sampling (see Section~\ref{sec:inf:foundation:rejection}) as done in
Church.\footnote{Note, though, that this
is not the only way to generate exact samples from an unnormalized target distribution (see e.g. \citep{craiu2011perfection}).}
However, even this opens up a minefield of theoretical
and practical questions.  For example, because the acceptance rate of an inner query might depend on the
value of the input variable, it would presumably not be valid to generate all the variables at once
and then only accept the sample if every condition is satisfied (this would require more formally demonstration though).
Instead, Church takes an approach whereby no sample ever returns until it passes its local acceptance criterion.
Although doing this automatically for soft conditioning in a automated way might be insurmountably problematic (because of the
need to choose a valid $C$ in~\eqref{eq:inf:rej-acc-criteria}), it does open up the intriguing prospect
of providing a standard Monte Carlo convergence rate for a set of inherently nested problems.  This is because although the performance clearly
gets exponentially worse with increased nesting (as the probability of a successful return is the
product of an increasing number of expected acceptance rates), it is a change in the constant factor of the
computation.  This perhaps easiest to see by noting that 
generating a single exact sample of the distribution
can be done in finite time using rejection sampling (and thus effectively requires a fixed $N_1,\dots,N_D$ for
convergence), whereas in general NMC we had that $N_1,\dots,N_D$ need to increase as $N_0$ increases for convergence.
If correct, this assertion is a somewhat startling result: it suggests that Monte Carlo estimates made use 
nested rejection sampling have a fundamentally different
convergence rate for nested inference problems (though not nested estimation problems more generally) than, say,
nested self-normalized importance sampling (SNIS).  Amongst other things, this could lead, perhaps through combination of 
the work of~\cite{moller2006efficient}, to a contradiction of the, ostensibly very reasonable, conjecture
made by~\cite{murray2004bayesian} that there are no generic tractable MCMC schemes for doubly intractable distributions.
It is beyond the scope of the work to analyze this hypothesis more formally, or the rabbit hole of
nested exact sampling methods more generally, but highlight that this constitutes a potentially fascinating line of inquiry.

%From a practical perspective, it becomes much easier to write programs which will never terminate
%as, using the notation from our previous example, then for all possible $y$, there must be some $z$
%for which $p(D|y,z)\neq0$ (or for which the condition holds true in the context of Church).

%WRONG!
%Imagine we take
%a rejection sampling approach whereby no sample ever returns until
%it passes the acceptance criterion (which seems to be the approach taken by Church).  One clear issues this throws up is that it might be the case
%that the inner query has no valid outputs for a particular input.  Using the notation from our earlier example, we
%might have that $p(z|y,D) = 0, \forall z$ for a particular value of $y$, e.g. if $p(D|y,z) = \mathbb{I}(|y|<1 \cap |z|<1)$
%then any $|y|\ge1$ outside this range has zero probability.  
%Our outer distribution here $\gamma_2(z,y,D)$
%is still well defined, it just happens to be zero unless $|y|<1$.  Thus our depth-first rejection 
%strategy could get permanently stuck for a fundamentally computable distribution.  On the other hand,
%we could instead 

Without access to exact sampling methods, it is straightforward to see from our NMC results and
the fact that normalization constitutes a nonlinear mapping, that asymptotic bias (i.e. bias even if we take
an infinite number of samples from the outer query) is, in general, unavoidable for
query nesting if each call of \conditional only carries out finite computation.  For example, imagine we
use SNIS to generate outputs from \conditional
with a fixed number of internal samples proposed for each sample output.  As we showed in 
Section~\ref{sec:inf:foundation:importance:self-norm} then expectations with respect to such samples will be
biased, which we can think of in the NMC context as occurring because the inversion of the unbiased marginal likelihood estimate
induces a bias.  It may therefore be beneficial to adapt the behavior of \conditional to account for this.\todo{Come back to this}

One special case were consistency can be maintained when using \conditional (albeit not with Anglican's current
setup) is if the inner query takes no random variables as inputs (as is the case for
the Schelling coordinate example in~\cite[Figure 6]{stuhlmuller2014reasoning}) presuming that
we use a method that provides an unbiased estimate of the marginal likelihood and weight the trace appropriately.  Using the notation
from our earlier example we have here that $p(D|y)=p(D)$ and thus the normalization constant
of the nested query can just be factored into that of the outer query as it is no longer parameter
dependent -- the nested query now defines the same distribution as if conditional query were replaced
simply with a \defm.  We can therefore actually view inference on such problems as taking a pseudo-marginal approach, provided
that all samples returned from conditional are properly weighted~\cite{naessethLS2015nested} using the
product of their local weight and the marginal likelihood estimate from whatever process was used to generate them.
The correctness of this approach clearly follows from a combination of Theorems~\ref{the:finite-res} and \ref{the:prod}.
However, the probabilistic semantics of \conditional need to be changed appropriately, such that it
both produces a sample and simultaneously weights the trace using the unnormalized weight of the returned
particle in the inner query.  For example we could redefine \conditional as following
\todo[inline]{Add this, using a do block with a sampling and an observation of the probability}

Another special case is problems where the variables passed to the inner query can only take on, say $C$, finite
possible values. As
explained in Section~\ref{sec:linear_case}, such problems can always be rearranged to
$C$ separate estimators such that the standard Monte Carlo error rate can be achieved.
For repeated nesting, this rearrangement can be recursively applied until one achieves
a complete set of non-nested estimators.  This special case is one which requires rearrangement
or a specialist consideration by the language back-end (as done by \cite{stuhlmuller2012dynamic,stuhlmuller2014reasoning,cornish2017efficient}),
with na\"{i}ve implementations leading to NMC convergence rates.  It emphasizes that
nested problems with continuous variables are a fundamentally different problem
class to those with only discrete bounded variables (or only continuous variables at the bottom layer)
and we must therefore be wary of presuming that results from the discrete case carry over.
Nonetheless, in the same way it is often necessary to use Monte Carlo rather than enumeration
for inference in discrete problems because there are too many possible outcomes to enumerate, 
it will still often be necessary from a practical perspective
to use NMC for discrete problems. Ideally one should then do this in a manner that identifies and exploits the fact that
the inner estimator is often called multiple times with the same inputs such that the samples
from these calling can be combined (e.g. as done by~\cite{stuhlmuller2012dynamic}).

\subsection{Nested Observation}
\label{sec:design:imp:obs}

Using the BOPP transform to "observe" a program

\subsection{Estimates as Variables}
\label{sec:design:imp:expt}



\todo[inline]{Voting probprog example?}

%We have shown that it is theoretically possible for a nested Monte Carlo scheme to yield a
%consistent estimator, and have quantified the convergence error associated with doing so.
%However, we have also revealed a number of pitfalls that can arise if nesting is applied
%na\"{i}vely, such as the resulting estimator becoming necessarily biased, requiring additional
%assumptions on $f$, being unlikely to converge unless the number of samples used in the inner
%estimator is driven to infinity,
%and is likely to converge at a significantly slower rate than un-nested Monte
%Carlo. These results have implications for applications ranging from experimental design
%to probabilistic programming, and serve both as an invitation for further inquiry and a
%caveat against careless use.

%We have shown that although consistent nested inference is still possible, it is inherently biased, requires Lipshitz continuity, and has a convergence rate that reduces exponentially in the nesting depth.  
%These results have implications for many applications such as experimental design and probabilistic programming.
%For the latter, it shows that when there is only a linear dependence on the nested query, the problem can be unravelled to a single inference, but that otherwise there are additional severe restrictions on the problems that can be solved and the performance that can be achieved.
%When there is only a linear dependence of the outer integration on the nested expectation, the problem can be unravelled to a single inference, therefore, although nested queries in probabilistic programs do increase the scope of models which can be defined, these models have continuity restrictions and only permit MC inference at prohibitively slower convergence rates than ordinary inference.

% !TEX root =  main.tex

\section{Background}
\label{sec:intro}

%Although interesting alternatives have recently been suggested \cite{briol2015probabilistic}, \mc integration is almost exclusively used to calculate the expectation, given the generated samples.
%the method used in practise for calculating these expectations, given the generated samples, is almost exclusively \mc integration.  
%The calculation of expectations using \mc can be considered intertwined with that of \mc inference.  
%After all, convergence rates that are often quoted for \mc inference schemes, actually correspond to the convergence of the final \mc integration estimate.

There are various problems involving nested expectations that require the use of nested
estimation schemes such as NMC. For example, the expected information gain used in
Bayesian experimental design requires
the calculation of an entropy of a marginal distribution (see Chapter~\ref{chp:design}), and therefore includes the
expectation of the logarithm of an expectation.  By extension, any Kullback-Leibler
divergence where one of the terms is a marginal distribution also involves a nested expectation.  Hence, our results have important implications for relaxing mean field assumptions in variational
inference \citep{hoffman2015stochastic,naesseth2017variational,maddison2017filtering} and deep generative models
\citep{burda2015importance,maaloe2016auxiliary,le2017auto}.
%Here the nonlinearity provided by the logarithm prevents a simple reformulation to a
%single expectation in the general case, and thus presents conventional \mc estimation.
Another common nested estimation scenario is calculating expectations with respect to so-called
doubly-intractable distributions~\citep{moller2006efficient,murray2006mcmc,liang2010double}, whereby
the target distribution is only known up to a parameter-dependent normalizing constant.  Here
the normalization constant itself represents an intractable expectation
nested within the overall expectation.
NMC can also arise in contexts demanding the use
of direct \mc simulations, for example in portfolio risk management
\citep{gordy2010nested} and stochastic control \citep{belomestny2010regression}. 
In particular, 
simulations of agents that reason about decisions of other agents or which
reason about sequential decision making processes tend to include nested expectations.
Thus problems like planning in games and ``theory-of-mind'' type meta-reasoning
tend to require some form of nested
 estimation~\citep{stuhlmuller2014reasoning,evans2017models}.

Certain nested estimation problems can be tackled by so-called pseudo-marginal methods
\citep{beaumont2003estimation,andrieu2009pseudo,andrieu2010particle,
	andrieu2015convergence,naessethLS2015nested}.
These consider cases of Bayesian inference where the likelihood is intractable, 
%such as
%when it originates from an Approximate Bayesian Computation (ABC)
%\citep{csillery2010approximate}, 
but can be estimated unbiasedly.
% or when sequential Monte Carlo \cite{smith2013sequential} is used to approximate a high
% dimensional distribution.  
From a theoretical perspective, they involve reformulating the problem in an extended space with auxiliary variables that
are used to represent the stochasticity in the likelihood computation. This then enables the
problem to be expressed as a single expectation.
Our work goes beyond this by also considering cases in which a non-linear mapping is
applied to the output of the inner expectation (such as the logarithm in the 
experimental design case or the inverse for general doubly-intractable distributions), so that 
this reformulation to a single expectation is no longer possible.

NMC with non-linear mappings of the inner expectation has been previously considered in
the financial statistics literature, for example in the pricing of American
options \citep{longstaff2001valuing}. Though most of this literature focuses on
particular application-specific non-linear mappings \citep{broadie2011efficient,gordy2010nested},
convergence bounds for a more general class of models
has been shown by \citet{hong2009estimating}.
We build on these results and outline the opportunities and pitfalls of nesting Monte Carlo
estimators in a machine learning context.
Our proofs apply to a more general class of problems than existing approaches.
For example, we provide the first convergence bounds for cases of multiple levels of estimator nesting
as might occur in a probabilistic programming system.
We further lay out novel methods for reformulating certain classes of nested expectation problems
into a single expectation, allowing the use of conventional \mc estimation 
schemes with superior convergence rates than the na\"{i}ve use of NMC, one of which we
exploit in Chapter~\ref{chp:design} to derive an improved estimator for discrete
Bayesian experimental design problems.
Meanwhile, we provide theoretical results showing that any \mc
estimator using imperfect nested estimates is, in general, biased.  Finally we provide numerical
experiments, corroborating our theoretical results.

%\tom{Policy search?}

%We now arrive at the crux of this paper: we aim to establish in what nested inference scenarios one can guarentee convergence; we demonstrate that even when a general purpose scheme convergences, it must be biased; and we provide upper bounds on the convergence rate that such a scheme can achieve, showing that it decreases exponentially with the depth of nesting.


% !TEX root =  main.tex

\section{Special Cases}
\label{sec:special_cases}

We begin by discussing some special cases where it is possible to achieve a
convergence rate of $O(1/N)$ in the mean square error (MSE) as per conventional
MC estimation \citep{robert2004monte}.  
Establishing these cases is important because it identifies what problems we can use existing results for,
when we can achieve an improved convergence rate, and what precautions we must take to ensure this.
The first case is that the top-level integrand $f$ in our estimation problem is linear. 
Section~\ref{sec:linear_case} summarizes the well-known result in this case, which forms the basis for pseudo-marginal, 
nested sequential MC \citep{andersson2015nested}, and ABC methods. The next is that $y$ has finite possible realizations.
The result for this case is given in Section~\ref{sec:discrete}. Though intuitively straight-forward, it 
requires significant care to prove. The third case in Section~\ref{sec:products} is concerned with the product of expectations.
Our result for this case is, to the best of our knowledge, new and applies to many latent variable models and a number 
of probabilistic program examples, e.g. when the probability of a program trace
is weighted by marginal likelihood estimates from other programs.
The last case is that the integrand $f$ is a polynomial (Section~\ref{sec:polynomial}). It 
covers cases such as moment estimation and opens up interesting possibilities in using tractable 
approximations to the integrand $f$.

\subsection{Linear $f$}
\label{sec:linear_case}

Suppose that $f$ 
is integrable and linear in its second argument, i.e. $f(y,\alpha v + \beta w) = 
\alpha f(y,v)+ \beta f(y,w)$.
%or equivalently $f(y,v) = g(y)v$ where
%$g(y)$ can be calculated without an approximate integration.
In this case, we can rearrange the problem to a single expectation:
\begin{align*}
I
& = \mathbb{E}_{y \sim p(y)}\left[f(y,\gamma(y))\right] = \mathbb{E}_{y \sim p(y)}\left[f\left(y,\mathbb{E}_{z\sim p(z|y)}\left[\phi(y,z)\right]\right)\right]
\\
& = \mathbb{E}_{y \sim p(y)}\left[ \mathbb{E}_{z\sim p(z|y)}\left[f(y,\phi(y,z))\right]\right]
 \approx\frac{1}{N} \sum_{n=1}^{N} f(y_n,\phi(y_n,z_n))
\end{align*}
where $(y_n, z_n) \sim p(y)p(z|y)$ if $\gamma(y)$ is of the form of~\eqref{eq:gamma_1} and
$y_n \sim p(y)$ and $z_n \sim p(z)$ are independently drawn if $\gamma(y)$ is of the form of~\eqref{eq:gamma_2}.

% or the use of nested queries \cite{goodman2008church,rainforth2016nips} in probabilistic
% programming when the outer query does not depend on any nonlinear functional mapping of
% the marginal probability from the inner query.  
%In these scenarios, nesting provides a convenient means of expressing the problem, but is
%not a fundamental component of its solution.

%Another important case occurs when $\gamma(y)$ is independent of $y$. \todo{Isn't this
%just saying $\gamma$ is constant?}  In this case, we can estimate $\tilde{\gamma} =
%\gamma(y)$ and $(I(f) \,|\, \tilde{\gamma})$ separately, each converging at
%$O(1/\sqrt{N})$, presuming that the same number of samples $N$ is used for both. Provided
%that $f(y,v)$ is Lipschitz continuous in $v$, the root mean squared error of the
%estimation for $I(f)$ will also have same convergence rate.
%\footnote{The proof for this
%follows the same lines as used to link the error in $\gamma$ to $f$ in
%Theorem~\ref{the:Consistent}.} \todo{Do we want this footnote?}

% !TEX root =  ../main.tex

\subsection{Finite Possible Realizations of $y$}
\label{sec:discrete}

If $y$ must take one of finitely many values $y_1, \cdots, y_C$, then it is possible to
use another approach to ensure the same convergence rate as standard MC.
The key observation is to note that in this case we can convert the nested problem
\eqref{eq:MC} into $C$ separate non-nested problems
\begin{align}
         I = \sum_{c=1}^C P(y = y_c) \, f(y_c, \gamma(y_c))
\end{align}
which can then be estimated using
\begin{align}
        I_N  &= \sum_{c=1}^C (\hat{P}_N)_c \, (\hat{f}_N)_c \quad \text{where}
        \label{eq:IN} \\
        \label{eq:PN}
        P(y = y_c) &\approx (\hat{P}_N)_c = \frac{1}{N} \sum_{n=1}^N \mathbbm{1}(y_n = y_c)  \\
        \label{eq:fn}
        f(y_c, \gamma(y_c)) &\approx (\hat{f}_N)_c = f\left(y_c, \frac{1}{N} \sum_{n=1}^N \phi(y_c, z_{n,c})\right),
\end{align}
with $y_n \iid p(y)$ and $z_{n,c} \sim p(z|y_c)$ or $z_{n,c} \sim p(z)$  
depending on whether our formulation uses
\eqref{eq:gamma_1} or \eqref{eq:gamma_2}.  
We can now show the following result, noting that as the same set of $y_n$'s is used for every
$(\hat{P}_N)_c$, calculating $I_N$ requires $N$ samples of $y$, 
$CN$ samples of $z$ (each
drawn independently to the samples of $y$), $CN$ evaluations of $\phi$, and $C$ evaluations of $f$.
\begin{restatable}{theorem}{thefiniteres}
	\label{the:finite-res}
  If $f$ is Lipschitz continuous, then the mean squared error of $I_N  = \sum_{c=1}^C (\hat{P}_N)_c \, (\hat{f}_N)_c$ 
  as an estimator for $I= \sum_{c=1}^C P(y = y_c) \, f(y_c, \gamma(y_c))$ converges at rate $O(1/N)$.
\end{restatable}
%
%
%If there are only finite possible values which $y$ can take on, there is another approach
%we can take to ensure standard MC results apply.  Unlike the previous special cases, here
%it will be necessary to carry out the estimation in a particular way for the standard
%MC results to apply, with na\"{i}ve application of~\eqref{eq:nested-mc} reverting the 
%problem to the more general NMC case discussed in the next section.  The key realisation
%is to note that if there are only finite, say $T$, possible realisations of $y$, then we
%can convert the single nested estimation problem into $T$ separate un-nested problems
%\begin{align*}
%I
%& = \sum_{t=1}^{T} P(y=y_t) f(y_t, \gamma(y_t)).
%\end{align*}
%Here the calculation of $I$ is deterministic given $\left\{\gamma (y_t)\right\}_{t=1:T}$
%and so, assuming appropriate continuity on $f$, we can bound the mean square
%error as 
%\begin{align*}
%	\norm{I - I_{N,M}}_2^2 & \le K \sum_{t=1}^T \left\lVert\gamma (y_{t}) - \frac{1}{M}  \sum_{m=1}^{M}  \phi(y_{t},z_{{t},m})\right\rVert_2^2 \\
%	& \le KT \max_{t\in1:T} \left\lVert\gamma (y_{t}) - \frac{1}{M}  \sum_{m=1}^{M}  \phi(y_{t},z_{t,m})\right\rVert_2^2
%\end{align*}
%where $K$ is an unknown constant associated with mapping the error through $f$ and the
%multiplication by $P(y=y_{t})$.
%Here our error is a constant times the largest error for the finite number of inner estimators, 
%and so we can bound our error using the MC error from this estimator, which is $O(1/M)$,
%noting that this is independent of $N$.
%
%We can also think about this approach as grouping the samples from each evaluation of the
%inner estimator, exploiting the fact that we are only evaluating $T$ distinct inner estimations.
%This is why na\"{i}ve application of~\eqref{eq:nested-mc} is not sufficient, as this involves
%no sharing of the inner samples $z_{n,m}$.


\subsection{Products of Expectations}
\label{sec:products}

In this section we consider the scenario  where $\gamma (y)$ is equal to the product of 
multiple expectations, rather than just a single expectation as per~\eqref{eq:gamma_1}.
That is,
\begin{equation}
\label{eq:prod-mc}
I = \mathbb{E}_{y\sim p(y)}\left[ f\left(y,\prod_{\ell=1}^{L}\mathbb{E}_{z_{\ell}\sim p(z_{\ell}|y)}\left[\psi_{\ell}(y,z_{\ell})\right]\right) \right].
\end{equation}
This product of expectations can be rearranged to a single expectation and so the estimation of $I$ 
corresponds to the formulation laid out in Section~\ref{sec:prob-form}. To see this, consider the set of 
random variables $\{z'_{\ell}\}_{\ell=1:L}$ such that each 
$z'_{\ell}|y \sim p(z_\ell | y)$ and the $z'_{\ell}$ are independent of one another. 
This can be achieved by, for example, taking $L$ independent samples from the joint $Z_{\ell} \iid p(z_{1:L} | y)$ 
and using the $\ell^{\mathrm{th}}$ such draw for the $\ell^{\mathrm{th}}$ dimension of $z'$, i.e.
setting $z'_{\ell}= \{Z_{\ell}\}_{\ell}$.
For every $y \in \mathcal{Y}$ we now have
%Consider the set of random variables $\{Z_{\ell}\}_{\ell=1:L}$  such that each $Z_{\ell} \sim p(z_\ell | y)$,
%noting that by definition, the $Z_{\ell}$ are independent of one another.
%Now define the set of variables  $\{z'_{\ell}\}_{\ell=1:L}$ such that $z'_{\ell}= \{Z_{\ell}\}_{\ell}$.
%We can now see that each $z'_{\ell} | y \sim p(z_{\ell} | y)$, but unlike $z_{\ell}$, the $z'_{\ell}$ must be independent of one another and thus
\begin{align*}
 \prod_{\ell=1}^{L}&\mathbb{E}_{z_{\ell}\sim p(z_{\ell}|y)}[\psi_{\ell}(y,z_{\ell})] =  \prod_{\ell=1}^{L}\mathbb{E}_{z'_{\ell}\sim p(z'_{\ell}|y)}[\psi_{\ell}(y,z'_{\ell})]
  = \mathbb{E}_{\{z'_{\ell}\}_{\ell=1:L} \sim p(\{z'_{\ell}\}_{\ell=1:L}|y)}\left[ \prod_{\ell=1}^{L} \psi_{\ell}(y,z'_{\ell}) \right]
\end{align*}
which is a single expectation on an extended space and implies that \eqref{eq:prod-mc} 
fits the NMC formulation.
Furthermore, we can now show that if $f$ is linear, the MSE of the NMC estimator~\eqref{eq:prod-mc}
converges at the standard MC rate $O(1/N)$, provided that 
the number of samples used in the inner estimators remains fixed.  
%More formally we can state the following result.

\begin{restatable}{theorem}{theprod}
	\label{the:prod}
	Consider the NMC estimator
	\begin{align*}
	I_{N} &= \frac{1}{N}\sum_{n=1}^N f\left(y_n,\prod_{\ell=1}^{L} \frac{1}{M_{\ell}} \sum_{m=1}^{M_{\ell}} \psi_{\ell}(y_n,z_{n,\ell,m}')\right)
	%&= \frac{1}{N}\sum_{n=1}^N \prod_{\ell=1}^{L} \frac{1}{M_{\ell}} \sum_{m=1}^{M_{\ell}} f(y_n,\psi_{\ell}(y_n,z_{n,\ell,m}')) \\
	\end{align*}
	where each $y_n \in \mathcal{Y}$ and $z_{n,\ell,m}' \in \mathcal{Z}_{\ell}$ are independently drawn from 
	$y_n \sim p(y)$ and $z_{n,\ell,m}' | y_n \sim p(z_{\ell} | y_n)$, respectively. If $f$ is linear, 
        the estimator converges almost surely to $I$,
	with a convergence rate of $O(1/N)$ in the mean square error for any fixed choice of $\{M_{\ell}\}_{\ell = 1:L}$.
\end{restatable}
\begin{proof}
See Appendix~\ref{sec:app-prod}.
\end{proof}

As this result holds in the case $L=1$, an important consequence is that
whenever $f$ is linear, the same convergence rate is achieved regardless of whether we
reformulate the problem to a single expectation or not, provided that the number of samples
used by the inner estimator remains fixed.  Conversely, as we will show in Section~\ref{sec:convergence},
if $f$ is nonlinear, it will be necessary for the number of samples in \emph{both} the inner and the 
outer estimators to increase in order to achieve the convergence of~\eqref{eq:nested-mc}. 

\subsection{Polynomial $f$}
\label{sec:polynomial}

Perhaps surprisingly, when $f$ is of the form
\begin{align}
  f(y,\gamma(y)) &= \sum_{j=1}^{J} g_j(y) \, \gamma_j(y)^{\alpha_j} \quad \text{where}
  \quad \alpha_j \in \mathbb{Z}_{\geq 0}, \; \forall j \in \{1,\dots,J\}
\end{align}
then it is also be possible to construct a standard MC estimator by building on the ideas
introduced in Section~\ref{sec:products} and those introduced by \cite{goda2016computing}.
The key idea is to note that
\begin{align*}
\left(\E \left[z\right]\right)^2 = \E \left[z\right]\E \left[z'\right] = \E \left[z z'\right]
\end{align*}
where $z$ and $z'$ are i.i.d.  Therefore, assuming appropriate integrability requirements,
we can construct the following non-nested MC estimator:
\begin{align}
\E\left[\sum_{j=1}^{J} g_j(y) \, \gamma_j(y)^{\alpha_j}\right] &= 
\sum_{j=1}^{J} \E\left[g_j(y) \, \E\left[\phi(y,z_{j}) | y\right]^{\alpha_j}\right]
=\sum_{j=1}^{J} \E\left[g_j(y) \E\left[\prod_{\ell=1}^{\alpha_j} \phi(y,z_{j,\ell}) \middle|y\right]\right] \nonumber \\
=\sum_{j=1}^{J} \E&\left[g_j(y)\prod_{\ell=1}^{\alpha_j} \phi(y,z_{j,\ell})\right]
\approx \frac{1}{N} \sum_{n=1}^{N} \sum_{j=1}^{J} g(y_n)\prod_{\ell=1}^{\alpha_j} \phi(y_n,z_{j,\ell,n}).
\end{align}
Here each $z_{j,\ell,n} \sim p(z_j | y)$ is drawn i.i.d. from the corresponding conditional distribution.

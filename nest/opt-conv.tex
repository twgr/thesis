% !TEX root =  ../main.tex

\section{Optimizing the Convergence Rates}
\label{sec:app:opt-conv}

We have shown that the mean squared error converges at a rate $O(1/N + 1/M)$ or
$O(1/N + 1/M^2)$ depending on the smoothness assumptions that can be made about
$f$.  Here we show that given a sample budget $T=MN$, then these bounds are
optimized by setting $N \propto M$ and $N \propto M^2$ respectively and that this
gives bounds of $O(1/\sqrt{T})$ and $O(1/T^{2/3})$ respectively.

Consider setting $N = \tau(M)$ then $T = \tau(M) \cdot M$ and our bounds become $O(R)$, where
\[
  R = 1/\tau(M) + 1/M \quad \text{and} \quad R = 1/\tau(M) + 1/M^2.
\]
for the two cases respectively.

In this first case supposing $\tau(M) = O(M)$ easily gives
\[
  T = M \tau(M)
    = O\left(M^2\right)
\]
and as such 
\begin{equation} \label{eq:rm}
  R = O\left(\frac{1}{M}\right) = O\left(\frac{1}{\sqrt{T}}\right)
\end{equation}
as $M \to \infty$.  In contrast, consider the case that $\tau(M) \gg M$ as $M \to \infty$. We then have
$\frac{1}{\sqrt{M}} \gg \frac{1}{\sqrt{\tau(M)}}$ as $M \to \infty$, so that
\[
  R = O\left(\frac{1}{M}\right) \gg \frac{1}{\sqrt{M}} \frac{1}{\sqrt{\tau(M)}} = \frac{1}{\sqrt{T}}.
\]
Comparing with \eqref{eq:rm}, we observe that, for the same total
budget of samples $T$, this choice of $\tau$ provides a strictly weaker convergence
guarantee than in the previous case. When $M \gg \tau(M)$ also then we have
$\frac{1}{\sqrt{\tau(M)}} \gg \frac{1}{\sqrt{M}}$ as $M \to \infty$ and so
\[
R = O\left(\frac{1}{\tau(M)}\right) \gg \frac{1}{\sqrt{M}} \frac{1}{\sqrt{\tau(M)}} = \frac{1}{\sqrt{T}}
\]
which is again a weaker bound.  We thus see that the $O(1/N + 1/M)$ bound is optimized when
$N \propto M$, giving a convergence rate of $O(1/\sqrt{T})$.

In the second case suppose that $\tau(M) = O(M^2)$ as $M \to \infty$.  This now gives
\[
  T = M \tau(M)
  = O\left(M^3\right)
\]
and therefore
\[
R = O\left(\frac{1}{M^2}\right) = O\left(\frac{1}{T^{2/3}}\right)
\]
as $M \to \infty$.  Now considering the cases $\tau(M) \gg M^2$ leads to $\frac{1}{M^{4/3}}  \gg \frac{1}{\tau(M)^{2/3}}$ and thus
\[
R = O\left(\frac{1}{M^2}\right) \gg \frac{1}{M^{2/3}} \frac{1}{\tau(M)^{2/3}} = \frac{1}{T^{2/3}}.
\]
Similarly, if $\tau(M) \ll M^2$ then $\frac{1}{\tau(M)^{1/3}}\gg \frac{1}{M^{2/3}}$ and thus
\[
R = O\left(\frac{1}{\tau(M)}\right) \gg \frac{1}{M^{2/3}} \frac{1}{\tau(M)^{2/3}} = \frac{1}{T^{2/3}}.
\]
Both of these cases lead to weaker bounds and so we see that the $O(1/N + 1/M^2)$ bound
is tightest when $N \propto M^2$, giving a convergence rate of $O(1/T^{2/3})$.
% !TEX root =  main.tex

\subsection{`Almost Almost Sure'' Convergence}
\label{sec:app:consistent}

\theConsistent*

\begin{proof}
	For all $N, M$, we have by the triangle inequality that
		\begin{align*}
	\left|I_{N,M} - I\right| \leq V_{N,M} + U_N, &\quad \text{where} \\
		V_{N,M} = \left|\frac{1}{N} \sum_{n=1}^N f(y_n, \gamma(y_n)) - I_{N,M} \right| 
		\quad \text{and}& \quad
		U_N = \left|I - \frac{1}{N} \sum_{n=1}^N f(y_n, \gamma(y_n)) \right|.
	\end{align*}
	A second application of the triangle inequality then allows us to write
	\[
    V_{N,M} \leq \frac{1}{N} \sum_{n=1}^N (\epsilon_M)_n
	\]
	where we recall that $(\epsilon_M)_n = |f(y_n, \gamma(y_n)) - f(y_n, \hat{\gamma}_n)|$.
	Now, for all fixed $M$, each $(\epsilon_M)_n$ is i.i.d. Furthermore, since
        $\E
	\left[(\epsilon_M)_1\right] \to 0$ as $M \to \infty$ by our assumption and 
	$(\epsilon_M)_n$ is nonnegative, there exists some $L \in \N$
	such that $\E\left[\left|(\epsilon_M)_n \right|\right] < \infty$ for all $M \geq L$.
	Consequently, the strong law of large numbers means that as $N \to \infty$ then for all $M \geq L$
	\begin{align}
	\label{eq:epas}
	\frac{1}{N} \sum_{n=1}^N (\epsilon_M)_n \asto \E\left[(\epsilon_M)_1\right].
	\end{align}      
        For any fixed $\delta > 0$ then by repeatedly applying Egorov's theorem to each $M \geq L$,
        we can find a sequence of events $B_{L}, B_{L+1}, B_{L+2}, \ldots$
        such that for every $M \geq L$,
        \[
                \mathbb{P}(B_{M}) < \frac{\delta}{4} \cdot \frac{1}{2^{M-L}}
        \]
        and outside of $B_{M}$, the sequence $\frac{1}{N} \sum_{n=1}^N (\epsilon_M)_n$
        converges \emph{uniformly} to $\E\left[(\epsilon_M)_1\right]$. 
        This uniform convergence (as opposed to just the piecewise convergence implied
        by~\eqref{eq:epas}) now guarantees that we can define some function
        $\tau^1_\delta : \N \to \N$ such that 
	\begin{align}
	\label{eq:ombo}
                \left|\frac{1}{M'} \sum_{n=1}^{M'} (\epsilon_M)_n(\omega) - \E\left[(\epsilon_M)_1\right]\right| < \frac{1}{M} 
        \end{align}
        for all $M \geq L$, $M' \geq \tau^1_\delta(M)$, and $\omega \not\in B_M$, remembering that $\omega$ is
        a point in our sample space.  We further have that~\eqref{eq:ombo} holds for all 
        $M \geq M_0$, $M' \geq \tau^1_\delta(M)$, and $\omega \not\in B_\delta := \bigcup_{M \geq L} B_M$.
%        Such a $\tau^1_\delta$ exists
%        because of the uniform convergence guarantee that we just mentioned. 
%        Defining 
%        \[
%                B_\delta := \bigcup_{M \geq L} B_M,
%        \]
%        then~\eqref{eq:ombo} hold for all $M \geq M_0$, $M' \geq \tau^1_\delta(M)$, and $\omega \not\in B_\delta$.
        Consequently, for all such $M$, $M'$ and $\omega$,
        \begin{equation} 
                \label{eqn:almostsure:1}
                V_{M',M}(\omega)
                \leq
                \frac{1}{M'} \sum_{n=1}^{M'} (\epsilon_M)_n(\omega) 
                < 
                \frac{1}{M} + \E\left[(\epsilon_M)_1\right],
        \end{equation}
        while we also have
        \begin{equation} 
                \label{eqn:almostsure:2}
                \mathbb{P}(B_\delta) 
                %=\mathbb{P}\left(\bigcup_{M \geq L} B_M\right)
                \leq 
                \sum_{M \geq L} \mathbb{P}\left(B_M\right)
                < \sum_{M \geq L} \frac{\delta}{4} \cdot \frac{1}{2^{M-L}}
                = \frac{\delta}{2}.
        \end{equation}
      To complete the proof, we must remove the dependence of $U_N$ on $N$ as well. This is
	straightforward once we observe that $U_N \asto 0$ as $N \to \infty$ by the strong law of 
        large numbers. So, by Egorov's theorem again, there exists an event $C_\delta$ such that
        \begin{equation} 
                \label{eqn:almostsure:3}
                \mathbb{P}(C_\delta) < \frac{\delta}{2}
        \end{equation}
        and outside of $C_\delta$, the sequence $U_N$ converges uniformly to $0$. This uniform convergence,
        in turn, ensures the existence of a function $\tau^2_\delta : \N \to \N$ such that
        \begin{equation} 
                \label{eqn:almostsure:4}
                U_{M'}(\omega) < \frac{1}{M} 
        \end{equation}
        for all $M \in \N$, $M' \geq \tau^2_\delta(M)$, and $\omega \not\in C_\delta$.
	
        We can now define $\tau_\delta(M) = \max(\tau^1_\delta(M), \tau^2_\delta(M))$, and 
        $A_\delta = B_\delta \cup C_\delta$. By the inequalities in \eqref{eqn:almostsure:2}
        and \eqref{eqn:almostsure:3},
        \[ 
                \mathbb{P}(A_\delta) \leq \mathbb{P}(B_\delta) + \mathbb{P}(C_\delta) < \delta.  
        \] 
        Also, by the inequalities in \eqref{eqn:almostsure:1} and \eqref{eqn:almostsure:4},
	\[ 
                \left|I - I_{\tau_\delta(M),M}(\omega)\right| 
                \leq
                V_{\tau_\delta(M),M}(\omega) 
                +
                U_{\tau_\delta(M)}(\omega) 
                \leq 
                \frac{1}{M} + \frac{1}{M} + \E\left[(\epsilon_M)_1\right]
	\]
        for all $M \geq L$ and $\omega \notin A_\delta$. Since $\E\left[(\epsilon_M)_1\right] \to 0$, we have that 
        $I_{\tau_\delta(M),M}(\omega) \to I$ as desired.
\end{proof}

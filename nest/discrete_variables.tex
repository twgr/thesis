% !TEX root =  ../main.tex

\subsection{Finite Possible Realizations of $y$}
\label{sec:discrete}

If $y$ must take one of finitely many values $y_1, \cdots, y_C$, then it is possible to
use another approach to ensure the same convergence rate as standard \mc.
The key observation is to note that in this case we can convert the nested problem
\eqref{eq:MC} into $C$ separate non-nested problems
\begin{align}
         I = \sum_{c=1}^C P(y = y_c) \, f(y_c, \gamma(y_c))
\end{align}
which can then be estimated using
\begin{align}
        I_N  &= \sum_{c=1}^C (\hat{P}_N)_c \, (\hat{f}_N)_c \quad \text{where}
        \label{eq:IN} \\
        \label{eq:PN}
        P(y = y_c) &\approx (\hat{P}_N)_c = \frac{1}{N} \sum_{n=1}^N \mathbbm{1}(y_n = y_c)  \\
        \label{eq:fn}
        f(y_c, \gamma(y_c)) &\approx (\hat{f}_N)_c = f\left(y_c, \frac{1}{N} \sum_{n=1}^N \phi(y_c, z_{n,c})\right),
\end{align}
with $y_n \iid p(y)$ and $z_{n,c} \sim p(z|y_c)$ or $z_{n,c} \sim p(z)$  
depending on whether our formulation uses
\eqref{eq:gamma_1} or \eqref{eq:gamma_2}.  
We can now show the following result, noting that as the same set of $y_n$'s is used for every
$(\hat{P}_N)_c$, calculating $I_N$ requires $N$ samples of $y$, 
$CN$ samples of $z$ (each
drawn independently to the samples of $y$), $CN$ evaluations of $\phi$, and $C$ evaluations of $f$.
\vspace{-5pt}
\begin{restatable}{theorem}{thefiniteres}
	\label{the:finite-res}
  If $f$ is Lipschitz continuous, then the mean squared error of $I_N  = \sum_{c=1}^C (\hat{P}_N)_c \, (\hat{f}_N)_c$ 
  as an estimator for $I= \sum_{c=1}^C P(y = y_c) \, f(y_c, \gamma(y_c))$ converges at rate $O(1/N)$.
\end{restatable}
\vspace{-12pt}
\begin{proof}
	See~\citep{rainforth2017pitfalls}.
\end{proof}
%
%
%If there are only finite possible values which $y$ can take on, there is another approach
%we can take to ensure standard \mc results apply.  Unlike the previous special cases, here
%it will be necessary to carry out the estimation in a particular way for the standard
%\mc results to apply, with na\"{i}ve application of~\eqref{eq:nested-mc} reverting the 
%problem to the more general NMC case discussed in the next section.  The key realisation
%is to note that if there are only finite, say $T$, possible realisations of $y$, then we
%can convert the single nested estimation problem into $T$ separate un-nested problems
%\begin{align*}
%I
%& = \sum_{t=1}^{T} P(y=y_t) f(y_t, \gamma(y_t)).
%\end{align*}
%Here the calculation of $I$ is deterministic given $\left\{\gamma (y_t)\right\}_{t=1:T}$
%and so, assuming appropriate continuity on $f$, we can bound the mean square
%error as 
%\begin{align*}
%	\norm{I - I_{N,M}}_2^2 & \le K \sum_{t=1}^T \left\lVert\gamma (y_{t}) - \frac{1}{M}  \sum_{m=1}^{M}  \phi(y_{t},z_{{t},m})\right\rVert_2^2 \\
%	& \le KT \max_{t\in1:T} \left\lVert\gamma (y_{t}) - \frac{1}{M}  \sum_{m=1}^{M}  \phi(y_{t},z_{t,m})\right\rVert_2^2
%\end{align*}
%where $K$ is an unknown constant associated with mapping the error through $f$ and the
%multiplication by $P(y=y_{t})$.
%Here our error is a constant times the largest error for the finite number of inner estimators, 
%and so we can bound our error using the \mc error from this estimator, which is $O(1/M)$,
%noting that this is independent of $N$.
%
%We can also think about this approach as grouping the samples from each evaluation of the
%inner estimator, exploiting the fact that we are only evaluating $T$ distinct inner estimations.
%This is why na\"{i}ve application of~\eqref{eq:nested-mc} is not sufficient, as this involves
%no sharing of the inner samples $z_{n,m}$.

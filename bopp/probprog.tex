% !TEX root =  bopp.tex

\subsection{Probabilistic Programming}
\label{sec:prob-prog}

Probabilistic programming systems allow users to define probabilistic models using a domain-specific programming language. A probabilistic program implicitly defines a distribution on random variables, whilst the system back-end implements general-purpose inference methods.  

PPS such as Infer.Net \citep{minka_software_2010} and Stan \citep{carpenter2015stan} can be thought of as defining graphical models or factor graphs.  Our focus will instead be on systems such as Church \citep{goodman_uai_2008}, Venture \citep{mansinghka2014venture}, WebPPL \citep{goodman_book_2014}, and Anglican \citep{wood2014new}, which employ a general-purpose programming language for model specification. In these systems, the set of random variables is dynamically typed, such that it is possible to write programs in which this set differs from execution to execution.  This allows an unspecified number of random variables and incorporation of arbitrary black box deterministic functions, such as was exploited by the \simulatec function in Figure \ref{fig:house-heating-code}. The price for this expressivity is that inference methods must be formulated in such a manner that they are applicable to models where the density function is intractable and can only be evaluated during forwards simulation of the program. 

One such general purpose system, \emph{Anglican}, will be used as a reference in this paper.  In Anglican, models are defined using the inference macro \defquery. These models, which we refer to as queries \citep{goodman_uai_2008}, specify a joint distribution $p(Y,X)$ over data $Y$ and variables $X$. Inference on the model is performed using the macro \doquery, which produces a sequence of approximate samples from the conditional distribution $p(X|Y)$ and, for importance sampling based inference algorithms (e.g. sequential Monte Carlo), a marginal likelihood estimate $p(Y)$.  

Random variables in an Anglican program are specified using \sample statements, which can be thought of as terms in the prior. Conditioning is specified using \observe statements which can be thought of as likelihood terms.  Outputs of the program, taking the form of posterior samples, are indicated by the return values.  There is a finite set of \sample and \observe statements in a program source code, but the number of times each statement is called can vary between executions.  We refer the reader to  \href{http://www.robots.ox.ac.uk/~fwood/anglican/}{\small\url{http://www.robots.ox.ac.uk/~fwood/anglican/}} for more details.
% !TEX root =  bopp.tex

In this case study, illustrated in Figure~\ref{fig:houses}, we optimize the parameters of a stochastic engineering simulation. We use the Energy2D system from \cite{xie2012energy2d} to perform finite-difference numerical simulation of the heat equation and Navier-Stokes equations in a user-defined geometry. 

In our setup, we designed a 2-dimensional representation of a house with 4 interconnected rooms using the GUI provided by Energy2D. The left side of the house receives morning sun, modelled at a constant incident angle of $30^\circ$. We assume a randomly distributed solar intensity and simulate the heating of a cold house in the morning by 4 radiators, one in each of the rooms. The radiators are given a fixed budget of total power density $P_{\text{budget}}$. The optimization problem is to distribute this power budget across radiators in a manner that minimizes the variance in temperatures across 8 locations in the house. 

Energy2D is written in Java, which allows the simulation to be integrated directly into an Anglican program that defines a prior on model parameters and an ABC likelihood for evaluating the utility of the simulation outputs. Figure \ref{fig:house-heating-code} shows the corresponding program query. In this, we define a Clojure function \lsi{simulate} that accepts a solar power intensity $I_{\text{sun}}$ and power densities for the radiators $P_{\text{r}}$, returning the thermometer temperature readings $\{T_{i, t}\}$. We place a symmetric Dirichlet prior on $\frac{P_r}{P_{\text{budget}}}$ and a gamma prior on $\frac{I_{\text{sun}}}{I_{base}}$, where $P_{\text{budget}}$ and $I_{base}$ are constants. This gives the generative model:
 \begin{align}
 p_r &\sim \Dirichlet([1,1,1,1]) \\
 P_r &\leftarrow P_{\text{budget}} \cdot p_r \\
 \upsilon &\sim \text{Gamma}(5,1) \\
 I_{\text{sun}} &\leftarrow I_{\text{base}} \cdot \upsilon.
 \end{align}
After using these to call \lsi{simulate}, the standard deviations of the returned temperatures is calculated for each time point,
\begin{align}
\omega_t = \sqrt{\sum_{i=1}^8 T_{i, t}^2 -\left(\sum_{i=1}^8 T_{i, t}\right)^2}
\end{align}
and used in the ABC likelihood \lsi{abc-likelihood} to weight the execution trace using a multivariate Gaussian:
\begin{align*}
p\left(\{T_{i, t}\}_{i = 1:8, t = 1:\tau}\right) = \text{Normal}\left(\omega_{t=1:\tau};\mathbf{0},\sigma_T^{2}\mathbf{I}\right)
\end{align*}
where $\mathbf{I}$ is the identity matrix and $\sigma_T = 0.8 ^{\circ} \mathrm{C}$ is the observation standard deviation.

Figure \ref{fig:houses} demonstrates the improvement in homogeneity of temperatures as a function of total number of simulation evaluations. Visual inspection of the heat distributions also shown in Figure \ref{fig:houses} confirms this result, which serves as an exemplar of how BOPP can be used to estimate marginally optimal simulation parameters.

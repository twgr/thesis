% !TEX root =  ../main.tex

So far in this thesis, and in the literature more generally, the focus has been on automating
inference for probabilistic programs.  However, as we explained in Section~\ref{sec:opt:intro},
optimization is also a very important tool in the machine learning toolbox.  Further, many problems
require concurrent inference and optimization in the form of marginal maximum a posteriori (MMAP) estimation,
maximum marginal likelihood (MML) estimation, or risk minimization.  In this chapter, we develop a probabilistic programming framework
for automating the solution of these mixed inference-optimization problems.
We will focus for the most part on
MMAP estimation, highlighting differences for MML estimation and risk minimization when necessary.

MMAP estimation is challenging as it corresponds to the optimization of an intractable integral, such that the 
optimization target is expensive to evaluate and gives noisy results.  Current PPS inference engines are 
typically unsuited to such settings.  We therefore introduce BOPP\footnote{Code available at \url{http://www.github.com/probprog/bopp/}}
(Bayesian optimization for probabilistic programs)~\citep{rainforth2015workshopbopp,rainforth2016bayesian}, 
which uses a series of code transformations, existing inference algorithms,
and a bespoke Gaussian process (GP) based Bayesian optimization (BO) package, which we call Deodorant\footnote{Code available
at \url{http://www.github.com/probprog/deodorant/}}, to optimize the evidence of a query with respect to
an arbitrary subset of its internally sampled variables.\footnote{As we will explain later, there is no need to
	provide separate consideration for optimizing with respect to the inputs of the query as, if required,
	variables can be trivially internalized.}

BOPP can be viewed from two different perspectives, providing distinct contributions to both the probabilistic
programming and Bayesian optimization literatures.  From a probabilistic programming perspective, we can
view BOPP as extending PPSs beyond their typical inference setting to a more
general mixed inference-optimization framework.  It allows users to specify a model in the same manner 
as existing systems, but then select some subset of the sampled variables in the query to be optimized, 
with the rest marginalized out using existing inference algorithms.  Though the exact code transformations we
introduce are inevitably language specific and have been implemented for Anglican, the concepts we
introduce apply more widely and the BO package itself, Deodorant, is not PPL specific and is provided
as a stand-alone package.  More generally, the \textit{optimization query} we 
introduce can be implemented and utilized in any PPL that supports an inference method returning a 
partition function estimate.  This framework increases the scope of models that can be expressed in
PPSs and gives additional flexibility in the outputs a user can request from a program.

From a BO perspective, we can view BOPP as the first package to directly exploit the source code of its target.
This leads to a number of novel contributions to the BO literature, such as innovations in 
problem-independent hyperpriors, unbounded optimization, and implicit constraint satisfaction.  
The ability to do these is rooted
in the ability provided by PPLs to manipulate the target source code and return artifacts suitable for carrying
out certain tasks on a model.  For example, we use a code transformation to produce a problem-specific
optimizer for the acquisition function that ensures the implicit constraints specified by the query,
including equality constraints not before considered by the BO literature, are automatically and exactly
satisfied.  As a consequence, BOPP provides a significant novel features and performance
improvements even when used simply as a conventional BO package.
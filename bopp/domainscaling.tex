% !TEX root =  ../main.tex

\subsection{Automatic and Adaptive Domain Scaling}
\label{sec:bopp:domain}

Domain scaling, by mapping to a common space, is crucial for BOPP to operate in the required black-box fashion as it allows a general purpose and problem independent hyperprior to be placed on the GP hyperparameters.  BOPP therefore employs an affine scaling to a $[-1,1]$ hypercube for both the inputs and outputs of the GPs.  To initialize scaling for the input variables, we sample directly from the generative model defined by the program. %\footnote{Note that Anglican's ability to include statements for conditioning on generated variables, for example to truncate distributions, means this does not always represent $p(\theta)$ and is only a prior in a more abstracted sense.}
This is achieved using a second transformed program, \qprior, which removes all conditioning, i.e. \observe statements, and returns $\theta$.  This transformation also introduces code to terminate execution of the query once all $\theta$ are sampled, in order to avoid unnecessary computation.
As \observe statements return \lsi{nil}, this transformation trivially preserves the generative model of the program, 
but the probability of the execution changes. Specifically, if we denote $n_{\theta}$ as the number of non-target \sample 
statements that have been invoked by the time all $\phi_{1:L}$ are sampled, then \qprior more formally defines the
\emph{unconditional} distribution $p_{\lambda}(\mT = \{\phi_{1:L},x_{1:n_{\theta}}\}) \propto 
\gamma_{\text{prior}}(\theta,x_{1:n_{\theta}},\lambda)$ where
\begin{align}
\label{eq:bopp:qprior}
\gamma_{\text{prior}}(\theta,x_{1:n_{\theta}},\lambda)= \begin{cases}
\prod_{\ell=1}^{L}
h_{\ell,c_{\ell}} (\phi_{\ell} | \xi_{\ell})
\prod_{j=1}^{n_{\theta}} 
f_{a_j}(x_j | \eta_j) \;\;\; \text{if} \;\;\; \mathcal{B}(\theta,x_{1:n_\theta},\lambda)=1 \\
0 \quad \text{otherwise}
\end{cases}
\end{align}
and the trace validity function $\mathcal{B}(\theta,x_{1:n_\theta},\lambda)$ is redefined appropriately.
Because~\eqref{eq:bopp:qprior} is an unconditional distribution, it can be sampled from directly by
running the program forwards, returning exact samples from the corresponding marginal distribution on $\theta$.
This is computationally inexpensive, as it does not require inference or calling potentially expensive 
likelihood functions.  It can thus be cheaply sampled from a number of times to initialize the input scaling.
By further running inference on \qmarg~given a small number of these samples as arguments, a rough initial characterization of output scaling can also be achieved and initial samples for the BO algorithm generated. 

If points are later observed that fall outside the hypercube under this initial scaling, the domain scaling 
is appropriately updated so that the target for the GP remains the $[-1,1]$ hypercube.  
An important exception to this is that the output mapping to the bottom of the hypercube remains 
fixed and any points with partition function estimates lower than this are not incorporated into the scaling in any way,
i.e. the input scaling is not updated to incorporate these points either.
For MMAP estimation, this ensures stability for unbounded problems as there can only be a finite region
of the input space where the true value of the partition function is above any given value because its integral
over $\theta$ must be finite.  Similarly, the
maximum possible estimate the inference algorithm might return will be bounded
given some weak assumptions (roughly that $p(Y,X,\theta)$ is itself bounded).
Consequently, the fixed base of the hypercube (as dictated by the initial samples) ensures that the 
there is a maximum possible size the hypercube can reach.
For risk minimization (where our target is $-\log p(Y,\theta)$) and  MML estimation
(where our target is $\log p(Y|\theta)$) problems then we have no such guarantee that the adaptation will 
eventually cease.  However, this is somewhat inherent to unbounded global optimization problems,
rather than being a specific issue of BOPP.

\subsection{Unbounded Bayesian Optimization via Non-Stationary Mean Function Adaptation}
\label{sec:bopp:unbounded}

Unlike standard BO implementations, BOPP is not provided with external constraints and we therefore 
develop a scheme for operating on targets with potentially unbounded support.  For MMAP estimation,
the target function is an unnormalized density, implying that the area that must 
be searched in practice to find the optimum is finite.  For MML estimation and risk minimization this
assumption is still reasonable in practice as if it is not true, we are effectively doomed to fail anyway.
We, therefore, exploit this assumption by defining a non-stationary prior mean function.  
This takes the form of a bump function that is constant within a region of interest, but decays rapidly 
outside.  Specifically we define this bump function in the transformed space as
\begin{align}
\label{eq:BUMP}
\mu_{\mathrm{prior}}\left(r;r_e,r_{\mathrm{\infty}}\right) = \begin{cases} 0  \hfill & \mathrm{if} \; r \leq r_{\mathrm{e}} \\ 
\log \left(\frac{r-r_{\mathrm{e}}}{r_{\mathrm{\infty}}-r_{\mathrm{e}}}\right)+\frac{r-r_{\mathrm{e}}}{r_{\mathrm{\infty}}-r_{\mathrm{e}}} & \mathrm{otherwise} \end{cases}
\end{align}
where $r$ is the radius from the origin, $r_e$ is the maximum radius of any point generated 
in the initial scaling or subsequent evaluations, and $r_{\mathrm{\infty}}$ is a parameter 
set to $1.5 r_{e}$ by default.  Consequently, the acquisition function also decays and new points 
are never suggested arbitrarily far away.  Adaptation of the scaling will automatically update this 
mean function appropriately, learning a region of interest that matches that of the true problem, 
without complicating the optimization by over-extending this region.  We note that our method 
is very similar to the independently developed work of \cite{shahriari2016unbounded}, but overcomes the 
sensitivity of their method upon a user-specified bounding box representing soft constraints, 
by initializing automatically and adapting as more data is observed.

An important consequence of this approach is that BOPP is not always an entirely
global optimizer as the adaptation can, at least in theory, become stuck around a single mode if
their is extreme prior-target mismatch.  Specifically, because only ``bad'' points 
are not incorporated into the rescaling as described in the last
section, it may be that region where the target is low blocks expansion to another mode.
In practice, such occurrences should be extremely rare (at least for MMAP estimation) as the
initial scaling is approximately set to the region where the generative model has reasonable density, such that
the problem would need to be both multi-modal and have extreme prior-target mismatch for BOPP to
get stuck.  One could in theory refine our method to provide better guarantees against such occurrences,
but given the inherent difficulty of such problems and the fact that BOPP, like other GP-based BO methods,
is heavily restricted in the number of iterations before the GP training cost becomes
prohibitive (usually in the hundreds of iterations), doing so seems more likely in practice to do harm than good.
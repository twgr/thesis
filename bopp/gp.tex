% !TEX root = bopp.tex

Remembering that the domain scaling introduced in Section~\ref{sec:domain} means that both the input and outputs of the GP are taken to vary between $\pm1$, we define the problem independent GP hyperprior as $p(\alpha)=p(\sigma_n)p(\sigma_{3/2})p(\sigma_{5/2})\prod_{i=1}^{D}p(\rho_i)p(\varrho_i)$ where
\begin{subequations}
	\begin{align}
	\label{eq:hyperPriorDef}
	\log \left(\sigma_n\right) & \sim \mathcal{N} \left(-5,2\right) \\
	\log\left(\sigma_{3/2}\right) & \sim \mathcal{N} \left(-7,0.5\right)\\
	\log\left(\sigma_{5/2}\right) & \sim \mathcal{N} \left(-0.5,0.15\right)\\
	\log \left(\rho_i\right) & \sim \mathcal{N} (-1.5,0.5) \quad \forall i \in \{1,\dots,D\}\\
	\log\left(\varrho_i\right) & \sim \mathcal{N} \left(-1,0.5\right) \quad \forall i \in \{1,\dots,D\}.
	\end{align}
\end{subequations}
The rationale of this hyperprior is that the smoother Mat\'{e}rn 5/2 kernel should be the dominant effect and model the higher length scale variations. The Mat\'{e}rn 3/2 kernel is included in case the evidence suggests that the target is less smooth than can be modelled with the Mat\'{e}rn 5/2 kernel and to provide modelling of smaller scale variations around the optimum.
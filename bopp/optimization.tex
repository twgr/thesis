% !TEX root =  bopp.tex

In addition to the syntax introduced in the previous section, there are five main components to BOPP:
\begin{itemize}
	\setlength\itemsep{0.1em}
	\item[-] A program transformation, \texttt{q}$\rightarrow$\qmarg, allowing estimation of the evidence $p(Y,\theta)$ at a fixed $\theta$.
	\item[-] A high-performance, GP based, BO implementation for actively sampling $\theta$.
	\item[-] A program transformation, \texttt{q}$\rightarrow$\qprior,  used for automatic and adaptive domain scaling, such that a problem-independent hyperprior can be placed over the GP hyperparameters.
	\item[-] An adaptive non-stationary mean function to support unbounded optimization.
	\item[-] A program transformation, \texttt{q}$\rightarrow$\qacq, and annealing maximum likelihood estimation method to optimize the acquisition function subject the implicit constraints imposed by the generative model.
\end{itemize}
Together these allow BOPP to perform online MMAP estimation for arbitrary programs in a manner that is black-box from the user's perspective - requiring only the definition of the target program in the same way as existing PPS and identifying which variables to optimize.  The BO component of BOPP is both probabilistic programming and language independent, and is provided as a stand-alone package.\footnote{Code available at~\href{http://www.github.com/probprog/deodorant/}{\url{http://www.github.com/probprog/deodorant/}}}  It requires as input only a target function, a sampler to establish rough input scaling, and a problem specific optimizer for the acquisition function that imposes the problem constraints.  %We first provide a high-level overview of the algorithm before separately explaining these components.

%BOPP provides online MMAP estimation for arbitrary programs in a manner that is black-box from the user's perspective - requiring only the definition of the target program in the same way as existing PPS and identifying which variables to optimize. It has three main components: a series of program transformations, inference schemes for evaluating these transformed programs, and a BO scheme that uses them to provide the required MMAP estimation.  Implementation of the transformations is naturally language specific, but the required techniques can be applied to any system with general-purpose languages for model specification and which provides the required inference schemes.  Given functions for evaluating these transformed programs, the BO scheme for MMAP estimation can be abstracted from probabilistic programming and is provided as its own separate package\footnote{\url{http:\\www.bitbucket.org\twgr\bo-mapp}}.  This package requires three things: a target function which provides estimates of the marginal $p(Y,\theta)$, a sampler for cheaply generating a rough representation of the input scaling, and an optimizer for the acquisition function that imposes the constraints of the problem.

Figure \ref{fig:bopp_overview} provides a high level overview of the algorithm invoked when \doopt is called on a query \texttt{q} that defines a distribution $p\left(Y, a, \theta , b\right)$.  We wish to optimize $\theta$ whilst marginalizing out $a$ and $b$, as indicated by the the second input to \texttt{q}. In summary, BOPP performs iterative optimization in 5 steps

\begin{figure}[t]
	\centering
	\includegraphics[width=\textwidth]{"bopp_overview_figure"}
	%\vspace{20pt}
	\caption{
		\label{fig:bopp_overview}
		Overview of the BOPP algorithm, description given in main text. \texttt{p-a}, \texttt{p-$\theta$}, \texttt{p-b} and \texttt{lik} all represent distribution object constructors. \factor is a special distribution constructor that assigns probability $p(y) = y$, in this case $y = \zeta(\theta)$.}
\end{figure}

%where \footnote{Anglican's \defdist construct makes it possible to specify arbitrary code, including code external to Anglican, to be called when \sample or \observe is invoked, e.g. as used to define \abcl in Figure \ref{fig:house-heating-code}.}, and which 

\begin{itemize}
	\item[-] Step 1 (blue arrows) generates unweighted samples from the transformed prior program \texttt{q-prior} (\emph{top center}), constructed by removing all conditioning. This initializes the domain scaling for $\theta$.
	\item[-] Step 2 (red arrows) evaluates the marginal $p(Y,\theta)$ at a small number of the generated $\hth$ by performing inference on the marginal program \qmarg~ (\emph{middle centre}), which returns samples from the distribution $p\left(a,b | Y, \theta\right)$ along with an estimate of $p(Y, \theta)$.  The evaluated points (\emph{middle right}) provide an initial domain scaling of the outputs and starting points for the BO surrogate.
	\item[-] Step 3 (black arrow) fits a mixture of GPs posterior \cite{rasmussen2006gaussian} to the scaled data (\emph{bottom centre}) using a problem independent hyperprior. The solid blue line and shaded area show the posterior mean and $\pm2$ standard deviations respectively. The new estimate of the optimum $\hth^*$ is the value for which the mean estimate is largest, with $\hat{u}^*$ equal to the corresponding mean value.    
	\item[-] Step 4 (purple arrows) constructs an acquisition function $\zeta \colon \vartheta \rightarrow \real^+$ (\emph{bottom left}) using the GP posterior.  This is optimized, giving the next point to evaluate $\hth_{\mathrm{next}}$, by performing annealed importance sampling on a transformed program \texttt{q-acq} (\emph{middle left}) in which all \observe statements are removed and replaced with a single \observe assigning probability $\zeta(\theta)$ to the execution. % A non-stationary prior mean function for the GP, the AF is penalized away from a region of interest, allowing unbounded optimization.  
	%The AF is optimized by performing annealed importance sampling on a transformed program \texttt{q-acq} (\emph{middle left}) in which all \observe statements are removed and replaced with a single \observe that assigns probability $\zeta(\theta)$ to the execution. 
	\item[-] Step 5 (green arrow) evaluates $\hth_{\mathrm{next}}$ using \qmarg~and continues to step 3.
\end{itemize}

\subsection{Program Transformation to Generate the Target}
\label{sec:transform}
% !TEX root =  bopp.tex

Consider the \defopt query \texttt{q} in Figure \ref{fig:bopp_overview}, the body of which defines the joint distribution $p\left(Y,a,\theta,b\right)$.   Calculating \eqref{eq:MMAP} (defining $X=\left\{a,b\right\}$) using a standard optimization scheme presents two issues: $\theta$ is a random variable within the program rather than something we control and its probability distribution is only defined conditioned on $a$.

We deal with both these issues simultaneously using a program transformation similar to the disintegration transformation in Hakaru \citep{zinkov2016composing}. Our \emph{marginal} transformation returns a new \query object, \qmarg~ as shown in Figure~\ref{fig:bopp_overview}, that defines the same joint distribution on program variables and inputs, but now accepts the value for $\theta$ as an input.  This is done by replacing all \sample statements associated with $\theta$ with equivalent \observes statements, taking $\theta$ as the observed value, where \observes is identical to \observe except that it returns the observed value.  As both \sample and \observe operate on the same variable type - a distribution object - this transformation can always be made, while the identical returns of \sample and \observes trivially ensures validity of the transformed program.  


%We now build upon our optimization query to demonstrate how BOPP can optimize with respect to an arbitrary subset of variables sampled within a PP.  This is equivalent to optimizing with respect to an arbitrary subset of nodes in a graphical model, whilst marginalizing over the others, representing a new method beyond the scope of current BO algorithms.


%Consider the Anglican query \texttt{q} in figure \ref{fig:originalQuery} as a demonstrative example.  The marginal distribution defined by \texttt{q}, $p\left(Y,\theta\right) = \int_{U} \int_{V} p\left(U\right)p\left(\theta|U\right)p\left(V|\theta,U\right)p\left(Y|V,\theta,U\right)dUdV$, is the same objective function as in~\eqref{eq:hyperOpt} if we define $X= \left\{U,V\right\}$, but $\theta$ is no longer at the root of the dependency structure as it was in \eqref{eq:Joint}.  This causes two problems for optimizing with respect to $\theta$: it is sampled within the program and the corresponding probability distribution is only defined conditioned on one of the parameters we wish to marginalize over $U$.  

%We propose dealing with both these issues simultaneously using a program transformation by which we change any \sample statements for elements of $\theta$ into \observes statements, resulting in the transformed query \texttt{qT} shown in \ref{fig:transformedQuery}.  Here \observes is identical to \observe except that its return value is equal to its observation, in this case $\theta$.  The transformed query is a function of $\theta$ and can therefore be optimized.  When \doquery is called on \texttt{q} with the BOPP algorithm specified as the inference engine, it acts a macro which first makes this transformation before passing the transformed program to our BO wrapper.

%At a high level, the result of this transform is that we use use the defined probability distribution for sampling $\theta$ to condition the program to a particular value of $\theta$.  Critically, the distribution defined by the program has not changed.  This is easiest to assert by considering the program as defining a joint density on the sampled variables and the observations, and noting that whether these variables are fixed or sampled at runtime does not change the definition of this joint.  This simple but elegant solution means that we can transform any probabilistic program, and therefore any graphical model, to an optimization problem with respect to any of its sampled variables. 





%\subsection{Marginal Maximum A Posteriori Estimation}
%% !TEX root =  bopp.tex

%In this section we introduce a set of requirements for an ``optimization query", which returns an infinite lazy sequence of increasingly optimal estimates for some target variables $\theta \in \vartheta$.  For exposition purposes, we first consider the case where $\theta$ correspond to the inputs of a query $q$ and show how this can be extended to arbitrary variables within the program in section \ref{sec:transform}. We assume $q$ takes as inputs, along with $\theta$ data upon which the query is conditioned $Y$.


%As it is only possible to estimate $p(Y, \hth_m)$ such that
%\begin{align}
%\label{eq:BOPPoutput}
%E_f\left[\hat{p} \left(Y,\hth_m\right) | D_{m} \right] \ge E_f\left[\hat{p} \left(Y,\hth_j\right) | D_{m} \right] \quad \forall j=1,\dots,m-1
%\end{align}
%where $\hat{p}$ is used to indicate that the estimation of the marginal probability is itself probabilistic due to the approximation nature of inference. 

Given the above program transformation we can use a generic inference method provided by the back end to marginalize over the latent variables $X$ conditioned on $\theta$. We will here use sequential Monte Carlo for probablistic programs \citep{wood2014new} to obtain unnormalized estimates of the marginal conditional likelihood
\begin{align}
\hw \left(Y,\theta\right) \approx p\left(Y | \theta\right) =\int p\left(X,Y|\theta\right) dX.
\end{align}
Given these estimates we are now in a position to define the problem setting for MMAP estimation in probabilistic programs. Specifically we will define a macro \lsi{doopt} that accept a query defined using \lsi{defopt} and returns a lazy sequence of increasingly optimal estimates for the target variables $\theta$. We now formally define our optimization query to output an infinite lazy sequence $\{\hth_1,\hat{\Omega}_1\},\{\hth_2,\hat{\Omega}_2\},\dots$ where $\hat{\Omega}_i$ is the map of \predict values with the query when $\theta=\hth_i$ and
\begin{align}
\label{eq:BOPPoutput}
E\left[\hw \left(Y,\hth_m\right) p\left(\hth_m\right) | D_{m} \right] \ge E\left[\hw \left(Y,\hth_j\right) p\left(\hth_j\right) | D_{m} \right] \quad \forall j=1,\dots,m-1 \quad m=1,2,\dots
\end{align}
where the expectation is over the surrogate function posterior. $\hth_m$ corresponds to the point that is expected to be the most optimal of those evaluated under the posterior of our surrogate function. Since evaluations of are noisy, this need not be the $\theta$ value that produced the highest the $p(\theta)$-weighted marginal likelihood estimate. % Further as the observation of a new point affects the surrogate function posterior at all other points (as the expectation of both sides of~\eqref{eq:BOPPoutput} is conditioned on all data observed so far $D_m$), $\hth_m$ can change between different historical values when a new point is queried.


% Consider a generic query $q$.  Let the \sample statements within the $q$ define a generative distribution for a set of latent variables $X = \left\{x_{i}\right\}_{i=1,\dots,N}, \; X \in \mathcal{X}$ (note $x_i$ may have different support for different $i$) with prior $p\left(X | \theta\right) = p\left(x_1 | \theta\right) \prod_{i=2}^{N} p\left(x_i | x_1,\dots,x_{i-1},\theta\right)$, parametrized by a set of program inputs $\theta \in \vartheta$.  Let the \observe statements in the program define conditioning on observations $Y = \left\{y_i\right\}_{i=1,\dots,N}, \; Y \in \mathcal{Y}$ such that the query defines the joint factorization\footnote{Note, there is notational deficiency as in a higher-order PPS variable types, the order of the conditioning for the latent variables and even the number of latent variables can change depending on the program trace.}
% \begin{multline}
% \label{eq:Joint}
% p\left(X,Y|\theta\right) = p\left(x_1 | \theta\right) p\left(y_1 | x_1, \theta\right) \\ \prod_{i=2}^{N} p\left(y_i | x_1,\dots,x_{i},\theta\right) p\left(x_i | x_1,\dots,x_{i-1},\theta\right).
% \end{multline}
% We assume that the observations $Y$ are fixed and finite dimensional.  Our aim is to optimize the marginal likelihood of this joint scaled by a prior on $\theta$:
% \begin{align}
% \label{eq:hyperOpt}
% \theta^* = \argmax_{\theta \in \vartheta} p\left(\theta\right) \int_{X}^{} p\left(X,Y|\theta\right) dX,
% \end{align}

%Often the prior on $\theta$ will often correspond only to a set of bounds, giving a uniform distribution within the permissible input space.  If $p\left(\theta\right)$ is allowed to be potentially improper,~\eqref{eq:hyperOpt} also incorporates maximum likelihood estimation .  
%restricting the choice of inference algorithm. Anglican supports a number of suitable algorithms including importance sampling \citep{glynn1989importance}, sequential Monte Carlo (SMC) \citep{smith2013sequential,wood_aistats_2014} and the particle cascade \citep{paige2014asynchronous}.


For clarity we introduce the following notation of the rest of the paper.  We use $\theta_m$ to refer to the $\theta$ used to call the query at iteration $m$, and $\Omega_m$ and $W_m$ for the predicts and marginal likelihood estimate from this call respectively.  We define $\jsm \in \{1,\dots,m\}$ to be the index corresponding to the estimated best $\theta_m$ at iteration $m$ such that $\hth_m = \theta_{\jsm}$.  We further define $Z_m \coloneqq W (Y,\theta_j) p(\theta_j)$ and $\hz_m \coloneqq \hw (Y,\hth_j) p(\hth_j)$ as the corresponding estimates of the weighted marginal weights.

%We finally note that our optimization query includes as a special case independent calls to an inference query by setting $\ell (\cdot) = 0$ and by convention taking the most recent sample under equality of~\eqref{eq:BOPPoutput}.  Furthermore, one is free to choose the sequence of $\tilth$ in anyway desired.  For example, one may wish to explicitly control the trade off between improving our estimates for $\theta^*$, and refining the inference of the latent variables $p(z | y, \tilth_{\jsm})$ by recalling the original query with the same $\theta$.


\subsection{Bayesian Optimization of the Marginal}
\label{sec:BOPP}

% !TEX root =  bopp.tex

The target function for our BO scheme is $\log p(Y,\theta)$, noting $\argmax f\left(\theta\right) = \argmax \log f\left(\theta\right)$ for any $f : \vartheta \rightarrow \real^+$.  The log is taken because GPs have unbounded support, while $p\left(Y,\theta\right)$ is always positive, and because we expect variations over many orders of magnitude.  PPS with importance sampling based inference engines, e.g. sequential Monte Carlo \citep{wood2014new} or the particle cascade \citep{paige2014asynchronous}, can return noisy estimates of this target given the transformed program \qmarg.   

%By coupling \qmarg~with an existing inference engine that returns a marginal likelihood estimate, i.e.  we can form a function for evaluating (noisy) estimates of $p(Y,\theta)$ at a given $\theta$.  This provides the required target function for a BO scheme and its optimization will provide the required MMAP estimate as defined by~\eqref{eq:MMAP}.

%Given this transformation, we can convert any probabilistic program, and therefore any graphical model, to an optimization problem for the marginal probability with respect to any subset of the sampled variables, marginalizing out the rest.  We can evaluate the resulting target function at a given $\theta$ using any existing inference engine that returns a (noisy) estimate of the log ML $\hw$, such as sequential Monte Carlo (default behaviour used in all experiments), the particle cascade \citep{paige2014asynchronous} or importance sampling, and use GP based BO to actively sample the queried $\theta$.

Our BO scheme uses a GP prior and a Gaussian likelihood.  Though the rationale for the latter is predominantly computational, giving an analytic posterior, there are also theoretical results suggesting that this choice is appropriate \citep{berard2014lognormal}. We use as a default covariance function a combination of a Mat\'{e}rn-3/2 and Mat\'{e}rn-5/2 kernel.  Specifically, let $D = \lVert \theta \rVert_0$ be the dimensionality of $\theta$ and define
\begin{subequations}
	\begin{align}
	\label{eq:d_def}
	d_{3/2}(\theta,\theta') &= \sqrt{\sum_{i=1}^{D} \frac{\theta_i-\theta_i'}{\rho_i}} \displaybreak[0] \\
	d_{5/2}(\theta,\theta') &= \sqrt{\sum_{i=1}^{D} \frac{\theta_i-\theta_i'}{\varrho_i}}
	\end{align}
\end{subequations}
where $i$ indexes a dimension of $\theta$ and $\rho_i$ and $\varrho_i$ are dimension specific length scale hyperparameters. Our prior covariance function is now given by
\begin{align}
\label{eq:kprior}
\begin{split}
k_{\text{prior}}\left(\theta,\theta'\right) = & \sigma_{3/2}^2 \left(1+\sqrt{3}d_{3/2}\left(\theta,\theta'\right)\right)\exp\left(-\sqrt{3}d_{3/2}\left(\theta,\theta'\right)\right) +\\&\sigma_{5/2}^2 \left(1+\sqrt{5}d_{5/2}\left(\theta,\theta'\right)+\frac{5}{3}(d_{5/2}\left(\theta,\theta'\right))^2\right)\exp\left(-\sqrt{5}d_{5/2}\left(\theta,\theta'\right)\right) 
\end{split}
\end{align}
where $\sigma_{3/2}$ and $\sigma_{5/2}$ represent signal standard deviations for the two respective kernels.  The full set of GP hyperparameters is defined by $\alpha = \{\sigma_n,\sigma_{3/2},\sigma_{5/2},\rho_{i=1:D},\varrho_{i=1:D}\}$.  A key feature of this kernel is that it is only once differentiable and therefore makes relatively weak assumptions about the smoothness of $f$.  The ability to include branching in a probabilistic program means that, in some cases, an even less smooth kernel than~\eqref{eq:kprior} might be preferable.  However, there is clear a trade-off between generality of the associated reproducing kernel Hilbert space and modelling power.

As noted by \citep{snoek2012practical}, the performance of BO using a single GP posterior is heavily influenced by the choice of these hyperparameters.  We therefore exploit the automated domain scaling introduced in Section~\ref{sec:domain} to define a problem independent hyperprior $p(\alpha)$ and perform inference to give a mixture of GPs posterior.  Details on this hyperprior are given in Appendix~\ref{sec:app:hyperprior}.

Inference over $\alpha$ is performed using Hamiltonian Monte Carlo (HMC) \citep{duane1987hybrid}, giving an unweighted mixture of GPs.  Each term in this mixture has an analytic distribution fully specified by its mean function $\mu_m^i \colon \vartheta \rightarrow \real$ and covariance function $k_m^i \colon \vartheta \times \vartheta \rightarrow \real$, where $m$ indexes the BO iteration and $i$ the hyperparameter sample.  HMC was chosen because of the availability of analytic derivatives of the GP log marginal likelihoods.  As we found that the performance of HMC was often poor unless a good initialization point was used, BOPP runs a small number of independent chains and allocates part of the computational budget to their initialization using a L-BFGS optimizer \citep{broyden1970convergence}. 

The inferred posterior is first used to estimate which of the previously evaluated $\hth_j$ is the most optimal, by taking the point with highest expected value
%\footnote{One could trivially expand the algorithm beyond MMAP by instead using the approriate linear combination of the mean and marginal standard deviation, for example maximizing a lower confidence bound.}
, $\hat{u}^*_m = \max_{j\in1\dots m} \sum_{i=1}^{N} \mu_{m}^i (\hth_j)$.  This completes the definition of the output sequence returned by the \doopt macro.  Note that as the posterior updates globally with each new observation, the relative estimated optimality of previously evaluated points changes at each iteration.
% and $\hat{u}^*_a > \hat{u}^*_b$ is possible for $a<b$, even though with the extra information gather by iteration $b$ we believe $\hth_b^*$ is more optimal that $\hth_a^*$.
Secondly it is used to define the acquisition function $\zeta$, for which we take the expected improvement \citep{snoek2012practical}, defining $\sigma_m^i\left(\theta\right) = \sqrt{k_m^i\left(\theta,\theta\right)}$ and $\gamma_m^i\left(\theta\right) = \frac{\mu_m^i \left(\theta\right)-\hat{u}_m^*}{\sigma_m^i\left(\theta\right)}$,
\begin{align}
\label{eq:exp-imp}
\zeta \left(\theta\right) = \sum_{i=1}^{N} \left(\mu_m^i\left(\theta\right)-\hat{u}_m^*\right)\Phi \left(\gamma_m^i\left(\theta\right)\right)+\sigma_m^i\left(\theta\right)\phi\left(\gamma_m^i\left(\theta\right)\right)
\end{align}
where $\phi$ and $\Phi$ represent the pdf and cdf of a unit normal distribution respectively.   We note that more powerful, but more involved, acquisition functions, e.g. \citep{hernandez2014predictive}, could be used instead.



\label{sec:bopp-for-ml}

% !TEX root =  ../main.tex

\subsection{Automatic and Adaptive Domain Scaling}
\label{sec:bopp:domain}

Domain scaling, by mapping to a common space, is crucial for BOPP to operate in the required black-box fashion as it allows a general purpose and problem independent hyperprior to be placed on the GP hyperparameters.  BOPP, therefore, employs an affine scaling to a $[-1,1]$ hypercube for both the inputs and outputs of the GPs.  To initialize scaling for the input variables, we sample directly from the generative model defined by the program. %\footnote{Note that Anglican's ability to include statements for conditioning on generated variables, for example to truncate distributions, means this does not always represent $p(\theta)$ and is only a prior in a more abstracted sense.}
This is achieved using a second transformed program, \qprior, which removes all conditioning, i.e. \observe statements, and returns $\theta$.  This transformation also introduces code to terminate execution of the query once all $\theta$ are sampled, in order to avoid unnecessary computation.
As \observe statements return \lsi{nil}, this transformation trivially preserves the generative model of the program, 
but the probability of the execution changes. Specifically, if we denote $n_{\theta}$ as the number of non-target \sample 
statements that have been invoked by the time all $\phi_{1:L}$ are sampled, then \qprior more formally defines the
\emph{unconditional} distribution $p_{\lambda}(\mT = \{\phi_{1:L},x_{1:n_{\theta}}\}) \propto 
\gamma_{\text{prior}}(\theta,x_{1:n_{\theta}},\lambda)$ where
\begin{align}
\label{eq:bopp:qprior}
\gamma_{\text{prior}}(\theta,x_{1:n_{\theta}},\lambda)= \begin{cases}
\prod_{\ell=1}^{L}
h_{\ell,c_{\ell}} (\phi_{\ell} | \xi_{\ell})
\prod_{j=1}^{n_{\theta}} 
f_{a_j}(x_j | \eta_j) \;\;\; \text{if} \;\;\; \mathcal{B}(\theta,x_{1:n_\theta},\lambda)=1 \\
0 \quad \text{otherwise}
\end{cases}
\end{align}
and the trace validity function $\mathcal{B}(\theta,x_{1:n_\theta},\lambda)$ is redefined appropriately.
Because~\eqref{eq:bopp:qprior} is an unconditional distribution, it can be sampled from directly by
running the program forwards, returning exact samples from the corresponding marginal distribution on $\theta$.
This is computationally inexpensive, as it does not require inference or calling potentially expensive 
likelihood functions.  It can thus be cheaply sampled from a number of times to initialize the input scaling.
By further running inference on \qmarg~given a small number of these samples as arguments, a rough initial characterization of output scaling can also be achieved.

If points are later observed that fall outside the hypercube under this initial scaling, the domain scaling 
is appropriately updated so that the target for the GP remains the $[-1,1]$ hypercube.  
An important exception to this is that the output mapping to the bottom of the hypercube remains 
fixed and any points with partition function estimates lower than this are not incorporated into the scaling in any way,
i.e. the input scaling is not updated to incorporate these points either.
For MMAP estimation, this ensures stability for unbounded problems as there can only be a finite region
of the input space where the true value of the partition function is above any given value because its integral
over $\theta$ must be finite.  Similarly, the
maximum possible estimate the inference algorithm might return will be bounded
given some weak assumptions (roughly that $p(Y,X,\theta)$ is itself bounded).
Consequently, the fixed base of the hypercube ensures that the 
there is a maximum possible size the hypercube can reach.
For risk minimization (where our target is $-\log p(Y|\theta)$) and  MML estimation
(where our target is $\log p(Y|\theta)$) problems then we have no such guarantee that the adaptation will 
eventually cease.  However, this is somewhat inherent to unbounded global optimization problems,
rather than being a specific issue of BOPP.

\subsection{Unbounded Bayesian Optimization via Mean Function Adaptation}
\label{sec:bopp:unbounded}

Unlike standard BO implementations, BOPP is not provided with external constraints and we, therefore, 
develop a scheme for operating on targets with potentially unbounded support.  For MMAP estimation,
the target function is an unnormalized density, implying that the area that must 
be searched in practice to find the optimum is finite.  For MML estimation and risk minimization this
assumption is still reasonable in practice as if it is not true, we are effectively doomed to fail anyway.
We, therefore, exploit this assumption by defining a non-stationary prior mean function.  
This takes the form of a bump function that is constant within a region of interest, but decays rapidly 
outside.  Specifically, we define this bump function in the transformed space as
\begin{align}
\label{eq:BUMP}
\mu_{\mathrm{prior}}\left(r;r_e,r_{\mathrm{\infty}}\right) = \begin{cases} 0  \hfill & \mathrm{if} \; r \leq r_{\mathrm{e}} \\ 
\log \left(\frac{r-r_{\mathrm{e}}}{r_{\mathrm{\infty}}-r_{\mathrm{e}}}\right)+\frac{r-r_{\mathrm{e}}}{r_{\mathrm{\infty}}-r_{\mathrm{e}}} & \mathrm{otherwise} \end{cases}
\end{align}
where $r$ is the radius from the origin, $r_e$ is the maximum radius of any point generated 
in the initial scaling or subsequent evaluations, and $r_{\mathrm{\infty}}$ is a parameter 
set to $1.5 r_{e}$ by default.  Consequently, the acquisition function also decays and new points 
are never suggested arbitrarily far away.  Adaptation of the scaling will automatically update this 
mean function appropriately, learning a region of interest that matches that of the true problem, 
without complicating the optimization by over-extending this region.  We note that our method 
is very similar to the independently developed work of \cite{shahriari2016unbounded}, but overcomes the 
sensitivity of their method upon a user-specified bounding box representing soft constraints, 
by initializing automatically and adapting as more data is observed.

An important consequence of this approach is that BOPP is not always an entirely
global optimizer as the adaptation can, at least in theory, become stuck around a single mode if
there is extreme prior-target mismatch.  Specifically, because ``bad'' points 
are not incorporated into the rescaling as described in the last
section, we might have a ``bad'' region blocking expansion to another mode.
In practice, such occurrences should be extremely rare (at least for MMAP estimation) as the
initial scaling is approximately set to the region where the generative model has significant density, such that
the problem would need to be both multi-modal and have extreme prior-target mismatch for BOPP to
get stuck.  One could, in theory, refine our method to provide better guarantees against such occurrences,
but given the inherent difficulty of such problems and the fact that BOPP, like other GP-based BO methods,
is heavily restricted in the number of iterations before the GP training cost becomes
prohibitive (usually in the hundreds of iterations), doing so seems more likely to do harm than good in practice.

% !TEX root =  bopp.tex

\subsection{Optimizing the Acquisition Function}
\label{sec:optacqfunc}

Optimizing the acquisition function for BOPP presents the issue that the query contains implicit constraints that are unknown to the surrogate function.  The problem of unknown constraints has been previously covered in the literature \citep{gardner2014bayesian,hernandez2015general} by assuming that constraints take the form of a black-box function which is modeled with a second surrogate function and must be evaluated in guess-and-check strategy to establish whether a point is valid. Along with the potentially significant expense such a method incurs, this approach is inappropriate for \emph{equality} constraints or when the target variables are potentially discrete.  For example, the Dirichlet distribution in Figure~\ref{fig:house-heating-code} introduces an equality constraint on \lsi{powers}, namely that its components must sum to $1$.

We therefore take an alternative approach based on directly using the program to optimize the acquisition function.  To do so we consider a transformed program \lsi{q-acq} that is identical to \lsi{q-prior} (see Section \ref{sec:domain}), but adds an additional \observe statement that assigns a weight $\zeta(\theta)$ to the execution.  By setting $\zeta(\theta)$ to the acquisition function, the maximum likelihood corresponds to the optimum of the acquisition function subject to the implicit program constraints.  %Critically, this evaluation does not require inference and so can be evaluated cheaply.
We obtain a maximum likelihood estimate for \lsi{q-acq} using a variant of the simulated annealing algorithm 
discussed in Section~\ref{sec:bopp:related} in which RMH (see Section~\ref{sec:proginf:str:lmh}) is used for
the transition kernel.  Using an RMH, rather than LMH, transition kernel here is important as it will often
be the case that the optimum of the transition function is in a position of very low prior mass (see e.g. Figure~\ref{fig:domainAdpat}),
such that LMH could take an unreasonably long time to propose an appropriate set of parameters.
%The latter of these, which we refer to as random-walk Metropolis Hastings (RMH), is made possible by examining the type of the relevant distribution object at runtime to generate an appropriate local proposal kernel given the distribution type.

%Our final transformation generates a program, \qacq, which is identical to \qprior, except for adding an additional \observe statement that assigns a weight $\zeta(\theta)$ to the execution, where $\zeta$ is a function provided as an input.  By setting $\zeta(\theta)$ to the acquisition function and using a maximum likelihood algorithm to optimize the program, the optimum of the acquisition function subject to the implicit program constraints can be found as detailed in Section \ref{sec:optacqfunc}.